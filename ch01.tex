\chapter{Peto}

\begin{quotation}
\noindent Ihmispedot kansoittavat yhteiskuntaamme.\newline ---Stefan Verstappen
\end{quotation}

\section{Ryöstetty pilotti}

\begin{tarina}

Keith puhuu miehelle huoneen nurkassa. Keith on vanha tuttuni, joten menen moikkaamaan häntä. Hän on levoton ja stressaantunut; ei ollenkaan oma itsensä. Hän on aina ollut rauhallinen mies, itsevarma ja hiljainen. Hän omistaa pienen Cessna-lentokoneen. Hän tekee pieniä kaupallisia keikkoja, kuten lentää turisteja Grand Canyonin yli. Kun hänelle jää käteen hiukan ekstraa, hän laittaa sen talteen. ``Vielä joku päivä ostan meille ikioman maatilan\vmq{,}'' hän sanoo minulle. Keith ja Alexis ja heidän eläkepäiviensä unelma.

``Hei, Keith, miten menee?'' kysyn häneltä. Toinen mies istuu vieressä sanomatta mitään. Tummat kiharat, tummennetut lasit. Hieno puku, raskas kultakello. Keith ravistaa päätään ja olkapäitään kiukkuisena. ``Hyvin! Menee ihan hyvin!'' hän sanoo minulle. ``Voisitko nyt mitenkään jättää meidät kahden? Jooko?''

Olen shokissa. En ole koskaan nähnyt häntä tällaisena. ``Totta kai, nähdään myöhemmin\vmq{,}'' sanon ja palaan takaisin pöytääni baarissa. Katselen heitä; he riitelevät. Mies kohauttaa olkiaan ja puhuu. Hän on hiljainen ja kova. Keith rauhoittuu ja lopettaa päänsä pudistelemisen. Nyt hän jo nyökkäilee. Pienoinen draama päättyy heidän kädenpuristukseen. Keith allekirjoittaa jonkinlaisen paperin, jonka mies taittelee, pujottaa takkinsa taskuun. Hän seisoo ja sitten lähtee. Odotan hetken, otan juomani ja istun Keithin viereen.

``Sopiiko istahtaa?'' kysyn häneltä jo istuttuani alas. Meidän oma running joke. Hän katsoo minuun nauramatta ja huokaisee. ``Mitä tuo oikein oli?'' kysyn. Olen edelleen ärsyyntynyt siitä, että hän komensi minut pois aikaisemmin. ``Ei mitään\vmq{,}'' hän sanoo. ``Bisnestä\vmq{,}'' hän tarkentaa ja vaihtaa puheenaihetta. Hän kyselee lapsistani ja juttelemme hetken. Hän on etäinen, laihempi, eikä hän ole ajanut partaansa. Minun tekee mieli kysellä lisää pukumiehestä. Tai ei sittenkään. Ei kannata heittää vettä myllyyn, eihän?

Tämä on viimeinen kerta kun näen Keithin. Kaksi viikkoa myöhemmin saan hautajaiskutsun Alexisilta, hänen leskeltänsä. Soitan hänelle välittömästi. Keith on kuollut. Hänen koneensa syöksyi alas. Ei matkustajia, vain hän. Olen sanaton. ``Olen niin pahoillani\vmq{.}'' En osaa sanoa mitään muuta. Keith?

Tutkija ei löydä koneesta teknistä vikaa. Taivas oli kirkas. Niinpä hän päättelee, että kyseessä oli itsemurha. Keith syöksyi maahan tarkoituksellisesti. Ei henkivakuutusta. Pahinta on, että Keith tyhjensi heidän yhteisen säästötilin aivan pari päivää ennen kuin olin nähnyt hänet. Yli \$180,000 katosi, ilman mitään selitystä.

\end{tarina}

\section{Psykopatia adaptaationa}

Jotkut tutkijat ovat esittäneet, että psykopatia olisi häiriön sijaan adaptaatio.\linkki{sfsd} Sain saman idean vuosia sitten ollessani tekemisissä nuoren hankalan naisen kanssa. Hän kävi terapiassa epävakaan persoonallisuushäiriön takia. Hän liikkui työpaikasta työpaikkaan, syyttäen aina muita. Hänen elämänsä oli pitkä tarina hyväksikäyttäviä vanhempia, ex-puolisoja ja ystäviä.

Hän eli kaaoksen ja emotionaalisen tuskan pilvessä. Kuitenkin ne olivat aina toiset ihmiset, joissa todellinen kärsimys esiintyi. Olipa tilanne mikä tahansa, hän onnistui löytämään tiensä ulos ja vetämään puoleensa uusia ystäviä. Jälkeensä hän jätti vaurioita ja traumoja. Hänellä oli lääkkeensä, joita hän harvoin otti. Hänellä oli terapeutti, jonka hän myöhemmin vietteli. ``Kärsijä'' sai aina sen, mitä hän halusi.

Ennen kaikkea hän oli \emph{petomainen} varmalla ja viattomalla tavalla. Minulla kesti kauan tajuta, että olin maannut psykopaatin kanssa. Hän piti yllään ``epävakaan persoonallisuuden'' maskia ollakseen menestyksekkäämpi uhri. Niihin aikoihin otin muistiinpanoja. Ne dokumentoivat matkani alas toiseen universumiin. Sosiaalisen kanssakäymisen ydinlait loistavat poissaoloaan suhteessa psykopaatin kanssa. Niiden tilalla kasvaa jotakin vierasta ja nälkäistä.

Moni ihminen tutkii psykopatiaa häiriönä. Halusin tutkia psykopatiaa adaptaationa, koska se sopii dataani paremmin. Lisäksi se näytti johtavan positiivisempiin ja hyödyllisempiin päätelmiin.

Psykopatian mallintaminen adaptaationa häiriön sijaan avaa oven uuteen maailmaan. Kysymyksemme vaihtuvat. Kysymme, mikä kamppailu on ajanut tätä kehitystä eteenpäin. Kysymme, millaisia adaptaatioita psykopaateilla tarkalleen ottaen on. Onko heillä pitemmät hampaat? Terävämmät kynnet? Vai onko heidän kykynsä hienovaraisempia?

Kysymme myös, millaisia vasta-adaptaatiota voi olla olemassa sosiaalisissa ihmisissä (ei-psykopaattisessa valtaosassa). Kysymme: ``Voisiko psykopattinen-sosiaalinen -suhde olla itseasiassa peto-saalis -tarina?'' Ja vastauksen havaitsemme olevan painokas ``Kyllä!'' Kysymme: ``Kuinka vanha tämä tarina on?'' ja vastaus on: ``Miljoonia vuosia\vmq{.}''

\section{Petomalli}

\emph{Draculan} kirjoittaja Bram Stoker maalasi psykopaatin pedoksi. Ehkä hänellä oli henkilökohtaista kokemusta. Tarinaa ei ole tarkoitettu kirjaimelliseksi totuudeksi. Se on metafora, ja hyvä sellainen. Dracula tulee paikalle yöllä, pukeutuneena tappamaan. Hän imee sinut kuiviin elämän verestä, vietellen sinut siinä samassa viehätysvoimallansa ja seksuaalisuudellansa.

Dracula ei tapa kokonaan. Sen sijaan hän muuttaa sinut heikoksi kopioksi itsestään. Hän on mahtava ja eläimellinen. Hän kykenee lukemaan ajatuksesi, silloinkin kun kompuroit yrittäessäsi paeta. Paras jännite on aina vampyyrin ja ihmisen välisessä taistossa, vampyyrien välisen konfliktin ollessa vain kirsikka kakun päällä.o

Tohtori Robert Hare alkoi kuvata psykopaatteja sosiaalisina petoina hänen vuoden 1994 \emph{Psycholology Today}-lehden virstanpylväsartikkelissaan.\linkki{sdfds} Aliotsikko kuuluu: \emph{This Charming Psychopath---How to spot social predators before they attack} \emph{(Tämä hurmaava psykopaatti---kuinka huomata sosiaaliset pedot ennen kuin ne hyökkäävät)}. Artikkelissa ja työssään hän keskittyy psykopaattien tunnistamiseen. Moni meistä tuntee hänen ``Psychopathy Checklist''-diagnoosityökalun.

Petojen ydinstrategia on saaliin pettäminen. Ihmispetojen ydinstrategia on uhrien huijaaminen. Kyse on samasta asiasta. Eläinmallit ovat oleellisia ihmiskäytöksen ymmärtämisessä ja ennustamisessa. Meidän on vaikea katsoa itseämme valehtelematta. Itseanalyysimme pysähtyy vapaan tahdon ja tietoisuuden konsepteihin. Emme kykene parantamaan niitä tai heittämään niitä romukoppaan, joten ne riivaavat meitä. Eläimiä taas osaamme analysoida ilman merkkiäkään vastaavista ongelmista.

Olen tässä kirjassa käyttänyt petomallia selkärankana, johon kaikki muut asiat ripustuvat. Aloitamme pedoista, jotka huijaavat tiensä läpi ihmisten sosiaalisen universumin. Kaikki muut asiat lähtevät tästä, ja ne käyvät järkeen siinä kontekstissa.

Ensimmäinen altistumiseni petomallille oli Stefan Verstappenin erinomainen työ \emph{Defence Against the Psychopath}.\linkki{sdfds} Se oli ensimmäinen lukemani teksti, joka esitti strategioita psykopaattien kohtaamiseksi ja päihittämiseksi. Olen valinnut saman polun tässä kirjassa.

\section{Jotakin muuta kuin tuttuja muurahaisia}

\begin{quotation}
\noindent Australiassa on hämähäkkejä, jotka tuoksuvat ja käyttäytyvät kuin muurahaiset: jotkut niistä ovat niin vakuuttavia, että muurahaiset antavat hämähäkin elää pysyvästi yhtenä heistä. Hämähäkki herkuttelee uusilla ystävillänsä, mutta se ei syö kaikkia muurahaisia, tai edes merkittävää osaa niistä; sen sijaan se louhii resurseja hitaasti, kestävästi, ajan kanssa.\newline---Daniel N. Jones, \emph{Snake in the grass}\linkki{sdfds}
\end{quotation}
Kyse ei ole vain jokusesta hämähäkistä. Tuhannet eri hyönteiset ovat hakkeroineet sisään muurahaisyhdyskuntaan tavalla tai toisella. Eräs toukka matkii muurahaiskuningattaren ääntä höynäyttääkseen työläisiä. \emph{Paussus}-kuoriainen syntyy, elää ja kuolee yhdyskunnan sisällä. Se ei pelkästään tuoksu oikealta (muurahaiset käyttävät hajuja ystävien erottamiseksi vihollisista), se myös matkii ääniä, joita muurahaiset päästelevät. Se kirjaimellisesti imitoi kuningatarmuurahaista sanoen ``Kaikki on hyvin'' muurahaisille ja toukille, silloinkin kun se hotkaisee niitä suihinsa.

Muurahaiset kehittyivät työskentelemään yhdessä kerääkseen ruokaa ja suojellakseen sitä varkailta. Ne jakavat työn, pitävät huolta jälkikasvusta yhdessä ja elävät suurissa yhdyskunnissa. Ne kommunkioivat ja ne ajattelevat kollektiivisesti. Muurahaisyhdyskunnan käytöksessä näkyy älykkyys.

Muurahaiset kukoistavat huolimatta niitä, niiden suojelumekanismeja ja niiden ruokaa jahtaavista loisista ja pedoista. Muurahaiset on yksi kaikista menestyneimmistä lajeista. Muurahaisilla on kieliä, heimoidentiteetti, sosiaalinen organisaatio, kyky työskennellä yhdessä. Nämä ovat adaptaatioita. On syytä kysyä, että \emph{minkä ongelmien ratkaisemiseksi ne ovat kehittyneet?} Vastaus näyttää olevan, että täsmälleen tuota huijareiden lauman kanssa pärjäämiseksi.

Muurahaiset aloittivat jakamalla kausittaisen ravinnonlähteen riskiä. Useampi muurahainen kykenee korjaamaan satoa laajemmalta alueelta kuin yksittäinen muurahainen. Onnekas muurahainen voi jakaa löytönsä onnettomien kanssa. Onneton muurahainen selviytyy bad spellistä. Ja niin muurahaisille kehittyi altruismi, mikä on hyvä ratkaisu, jos ravinnonlähde on riskialtis. Muita ratkaisuja ovat muuttaminen, horrostaminen ja synkronoidut parittelukaudet.

Mutta altruismilla on heikkoutensa. Se on huijaava käytös. Muurahaiste ravinnonlähde on auki jokaiselle joka sitä tarvitsee. Jos pääset sisään yhdyskuntaan, voit syödä muurahaisia, toukkia ja ruokaa tekemättä työtä. Joten altruistien täytyi joko kehittää puolustusta huijareiden varalle tai kuolla sukupuuttoon. Muurahaisyhdyskunnalle tämä tarkoittaa tunkeilijoiden tunnistamista ja tappamista. Altruismin geenit voivat selviytyä vain, mikäli ne samalla valvovat vastavuoroisuutta.

Kuten Daniel Jones kirjoittaa:\linkki{sdfds}
\begin{quotation}
\noindent Jotkut pedot ovat nopeita, liikkuvaisia ja wide-ranging, huijataksensa niin montaa uhria kuin mahdollista; ne muistuttavat ihmispsykopaatteja. Toiset ovat hitaita, vaanien petoansa erityisellä, strategisella (melkein Machiavellianisella) tavalla{\ldots} Useimpien elävien asioiden keskuudessa huijarin ja huijatun välillä on kilpavarustelutilanne, joka ei lopu koskaan.
\end{quotation}
Ja niinpä muurahaisille kehittyi hajuin, kosketksin ja äänin toimivia kieliä, jotta he kykenevät tunnistamaan toisensa. Huijareille kehittyi kyky imitoida näitä kieliä. Muurahaisten kielet muuttuivat hienostuneemmiksi. Huijarit kehittyivät paremmiksi. Ja niin edelleen, satojen miljoonien vuosien ajan, antaakseen meille tuon modernin muurahaisen. Yksi muurahaisperhe Argentiinasta peittää suuren osan maailmaa ``mannertenvälisenä superyhdyskuntana\vmq{.}'' Tämä superyhdyskunta on päällekäyvä, hallitseva, ja se ajaa paikalliset muurahaislajit pois. Portugalilainen muurahainen voi astua sisään muurahaispesään Uudessa-Seelannissa ja tulla hyväksytyksi.

Tämän pitäisi olla sinulle tuttua. Yhteistyöhenkistä altruismia esiintyy muissakin lajeissa. Termiitit, mehiläiset ja ampiaislajit kehittyivät pitkin samaa polkua. Samoin kehittyivät vampyyrilepakot, miekkavalaat ja ihmiset. Myös me muodostamme mannertenvälisen superyhdyskunnan, joka on päällekäyvä, hallitseva, ja joka usein käyttäytyy kuin yksi perhe.

Vuoden 2012 kirjassaan \emph{The Social Conquest of the Earth} Edward Wilson kuvasi ihmisiä aitososiaalisina ihmisapinoina. Työnjakomme, päällekäiset sukupolvet ja yhdessä tapahtuva jälkeläisistä huolehtiminen antavat meille ``supervoiman'' johon harva muu laji kykenee.

Ihmisyys ei kehittynyt Eedenin puutarhassa. Useat ilmastonvaihdokset moukaroivat meitä, yhä uudelleen ja uudelleen. Selvisimme monesta sukupuuttoa liippaavasta pullonkaulasta, vain muutaman tuhannen yksilön populaatioista, kerta toisensa jälkeen. Nämä tapahtumat eivät tappaneet meitä. Kuten Argentiinalaiset muurahaiset, mekin polveudumme yhdestä pienestä joukosta geneettisesti samankaltaisia ihmisiä. Tämä antaa meille kyvyn tunnistaa toisemme saman heimon jäseniksi.

Selviydyimme katastrofista katastrofin jälkeen työskentelemällä yhdessä. Kehitimme kyvyn siirtää tietämystä sukupolvelta toiselle. Kehitimme altruismin, kyvyn jakaa riskin heimojen ja sukupolvien kautta.

Aikaiset altruistit kärsivät monista huijareista: haaskansyöjistä, loisista ja ennenkaikkea toisista ihmisistä. Jokaista kehittämäämme sosiaalista vaistoamme kohtaan kehitimme kyvyn huijata toisia. Ja kun huijaukset kehittyivät paremmiksi, sosiaaliset ihmiset kehittyivät paremmiksi niiden tunnistamisessa ja rankaisemisessa.

Ihmiset muodostavat suhteiden verkkoja. Joskus ne ovat hierarkisia. Yleensä luomme siteitä yksilöihin ja ryhmiin. Nuo suhteet eivät ole mielivaltaisia. Ne rakentuvat huolellisen kirjanpidon varaan. Laskemme kauppoja geeneisssä, ruuassa, suojassa, seksissä, kiintymyksessä, informaatiossa, ajassa. Tämä kaikki on yleensä alitajuntaista. Se on myös jatkuvaa ja hallitsevaa.

Meillä on hienostuneet mentaaliset työkalut näiden suhteiden seuraamista varten. Kykenemme muistamaan kasvot koko elämän ajan. Muistamme hyvän ja pahan yksityiskohtaisesti. Voimme arvata minkä tahansa palveluksen tai tavaran suhteellisen arvon annetussa paikassa tai ajassa. Se paahdettu kana jonka jaoit kanssani lounaaksi tarkoittaa kolmea olutta huomenna, tai yhtä kahden viikon aikana. Muistamme huijaukset ikuisuuden, emmekä anna niille anteeksi.

Meillä on mielikuvituksemme, joten voimmme suunnitella, kuinka työskennellä yhdessä. Meillä on kieli, jotta voimme vaihtaa tietämystä. Ilmaisemme tunteitamme kasvoillamme, äänillämme, kehonkielellämme, ja verisellä punastumisela kasvoissamme, korvissamme ja kehoissamme.

Kaikki nämä ovat adaptaatioita huijauksilta puolustautumiseksi. Aivan kuten muurahaisyhdyskunta on kilpavarustelun tulos, niin on ihmisten yhteiskuntakin. Se, mitä olemme, polveutuu loputtomasta sodasta yhtesityön ja lupauksen ``shekki on postissa!'' välillä.

\section{Ikuinen sota}
\begin{quotation}
\noindent Sinä ja minä: me olemme olleet sodassa ajasta ennen kuin kumpikaan meistä oli edes olemassa.\newline---John Connor elokuvassa \emph{Terminator Salvation}
\end{quotation}
Nyt kun puhumme evoluutiosta ja pitkästä kilpavarustelusta, eräs kysymys poksahtaa ilmoilla. Milloin ihmispsykopatia alkoi kehittyä? Mitä aikaväliä meidän tulisi tarkastella? Googlen mukaan kukaan ei ole koskaan kysynyt tätä kysymysta. Yritän vastata siihen.

Ensinnäkin voimme sulkea pois äskettäisen kehittymisen. Psykopatia on konsistentti piirre ympäri maailman. Se on ihmisluonnon universaali ominaisuus. Niinpä se edeltää meidän leviämistämme pois Afrikasta 150 000 vuotta sitten.

Ihmisyyden alkulähteet painuvat kauemmas ja kauemmas taakse ajassa. Homo naledin rituaaliset hautaamiset Etelä-Afrikassa ajoittuvat noin kolmen miljoonan vuoden taakse. Vanhimmat kivityökalut ovat 3.3 miljoonaa vuotta vanhoja.\linkki{sdfs}

Minäpä selitän. Kiven muuttaminen käytännölliseksi työkaluksi vaatii kasautuvaa taitoa ja oppimista, tekniikoiden hidasta evoluutiota seuraavaa adaptoitumista. Tämä tarkoittaa tietämyksen siirtymistä sukupolvelta toiselle, mikä taas tarkoittaa erikoistuneita yksilöitä, tykalujen valmistajien kastia.

Kuten Scientific American kirjoittaa:\linkki{sdfds}
\begin{quotation}
\noindent Lomekwi knapper?s kykenivät toimittamaan riittoisaa tarkoituksellista voimaa irrottaakseen tostuvasti sarjan rinnakkaisia ja kerrostuneita hiutaleita ja jatkamaan knappaamista? pyörittämällä ytimiä. [He] tarkoituksellisesti valitsivat suuria, painavia palikoita hyvin kovaa raakamateriaalia läheisistä lähteistä vaikka pienempiä palikoita olisi ollut saatavilla. He käyttivät monia knappaustekniikoita irroittaakseen teräväreunaisit hiutaleet ytimistä.
\end{quotation}
Raakamateriaalieja ei ole joka paikassa. Työkaluntekijöiden täytyi matkustaa paikkoihin, joista oikeanlaisia kiviä löytyi. Heidän täytyi valmistaa työkalut. Heidän täytyi kuljettaa työkalut takaisin niille, jotka niitä tarvitsivat. Tämä tarkoitti ruoan ja veden, säkkien ja köysin ja niin edelleen kuljettamista??.

Se tarkoitti myös kykyä suunnitella asioita etukäteen ja organisoitua toisten kanssa. Tämä tarkoitti kieltä, tarpeeksi rikasta ilmaistakseen futuureja ja epämääräisiä lupauksia. Tämä kuulostaa kehittyneeltä pieniaivoiselle homidille, kunnes sitä tajuaa, että muurahaiset tekevät pitkälti samaa. Tällaisen käytöksen ei tarvitse olla tietoista. Se voi olla vaistonvaraista.

Kivityökalujen tuotantoketjussa on ytimiä, ketjuja ja alasimia. Se menee paljon yksittäisen ihmisen mentaalista kapasiteettia pidemmälle. Se kertoo, että maailmassa oli sosiaalinen rakenne. Jotkut erikoistuivat työkalujen valmistamiseen. Toiset erikoistuivat työkalujen käyttämiseen metsästyksessä, lihan puhdistamisessa, luiden hajottamisessa, puun leikkaamisessa. Tällainen sosiaalinen rakenne tarkoittaa altruismia eli kykyä jakaa muiden kanssa. Ja aina siellä, missä on altruismia, on huijaamista.

Vastakkaisen näkemyksen mukaan aikaiset ihmiset olivat generalisteja, ja he tekivät itelleen työkaluja sen mukaan, kun he tarvitsivat niitä. Tämän näkemyksen mukaan erikoistumien ja vaihtokauppa tulevat kuvaan vasta paljon myöhemmin. Mutta tämä malli on helppo osoittaa vääräksi. Metsästäminen vaatii omanlaisensa erikoistuneet taidot. Miesten välillä olisi äärimmäistä kilpailua siitä, kuka on paras metsästäjä, tai kuka on paras työkalunvalmistaja. Naiset tarvitsevat työkaluja siinä missä miehetkin, mutta he tuskin ovat yleensä työkalunvalmistajia. Kaksi erityisosaajaa jotka kykenevät vaihtokauppaan pärjäävät \emph{aina} paremmin kuin kaksi generalistia. Joten generalistimalli ei selivä seksuaalisesta valinnasta, ei ekonomiasta eikä misten ja naisten välisestä työnjaosta.

Joten olen sitä mieltä, että voimme ajoittaa ihmispykopatian ainakin kolmen miljoonan vuoden taakse ajassa.

\section{Isojen aivojen ongelma}

Eräs tunnusomaisimmista inhimillisistä piirteistä on ylisuuret aivomme. Fossiilit näyttävät aivojemme kasvaneen yhtäkkiä suuremmiksi ja suuremmiksi, alkaen noin kaksi miljoonaa vuotta sitten. Mikä ajoi tätä laajenemista eteenpäin? Vastauksen huomataan olevan toiset ihmiset. Kuten Davide Geary kirjoittaa:\linkki{sdfds}
\begin{quotation}
\noindent Aivojen koossa tapahtui hyvin vähän muutoksia fossiilikallonäytteissämme ennenkuin osuimme tiettyyn populaatioon. Kun tämä populaatiotiheys oli saavutettu, aivojen koossa tapahtui nopea kasvupyrähdys.
\end{quotation}
Miksi enemmän ihmisiä tarkoittaisi suurempia aivoja? Geary antaa tunnustusta ``sosiaaliselle kilpailulle\vmq{,}'' missä useammat ihmiset kilpailevat samasta ruuasta. Fiksuimmat voittavat, saavat eniten lapsia, ja geenit pienemmille, typerämmille aivoille kuolevat pois, hän argumentoi.

Mutta ihmisten ravinnonlähde ei ole staattinen resurssi. Sen sijaan se on ihmistoiminnan suora seuraus. Enemmän ihmisiä tarkoittaa enemmän ruokaa, ei vähempää. Kalat eivät ryhmity matalaan veteen, odottamaan sitä, että joku kerää ne talteen. Hirvet eivät tule kahdentoista kappaleen paketeissa. Ruoka taistelee vastaan. Kyse on syvästä ja monimutkaisesta pähkinästä, joka ratkeaa teknologian, tietämyksen jakamisen, altruismin ja vaihtokaupan avulla. Sosiaalinen mallimme muuttuu tehokkaammaksi, ei vähemmän tehokkaaksi, kun ihmisiä on enemmän. Joten, mitä enemmän ihmisiä, sitä enemmän ruokaa.

Tämä pitää paikkansa kaikissa muissa tilanteissa, pitsi silloin, kun saavutamme ympäristömme rajat. Tämä tarkoittaa populaation romahdusta, ei räjähdystä. Ainoastaan katasrofaalisissa tilanteissa ihmiset kilpailevat ruuasta.

Voisimme myös kysyä, että miksi älykkyys saisi palkinnoksi ruokaa? Muissa eläimissä sitä ei tapahdu. Suuret aivot ovat kallis ja vaarallinen elin äidille ja lapselle. Miksi laji ei vain kehittäisi suurempia hampaita, tai vahvempia lihaksia, tai pidempiä jalkoja? Vaikuttaa mielivaltaiselta väittää, että älykkyys olisi avain suurien ruokamäärien keräämiseen ilman lisäselostusta.

Saamme Gearyn mallin toimimaan, kun vaihdamme ``sosiaalisen kilpailun'' ``kilpavarusteluun altruistien ja huijarien välillä\vmq{.}'' Kun populaatio koostuu pienistä, eristyksissä elävistä perheistä, huijaaminen on huono strategia. Huijareiden tunnistaminen ja rankaiseminen on helppoa. Pedot tarvitsevat tietyn populaatiotiheyden. Heidän täytyy kyetä liikkumaan tietyn alueen tyhjentämisen jälkeen.

Joten ekonomiset kannustimet huijaamiselle kasvoivat kun muinaiset ihmispopulaatiot kasvoivat. Jossakin pisteessä kilpavarustelu kävi kuumaksi. Yhteistyöhenkisille ihmisille kehittyi sosiaalisia emootioita huijareiden tunnistamista ja rankaisemista varten. Huijareille kehittyi kyky manipuloida ja matkia emootioita, jotta he kykenevät hakkeroimaan emotionaalisia kieliä. Sosiaaliset emootiot muuttuivat monimutkaisemmiksi, kun huijaava matkiminen kehittyi paremmaksi. Siinä missä yhteistyöhenkisten ihmiste sosiaalinen muisti kehittyi paremmaksi, huijarit kehittyivät paremmiksi valehtelijoiksi.

Ja niin edelleen ja niin edelleen. Aivomme ovat pullollaan psykopaattisia kykyjä ja psykopatian tunnistimia. Ei ole kyse siitä, että älykkäämmät ihmiset saivat enemmän lapsia. Kilpavarustaja teki pincer? liikkeen pienille aivoille. Olemme joko loistavia altruisteja, tai olemme loistavia huijareita. Kumpikin vaatii paljon älyä: mitä enemmän, sen parempi. Turvallista välimaastoa ei ole olemassa.

\section{Someone Stole my Lamp. I'm Delighted.}

Tutkailkaamme joitakin näistä psykopaattitunnistimista. Yksi on huumorintajumme. Huumori on ihmisyyden universaali piirre, joka näkyy jo pienissä lapsissa. Vauvat kikattavat ilosta kun he leikkivät vanhempiensa kanssa. Luotamme vaistonvaraisesti ihmisiin, jotka saavat meidät nauramaan. Emme luota niihin, jotka eivät pidä vitseistämme, tai joilla näyttää olevan ongelmia huumorintajussa.

Käytämme huumoria enemmän stressaavissa tilanteissa. Arvostamme omintakeista huumoiria ja palkitsemme ``kertomisen'' paremmin kuin itse vitsin. Kauhuelokuvissamme hirviöt eivät naura muuten kuin karmivalla, pieniä lapsia pelottavalla tavalla. Hirviöillä ei ole huumorintajua.

Vitsi on rakennelma, tarina, jolla on erityinen ja johdonmukainen muoto. Jokainen vitsi, jopa sanaleikit, riippuu mysteeristä. Emme kerro mysteeriä. Se olisi ``vitsin selostamista\vmq{.}'' Sen sijaan kerromme vitsin ja odotamme, että toinen ``tajuaa'' sen. Kun hän tajuaa vitsin, hän nauraa, ja tapahtuma on saatettu loppuun. Tai, riippuen vitsistä, saatamme odottaa huokausta.

Pelkkä nauraminen ei riitä. Molempien osapuolten on naurettavat oikealla hetkellä, ei liian aikaisin eikä liian myöhään. Naurun täytyy kestää tarpeeksi pitkään. Sen ei pidä olla liian kovaäänistä eikä liian hiljaista. Hyvä vitsi saa sekä kertojan että kuulijan iloiseksi. Epäonnistunut vitsi häiritsee ja ärsyttää meitä. Huumori on niin syvästy yhteydessä emootioihimme.

Huumoriprotokolla. Kuinka arvokas asia se onkaan. Tämä ei ole sattumaa.

Se, mitä varten huumori on kehittynyt, on empatian tunnistaminen. Vitsi on kortti, jolla on kaksi puolta. Näytämme toisen, ja pidämme toisen salassa. Jos kuulija on empaattinen tarinan hahmolle, hän näkee kortin piilotetun puolen. Tämä liipaisee naurureaktion. Jos kuulijalla ei ole empatiaa, hän on ymmällään.

Psykopaatti ei kykene nauramaan ``oikealla'' tavalla. Hän ei naura, tai hän nauraa liikaa, tai hän nauraa liian pitkään. Olemme varoisempia sellaisten ihmisten kohdalla jotka nauravat liikaa, kuin sellaisten, jotka eivät naua ollenkaan. Mitä hän piilotteleekaan, pohdiskelemme?

\section{Miltä se tuntuu sinusta?}

Toinen ``ihmisten uniikki kyky'' on taide. Miksi edeltäjämme tykkäsivät maalata kalliolle ja luolien seinämille? Perinteiset selitykset kuuluvat, että maalaukset on tehty taiteellisista syistä, tylsyydestä, huumeiden vaikutuksen alaisina, tai osana metsästysseremonioita.

Mutta esimerkiksi 40 000 vuotta vana norsunluinen Venus ei palvele mitään muuta funktionaalista tarkoitusperää kuin tunteiden saaminen aikaan katselijassa. Kyky luoda on niin laajalle levinnyt, että se soittaa jokaisella kadunkulmalla penneistä. Silti arvostamme sitä, ja vaikuttaa siltä, että lajimme on tehnyt sitä jo kauan. Ylise kaiken, me oletamme, että taide saa meidät ``tuntemaan'' jotakin. Me kysymme tätä toisilta: ``miltä se sinusta tuntuu?'' Ja skannaamme heidän kasvojansa kun he vastaavat.

Psykopaateilla on monia mielenkiintoisia piirteitä, joihin tulen myöhemmin tässä kirjassa. Yksi on heidän kiinnostuksen puute luoviin taiteisiin. He ievät piirrä, maalaa, muotoile tai kaiverra. He eivät ota valokuvia, paitsi heidä itsestään ja heidän tavaroistaan. He eivät kokkaa huvin vuoksi, keksi reseptejä, eivätkä he tee leipää itselleen harrastuksen vuoksi. He eivät luo musiikkia, vaikkakin he voivat olla erinomaisia toisten töiden esittäjiä.

Tämä luovan voiman puute on vinha juttu, kun se ensimmäisen kerran tulee vastaan. Se sopii yhteen heidän yleisellä tasolla tyhjän huumorintajun kanssa. Heidän harrastuksensa ovat matkustaminen, shoppailu, ravintoloissa syöminen, uusien ihmisten tapaaminen. Tämä on kuluttamista, ei luomista.

Taide on kallisarvoinen asia. Se on universaali ihmisten kieli. Aivan kuten komedian kanssa, me palkitsemme omaperäisyyttä enemmän kuin teknistä taitavuutta. Kuten komedian kanssa, nautimme taiteesta mieluummin seurassa kuin yksin. Ja kuten koomikkojen kanssa, ylistämme ja arvostamme taiteilijoita, vaikka kyvyllä ei ole selviytymisen kannalta arvoa. Viimeksi, mittaamme taiteilijoita heidän historiansa mukaan: yksi menestys ei ole tarpeeksi. Yksi menstys voi olla väärennös, varastettu, tai sattumaa. Kun taas urheilijan tai tieteilijän saavutus voi kestää elämän loppuun saakka.

Olen varma, että luovuus on toinen salainen empatian kieli. Se kysyy maailmalta, ``ystävä vai vihollinen? Katso tätä ja kerro minulle, että tunnet jotakin!'' ja katsoja vastaa, tai epäonnistuu testissä. Kyse on hyvin samanlaisesta asiasta kuin vitsin kertomisessa. Kuten loistavat vitsit, loistavan luovan työn täytyy puhua emootioista emootioille. Sen täytyy kertoa puolet tarinasta, jonka ainoastaa sosiaaliset aivot voivat täydentää ja ``tajuta\vmq{.}''

Nuori lapsi oppii piirtämään koulussa ja tuo töitään näytille vanhemille. Nämä lahjat eivät ole hyödyllisiä missää konkreettisessa mielessä. Silti ne ovat tärkeitä ja erityisiä siinä hetkessä. Lapsi katsoo hänen vanhempiensa reaktioita. Kun he näkevät iloa puristuneissa? kasvoissa ja kummallisissa väreissä, lapsikin tuntee iloa. He jakavat hetken, vahvistaen toistensa sosiaalisen ihmisyyden. Lapsi ajattelee, ''Katso, Äiti, olen normaali. Älä torju minua!''

Luomme itsellemme ja toisillemme. Luomme, jotta toiset tuntisivat jotakin. Useimmiten se on onnellisuutta, vaikkakin joskus se on menetystä, surua, tai muita emootioita. Luova taide on empatian viesti. Ja me mittaamme taiteemme laatua aivan samoin kuin mittaamme huumoria: se omaperäisyyden ja siten sen autenttisuuden kautta.

Tämän takia imitatiivinen taide on ``feikkiä\vmq{.}'' Se on syy, miksi insinöörimäinen suunnitteleminen ei ole taidetta, ja miksi pikaruoka tuntuu ``halvalta\vmq{.}'' Sen takia emme selosta vitsejä, ja sen takia taitelija ei voi selittää teoksensa ``pointtia\vmq{.}'' Se on testi, ja jos et tiedä vastausta, se jo itsessään on merkittävää. Sen takia kasa palikoita Tate Galleryssä? on arvoltaan miljoona puntaa. Siinä se vitsi piilee.

\section{Johtopäätökset}

Olemme alkaneet avata psykopaattien mysteeriä käsittelemällä heitä petioina, sen sijana, että käsittelisimme heitä rikkinäisinä ihmisinä. Petomalli saa aikaan muutakin kuin vain selittää psykopatian. Se myöskin selittää ihmismielen evolutiivisen kehityksen lopputuloksena iänikuisen altruistien ja huijareiden välisenä kilpavarusteluna. Kolmen miljoonan vuoden ajan altruistit ovat kehittyneet paremmaksi yhteistyön tekemisessä ja huijarit ovat kehittyneet paremmaksi sen jäljittelemisessä. Seuraavassa luvussa selostan, kuinka psykopaatit metsästävät.
