\chapter{Kreetalainen menetelmä}\label{the-cretan-method}

Totuuden etsiminen on ikivanha ja vaikea matka. Moderni yhteiskunta kylpee edelleen valheiden meressä politiikassa, markkinoilla ja etenkin ohjelmistoalalla, missä suuret valheet jalostuvat kuin kultit. Minun työtäni on tien leikkaaminen valheiden läpi, tavoitteena kehitettää parempia teorioita totuudesta. Käy ilmi, että valheet aiheutavat kipua, ja kun lähestymme totuutta, muutumme onnellisemmiksi. Tässä esseessä selostan, kuinka teen sen, ja annan sarjan työkaluja ja opetuksia avuksesi.

\section{Totuuden lähestyminen}

Ihmisyys on voimakas aitososiaalinen eläinlaji. Eräs supervoimistamme on kokoelma todellisesta maailmasta kertovaa tietämystä. Parannamme sitä rakentamalla malleja tai teorioita, jalostamme niitä käytännön kautta ja opettamalla niitä jälkeläisillemme. Se, mikä kaikista mallinnettavissa olevista todellisuuden kerroksista vaikuttaa meihin eniten, joka ikinen päivä, on toisten ihmisten todellisuus. Karl Popper kirjoitti: ``Kaikista syvimmällä olemassaolon tasolla olemme sosiaalisia olentoja\vmq{.}''

Teorioiden muotoilu, testaaminen ja soveltaminen on taiteenlaji. Totuus on merkillinen asia. Aivan kuin irrationaaliluku, totuus on olemassa, siitä ei voida neuvotella, se ei ole subjektiivinen, eikä sitä voida koskaan saavuttaa. Totuus on neliulotteisen avaruusajan absoluuttinen ominaisuus. Emme voi koskaan saavuttaa totuutta; voimme vain rakentaa teorioita, jotka approksimoivat sitä paremmin ja paremmin.

Rakennamme teorioita näennäisesti tyhjästä. Otamme havaintoja ja tuntemuksia ja sen loputtoman teorioiden perinnön, jonka aikaisemmat sukupolvet ovat meille jättäneet. Suunnittelemme uusia tai paranneltuja teorioita ja enkoodaamme ne kieleen ja sanoihin argumentointia, muistamista ja jakamista varten.

Popper järkeili että teorioita on oleellisesti kahta sorttia. On tieteellisiä teorioita, jotka voidaan todistaa vääräksi datalla tai havainnoilla, ja on mystisiä teorioita, joita ei voida falsifioida. Toisin sanoen teoriaa ei voida koskaan todistaa oikeaksi, sillä totuutta ei voida koskaan saavuttaa. Voimme kuitenkin yrittää todistaa teoria vääräksi ja epäonnistua. Kun poistat kaiken, mikä on todistettavasti väärin, se, mikä jää jäljelle, lähestyy totuutta.

Kaikki teoriat ovat jossakin määrin epätarkkoja, mutta falsifioitavissa olevilla tieteellisillä teorioilla on mielenkiintoinen ominaisuus, joka mystisiltä teorioilta puuttuu: tieteellisiä teorioita voidaan kehittää kohti hienompia ja hienompia totuuden approksimointeja. Tässä on joukko teorioita, jotka tunnistat:
%\begin{itemize}
%\item Ympyrän ympärysmitan suhde sen halkaisijaan, \(\pi\), on vakio.
%\item \(\pi=3\).
%\item \(\pi=\frac{22}{7}\).
%\item \(\pi={3,}141592\).
%\end{itemize}
\begin{itemize}
\item Ympyrän ympärysmitan suhde sen halkaisijaan, Pii, on vakio.
\item Piin arvo on kolme.
\item Piin arvo on \(\frac{22}{7}\).
\item Piin arvo on {3,}141592.
\end{itemize}
Nämä ovat kaikki falsifioitavissa, eikä mitään niistä voi todistaa oikeaksi. Otetaan listan ensimmäin teoria. Voimme mitata niin monta ympyrää kuin haluamme, ja saamme joka kerta suurin piirtein saman vastauksen. Emme saa mitään dataa, mikä todistaisi teorian vääräksi. Emme myöskään saa dataa, mikä todistaisi sen oikeaksi. Toistaiseksi kaikki hyvin. Toinen teoria vaikuttaa hyvältä, ja sen mitätöiminen on triviaalia. Väärässä oleminen on tieteellistä! Kolmas teoria on parempi; joskin mittaustyökalujemme parantaminen paljastaa, että Pii näyttää enemmän luvulta {3,}1416 kuin luvulta {3,}1429. Neljäs teoria on vielä paljon parempi ja sen mitätöiminen vaatii paljon työtä. Ei ole olemassa Piin teoriaa, jonka voisimme absoluuttisesti sanoa olevan totta. Meillä on ainoastaan tarkempia ja tarkempia malleja.

Tässä taas tulee maagisia teorioita, jotka saatat tunnistaa: ``C++ on mahtava, koska siinä on Standardimallikirjasto\vmq{,}'' ja ``Java on mahtava, koska se estää koodareita tekemästä pahoja virheitä\vmq{.}'' Kuinka falsifioida nämä teoriat? Mahdotonta. Ne eivät ole pelkästään epätotta: ne eivät ole edes väärässä,\linkki{https://en.wikipedia.org/wiki/Not_even_wrong} kuten Wolfgang Pauli sanoisi.

Tieteellinen teoria on aina jonkin verran väärässä, ja sitä voidaan parantaa vähentämällä sen virheellisyyttä testaamisen, havainnoimisen ja mittaamisen avulla. Mystinen teoria on epätotta, eikä sitä voida parantaa.

\section{Valheiden teoria}

Jos haluamme etsiä totuutta, kannattaa meidän ymmärtää harhautuksen luonnetta. Kreikkalaiset ajattelivat (tai teeskentelivät), että harhautus oli paha henki nimeltään Apate, joka pakeni Pandoran avatessa pahamaineisen lippaansa. Sun Tzu kirjoitti, että ``kaikki sodankäynti perustuu harhautukseen\vmq{,}`` millä hän tarkoitti, että harhautus on menestyksekäs strategia konfliktitilanteessa.

Useimmat mielet ovat rehellisiä valehtelijoita. Voimme olettaa, että valehtelu on ihmismielen sisäänrakennettu toiminto. Kyky valehdella, tai vähintäänkin bluffata, on välttämätöntä maailmaa kuvaavien teorioiden rakentamisessa, sillä kaikki teoriat ovat jossakin määrin valheellisia. Ainoastaan toimintahäiriöinen mieli on kyvytön valehtelemaan. Kun mieli muotoilee teoriaa, se remiksaa tietämiään olemassaolevia teorioita uusiin havaintoihin, lisää sen approksimaatiota, arvauksia, otaksumia ja uskomuksia, yrittäen yhdistää ne johdonmukaiseksi tarinaksi. Useimmat mielet tekevät tätä hitaasti, viikkojen tai vuosien kuluessa. Tällaisissa teorioissa valheet ovat väliaikaisia rakennustelineitä, jotka voidaan poistaa ajan myötä.

Jotkut mielet jatkuvassa sodassaan muuta ihmisyyttä vastaan valehtelevat strategisesti manipuloidakseen ja aiheuttaakseen sekaannusta. Tällaiset mielet kykenevät rakentamaan täysin mystisiä teorioita niin nopeasti, että kaikki tapahtuu reaaliajassa, ja kertomaan ne kuuntelijalle äärimmäisen uskottavina valheina. Kyse on hyökkäyksestä, sodankäynnin aseesta. Voimme kutsua tällaisia mieliä ``psykopaattisiksi\vmq{,}'' sillä heidän tavoitteena on toisten saalistaminen. Heidän teorioissaan totuus on väliaikainen rakennusteline, joka vaihdetaan myöhemmin sepityksiin.

Valheen arvo aseena on selvä. Peto-saalis-suhteessa pedon on pidettävä saalis sekaantuneena ja liikkumiskyvyttömänä, jotta se voi ruokailla turvallisesti. Ihmisten tapauksessa emotionaaliset siteet ovat tehokkaampia ja vähemmän riskialttiita kuin fyysiset siteet. Injektoimalla mystisiä teorioita saaliin mieleen peto rampauttaa saaliin teoriaprosessin. Tämä on yksi tapa, jolla kultit pyydystäväþ uhrejansa: injektoimalla suuria mystisiä teorioita, jotka häiritsevät loogista ajattelua.

Ensimmäinen käytännön opetus kuuluu näin: \emph{jokainen valehtelee,} kuten Tohtori House (tai pikemminkin hänen käsikirjoittajansa) selostaa. Oletan, että on rehellisiä valehtelijoita (kuten minä, vakuutan), ja psykopaatteja. Olen niin skeptinen, että pilkkaan koko universumia, ja lopetan vasta, kun se näyttää minulle dataa. Ja silloinkin paras, mitä se voi saada aikaan, on ``väärin'' ``epätoden'' sijaan.

\section{Mieli evolutiivisesti kehittyneenä strategiana}

Kun seurasin kolmevuotista poikaani opettelemassa ja siten pelaamassa Minecraftia, valaistuin. Olen aina kohdellut lapsiani protoaikuisina: keskeneräisinä, mutta toimintavalmiina. Jos et ole nähnyt Minecraftia, kyse on rakennussimulaattorista, joka yksinkertainen, syvällinen ja sosiaalinen. Se on paras malli idealiselle tietokoneohjelman kehitysprosessille, mitä olen tähän mennessä nähnyt.

Vanha ``mieli on savea\vmq{,}'' jonka yhteiskunta muovailee -teoria on mitätöity kaksostutkimuksilla. Tiedämme, että mielemme on määritelty pitkälle jo ennen syntymää. ``Mieli evoluutiivisesti kehittyneenä strategiana, jota yhteiskunta muovaa ja kalibroi'' vaikuttaa paljon tarkemmalta. Haluaisin nähdä tutkimuksia kaksosista, jotka ovat kasvaneet dramaattisen erilaisissa kulttuureissa. Kalibroiko vaikkaka Kinshasan kaduilla kasvaminen paranoiaa suuremmalle kuin esimerkiksi Lontoossa kasvaminen?

Mentaaliset työkalumme ovat selvästi teräviä ja funktionaalisia hyvin varhaisesta iästä lähtien. Noin vuoden vanha lapsi alkaa vuorovaikuttaa äidin lisäksi muiden ihmisten kanssa, ja useimmiten hän tekee sen täsmällisesti. Jos haluat ymmärtää sellaista käsittettä kuin ``luottamus\vmq{,}'' havainnoi nuoria lapsia. Lapset aloittavat luottamalta ainoastaan vanhempiinsa ja sisaruksiinsa. He laajentavat luottamustansa muihin aikuisiin, kuin heidän vanhempansa kertovat, että se on OK. He luottavat toisiin lapsiin implisiittisesti, paitsi jos ikäero on liian suuri. He eivät luota tuntemattomiin eläimiin, paitsi silloin, kun he ovat vanhempiensa kanssa.

Samalla tava ``vapauden'' käsite, joka inspiroinut miljardeja kirjoitettuja sanoja, ja jonka kuitenkin olen itse summannut ``mahdollisuudeksi tehdä mielenkiintoisia asioita toisten ihmisten kanssa\vmq{.}'' Tämä käy selvästi ilmi tyypillisen lapsen pikaisesta havainnoimisesta, ja sillä on syvällisestä arvoa toimivana teoriana. Olen puhunuttästä paljon enemmän kirjassani \emph{Culture and Empire: Digital Revolution.}

Lapsen maailma on pakosta yksinkertainen, ja lapsen mieltä eivät sotke mystiset teoriat. Me olemme tehokkaita teoreetikkoja jo nuorena. Lapset ovat luonnonltansa tieteilijöitä.

Tämä tuo meidät toiseen käytännön opetukseen: \emph{etsi lapsenmielistä intuitiota.} Työkalut, joita tarvitsemme maailman ymmärtämiseen ovat meissä sisäänrakennettuna. Ne ovat kehittyneet pitkän ja heltymättömän tietämysmarkkinoiden kilpailun seurauksena. Olemme hyviä kehittämään teorioitamme, kunnes mystinen ajattelu myrkyttää mielemme.

\section{Kipu on pätevää dataa}

Yksi juttu lapsissa on se, kuinka suodattamattomia he ovat. Kun he eivät pidä jostakin, he kertovat sen. Kohtaamme synkeitä kasvoja, kovaäänistä valitusta ja jopa kirkumista ja itkua. Vanhemmalle nämä voivat tuntua turhauttavilta. On kuitenkin kiehtovaa verrata sitä siihen, kuinka paljon aikuiset hyväksyvät ärsytystä ja kipua aikuiset hyväksyvät ilman silmänräpäytystäkään.

Kaikki teoriat ovat savua niin kauan, kun niitä ei testata, mikä tarkoittaa niiden soveltamista todellisuuteen ja tulosten havainnointia. Hyvä teoria toimii sulavasti ja lähes ääneti. Huono teoria luo jotain, mitä kutsun ``kitkaksi\vmq{,}'' joka näyttäytyy ärsytyksenä, kustannuksina, viiveinä, stressinä. Esimerkiksi kotitaloudessani on kaksi keskenään ristiriitaista teoriaa siitä, mikä on paras tapa mennä kouluun:
\begin{itemize}
\item Paras tapa mennä kouluun on autolla (kolmen minuutin ajomatka).
\item Paras tapa mennä kouluun on kävellä (kymmenen minuutin kävelymatka).
\end{itemize}
Tänä aamuna olimme myöhässä ja meistä tuntui laiskalta. Menimme autolla. Matkassa meni kaksikymmentä minuutti, josta suurimman osan vietin kihisten liikenteelle ja huonoille kuskeille. Sitten tajusin, että jokainen hetki on koe, ja että kipu on pätevää dataa, ja olin taas tyytyväinen. Autoteoria on mitätöity ja voimme taas kävellä.

Kun käytät teoriaa---ja muista, kaikki teoriat ovat jossakin suhteessa väärässä---koet aina jonkinlaista ärsytystä. Joskus ärsytyksen pyyhkii pois suuremmat huolenaiheet. Esimerkiksi ``OK, olen jumissa liikenteessä, mutta ainakin olen suojassa rankkasateelta\vmq{.}'' Tai ``OK, pomoni persereikä, mutta onpahan minulla ainakin joku työpaikka\vmq{.}''

Varsinainen henkinen kipu (vertaa rikkoutuneen lasin päällä istumiseen) on merkki vakavasta kitkasa, kuten selvistä valheista. Pieni ärsytys indikoi kevyttä kitkaa, kuten epätarkkoja oletuksia. Jokainen teoria on joko tieteellinen, ja määritelmän mukaan väärässä ja paranneltavissa, tai se on mystinen ja mahdoton korjata. Mystinen teoria aiheuttaa kipua aina, kun yrität käyttää sitä mihinkään vakavaan tarkoitukseen.

Tämä vie meidät opetukseen numero kolme: \emph{kehitä teorioitasi tai heitä ne menemään.} Mystisen teorian avainindikaattori on se, että et voi poistaa kitkaa parantamalla teoriaa. Joko hyväksyt teorian täysin ja kokonaan, tai et ollenkaan. Mystisiä teorioita voi pitää tarttuvina henkisinä sairauksina.

\section{Psykopaattinen side}

Lienet huomannut, että ihmiset pysyvät psykopaattisissa suhteissa paljon pidempään, kuin mitä olettaisit. Kun aloitimme ZeroMQ-projektin, pääkehittäjät halusivat käyttää C++:aa oman mieltymykseni C:n sijaan. He sanoivat minulle: ``On totta, että kielen opettelemiseen menee kymmenen vuotta, mutta se on paljon tehokkaampi\vmq{.}'' Niihin aikoihin minulla ei ollut antaa vasta-argumentteja. He tarvitsivat viisi vuotta ja yhden oikean open surce -projektin, ennen kuin he tekivät U-käännöksen ja hylkäsivät C++:n.

Psykopaattinen riippuvuus on intuition vastainen mekanismi, joka kannattaa ymmärtää. Arvostamme suhteitamme sen mukaan, kuinka paljon olemme investoineet niihin aikaa, vaivaa, rahaa, kiintumystä, resursseja. Normaalissa sihteessa tämä tapahtuu kahdella tavalla: Aleksandra ja Bob vaihtavat vaihtelevan hienovaraisia lahjoja ja tekevät päänsisäisiä laskuja pitääkseen kirjaa saldoistaan. ``Terveessä'' suhteessa tase on lähellä nollaa, ja laskennon vaiva merkitsee suhteen syvyyttä. ``Sairaassa'' suhteessa toisella puolella on rankasti velkaa (ja laskennon vaiva lähentelee 100\% CPU-aikaa, toimien DoS-hyökkäyksenä).

On olemassa huijaava strategia, joka perustuu tulevaisuuden lupauksiin. Jos Mallory lupaa suuren voiton tulevaisuudessa, Bob investoi sen mukaan. Moni huijaus, muun muassa Espanjalainen Vanki,\linkki{https://en.wikipedia.org/wiki/Spanish_Prisoner} toiselta nimeltään Nigerialaiskirjeet, perustuu tähän ilmiöön.

Homma toimii tällä tavalla: Mallory tekee Bobille suuren lupauksen ja antaa samalla pienen lahjan todisteena hyvistä aikomuksista. Bob vastaa lahjoilla ja Mallory hyväksyy ne yllyttääkseen Bobia. Mallory epäonnistuu lupauksensa pitämisessä joka kerta johtuen aina traagisista ulkopuolisista olosuhteista. Mallory imartelee Bobia, leikkii uhria ja pyytää Bobilta poletti-investointeja.

Bobin mielessä suhde syvenee ja sen arvo nousee. Tilikirjassa on Malloryn suuri lupaus sekä kaikki Bobin investoinnit. Se tuntuu oikealta ja syvältä. Bobille kehittyy yhä vain suureneva kiintymys teoriaan Malloryn lupauksista ja sen sijaan, että hän perääntyisi, hän tekee suurempia ja suurempia investointeja.

Mallory siirtyy kolmansien osapuolien syyttämisestä Bobin syyttämiseen. Kaikki alkaa olla hänen syytänsä. Mallory uudelleenkirjoittaa historiaa selittääkseen, kuinka Bob on kaikkien Malloryn ongelmien aiheuttaja. Bob hyväksyy nämä velat ja tilikirja alkaa kääntyä massiivisesti häntä vastaan. Lupaukset hukkuvat ja unohtuvat. Bob tekee ylitöitä maksaakseen ``velkansa'' ja normalisoidakseen suhteen, mikä mahdollistaa yhä vain pahemmaksi kehittyvän Malloryn käytöksen. Mitä kovemmin Mallory pahoinpitelee Bobia, sitä enemmän Bob investoi, ja mitä enemmän hän investoi, sitä arvokkaampana hän pitää suhdetta ja sitä kovemmin hän sitoutuu Malloryyn. Malloryä ei tietysti kiinnosta pätkääkään. Hänellä on negatiivinen side Bobiin, joka koostuu pääasiasa halveksunnasta ja inhosta.

Tämä side voi kestää vuosia, jopa elämän loppuun saakka. Ulkopuolelle se näyttäytyy käsittämättömänä ja moraalittomana. Mutta kyse on vain siitä hinnasta, jonka maksamme aitososiaalista voimistamme: ne mahdollistavat petojen luokan, Malloryt. Mallory ei ole aina henkilö: se voi olla organisaatio tai joukko mystisiä teorioita.

Psykopaattisesta siteestä on vain kaksi tietä ulos. Ensimmäinen vaihtoehto on, että Bobilla ei ole enää mitään tarjota Mallorylle, joka hylkää Bobin kuin roskan selostaen samalla muille yksityiskohtaisesti, kuinka kaikki oli Bobin syytä. Toinen vaihtoehto on, että Mallory vaatii Bobilta jotakin liian suurta. Tässä pisteessä Bob saattaa herätä tai tuhota itsensä.

Mieltemme Espanjalaisen Vangin hyökkäykseen kehittynyt vastaus on pitää tulevaisuutta massiivisen paljon vähempiarvoisena. Sen piekseminen on kuitenkin triviaalin helppoa: lisäät vain tarpeeksi nollia. Niinpä modernit huijaukset lupaavat aina naurettavan suuria määriä rahaa tai valtaa.

Teoriat, jotka ovat ylen monimutkaisia ja jotka lupaavat tulevaisuudessa palkintoja ovat eräs Nigerialaiskirjeiden muoto. Jos sinun täytyy opetella kymmenen vuotta kieltä, joka lupaa sinulle ``tehokkuutta\vmq{,}'' sinulle valkenee, ettq C++-käyttäjät ovat psykopaattisessa suhteessa kielensä kanssa. C++ on ohjelmointikielten Skientologia. Java, rehottava massauskonto.

Merkillistä on, että kun haastat jonkun, joka on Espanjalaisen Vangin syleilyssä, hän taistelee sinua vastaan. Massiivisen, elämän kokoisen investoinnin kyseenalaistaminen tuntuu äärimmäisen vihamieliseltä teolta. Vain sitten, kun syleily on rikkoutumassa, Bobit nyökkäävät ja hyväksyvät, että he saattavat olla hukassa.

Tästä pääsemme opetukseen numero neljä: \emph{ihmiset puolustavat maagisia teorioita kaikista koviten.} Tietokoneohjelmien maailmassa on hyvin vähän teorioita, jotka eivät ole petollisia tällä tavalla. Suurin osa koodista perustuu mystisiin teorioihin, ja useimmat koodarit ovat Bobeja. Tieteellinen menetelmä puuttuu ohjelmoinnin maailmasta melkein kokonaan (ZeroMQ C4.1 -prosessi on yksi harvinaisista yrityksistä).

\section{Emootioiden rooli}

Lapset saattavat vaikuttaa hyvin emotionaalisilta, mutta tarkempi katsaus paljastaa, että useimmat lapset voivat kytkeä emootionsa päälle ja pois halunsa mukaan. On sanottu, että kaikki lapset ovat psykopaatteja. (Tässä on parempi teoria: kaikki psykopaatit ovat lapsellisia.) Lapset menettävät tämän kyvyn, kuin heille kehittyy empatia. Emootiot ovat sosiaalisia kommunikaatiotyökaluja, tapa manipuloida toiset käyttäytymään haluamallamme tavalla. Ne ovat alkuperäinen, ikivanha kielemme, joka näkyy kasvoillamme ja kehoissamme.

Tätä on helppo demonstroida. Jos joku kävelee eteesi jalkakäytävällä, astut sivuun, hymyilet tai nyökkäät, ja aluillaan oleva ärsytys (se pienenpieni ripe, joka näkyy esimerkiksi kulmakarvoijen nousemisena) muuttuu pikkuruiseksi miellyttäväksi vuorovaikutukseksi. Mutta sama vuorovaikutus kahden auton välillä voi usein johtaa molempien kuskien kokemaan intensiiviseen raivoon. Ero on siinä, että auto muodostaa kuskin ympärille häkin, joka leikkaa verbaalisen ja nonverbaalisen kommunikaation poikki. Autokuskin ja jalankulkijan on helpompi ymmärtää toisiaan kuin kahden autokuskin.

Kun kasvava ärsytys ei saa mitään vastausta, aivot menevä ``taistele-tai-pakene'' -tilaan, johon saattaisit mennä, jos joku tarkoituksella astuisi eteesi jalkakäytävällä. Rattiraivo on perustavanlaatuinen selviytymisvaisto toimimassa väärässä kontekstissa. Jyrkkä reaktio uhkaan on turvallisempi vaihtoehto kuin ei reaktiota ollenkaan. Paits tietysti silloin, kun uhkaa ei ole olemassa.

Jotkut ihmiset (ne Malloryt joista puhuin aikaisemmin) säilyttävät lapsenomaisen kykynsä kääntää tunteensa aikuiseksi kasvaessaan. He projisoivat väärennettyjä tunteita---kateutta, vihaa, pelkoa, raivoa, itsesääliä, surullisuutta---ajaakseen toisia Espanjalaisen Vangin syleilyyn. \emph{Katso, olen hullun mustasukkainen! Se todistaa, että rakastan sinua, joten anna minulle lisää huomiota!}

Jopa vieläkin mielenkiintoisempaa on se, kuinka itse käytämme emootioitamme kohdatessamme huonosti toimivia teorioita. Ilmaisemme usein kivun ja ärsytyksen vihana toisia kohtaan. Kuten ``paras tapa mennä kouluun''-esimerkkini näytti, emootiot ovat pätevää dataa, mutta prosessina ne ovat kelvottomia. Toiselle kuskille huutaminen ei ole Aristotelinen dialogi.

Emootiot saavat aikaan loistavaa taidetta ja karmeaa tiedettä. Voimme itseasiassa mitata mystiikan ja tieteen suhdetta emootioiden voimakkuudella. Huomasimme tämän selvästi ZeroMQ-yhteisössä: kun siirryimme mystisestä prosessista tieteelliseen alkuvuodesta 2011, kaikki emotionaaliset argumentit katosivat.

Selitän myöhemmin, kuinka säännellä tunteita, aktiviteetti jota kutsun ``maadoittamiseksi\vmq{.}'' Kyse on vaikeasta tekniikasta, joka kuitenkin auttaa äärimmäisen paljon etäisyyden ottamisessa. Ja etäisyyden ottaminen omista kokemuksista on ironisesti paras tapa ymmärtää ne kunnolla. Kipu on pätevää dataa vain silloin, kun emootiot ovat hiljaa.

Nähdäksesi kitkaa voit joko havainnoida kipua ja stressiä toisissa ihmisissä tai itsessäsi. Toisten seuraaminen reagoimatta heidän tunteisiinsa on ei-psykopaateille jo valmiiksi vaikeaa. Ja psykopaatit eivät voi ymmärtää tunteita; he kykyenevät ainoastaan lukemaan ja matkimaan niitä. (Arvaan, ilman kovaa dataa, että he ovat surkeita tieteessä.) Omien kokemusten havainnoiminen ilman niihin tunteella mukaan lähtemistä on äärimmäisen vaikeaa. Mutta siellä sinä olet, 24/7 sisällä omassa päässäsi. Jos onnistut kehittämään tämän kyvyn, voit nähdä kitkan käytännössä missä tilanteessa tahansa yksinkertaisesti osallistumalla siihen.

Tämä tuo meidät opetukseen numero viisi: \emph{olet itse paras instrumenttisi.}

\section{Ydinprosessi}

Olemme keränneet tarpeeksi aksioomia ja teorioita teorioista puhuaksemme itse ydinprosessista. Jos olet säännöllinen lukijani, tiedät tämän prosessin jo etukäteen. Kuten moni hyvä teoria, se rakentuu menestyksekkään harjoituksen varaan. Prosessi vaikuttaa yksinkertaiselta:
\begin{itemize}
\item Havaitse kitkaa sosiaalisessa tilanteessa.
\item Löydä tai arvaa taustalla vaikuttavat teoriat.
\item Identifioi teorioiden viat.
\item Parantele teorioita, tai hylkää niitä.
\item Kokeile paranneltuja teorioita.
\item Toista, kunnes tylsistyt.
\end{itemize}
Olemme nähneet, että jokaista tieteellistä teoriaa on mahdollista parannella. Ei ole olemassa ``lokaalia maksimia\vmq{,}'' vain loputtomiin kasvavaa tarkkuutta jonka voit saavuttaa käyttämällä aikaa ja näkemällä vaivaa. Se, mikä on yllättävämpää monille (ehkä heidän äitinsä kertoi heille usein, kuinka erityisiä he ovat) on, että tämä prosessi on täysin mekaaninen. Toinen tapa sanoa tämä on, että yksilöllinen älykkyys on jokseenkin yliarvostettua.

Yksilöllisen älykkyyden arvostaminen on kuin osoittaisi yhtä muurahaista muurahaispesässä ja sanoisi, ``Katsokaa, tuo muurahainen on superfiksu!'' Kyse on mystisestä ajattelusta. Kuten muurahaisetkin, me olemme kollektiivinen laji ja ajattelemme ryhmissä, emme yksilöllisesti. Fiksuin koskaan elänyt muurahainen työskentelemässä yksin ei ole mitään verrattuna muutamaan ``tavalliseen'' muurahaiseen, jotka työskentelevät yhdessä.

Itseasiassa tilanne on vielä tätäkin pahempi. Fiksut ihmiset käyttävät usein älykkyyttään kitkan kompensoimiseen. Ihmisten kyky rationalisoida pahimmatkin tilanteet voi olla hämmästyttävä. Jos on olemassa jokin yksilöllisyyden älykkyyden laji, jota arvostan, on se kyky havaita kitkaa ja kiusaantua siitä. Jopa sekin riippuu muista ihmisistä: yksinäisyydessä ei esiinny sosiaalista kitkaa.

Joten prosessi, jota juuri kuvailin, ei olekaan niin yksinkertainen. Se toimii ainoastaan ryhmän harjoittamana. Olen havainnut melko tyypillisen, lähes ideaalisen syklin ajatusprosessissamme: opi toisilta, kokeile uutta tietämystä (leiki), sovella sitä todellisiin ongelmiin (työskentele), ja sitten opeta sitä toisille. Kutsun tätä lyhyesti nimellä opi-leiki-työskentele-opeta (OLTO).

OLTOlla on mielenkiintoisia ominaisuuksia. Ensinnäkin se on synnynnäinen. Näet sen ilmaantuvan pienten lasten käytökseen ilman opastusta. Jos otetaan taas Minecraft esimerkiksi, prosessi menee näin: opi katselemalla YouTube-videoita, joissa toiset lapset pelaavat, sitten harjoittele yksin varmistuaksesi siitä, että kerätty tietämys toimii, sitten pelaa peliä toisten lasten kanssa, ja sitten opeta toisille lapsille se, mitä olet oppinut. Lapset oppivat pelaamaan Minecraftia toisilta lapsilta. Ei kirjoja, ei kouluja. Ja silti kyse on poikkeuksellisen rikkaasta ja syvästä tietämyksestä.

Toiseksi, se toimii erittäin hyvin. Avoimen koodin projektini rakentavat tietokoneohjelmia sen avulla. Vuosikymmenet prosessin hiomista, ja päädyimme sinne, missä lapseni ovat. Kirjoitamme koodia tiukoissa OLTO-sykleissä. Koodi ei ole ikinä täydellistä, mutta sitä voidaan täydellistettää: tieteellinen teoria, joka on aina väärässä, mutta ei koskaan epätotta.

OLTO on kertaluokkia tehokkaampi kuin klassinen yksisuuntainen oppiminen. Sen päälle rakentuu kokonainen teoria itseorganisaatiosta ongelmien ympärille. Olen puhunut siitä useita kertoja kirjoituksissani.

Ja OLTU hädin tuskin tuntuu työltä. Se on nautinnollista, lähes koukuttavaa. On erikoista nähdä ammattielämäni kaareutuvan takaisin lapselliseen maailmankuvaan. Luonnollisen matkan sijaan tämä tuntuu pakotetulta. Todellakin: pohdiskelen, mitä minulle tapahtuikaan kaikkina noina vuosina.

Ah, kyllä, massakoulutus ja massatyö, nuo teollisuus- ja massamedia-aikakauden kiroukset. Jossakin kohtaa 1800-luvulla vanhat OLTO-rakenteet rikkoutuivat ja ne korvattiin tarkasti eritellyillä kouluilla, yliopistoilla ja työpaikoilla. Leikimme lapsina, opimme nuorina, työskentelemme aikuisina, ja eläköidymme ja kuolemme vanhoina. Opettaminen on jonkin sortin työ joillekin harvoille.

Leiki, opi, tyskentele, kuole. Elämän neljä vaihetta. Tällä neljä-elämää-teorialla on monia ongelmia sen lisäksi, että se vain tekee ihmisistä onnettomia. Listaan muutamia:
\begin{itemize}
\item Se sallii meidän kontribuoida noin 45 vuotta, 60\% elämästämme. Kuulostaako tämä paljolta? ``Videopelin'' asun kautta seitsenvuotiaani opettaa nelivuotiaalleni arkkitehtuuria, fysiikkaa, kemiaa. Minun silmääni näyttää siltä, että OLTO antaa meidän kontribuoida kolme- tai neljävuotiaasta kuolemaan saakka.
\item Se sulkee pois suuria osia populaatiosta. Koulutuksen kallis hinta (ajassa mitattuna) suosii niitä, joilla ei ole mitään parempaa tekemistä, mikä tarkoittaa nuoria miehiä. Monessa kulttuurissa naisten voi olla vaikea investoida korkeampaan koulutukseen.
\item Se suosii mystisiä teorioita tieteellisten sijaan. Jos meinaat viettää vuosia opiskellen, ilman mitään tapaa testata oppimaasi oikeassa elämässä, olet jo pelaamassa Espanjalaista Vankia. Mystiset teoriat tulevat aina olemaan houkuttelevampia, sillä ne voivat yksinkertaisesti valehdella sinulle (``Kyllä, opettele minut ja tulet paljon rikkaammaksi!'').
\item Se kohtelee vanhoja ihmisiä jätteenä. ``Eläköitymisen'' käsite on merkki viha-viha-suhteesta työntekijän ja työnantajan välillä. OLTOn kanssa iällä ei ole merkitystä. Jos kykenet oppimaan, leikkimään, työskentelemäqn ja opettamaan nelivuotiaana, voit tehdä sitä myös 84-vuotiaana (olettaen, että kykenet keskustelemaan).
\item Se ei kykene käsittelemään muutosta ja mahdollisuuksia. Jos koulutus on pääsyvaatimus työhön, ei jo työelämään päätynyt voi enää oppia uusia asioita hänen työnsä ulkopuolelta maksamatta kovaa hintaa.
\item Se jakaa sukupolvet ja ikäluokat. Työssäkäyvät eivät voi opettaa nuoria. Sen sijaan opetus kulkee ``koulutusjärjestelmäksi'' kutsutun kerroksen kautta, joka päättää yhdessä poliittisen eliitin kanssa mitä tietämystä se opettaa. Luotan siihen, että näet mahdollisuudet vallan monopoleille ja kulttien luomiselle.
\end{itemize}
Yhteiskunnan 4-elämää-teoria on näkyvästi epätarkka ja se tuotta loputtomasti kitkaa, jota moni meistä kokee suurena osana elämästämme. Luulen, että jos eivät kaikki, niin monet tähän järjestelmään jumiutuneet haaveilevat paosta, paluusta siihen vapauteen, jota he kokivat lapsena. Jotkut määrittävät oman elämänsä. Se on kuitenkin harvinaista ja vaikeaa. Yhteiskunta tuppaa paheksua kulkureita ja opportunisteja.

Tästä pääsemme opetukseen numero kuusi: \emph{yhteiskunta on kylläinen mystisistä teorioista.} Se nalkuttava ääni, joka mielesi perukoilla sanoo jonkin olevan syvällisesti pielessä elämässäsi, on asian ytimessä. Moderni teknologia ja hintagravitaatio kompensoivat jonkin verran, mutta elämä voisi kuitenkin olla niin paljon parempaa.

OLTO on hyvissä voimissa muodollisten koulutus- ja työjärjestelmien ulkopuolella. Opimme sen avulla Internet-maailmassa. Massiiviset Internet-foorumit toimivat sen kautta. Kun kaikki voivat kontribuoida täydellä nopeudella, on OLTO erittäin tehokas kollektiivinen oppimismenetelmä. Kun ihmiset käyttävät sitä vihaisena, niinkuin Anonymous teki, he muuttuvat poliittiseksi voimaksi ja vakavaksi uhaksi vakiintuneille valtarakenteille.

\section{Iltasatuja}
\begin{quotation}
---Olipa kerran taikalinna korkeilla vuorilla\ldots

---Minkä värinen se oli?

---Onko sillä väliä? OK, sanotaan, että se oli valkoinen. Joka tapauksessa, tässä linnassa asusti prinsessa ypöyksin\ldots

---Ypöyksin? Kuinka prinsessa voisi asua yksin? Kuka teki hänelle ruokaa? Kuka siivosi linnan?

---Hei, haluatko kuulla tarinan vai et?

---OK, OK, olen pidän suuni kiinni.
\end{quotation}
Puhun monissa konferensseissa. Tyylini on muuttunut vuosien myötä. Käytin aikaisemmin kalvoja, kuten useimmat muutkin puhujat. Tyylikkäät kalvot, jotka korostavat avainasioita, herättäen tunteita, kertoen huolella rakentamani tarinan. Nykyään tulen ilman kalvoja ja käsikirjoitusta, ja sen sijaan, että kertoisin tarinani yleisölle monologina, improvisoin dialogin heidän kanssaan.

Dialogi on kovaa työtä, ja se uuvuttaa. Mutta se on helpompaa, kuin kalvojen tekeminen. Olen vaihtanut dialogiformaattiin yksinkertaisesta syystä. Kun käytin monologeja, ehkä jokunen ihminen sadasta ``tajusi'' juttuni. Jokainen näki, mitä selostin. Hyvin harva uskoi minua. Dialogin avulla saan huoneesta puolet tai yli tajuamaan ja reagoimaan.

Dialogi on muinainen muoto, joka on lähes kadonnut nykypäivänä. Olemme niin rakastuneita teknologiaan, että olemme unohtaneet sen. Katso videoni. Klikkaa linkkejäni. Ryhdy kaverikseni Linked-inissä. Mutta luoja varjelkoon jos pyydän sinua kumoamaan hypoteesiani livenä, yleisön edessä. 

Syy, miksi käyn konferensseissa, ei ole myydä tai evankelisoida. Syy on oppia. Dialogi on oleellisesti se tapa, millä suoritamme ydinprosessin testausvaiheen. Tässä on teoriani, anna kun selitän, ja sinä kerrot minulle, jos sinulla on dataa tai havaintoja, jotka falsifoivat sen. Jos kykenen kertomaan tarinani viidelle asiantuntijayleisölle, eikä sitä kumota, se seisoo. Jotta tämä prosessi toimisi, on minun ehdottoman pakko käyttää dialogia.

Mutta---ja tämä selittää, miksi useimmat puhujat rakastavat kalvojaan---jos haluan vakuuttaa sinut mystisestä teoriasta, silloin dialogi on viimeinen asia, mitä haluan. Sen sijaan turvaudun toiseen lapsuusajan skenaarioon, nimittäin iltasatuun. Nukkumaanmenoaikaan hyväksymme minkä tahansa sadun kysymättä. Emme välitä siitä, minkä värinen linna on, sillä olemme jo puolinukuksissa, unelmoiden korkeista muureista.

Tämä on monologin voima: se lähettää yleisön jonkinlaiseen nukkumistilaan missä yleisö hyväksyy mitä vain. Poliittiset puhuja, saarnaajat, luennoitsijat ja halvat myyjät tietävät tämän hyvin. Kuten tavallista, ongelma on siinä, että olemme jollakin tasolla immuuneja tällaiselle yksinkertaiselle manipuloinnille. Heräämme, ja uni on kadonnut. Parhaat kauppiaat kuuntelevat tarkasti yleisöään ennen kuin he askartelevat valheensa jotka vetoavat siihen.

Konferenssikalvot halventavat satua entisestään supistaen sen kaksiulotteiseksi visuaaliseksi pikaruoaksi. Tässä on iskusana. Tuolla on söpö kuva. Tässä on jokunen mieleenpainuva sana! Viihteenä tämä on oivallista, jopa eleganttia. Opetus- ja oppimistyökaluna se on traagisen surkea. Maailmanluokan puhujat taustallaan vuosikymmenten kokemus matkustavat kaukaisiin kohteisiin vain leikkiäkseen teksti-ääneksi-syntetisaattoria 45 minuutiksi. Jos menet konferenssiin, veikkaan, että muista satunnaisia kuvia, mutta et tarinoita. ``Parhaat keskustelut käydään käytävillä\vmq{.}'' Tämä lausahdus kertoo epäonnistumisesta.

Ja niinpä tulemme opetukseen numero seitsemän: \emph{paras oppimis- ja opetuskaava on dialogi.} Tarkemmin sanottuna kaava, jossa yksi henkilö esittää teorian ja toiset yrittävät kumota sen, ja missä sivustaseuraajat aploodeeraavat ja kannustavat. Tämä kuulostaa hyökkäävältä, mutta mikäli osallistuminen on tarpeeksi vapaata, ei ensimmäisellä kerralla ole pakko onnistua, ja niinpä väärässäolemisessa ei ole mitään hävettävää. Itseasiassa, kun joku ottaa työsi ja parantaa sitä, tunnet tyydytystä, et häpeää.

Tällä tavalla tykkään järjestää ohjelmistoprojektini. Sillä tavoin organisoin työpajani. Ja sillä tavoin organisoisin myös konferenssin (ja tulenkin organisoimaan vuonna 2015, jos kaikki menee hyvin): monituntisia sessioita, joissa puhujat esittelevät teoriansa (teknologiasta) ja pyytävät yleisöä kumoamaan ne. Moni muukin asia on rikki totutussa konferenssimallissa. Lihaa toista esseetä varten.

\section{Yksinkertainen ei ole helppoa}

Olen kuullut monimutkaisia malleja kutsuttavan ``ylisuunnitelluiksi\vmq{.}'' Olisi tarkempaa kutsua niitä ``alisuunnitelluiksi\vmq{,}'' sillä monimutkaisuuden muuttaminen yksinkertaisuudeksi vaatii oikeaa työtä. Voin kuvailla tämän prosessin, sillä kirjoitan ohjelmakoodia sitä käyttäen.

Otat olemassa olevan teorian ja sovellat sitä uusiin ongelmiin. Tämä luo kitkaa, minkä voit ratkaista laajentamalla teoriaa. Voit tehdä tämän kerta toisensa jälkeen, ja teoria muutta leveämmäksi ja monimutkaisemmaksi. Aivan kuin lisäisit pizzaan lisää täytteitä.

Jos olet tottunut kitkaan, kuten jotkut ihmiset näyttävät olevan, ei sinua haittaa, vaikka pizzasta tulisi kuinka suuri ja sotkuinen. Teet vain aina tarvittaessa hiukan lisää tilaa mentaaliselle pöydällesi. Mutta jos ärsyynnyt kitkasta helposti, kuten minä ärsyynnyn, kesyttämättömästä pizzakompleksisuudesta tulee ongelma, joka pitää ratkaista. Täytteiden lisääminen täytyy lopettaa, ja sen sijaan on luotava päälle uusi teoria, abstraktio, joka saa aikaan saman asian, mutta on paljon yksinkertaisempi.

Sen sijaan, että meillä olisi ``\emph{pizza, jonka päällä on tomaattia, mozzarellaa, anjovisfileitä ja kaprista}\vmq{,}'' saamme ``\emph{pizza Napoletana}\vmq{.}'' Nyt voimme sekoittaa kokojen teorian (``pieni\vmq{,}''  ``keskikikoinen\vmq{,}'' ``suuri\vmq{,}'' ``Americano'') reseptien teoriaan. Tällaisten abstraktioiden käyttäminen, todellisia pizzoja vastaan testaaminen (``En tilannut sipulia!'') ja parantaminen on helpompaa sekä asiakkaille että ravintolalle.

Tämä vie meidät opetukseen numero kahdeksan: \emph{yksinkertaisuus voittaa aina monimutkaisuuden.}

\section{Sukupuolien välisen kuilun selvittäminen}

Ei ole uusi havainto, että naisista tulee miehiä parempia tieteilijöitä. Niinpä sukupuolten välinen kuilu aiheuttaa joillakin aloilla, kuten vaikkapa ohjelmoinnissa, sekaannusta monissa ihmisissä. Olen alkanut uskoa, että vastaus piilee siinä, että suurin osa ohjelmoinnista ei yksinkertaisesti ole tiedettä. Asia on ollut näin siitä lähtien, kun aloin ohjelmoimaan: se valtava ohjelmistojen bulkki, jota kirjoitetaan ja käytetään, perustuu mystisiin teorioihin.

Useimmat ohjelmistoprojektit epäonnistuvat. Tämä ei ole suunnittelua. Se on käsien heiluttamista ankan sisäelimien yllä. Ooh, kyllä, me kaikki teeskentelemme tietävämme, mitä teemme. Ala on aina itsevarma, erityisesti silloin kun sillä ei ole hajuakaan. Totuus on, että me hädin tuskin saamme kiinni paikkansapitävyyden lautasta epäonnistumisen merellä.

Kuten jo selitin, mystiset teoriat tuppaavat syrjimään. Niiden oppiminen ottaa aikaa, vuosien opiskelua ja harjoittelua, ja ne levittäytyvät yhden tai useamman vuosikymmenen ajalle nuoren aikuisen elämässä. Kaikki eivät voi tehdä tällaista investointia. Eivätkä kaikki ole tarpeeksi tyhmiä hyväksyäkseen mystisiä teorioita ilman ankaraa ``Mitä helvettiä tämä on?''-tyyppistä henkistä oksennusrefleksiä. Erityisesti useimmat nuoret naiset rakentavat todellisia sosiaalisia verkostojaan liian kiireissään omistaakseen vuosia elämästään hölynpölyn opiskeluun.

Sille, miksi teknologian temppelit ovat täynnä nuoria, ei-niin-työllistyneitä miehiä, on syynsä. Mystiset teoriat houkuttelevat nuoria miehiä yhtä voimakkaasti, kuin ne yököttävät nuoria naisia. Kilpailkaamme siitä, kuka tietää salaisimmat faktat. Katsotaan, kuka voittaa yleisen äänestyksen. Hakataan toisemme väittelyssä, ei käyttämällä tieteellistä menetelmää, vaan lainaamalla mystiikkaa toisillemme. Minun näkymättömyysviittani voittaa tulimiekkasi!

Ja papit rakentavat sekaannuttavia pyramidihuijauksia, jotka lupaavat ``hyväksy vain tämä mystinen teoria ja opeta sitä muille, niin saat itsekin valtaa\vmq{.}'' Näin kultit toimivat. Sitä voisi argumentoida, että ainakin teknokultismi absorboi tämän typerän yksinkertaisen yhteiskunnan siivun, jota nuoret miehet edustavat, energian. Jokatapauksessa se tuntuu haaskaukselta.

Joten tulemme opetukseen numero yhdeksän: \emph{todellinen tiede toivottaa jokaisen tervetulleeks,} olipa henkilön ikä, sukupuoli ja syntyperä mikä tahansa.













