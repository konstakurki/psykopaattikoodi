\chapter{Esipuhe}\label{preface}

\section{Psykopaatin dekoodaaminen}

Elämässä tulee vastaan pelottavia ihmisiä. Ihmisiä, jotka ottavat mitä haluavat käyttäen oveluuttaan ja hurmausvoimaansa. Huijareita. Ammattimaisia valehtelijoita. He ottavat ystäviltä, kollegoilta, perheeltä ja tuntemattomilta ihmisiltä. He eivät koskaan pyydä anteeksi tai koe tunnontuskia niitä kohtaan, joita he satuttavat. Heillä on usein rikostaustaa. Kutsumme heitä monilla nimillä. Narsisti. Antisosiaalinen. Sosiopaatti. Toimitusjohtaja. Ja yhä useammin kutsumme heitä Psykopaateiksi.

Psykopaatit nostavat esiin monia kysymyksiä. Mikä näitä ihmisiä vaivaa? Ehkä heillä oli kylmät ja etäiset vanhemmat. Ehkä he kärsivät hyväksikäytöstä lapsena? Ehkä jokin heissä on rikki, kemiallinen epätasapaino, tai pahoja henkiä. Tai seuraava aste ihmisevoluutiossa. Uusi superihmisten rotu, ehkä? Voimmeko tunnistaa heitä? Voimmeko oppia bongaamaan heitä kodeissamme ja kaduilla? Mitä heidän mielessään liikkuu? Ovatko he tietoisia aiheuttamistaan vaurioista? Nukkuvatko he yönsä hyvin? Kuinka selviytyä heistä? Kuinka sellaisia \emph{hirvittäviä} ihmisiä voi olla olemassa? Olenko minä yksi heistä?

Onneksi on olemassa hyviä vastauksia, jotka irrottavat kauhean mysteerin psykopatiasta. Tämä oli tavoitteeni tässä kirjassa: dekoodata psykopaattinen mieli ja kirjoittaa manuaali muille. Materiaali perustuu minun ja monien muiden kokemukseen. Sitä on testattu todellisessa elämässä ja se näyttää toimivan. Sanottuani tämän ota huomioon seuraava osio.

\section{Vastuuvapauslauseke}

\emph{Kirjoittaja ei ole psykiatri tai lääketieteen ammattilainen. Kirjoittaja ei jaa lääketieteellisiä neuvoja tai määrää mitään tekniikkaa hoidoksi fyysisiin tai lääketieteellisiin ongelmiin ilman lääkärin neuvoa, ei suorasti eikä epäsuorasti. Kirjoittajan tarkoitusperä on tarjota yleisluontoista informaatiota auttaakseen lukijaa emotionaalisen tai henkisen hyvinvoinnin etsimisessä. Siinä tapauksessa, että lukija käyttää mitä tahansa tietoa tässä kirjassa itsellensä, mikä on lukijan oikeus, ei kirjailija eikä julkaisija ota minkäänlaista vastuuta lukijan teoista. Mikään tässä kirjassa esitetyistä ehdotuksista ei ole tarkoitettu korvaamaan lääkärin hoitoa tai sekaantumaan diagnoosin, määrättyjen lääkkeiden tai terapian kanssa.}

\section{Riko lasi}

Jos luet tätä kirjaa saadaksesi apua vaikeaan elämäntilanteeseen, aloita tästä. Selostan avainopetukset lyhyessä yhteenvedossa.

Sinun täytyy oivaltaa muutama asia. Ensimmäinen on: oletko psykologisen hyväksikäytön uhri? Sellainen on harvoin avointa. Mustelmat tuppaavat olemaan henkisiä, eivät fyysisiä. Hyväksikäyttävä suhde on piilossa valheiden alla. Niiden, joita hyväksikäyttäjä kertoo sinulle, sekä niiden, joita kerrot itsellesi. Tämä tekee hyväksikäytön selkeästä näkemisestä vaikeaa.

Aloitetaan tuntemuksistasi. Oletko usein surullinen, masentunut, jopa itsetuhoinen? Koetko olevasi tyhjä ja arvoton? Oletko taivuttanut elämäsi toisen ihmisen tarpeiden mukaan? Otatko epäonnistumisten syyt niskoillesi, ja yritätkö kerta toisensa jälkeen korjata asioita? Tuntuuko sinusta siltä, että saatat olla hullu? Tuntuuko sinusta siltä, että olet palanut loppuun? Oletko yksinäinen, ja oletko menettänyt vanhoja ystäviä ja suhteita? Juotko liikaa?

Jos nyökkäilet tälle listalle, olet luultavasti hyväksikäyttävässä suhteessa. Se on paljon yleisempää kuin luulet. Arvioisin, että 10\%--20\% ihmisistä on hyväksikäyttävässä suhteessa millä tahansa ajanhetkellä. Sitä voi olla vaikea tunnistaa, myöntää ja käsitellä sekä uhreille että heidän ystävilleen ja vanhemmilleen.

Seuraavaksi tarkastelemme suhdetta, joka stressaa sinua. Alkoiko se ``täydellisesti'' ja muuttui sitten ajan myötä painajaiseksi? Merkkaavatko sitä yhtäkkiset, odottamattomat kriisit? Luonnehtivatko sitä äärimmäiset tunteet? Onko siinä verbaalista tai fyysistä väkivaltaa? Oletko investoinut kaikkesi suhteeseen saaden hyvin vähän takaisin? Onko siitä tullut ainut ihmissuhteesi, joka merkitsee, jättäen varjoonsa ystäväsi ja perheesi? Oletko kykenemätön ajattelemaan vaihtoehtoja?

Nämä ovat hyväksikäyttävän siteen tunnusmerkit. Jos tämä kuvaa tilannettasi, olet hyökkäyksen kohteena. Oleta, että tätä tekevä henkilö on psykopaatti joko ilman muodollista diagnoosia tai sen kanssa. Tulemme yksityiskohtaiseen diagnoosiin myöhemmin. Se, mikä merkitsee tässä vaiheessa, on että tunnistat tilanteesi ja sen, kuinka sinua kohtaan hyökätään. Se voi vaikuttaa sattumanvaraiselta, mutta se on systemaattista. Tavoite on sekoittaa ja eristää sinut, riisua sinut voimavaroistasi, ja sitten tuhota ja hylätä sinut. Väkivalta on vain osa tätä kaikkea.

Jos tämä ei kuvaa tilannettasi, voit hypätä tämän osion loppuosan ohi.

Nyt ota uusi katse toiseen ihmiseen. Jos vastassasi on psykopaatti, hänen todellisen luonteensa näkeminen voi olla mahdotonta. Sinun täytyy katsoa sivuttain, toisissa ihmisissä näkyvien heijastusten kautta. Näetkö jonkun, joka välittää muista, vai jonkun, joka välittää itsestään? Tekeekö hän hiljaisia, huolellisia suunnitelmia, vai onko hän kaoottinen? Säästääkö hän ja investoiko hän, vai onko hänen raha-asiansa sekaisin? Ympäröikö häntä onnelliset ihmiset, vai pilvi stressaantuneita, pakkomielteisiä seuraajia? Onko hänellä eheä työ- ja sosiaalinen historia, vai onko hänen historiansa tyhjä mysteeri?

Kun saat tulokseksi ``Minä olen hyväksikäyttävän psykopaatin uhri\vmq{,}'' olet puolivälissä matkalla kohti ovea. Pakeneminen houkuttaa, ja kun kerrot kokemuksistasi muille, he kehottavat sinua lähtemään. Kansankulttuurissa ``psykopaatti'' tarkoittaaa ``sarjamurhaaja\vmq{.}''

Todellisuudessa ``psykopaatti'' tarkoittaa sitä hidasta elämänvoimasi näivettymistä, aivan kuin vampyyri imisi sinua kuivaksi viikkojen tai jopa vuosien ajan. Fyysistä väkivaltaa saattaa olla, vaikkakin se on pääasiassa merkityksetöntä verrattuna psykologisiin vaurioihin. Tämä tarkoittaa, että jos lähdet, viet samalla vauriot mukanasi.

Tässä on yleinen strategiani: kärsivällisyyttä, havaintoja ja suhteen kääntäminen ympäri hitaasti. Muutut uhrista ja mahdollistajasta?? liikkumattomaksi voimaksi, joka tunnistaa ja estää psykopaatin monet hyökkäykset. Pikkuhiljaa lamautat hyväksikäyttäjän, ja samalla saat voimasi takaisin. Lopulta päätät suhteen omilla ehdoillasi, kokonaisena ihmisenä.

Joskus voit vain sanoa hyväksikäyttäjälle: ``Se on ohi, älä ota minuun yhteyttä\vmq{.}'' Mutta usein suhteen rikkominen vaatii voimaa ja aikaa.

Laki tuppaa jättämään aikuisten välisen psykologisen hyväksikäytön huomiotta. Useimmat psykopaatit ovat huolellisia eivätkä he jätä todisteita. Poliisi ja oikeusistuimet ovat taipuvaisia kyynisyyteen ``Hän sanoi, tuo sanoi'' -tyyppisten syytösten kohdalla. Ja mitä ikinä sanotkaan, psykopaatilla on aina parempi vale. Tällä tavoin kultit kykenevät toiminaan kirkkaassa päivänvalossa.

Joten et voi tehdä verbaalisia syytöksiä. Jos teet niitä, toimivat ne luultavasti sinua itseäsi vastaan. Kun kyse on sanasodasta, psykopaatit ovat vahvoilla. Sen sijaan ole kärsivällinen ja kerää materiaalisia todisteita. On olemassa tapoja, joilla psykopaattia voi provosoida tekemään ja sanomaan itsetuhoisia asioita.

Kun kohtaat psykopaatin, tai edes muutat käytöstäsi hiukan, reaktio on yleensä lisää hyväksikäyttöä. Tulet kauhistumaan ja sinuun sattuu. Haluat, että asiat muuttuvat ``normaaliksi\vmq{,}'' ja osa sinua huutaa: ``Älä provosoi häntä, se tekee asioista vain vaikeampia!''

Tämä on se piste missä moni luovuttaa ja palaa hyväksikäyttäjänsä huomaan. On helpompaa hyväksyä kuin taistella vastaan tuskaisesti. Mutta hyväksikäytön hyväksyminen tarkoittaa hidasta kuolemaa.

Näemme kokemuksesta että useimmat uhat ovat bluffia ja uhoa. Pedot ovat hauraita. Ne eivät kykene selviytymään paljastumisesta. Ne murisevat ja kiusaavat, mutta kun ne kohtaavat todellista vastustusta ja laajempien sanktioiden riskin, ne useimmiten perääntyvät.

Opettele lait jotka koskevat hyväksikäyttöä ja ahdistelua. Ystävysty paikallisen poliisin kanssa. Selvitä, millaisia ilmoituksia voit jättää. Lasketaanko väkivaltainen kielenkäyttö ahdisteluksi? Vai tarvitsetko mustelmia ja lääkärin lausunnon? Onko sinulla laillinen oikeus äänittää puhelinsoittoja ja keskusteluja? Tee tarvittavat tutkimukset. ??

Jos teillä on yhteistä omaisuutta, yritys tai lapsia, ota yhteyttä juristiin. Poliisi antaa sinulle paikallisten uhrien tukiryhmien osoitteet. Jos olet vaikeassa perhetilanteessa, hyväksikäytön uhreihin erikoistunut psykologi auttaa sinua taistelemaan ulos suhteesta. Olitpa mies tai nainen, avun pyytämisessä hyväksikäyttäjää vastaan taistelemisessa ei ole mitään hävettävää.

Ja tässä tulee supervoimasi: toiset ihmiset. Kun puhut toisten kanssa, huomaat, että monilla on samankaltaisia kokemuksia. Kun saat todistusaineistoa hyväksikäyttävästä käytöksestä, voit julkaista sen ja tehdä rikosilmoituksen??. Hyväksikäyttäjäsi voi piiloutua vain, jos muut anteeksi antavat ja unohtavat hänen käytöksensä.

Ennen kaikkea: kärsivällisyyttä ja rauhallisuutta. Sinun täytyy oppia paljon ja muuttaa syviä olettamuksia elämästäsi. Sinä et ole syyllinen. Hyväksikäyttäjät valitsevat uhrinsa eikä toisin päin. Lue kirja hitaasti ja suhtaudu nykyiseen tilanteeseesi tilaisuutena muuttua vahvemmaksi ja viisaammaksi ihmiseksi.

\section{Kuinka kirja toimii}

\emph{Psykopaattikoodissa} on kahdeksan lukua, joista jokainen kertoo osan tarinaa. Voit lukea ne missä järjestyksessä tahansa. Suosittelen, että silmäilet tekstin ensin nopeasti läpi ja luet sen sitten huolellisesti muutaman kerran. Sitten keskustele luottamasi ihmisten kanssa, ja anna uuden tietämyksen upota hitaasti. Tee tutkimuksta, lue psykopaattifoorumeja ja muita kirjoja. Opittavaa on paljon, ja sen kaiken sulattaminen ottaa aikaa, jopa vuosia.

Luvussa \ref{predator} otamme esille kirjan ydinhypoteesin, mikä on, että psykopaatit ovat sosiaalisia petoja jotka vaanivat toisia ihmisiä. Idea ei ole uusi, ja siitä on tulossa valtavirtaa. Olen vain vienyt sen muita pidemmälle.

Luvussa \ref{the-hunt} näemme, kuinka psykopaatit metsästävät. Sukupuoli ja ikä ajavat metsästystä vahvasti. Jokaisessa tapauksessa psykopaatti käyttää häivetekniikoita ja harhautuksia päästäkseen saaliinsa lähelle ja voittaakseen sen luottamuksen. Opettele nämä kaavat ja muutut immuuniksi niille.

Luvussa \ref{attack-and-capture} näemme, kuinka psykopaatit vangitsevat uhrinsa ja rakentavat hyväksikäyttävän siteen. Psykopaatti eristää ja manipuloi kohteensa antamaan mitä tahansa. Taas kerran tietämällä kaavat muutumme immuuniksi niille.

Luvussa \ref{the-feeding} näemme psykopaattisen suhteen raaimman vaiheen. Tässä vaiheessa psykopaatti imee kohteensa kuiviin samalla, kun hän hyväksikäyttää kohteen hiljaisuuteen ja hyväksyntään.

Luvussa \ref{hunting-mallory} alamme kääntää pöytiä seuraamalla ja tunnistamalla psykopaatteja. Näemme yli sata piirrettä ja käytösmallia, jotka voit tunnistaa, mukaan lukien sen, miltä sinusta itsestäsi tuntuu, jos olet psykopaatin syleilyssä.

Luvussa \ref{the-dance-of-emotions} tarkastelemme ihmisen emootioita. Ne ovat avainasemassa, kun pyrimme ymmärtämään psykopatiaa ja tapaamme reagoida siihen. Käymme läpi noin viisikymmentä universaalia ihmistunnetta, joista psykopaatti tuntee yhdeksän.

Luvussa \ref{escape-from-jonestown} näemme, kuinka vapautua psykopaatin syleilystä. Materiaali selittää askel askeleelta kuinka saada voimat takaisin ja lamauttaa hyväksikäyttäjä. Kyse ei ole yhden yön prosessista, joten kärsivällisyys ja rauhallisuus ovat tarpeen.

Luvussa \ref{questions-to-the-author} vastaan usein kysyttyihin kysymyksiin jotka seuraavat materiaalista.

Kirja on saatavilla Amazon.com:sta, Kindlelle, sekä ilmaiseksi osoitteesta psychopathcode.com. Jaa vapaita PDF-tiedostoja ja sähkökirjoja ystävillesi ja perheellesi.

\section{Kuinka kirja syntyi}

Olen nörtti, joka kirjoittaa koodia, artikkeleita ja kirjoja. Tutkintoni sain tietokonetieteestä. Opiskelin psykologiaa vain hieman yliopistossa. Tämä ei ole tavanomainen lähtökohta kirjalle psykopaateista. Joten selostan, kuinka päädyin tähän.

Urani aikana olen työskennellyt tuhansien ihmisten kanssa. Olen rakentanut satoja tiimejä ja monia pieniä yrityksiä, voittoa tavoittelemattomia organisaatioita ja Internetyheisöjä. Minun on täytynyt oppia tuntemaan ihmisluonto. Jotkut opetukset ovat ilmeisiä. Toiset ovat hyvin piilossa. Olemme niin monimutkainen laji. Ja kuitenkin ihmisluontoa on mahdollista ymmärtää, dekoodata ja ennustaa.

Kirjassani \emph{Culture \& Empire} aloin kirjoittaa psykologiasta. Sosiaalinen psykologia kiehtoo minua ja on erityisosaamistani. Tarkoitan sitä, kuinka ryhmät toimivat ja kuinka ihmiset toimivat ryhmissä. Se on \emph{open source} -yhteisönrakennukseni ydinasia. Tietokoneohjelmissa on kyse ihmisistä, käy lopulta ilmi.

Kirjassa \emph{Culture \& Empire} katsahdin myös sitä, kuinka konfliktit toimivat muutosta ajavana voimana.

Unelmoimme rauhasta ja vakaudesta. Mutta läpi historian kaikista suurimmat harppaukset ovat nousseet konfliktista ja kaaoksesta. Rakennamme ihmisoikeuksien maailman. Työskentelemme suojellaksemme ympäristöä. Rakennamme lakijärjestelmiä ja oikeusistuimia ja poliisivoimia suojelemaan vaurautta ja rauhaa. Et välttämättä huomaa sitä jokapäiväisistä uutisista, mutta väkivalta vähenee vuosi vuodelta, ja on aina vähentynyt.\linkki{sdfd}

Ihmiset eivät ole hyviä eivätkä pahoja. Me olemme selviytyjiä. Teemme mitä meidän ikinä tarvitseekaan tehdä lisääntyäksemme ja varmistaaksemme että lapsemme menestyvät. Useimmat onnistuvat tekemällä kovaa työtä. Muutamat elävä loisina, ottaen toisilta, kuin vampyyrit. Kutsun näitä ``hyviksiksi'' ja ``pahiksiksi\vmq{.}''

Kun olin saanut \emph{Culture \& Empiren} valmiiksi, kohtasin yksityiselämässäni pahiksia. Kirjoittajana en juossut karkuun enkä kiistänyt sitä, mitä tapahtui. Sen sijaan aloin tehdä muistiinpanoja ja suorittamaan pieniä kokeita. Ei sitä joka päivä pääse leikkimään psykopaateilla.

Havaitsin, että työni sosiaalisessa psykologiassa oli jäänyt puolitiehen. Olin keskittynyt hyviksiin ja vain raapaissut pahisten maailman pintaa. Mutta nämä strategiat eivät ole vaihtoehtoja. Ne kietoutuvat toisiinsa ja toimivat yhdessä pitkässä, mystisessä kilpavarustelussa. Tämä kilpavarustelu ytimessä siinä, mitä ``olla ihminen'' tarkoittaa.

Internetissä on paljon materiaalia psykopaateista. Aihe trendaa joka vuosi.\linkki{sdfds} Tuhansia tarinoita, joita aihe on koskettanut. On tutkimusta psykologeilta ja psykiatreilta. On papereita kriminologeilta ja nettideittailueksperteiltä. On jopa diagnosoitujen psykopaattien kirjoittamia blogeja ja foorumeja.

Joihinkin valtaviin kysymyksiin ei ole olemassa yleisesti hyväksyttyjä vastauksia. Aloitetaan vaikkapa siitä, että mikä aiheuttaa psykopatiaa? Voimmeko parantaa sitä; haluammeko edes yrittää sitä? Voimmeko tunnistaa psykopaatteja ``villissä luonnossa\vmq{,}'' siis ilman psykiatrista tutkimusta? Ovatko he aina väkivaltaisia ja vaarallisia, ja jos eivät, mitkä tekijät vaikuttavat näihin seikoihin? Onko kyseessä käyttäytymisen spektri, kuten älykkyys? Vai onko psykopatia enemmänkin binäärinen ilmiö, kuten sukupuoli? Kuinka psykopaatit ajattelevat ja operoivat? Voimmeko suojella itseämme niiltä?

Tohtori Robert Hare on psykopatian auktoriteetti, ja käytän termiä samoin kuin hän, vaikkakin laajemmin. Hänen tarkastuslistansa\linkki{sdfs} keskittyy miehiin ja jättää monet naispsykopaatit pimentoon. Klassinen psykopatiatutkimus on keskittynyt tuomittuihin rikollisiin: taparikollisiin ja väkivaltarikollisiin. 1990-luvun puolen välin tienoilla tutkijat alkoivat tarkastella psykopaatteja, jotka piilottelevat tavallisessa kansassa. Nämä ovat ``subkliinisiä'' tai ``menestyviä'' psykopaatteja jotka pysyvät kaukana väkivaltaisesta rikollisuudesta. He joutuvat harvoin pidätetyksi.

Lukema, jonka näemme on suurinpiirtein 10\% kummassakin sukupuolessa. ``Kourallinen empiirisiä tutkimuksia antavat ymmärtää\ldots subkliininen psykopatia on paljon yleisempää kuin sen kliininen pari, perustiheyden liikkuessa viiden ja viidentoista prosentin välillä\vmq{.}'' ---Jay C. Thomas, Daniel L. Segal, puhuessaan Gustafsonin ja Ritzerin vuoden 1995 työstä ja Pethmanin ja Earlandsonin vuoden 2002 työstä kirjassa \emph{Comprehensive Handbook of Personality and Psychopathology: Personality and Everyday Functioning} (Google books).

Selvästikin jotkut psykopaatit ovat haitallisempia kuin toiset. Käytän itse neljää prosenttia leikkauspisteenä. Tämä on jokseenkin mielivaltaista. Liian pieni numero ja suurin osa psykopaateista ja heidän aiheuttamistaan vaurioista jää näkemättä. Liian suuri numero ja trivialisoimme ilmiön. \emph{Suomentajan huomio: tässä kohdassa Hintjens on jollakin tapaa kujalla. Jos psykopatia on kutakuinkin binäärinen ilmiö, niin kuin Hintjens selvästi kaiken muun tekstin perusteella ajattelee, ei minkäänlaista rajanveto-ongelmaa pitäisi olla olemassa.}

Siinä missä moni psykopaatti vaikuttaa täysin ``normaalilta\vmq{,}'' olen tullut siihen käsitykseen että jotkut piilottelevat suoraan silmiemme alla, tiettyjen persoonallisuushäiriöiden takana.Nämä ovat nasistinen, epävakaa ja huomiohakuinen persoonallisuushäiriö. Kun otat nämä mukaan laskuihin, muuttuu kuva selvemmäksi ja yksityiskohtaisemmaksi. Plussana saamme käyttökelpoisia malleja näiden häiriöiden aiheuttamista vaurioiden selviytymiseen.

Teoriassa persoonallisuushäiriöt kuten epävakaa persoonallisuus ovat hoidettavissa lääkkeillä ja terapialla. Käytännössä tämä toimii huonosti tai ei ollenkaan. Samoin pätee diagnosoituihin psykopaatteihin (heihin, joilla on ``antisosialinen persoonallisuushäiriö''). Terapia näyttää tekevän heistä vain taitavampia ihmisten manipuloinnissa.

Se, mikä voi toimia, on rajoittaa ja korjata vaurioita, joita psukopaatit aiheuttavat. Aivan kuten koulukiusaajatkaan, eivät psykopaatit kärsi masennuksesta. Se on perhe, ystävät ja kollegat jotka maksavat viulut. Sitten, kun näet prosessin psykopaatin törmäyskraatterien takana, voit puuttua asioihin.

Puuttuminen ei ole yksinkertaista. Sellaisten ihmisten, jotka ovat viettäneet koko elämänsä hurmaten, manipuloiden ja kiusaten toisia, käsitteleminen on oletusarvoisesti mahdotonta. Jos yrität varoittaa ryhmää, huomaat joutuvasi itse syytetyksi. Jos yrität varoittaa yskittäisiä ihmisiä, huomaat heidän kääntyvän sinua vastaan. Sinun täytyy liikkua hitaasti, huolellisesti ja käyttäen oikeaa tietämystä.

Joten tämä kirja keskittyy psykopaatin prosessiin, siihen, kuinka tunnistaa se, kuinka se toimii, ja kuinka lamauttaa se.

En ole pätevä psykologi tai psykiatri. En voi argumentoida auktoriteetin suojasta. Se, mitä voin tehdä, on kehittää malleja ja testata niitä psykopaateilla joihin minulla on pääsy. Voin testata niitä vasten epätavallisia tilanteita menneisyydestä. Ja voin testata niitä toisten psykopaatteihin kietoutuneiden ihmisten kautta.

Olen käyttänyt vuosia lukien kirjallisuutta. Foorumeja, kirjoja ja artikkeleita. Mitä tahansa, mikä voi näyttää valoa tähän kummallisten kanssaeläjien mysteeriin ja siihen, kuinka he toimivat. Olen puhunut sadoille ihmisille aiheesta. Joukko sisältävää psykologeja, jotka ovat erikoistuneet hyväksikäyttöön, ja kehityshäiriöihin. Se sisältää ihmisiä, jotka ovat selviytyneet hyväksikäyttävistä puolisoista. Ihmisiä, jotka ovat yrittäneet itsemurhaa paetakseen. Ihmisiä, joden vanhemmat käyttivät heitä hyväkseen ja sopivat psykopaatin profiiliin.

Olen käyttänyt näitä malleja rakentaakseni järjen puutarhoja yksityis- ja työelämässäni. Ne, jotka työskentelevät kanssani, tietävät, että Internet-yhteisömme ovat ennenkaikkea \emph{iloisia} paikkoja. Tämä ei ole sattumaa. Se kumpuaa pitkästä, huolellisesta työstä pahisten pitämiseksi loitolla.

Niin pitkälle, kuin mahdollista, olen työskennellyt toistuvien havaintojen, todennettavan tutkimuksen ja konsensuksen kautta. Olen pysynyt kaukana spekulaatioista ja mielipiteistä niiden itsensä vuoksi. Sanottuani tämän, kerron paljon tarinoita, joista toiset ovat mielikuvituksellisempia kuin toiset. Tämä on välttämätöntä. Kokemukseni on kertonut minulle, että meillä on syviä, tärkeitä ongelmia ratkottavana. Halusin ratkaista nämä ongelmat itseni, ystävieni ja perheeni vuoksi.

Karl Popper kirjoitti kirjassa \emph{All Life is Problem Solving}:
\begin{quotation}
\noindent Tiede alkaa ongelmista. Se yrittää ratkaista niitä rohkeilla ja kekseliäillä teorioilla. Suuri enemmistö näistä on virheellisiä tai niitä ei ole mahdollista testata. Arvokkaat, testattavissa olevat teoriat etsivät virheitä. Yritämme löytää virheet ja eliminoida ne. Tämä on tiedettä: se koostuu villeistä, usein vastuuttomista ideoista, jotka se asettaa virheenkorjauksen ankaran kontrollin alle.
\end{quotation}
Pyydän, että etsit virheitä villeistä, vastuuttomista ideoistani, ja työskentelet kanssani niiden korvaamiseksi tai korjaamiseksi. Olen välttänyt jargonia ja vihjailua. Selvästi esitettyä ideaa on helpompi kritisoida. Emme voi koskaan saavuttaa totuutta, vain löytää parempia approksimaatioita sille. Joskus se ottaa suuria harppauksia ja valistuneita arvauksia. Joskus meidän täytyy olla halukkaita ajattelemaan epäortodoksisiin suuntiin. Teen monia hypoteeseja, ja esitän ne kuin ne olisivat faktoja. Pyydän anteeksi etukäteen tästä tyylistä. Pyydän myös anteeksi spekulointia, jossa olen väärässä. Toivon, että ne osat, mitkä ovat osuneet oikeaan, hyvittävät ne.

\section{Menneisyyden syleileminen}

Jos ex-puolisosi on saattanut olla psykopaatti, tämä kirja tulee palauttamaan muistoja. Se saa sinut kokemaan vahvoja tunteita. Saatat haluta välttää sen lukemista, välttääksesi kokemustesi uudelleenelämisen. Tämä on yleinen ja ymmärrettävä reaktio. Hallitseva mielipide traumasta on, että kokemustemme uudelleenajatteleminen pysäyttää paranemisprosessin.

Mutta vältteleminen tarkoittaa avuttomuutta. Ja avuttomuus johtaa masennukseen. Olen puhunut satoja tunteja muiden psykopaattisista suhteista selvinneiden kanssa. Olen kuunnellut heidän tarinoitansa loputtomasta emotionaalisesta kiusanteosta, huiputuksesta, manipuloinnista ja varkauksista. Olen jakanut omia tarinoitani. Ja olen kertonut tarinoita, jotka kerron tässä kirjassa. Mitä psykopaatit ovat. Kuinka he ajattelevat. Mistä he saavat voimansa. Miltä he näyttävät. Ja kaikista eniten siitä, kuinka taistella vastaan.

Kokemuksemme ovat niin samanlaisia. Aivan kuin Maan jokainen psykopaatti lukisi samaa käsikirjaa. Kun puhun psykopaateista uudessa ryhmässä, ainakin neljäsosalla porukasta syttyy lamppu. ``Puhut exästäni\vmq{,}'' he sanovat. Selostan, kuinka opin pärjäämään sellaisten ihmisten kanssa, menneisyydessä ja tulevaisuudessa. ``Sinun pitäisi kirjoittaa kirja\vmq{,}'' he sanovat. ``Teen sitä\vmq{,}'' vastaan.

Neuvoni on, että syleilet menneisyyttäsi. Älä vältä sitä. Kohtaa se ja ymmärrä se. Sitten käytä uutta tietoasi tullaksesi vahvemmaksi, onnellisemmaksi ihmiseksi. Tämä on se syy, miksi kirjoitin tämän kirjan. Halusin selostaa psykopatian positiivisella tavalla. Ei sillä, ettq psykopaatit olisivat kivoja ihmisiä. He ovat kivoja kuin käden läpi lyöty naula. Mutta jos pääset kirjan loppuun saakka, lupaan psykopatiasta kuvan, joka muuttaa kaiken.
