\documentclass[11pt,oneside]{memoir}

\usepackage[finnish]{babel}

\usepackage[utf8]{inputenc}
\usepackage[T1]{fontenc}
\usepackage{microtype}
\usepackage{gensymb}
\usepackage{textcomp}
\usepackage{color}
\usepackage{verbatim}
\usepackage{tocloft}
\usepackage{amsmath}
\usepackage{textcomp}
\usepackage{slashed}
\usepackage{amssymb}
\usepackage{braket}
\usepackage[hyphens]{url}
\usepackage{tikz}
\usepackage{pgfplots}
\usepackage{natbib}
\usepackage[colorlinks=true,urlcolor=blue,linkcolor=blue,linktocpage=true]{hyperref}

\setstocksize{8.4in}{6in}
\settrimmedsize{8.4in}{6in}{*}
\setlrmarginsandblock{0.8in}{*}{1}
\setulmarginsandblock{0.8in}{*}{1}
\setlength{\headsep}{0.215in}
\setlength{\footskip}{2.5em}
\fixpdflayout
\checkandfixthelayout

\setsecnumdepth{subsection}\maxtocdepth{subsection}

\makepagestyle{thphp}
\makeevenhead{thphp}{\thepage}{\rightmark}{\thepage}
\makeoddhead{thphp}{\thepage}{\rightmark}{\thepage}
\makeoddfoot{thphp}{}{{}}{}
\makeevenfoot{thphp}{}{{}}{}
\pagestyle{thphp}

\renewcommand{\cftdot}{}
\setlength\cftparskip{1pt}

\captiontitlefont{\small}
\captionnamefont{\bfseries}

\newenvironment{tarina}{\itshape}{}
\newcommand{\problem}[1]{\(\Psi\)\em{#1}}
\renewcommand{\problem}[1]{}
\newcommand{\vtila}{\vspace{0.2in}}
\newcommand{\linkki}[1]{\footnote{\url{#1}}}
\newcommand{\vmq}[1]{\rlap{#1}\,}


\begin{document}

\frontmatter

\thispagestyle{empty}

\begin{center}

\resizebox{4.4in}{!}{\LARGE\textbf{Psykopaattikoodi}}

\noindent\rule{\textwidth}{0.8pt}

\vspace{0.12in}

\resizebox{4.4in}{!}{\large\textbf{Meitä vaanivien petojen murtaminen}}

\vtila

\textsc{Pieter Hintjens}

\vtila

\noindent Suomentanut Konsta Kurki

\vtila

\noindent {\today} (KESKENERÄINEN)

\end{center}

\vfill

\noindent Älä suotta etsi sarjamurhaajia: joka kahdeskymmenes viides ihminen ympärilläsi on psykopaatti, joka piilottelee ja elää salaista elämäänsä. Psykopaatit ottavat mitä haluavat hurmausvoimansa ja oveluutensa avulla. He tuntevat ainoastaan pedon emootiot. Tässä kirjassa Pieter Hintjens dekoodaa psykopatian mysteerin. Miksi sellaisia ihmisiä on olemassa? Kuinka he toimivat? Ja kaikista kriittisimpänä kysymyksenä: voimmeko oppia välttämään heitä, tai pakenemaan heidän kynsistään? Vastaukset tulevat valaisemaan sinua. Tämä kirja tarjoaa käytännöllisiä työkaluja ja tekniikoita kaikista vaikeimmista ihmisistä selviämiseen.

\vfill

\noindent Copyright {\textcopyright} (2015--2018) Pieter Hintjens, Konsta Kurki

\vtila

\noindent Tekstiä saa käyttää CC BY-SA 3.0 -lisenssin ehtojen mukaisesti.\linkki{https://creativecommons.org/licenses/by-sa/3.0/}

%\vtila

%\noindent Suomennos on saatavilla myös vapaana äänikirjana.\linkki{http://konstakurki.org/psykopaattikoodi/}

\newpage

\thispagestyle{empty}

\noindent Pieter Hintjens (1962--2016) oli kirjoittaja, ohjelmoija ja ajattelija. Hän rakensi vuosien ajan suuria ohjelmistojärjestelmiä ja Internet-yhteisöjä, joita hän kutsui ``eläviksi järjestelmiksi\vmq{.}'' Hän oli hajautetun laskennan asiantuntija, ja hän kirjoitti yli kolmekymmentä protokollaa ja hajautettua ohjelmistojärjestelmää. Hän perusti vapaan koodin ZeroMQ-ohjelmistoprojektin vuonna 2007.

\vtila

\noindent Saman kirjoittajan muita kirjoja: \emph{ZeroMQ---Messaging for Many Applications} (O'Reilly), \emph{Culture and Empire: Digital Revolution} (Amazon.com).

\vtila

\noindent \url{http://hintjens.com/}

\vspace{1in}

\noindent Konsta Kurki on ajattelija, joka aloitti fysiikasta ja päätyi tavalliseen elämään. Hän ei pyri rakentamaan uraa tai mainetta tai saamaan menestystä tai rahaa. Hänen elämäntavoitteensa on auttaa kavereita eli empaattisia ihmisiä, jotka hän tuntee henkilökohtaisessa elämässä.

\vtila

\noindent \url{http://konstakurki.org/}

\newpage

\tableofcontents

%\input{saatteeksi}

\chapter{Esipuhe}\label{preface}

\section{Psykopaatin dekoodaaminen}

Elämässä tulee vastaan pelottavia ihmisiä. Ihmisiä, jotka ottavat mitä haluavat käyttäen oveluuttaan ja hurmausvoimaansa. Huijareita. Ammattimaisia valehtelijoita. He ottavat ystäviltä, kollegoilta, perheeltä ja tuntemattomilta ihmisiltä. He eivät koskaan pyydä anteeksi tai koe tunnontuskia niitä kohtaan, joita he satuttavat. Heillä on usein rikostaustaa. Kutsumme heitä monilla nimillä. Narsisti. Antisosiaalinen. Sosiopaatti. Toimitusjohtaja. Ja yhä useammin kutsumme heitä Psykopaateiksi.

Psykopaatit nostavat esiin monia kysymyksiä. Mikä näitä ihmisiä vaivaa? Ehkä heillä oli kylmät ja etäiset vanhemmat. Ehkä he kärsivät hyväksikäytöstä lapsena? Ehkä jokin heissä on rikki, kemiallinen epätasapaino, tai pahoja henkiä. Tai seuraava aste ihmisevoluutiossa. Uusi superihmisten rotu, ehkä? Voimmeko tunnistaa heitä? Voimmeko oppia bongaamaan heitä kodeissamme ja kaduilla? Mitä heidän mielessään liikkuu? Ovatko he tietoisia aiheuttamistaan vaurioista? Nukkuvatko he yönsä hyvin? Kuinka selviytyä heistä? Kuinka sellaisia \emph{hirvittäviä} ihmisiä voi olla olemassa? Olenko minä yksi heistä?

Onneksi on olemassa hyviä vastauksia, jotka irrottavat kauhean mysteerin psykopatiasta. Tämä oli tavoitteeni tässä kirjassa: dekoodata psykopaattinen mieli ja kirjoittaa manuaali muille. Materiaali perustuu minun ja monien muiden kokemukseen. Sitä on testattu todellisessa elämässä ja se näyttää toimivan. Sanottuani tämän ota huomioon seuraava osio.

\section{Vastuuvapauslauseke}

\emph{Kirjoittaja ei ole psykiatri tai lääketieteen ammattilainen. Kirjoittaja ei jaa lääketieteellisiä neuvoja tai määrää mitään tekniikkaa hoidoksi fyysisiin tai lääketieteellisiin ongelmiin ilman lääkärin neuvoa, ei suorasti eikä epäsuorasti. Kirjoittajan tarkoitusperä on tarjota yleisluontoista informaatiota auttaakseen lukijaa emotionaalisen tai henkisen hyvinvoinnin etsimisessä. Siinä tapauksessa, että lukija käyttää mitä tahansa tietoa tässä kirjassa itsellensä, mikä on lukijan oikeus, ei kirjailija eikä julkaisija ota minkäänlaista vastuuta lukijan teoista. Mikään tässä kirjassa esitetyistä ehdotuksista ei ole tarkoitettu korvaamaan lääkärin hoitoa tai sekaantumaan diagnoosin, määrättyjen lääkkeiden tai terapian kanssa.}

\section{Riko lasi}

Jos luet tätä kirjaa saadaksesi apua vaikeaan elämäntilanteeseen, aloita tästä. Selostan avainopetukset lyhyessä yhteenvedossa.

Sinun täytyy oivaltaa muutama asia. Ensimmäinen on: oletko psykologisen hyväksikäytön uhri? Sellainen on harvoin avointa. Mustelmat tuppaavat olemaan henkisiä, eivät fyysisiä. Hyväksikäyttävä suhde on piilossa valheiden alla. Niiden, joita hyväksikäyttäjä kertoo sinulle, sekä niiden, joita kerrot itsellesi. Tämä tekee hyväksikäytön selkeästä näkemisestä vaikeaa.

Aloitetaan tuntemuksistasi. Oletko usein surullinen, masentunut, jopa itsetuhoinen? Koetko olevasi tyhjä ja arvoton? Oletko taivuttanut elämäsi toisen ihmisen tarpeiden mukaan? Otatko epäonnistumisten syyt niskoillesi, ja yritätkö kerta toisensa jälkeen korjata asioita? Tuntuuko sinusta siltä, että saatat olla hullu? Tuntuuko sinusta siltä, että olet palanut loppuun? Oletko yksinäinen, ja oletko menettänyt vanhoja ystäviä ja suhteita? Juotko liikaa?

Jos nyökkäilet tälle listalle, olet luultavasti hyväksikäyttävässä suhteessa. Se on paljon yleisempää kuin luulet. Arvioisin, että 10\%--20\% ihmisistä on hyväksikäyttävässä suhteessa millä tahansa ajanhetkellä. Sitä voi olla vaikea tunnistaa, myöntää ja käsitellä sekä uhreille että heidän ystävilleen ja vanhemmilleen.

Seuraavaksi tarkastelemme suhdetta, joka stressaa sinua. Alkoiko se ``täydellisesti'' ja muuttui sitten ajan myötä painajaiseksi? Merkkaavatko sitä yhtäkkiset, odottamattomat kriisit? Luonnehtivatko sitä äärimmäiset tunteet? Onko siinä verbaalista tai fyysistä väkivaltaa? Oletko investoinut kaikkesi suhteeseen saaden hyvin vähän takaisin? Onko siitä tullut ainut ihmissuhteesi, joka merkitsee, jättäen varjoonsa ystäväsi ja perheesi? Oletko kykenemätön ajattelemaan vaihtoehtoja?

Nämä ovat hyväksikäyttävän siteen tunnusmerkit. Jos tämä kuvaa tilannettasi, olet hyökkäyksen kohteena. Oleta, että tätä tekevä henkilö on psykopaatti joko ilman muodollista diagnoosia tai sen kanssa. Tulemme yksityiskohtaiseen diagnoosiin myöhemmin. Se, mikä merkitsee tässä vaiheessa, on että tunnistat tilanteesi ja sen, kuinka sinua kohtaan hyökätään. Se voi vaikuttaa sattumanvaraiselta, mutta se on systemaattista. Tavoite on sekoittaa ja eristää sinut, riisua sinut voimavaroistasi, ja sitten tuhota ja hylätä sinut. Väkivalta on vain osa tätä kaikkea.

Jos tämä ei kuvaa tilannettasi, voit hypätä tämän osion loppuosan ohi.

Nyt ota uusi katse toiseen ihmiseen. Jos vastassasi on psykopaatti, hänen todellisen luonteensa näkeminen voi olla mahdotonta. Sinun täytyy katsoa sivuttain, toisissa ihmisissä näkyvien heijastusten kautta. Näetkö jonkun, joka välittää muista, vai jonkun, joka välittää itsestään? Tekeekö hän hiljaisia, huolellisia suunnitelmia, vai onko hän kaoottinen? Säästääkö hän ja investoiko hän, vai onko hänen raha-asiansa sekaisin? Ympäröikö häntä onnelliset ihmiset, vai pilvi stressaantuneita, pakkomielteisiä seuraajia? Onko hänellä eheä työ- ja sosiaalinen historia, vai onko hänen historiansa tyhjä mysteeri?

Kun saat tulokseksi ``Minä olen hyväksikäyttävän psykopaatin uhri\vmq{,}'' olet puolivälissä matkalla kohti ovea. Pakeneminen houkuttaa, ja kun kerrot kokemuksistasi muille, he kehottavat sinua lähtemään. Kansankulttuurissa ``psykopaatti'' tarkoittaaa ``sarjamurhaaja\vmq{.}''

Todellisuudessa ``psykopaatti'' tarkoittaa sitä hidasta elämänvoimasi näivettymistä, aivan kuin vampyyri imisi sinua kuivaksi viikkojen tai jopa vuosien ajan. Fyysistä väkivaltaa saattaa olla, vaikkakin se on pääasiassa merkityksetöntä verrattuna psykologisiin vaurioihin. Tämä tarkoittaa, että jos lähdet, viet samalla vauriot mukanasi.

Tässä on yleinen strategiani: kärsivällisyyttä, havaintoja ja suhteen kääntäminen ympäri hitaasti. Muutut uhrista ja mahdollistajasta?? liikkumattomaksi voimaksi, joka tunnistaa ja estää psykopaatin monet hyökkäykset. Pikkuhiljaa lamautat hyväksikäyttäjän, ja samalla saat voimasi takaisin. Lopulta päätät suhteen omilla ehdoillasi, kokonaisena ihmisenä.

Joskus voit vain sanoa hyväksikäyttäjälle: ``Se on ohi, älä ota minuun yhteyttä\vmq{.}'' Mutta usein suhteen rikkominen vaatii voimaa ja aikaa.

Laki tuppaa jättämään aikuisten välisen psykologisen hyväksikäytön huomiotta. Useimmat psykopaatit ovat huolellisia eivätkä he jätä todisteita. Poliisi ja oikeusistuimet ovat taipuvaisia kyynisyyteen ``Hän sanoi, tuo sanoi'' -tyyppisten syytösten kohdalla. Ja mitä ikinä sanotkaan, psykopaatilla on aina parempi vale. Tällä tavoin kultit kykenevät toiminaan kirkkaassa päivänvalossa.

Joten et voi tehdä verbaalisia syytöksiä. Jos teet niitä, toimivat ne luultavasti sinua itseäsi vastaan. Kun kyse on sanasodasta, psykopaatit ovat vahvoilla. Sen sijaan ole kärsivällinen ja kerää materiaalisia todisteita. On olemassa tapoja, joilla psykopaattia voi provosoida tekemään ja sanomaan itsetuhoisia asioita.

Kun kohtaat psykopaatin, tai edes muutat käytöstäsi hiukan, reaktio on yleensä lisää hyväksikäyttöä. Tulet kauhistumaan ja sinuun sattuu. Haluat, että asiat muuttuvat ``normaaliksi\vmq{,}'' ja osa sinua huutaa: ``Älä provosoi häntä, se tekee asioista vain vaikeampia!''

Tämä on se piste missä moni luovuttaa ja palaa hyväksikäyttäjänsä huomaan. On helpompaa hyväksyä kuin taistella vastaan tuskaisesti. Mutta hyväksikäytön hyväksyminen tarkoittaa hidasta kuolemaa.

Näemme kokemuksesta että useimmat uhat ovat bluffia ja uhoa. Pedot ovat hauraita. Ne eivät kykene selviytymään paljastumisesta. Ne murisevat ja kiusaavat, mutta kun ne kohtaavat todellista vastustusta ja laajempien sanktioiden riskin, ne useimmiten perääntyvät.

Opettele lait jotka koskevat hyväksikäyttöä ja ahdistelua. Ystävysty paikallisen poliisin kanssa. Selvitä, millaisia ilmoituksia voit jättää. Lasketaanko väkivaltainen kielenkäyttö ahdisteluksi? Vai tarvitsetko mustelmia ja lääkärin lausunnon? Onko sinulla laillinen oikeus äänittää puhelinsoittoja ja keskusteluja? Tee tarvittavat tutkimukset. ??

Jos teillä on yhteistä omaisuutta, yritys tai lapsia, ota yhteyttä juristiin. Poliisi antaa sinulle paikallisten uhrien tukiryhmien osoitteet. Jos olet vaikeassa perhetilanteessa, hyväksikäytön uhreihin erikoistunut psykologi auttaa sinua taistelemaan ulos suhteesta. Olitpa mies tai nainen, avun pyytämisessä hyväksikäyttäjää vastaan taistelemisessa ei ole mitään hävettävää.

Ja tässä tulee supervoimasi: toiset ihmiset. Kun puhut toisten kanssa, huomaat, että monilla on samankaltaisia kokemuksia. Kun saat todistusaineistoa hyväksikäyttävästä käytöksestä, voit julkaista sen ja tehdä rikosilmoituksen??. Hyväksikäyttäjäsi voi piiloutua vain, jos muut anteeksi antavat ja unohtavat hänen käytöksensä.

Ennen kaikkea: kärsivällisyyttä ja rauhallisuutta. Sinun täytyy oppia paljon ja muuttaa syviä olettamuksia elämästäsi. Sinä et ole syyllinen. Hyväksikäyttäjät valitsevat uhrinsa eikä toisin päin. Lue kirja hitaasti ja suhtaudu nykyiseen tilanteeseesi tilaisuutena muuttua vahvemmaksi ja viisaammaksi ihmiseksi.

\section{Kuinka kirja toimii}

\emph{Psykopaattikoodissa} on kahdeksan lukua, joista jokainen kertoo osan tarinaa. Voit lukea ne missä järjestyksessä tahansa. Suosittelen, että silmäilet tekstin ensin nopeasti läpi ja luet sen sitten huolellisesti muutaman kerran. Sitten keskustele luottamasi ihmisten kanssa, ja anna uuden tietämyksen upota hitaasti. Tee tutkimuksta, lue psykopaattifoorumeja ja muita kirjoja. Opittavaa on paljon, ja sen kaiken sulattaminen ottaa aikaa, jopa vuosia.

Luvussa \ref{predator} otamme esille kirjan ydinhypoteesin, mikä on, että psykopaatit ovat sosiaalisia petoja jotka vaanivat toisia ihmisiä. Idea ei ole uusi, ja siitä on tulossa valtavirtaa. Olen vain vienyt sen muita pidemmälle.

Luvussa \ref{the-hunt} näemme, kuinka psykopaatit metsästävät. Sukupuoli ja ikä ajavat metsästystä vahvasti. Jokaisessa tapauksessa psykopaatti käyttää häivetekniikoita ja harhautuksia päästäkseen saaliinsa lähelle ja voittaakseen sen luottamuksen. Opettele nämä kaavat ja muutut immuuniksi niille.

Luvussa \ref{attack-and-capture} näemme, kuinka psykopaatit vangitsevat uhrinsa ja rakentavat hyväksikäyttävän siteen. Psykopaatti eristää ja manipuloi kohteensa antamaan mitä tahansa. Taas kerran tietämällä kaavat muutumme immuuniksi niille.

Luvussa \ref{the-feeding} näemme psykopaattisen suhteen raaimman vaiheen. Tässä vaiheessa psykopaatti imee kohteensa kuiviin samalla, kun hän hyväksikäyttää kohteen hiljaisuuteen ja hyväksyntään.

Luvussa \ref{hunting-mallory} alamme kääntää pöytiä seuraamalla ja tunnistamalla psykopaatteja. Näemme yli sata piirrettä ja käytösmallia, jotka voit tunnistaa, mukaan lukien sen, miltä sinusta itsestäsi tuntuu, jos olet psykopaatin syleilyssä.

Luvussa \ref{the-dance-of-emotions} tarkastelemme ihmisen emootioita. Ne ovat avainasemassa, kun pyrimme ymmärtämään psykopatiaa ja tapaamme reagoida siihen. Käymme läpi noin viisikymmentä universaalia ihmistunnetta, joista psykopaatti tuntee yhdeksän.

Luvussa \ref{escape-from-jonestown} näemme, kuinka vapautua psykopaatin syleilystä. Materiaali selittää askel askeleelta kuinka saada voimat takaisin ja lamauttaa hyväksikäyttäjä. Kyse ei ole yhden yön prosessista, joten kärsivällisyys ja rauhallisuus ovat tarpeen.

Luvussa \ref{questions-to-the-author} vastaan usein kysyttyihin kysymyksiin jotka seuraavat materiaalista.

Kirja on saatavilla Amazon.com:sta, Kindlelle, sekä ilmaiseksi osoitteesta psychopathcode.com. Jaa vapaita PDF-tiedostoja ja sähkökirjoja ystävillesi ja perheellesi.

\section{Kuinka kirja syntyi}

Olen nörtti, joka kirjoittaa koodia, artikkeleita ja kirjoja. Tutkintoni sain tietokonetieteestä. Opiskelin psykologiaa vain hieman yliopistossa. Tämä ei ole tavanomainen lähtökohta kirjalle psykopaateista. Joten selostan, kuinka päädyin tähän.

Urani aikana olen työskennellyt tuhansien ihmisten kanssa. Olen rakentanut satoja tiimejä ja monia pieniä yrityksiä, voittoa tavoittelemattomia organisaatioita ja Internetyheisöjä. Minun on täytynyt oppia tuntemaan ihmisluonto. Jotkut opetukset ovat ilmeisiä. Toiset ovat hyvin piilossa. Olemme niin monimutkainen laji. Ja kuitenkin ihmisluontoa on mahdollista ymmärtää, dekoodata ja ennustaa.

Kirjassani \emph{Culture \& Empire} aloin kirjoittaa psykologiasta. Sosiaalinen psykologia kiehtoo minua ja on erityisosaamistani. Tarkoitan sitä, kuinka ryhmät toimivat ja kuinka ihmiset toimivat ryhmissä. Se on \emph{open source} -yhteisönrakennukseni ydinasia. Tietokoneohjelmissa on kyse ihmisistä, käy lopulta ilmi.

Kirjassa \emph{Culture \& Empire} katsahdin myös sitä, kuinka konfliktit toimivat muutosta ajavana voimana.

Unelmoimme rauhasta ja vakaudesta. Mutta läpi historian kaikista suurimmat harppaukset ovat nousseet konfliktista ja kaaoksesta. Rakennamme ihmisoikeuksien maailman. Työskentelemme suojellaksemme ympäristöä. Rakennamme lakijärjestelmiä ja oikeusistuimia ja poliisivoimia suojelemaan vaurautta ja rauhaa. Et välttämättä huomaa sitä jokapäiväisistä uutisista, mutta väkivalta vähenee vuosi vuodelta, ja on aina vähentynyt.\linkki{sdfd}

Ihmiset eivät ole hyviä eivätkä pahoja. Me olemme selviytyjiä. Teemme mitä meidän ikinä tarvitseekaan tehdä lisääntyäksemme ja varmistaaksemme että lapsemme menestyvät. Useimmat onnistuvat tekemällä kovaa työtä. Muutamat elävä loisina, ottaen toisilta, kuin vampyyrit. Kutsun näitä ``hyviksiksi'' ja ``pahiksiksi\vmq{.}''

Kun olin saanut \emph{Culture \& Empiren} valmiiksi, kohtasin yksityiselämässäni pahiksia. Kirjoittajana en juossut karkuun enkä kiistänyt sitä, mitä tapahtui. Sen sijaan aloin tehdä muistiinpanoja ja suorittamaan pieniä kokeita. Ei sitä joka päivä pääse leikkimään psykopaateilla.

Havaitsin, että työni sosiaalisessa psykologiassa oli jäänyt puolitiehen. Olin keskittynyt hyviksiin ja vain raapaissut pahisten maailman pintaa. Mutta nämä strategiat eivät ole vaihtoehtoja. Ne kietoutuvat toisiinsa ja toimivat yhdessä pitkässä, mystisessä kilpavarustelussa. Tämä kilpavarustelu ytimessä siinä, mitä ``olla ihminen'' tarkoittaa.

Internetissä on paljon materiaalia psykopaateista. Aihe trendaa joka vuosi.\linkki{sdfds} Tuhansia tarinoita, joita aihe on koskettanut. On tutkimusta psykologeilta ja psykiatreilta. On papereita kriminologeilta ja nettideittailueksperteiltä. On jopa diagnosoitujen psykopaattien kirjoittamia blogeja ja foorumeja.

Joihinkin valtaviin kysymyksiin ei ole olemassa yleisesti hyväksyttyjä vastauksia. Aloitetaan vaikkapa siitä, että mikä aiheuttaa psykopatiaa? Voimmeko parantaa sitä; haluammeko edes yrittää sitä? Voimmeko tunnistaa psykopaatteja ``villissä luonnossa\vmq{,}'' siis ilman psykiatrista tutkimusta? Ovatko he aina väkivaltaisia ja vaarallisia, ja jos eivät, mitkä tekijät vaikuttavat näihin seikoihin? Onko kyseessä käyttäytymisen spektri, kuten älykkyys? Vai onko psykopatia enemmänkin binäärinen ilmiö, kuten sukupuoli? Kuinka psykopaatit ajattelevat ja operoivat? Voimmeko suojella itseämme niiltä?

Tohtori Robert Hare on psykopatian auktoriteetti, ja käytän termiä samoin kuin hän, vaikkakin laajemmin. Hänen tarkastuslistansa\linkki{sdfs} keskittyy miehiin ja jättää monet naispsykopaatit pimentoon. Klassinen psykopatiatutkimus on keskittynyt tuomittuihin rikollisiin: taparikollisiin ja väkivaltarikollisiin. 1990-luvun puolen välin tienoilla tutkijat alkoivat tarkastella psykopaatteja, jotka piilottelevat tavallisessa kansassa. Nämä ovat ``subkliinisiä'' tai ``menestyviä'' psykopaatteja jotka pysyvät kaukana väkivaltaisesta rikollisuudesta. He joutuvat harvoin pidätetyksi.

Lukema, jonka näemme on suurinpiirtein 10\% kummassakin sukupuolessa. ``Kourallinen empiirisiä tutkimuksia antavat ymmärtää\ldots subkliininen psykopatia on paljon yleisempää kuin sen kliininen pari, perustiheyden liikkuessa viiden ja viidentoista prosentin välillä\vmq{.}'' ---Jay C. Thomas, Daniel L. Segal, puhuessaan Gustafsonin ja Ritzerin vuoden 1995 työstä ja Pethmanin ja Earlandsonin vuoden 2002 työstä kirjassa \emph{Comprehensive Handbook of Personality and Psychopathology: Personality and Everyday Functioning} (Google books).

Selvästikin jotkut psykopaatit ovat haitallisempia kuin toiset. Käytän itse neljää prosenttia leikkauspisteenä. Tämä on jokseenkin mielivaltaista. Liian pieni numero ja suurin osa psykopaateista ja heidän aiheuttamistaan vaurioista jää näkemättä. Liian suuri numero ja trivialisoimme ilmiön. \emph{Suomentajan huomio: tässä kohdassa Hintjens on jollakin tapaa kujalla. Jos psykopatia on kutakuinkin binäärinen ilmiö, niin kuin Hintjens selvästi kaiken muun tekstin perusteella ajattelee, ei minkäänlaista rajanveto-ongelmaa pitäisi olla olemassa.}

Siinä missä moni psykopaatti vaikuttaa täysin ``normaalilta\vmq{,}'' olen tullut siihen käsitykseen että jotkut piilottelevat suoraan silmiemme alla, tiettyjen persoonallisuushäiriöiden takana.Nämä ovat nasistinen, epävakaa ja huomiohakuinen persoonallisuushäiriö. Kun otat nämä mukaan laskuihin, muuttuu kuva selvemmäksi ja yksityiskohtaisemmaksi. Plussana saamme käyttökelpoisia malleja näiden häiriöiden aiheuttamista vaurioiden selviytymiseen.

Teoriassa persoonallisuushäiriöt kuten epävakaa persoonallisuus ovat hoidettavissa lääkkeillä ja terapialla. Käytännössä tämä toimii huonosti tai ei ollenkaan. Samoin pätee diagnosoituihin psykopaatteihin (heihin, joilla on ``antisosialinen persoonallisuushäiriö''). Terapia näyttää tekevän heistä vain taitavampia ihmisten manipuloinnissa.

Se, mikä voi toimia, on rajoittaa ja korjata vaurioita, joita psukopaatit aiheuttavat. Aivan kuten koulukiusaajatkaan, eivät psykopaatit kärsi masennuksesta. Se on perhe, ystävät ja kollegat jotka maksavat viulut. Sitten, kun näet prosessin psykopaatin törmäyskraatterien takana, voit puuttua asioihin.

Puuttuminen ei ole yksinkertaista. Sellaisten ihmisten, jotka ovat viettäneet koko elämänsä hurmaten, manipuloiden ja kiusaten toisia, käsitteleminen on oletusarvoisesti mahdotonta. Jos yrität varoittaa ryhmää, huomaat joutuvasi itse syytetyksi. Jos yrität varoittaa yskittäisiä ihmisiä, huomaat heidän kääntyvän sinua vastaan. Sinun täytyy liikkua hitaasti, huolellisesti ja käyttäen oikeaa tietämystä.

Joten tämä kirja keskittyy psykopaatin prosessiin, siihen, kuinka tunnistaa se, kuinka se toimii, ja kuinka lamauttaa se.

En ole pätevä psykologi tai psykiatri. En voi argumentoida auktoriteetin suojasta. Se, mitä voin tehdä, on kehittää malleja ja testata niitä psykopaateilla joihin minulla on pääsy. Voin testata niitä vasten epätavallisia tilanteita menneisyydestä. Ja voin testata niitä toisten psykopaatteihin kietoutuneiden ihmisten kautta.

Olen käyttänyt vuosia lukien kirjallisuutta. Foorumeja, kirjoja ja artikkeleita. Mitä tahansa, mikä voi näyttää valoa tähän kummallisten kanssaeläjien mysteeriin ja siihen, kuinka he toimivat. Olen puhunut sadoille ihmisille aiheesta. Joukko sisältävää psykologeja, jotka ovat erikoistuneet hyväksikäyttöön, ja kehityshäiriöihin. Se sisältää ihmisiä, jotka ovat selviytyneet hyväksikäyttävistä puolisoista. Ihmisiä, jotka ovat yrittäneet itsemurhaa paetakseen. Ihmisiä, joden vanhemmat käyttivät heitä hyväkseen ja sopivat psykopaatin profiiliin.

Olen käyttänyt näitä malleja rakentaakseni järjen puutarhoja yksityis- ja työelämässäni. Ne, jotka työskentelevät kanssani, tietävät, että Internet-yhteisömme ovat ennenkaikkea \emph{iloisia} paikkoja. Tämä ei ole sattumaa. Se kumpuaa pitkästä, huolellisesta työstä pahisten pitämiseksi loitolla.

Niin pitkälle, kuin mahdollista, olen työskennellyt toistuvien havaintojen, todennettavan tutkimuksen ja konsensuksen kautta. Olen pysynyt kaukana spekulaatioista ja mielipiteistä niiden itsensä vuoksi. Sanottuani tämän, kerron paljon tarinoita, joista toiset ovat mielikuvituksellisempia kuin toiset. Tämä on välttämätöntä. Kokemukseni on kertonut minulle, että meillä on syviä, tärkeitä ongelmia ratkottavana. Halusin ratkaista nämä ongelmat itseni, ystävieni ja perheeni vuoksi.

Karl Popper kirjoitti kirjassa \emph{All Life is Problem Solving}:
\begin{quotation}
\noindent Tiede alkaa ongelmista. Se yrittää ratkaista niitä rohkeilla ja kekseliäillä teorioilla. Suuri enemmistö näistä on virheellisiä tai niitä ei ole mahdollista testata. Arvokkaat, testattavissa olevat teoriat etsivät virheitä. Yritämme löytää virheet ja eliminoida ne. Tämä on tiedettä: se koostuu villeistä, usein vastuuttomista ideoista, jotka se asettaa virheenkorjauksen ankaran kontrollin alle.
\end{quotation}
Pyydän, että etsit virheitä villeistä, vastuuttomista ideoistani, ja työskentelet kanssani niiden korvaamiseksi tai korjaamiseksi. Olen välttänyt jargonia ja vihjailua. Selvästi esitettyä ideaa on helpompi kritisoida. Emme voi koskaan saavuttaa totuutta, vain löytää parempia approksimaatioita sille. Joskus se ottaa suuria harppauksia ja valistuneita arvauksia. Joskus meidän täytyy olla halukkaita ajattelemaan epäortodoksisiin suuntiin. Teen monia hypoteeseja, ja esitän ne kuin ne olisivat faktoja. Pyydän anteeksi etukäteen tästä tyylistä. Pyydän myös anteeksi spekulointia, jossa olen väärässä. Toivon, että ne osat, mitkä ovat osuneet oikeaan, hyvittävät ne.

\section{Menneisyyden syleileminen}

Jos ex-puolisosi on saattanut olla psykopaatti, tämä kirja tulee palauttamaan muistoja. Se saa sinut kokemaan vahvoja tunteita. Saatat haluta välttää sen lukemista, välttääksesi kokemustesi uudelleenelämisen. Tämä on yleinen ja ymmärrettävä reaktio. Hallitseva mielipide traumasta on, että kokemustemme uudelleenajatteleminen pysäyttää paranemisprosessin.

Mutta vältteleminen tarkoittaa avuttomuutta. Ja avuttomuus johtaa masennukseen. Olen puhunut satoja tunteja muiden psykopaattisista suhteista selvinneiden kanssa. Olen kuunnellut heidän tarinoitansa loputtomasta emotionaalisesta kiusanteosta, huiputuksesta, manipuloinnista ja varkauksista. Olen jakanut omia tarinoitani. Ja olen kertonut tarinoita, jotka kerron tässä kirjassa. Mitä psykopaatit ovat. Kuinka he ajattelevat. Mistä he saavat voimansa. Miltä he näyttävät. Ja kaikista eniten siitä, kuinka taistella vastaan.

Kokemuksemme ovat niin samanlaisia. Aivan kuin Maan jokainen psykopaatti lukisi samaa käsikirjaa. Kun puhun psykopaateista uudessa ryhmässä, ainakin neljäsosalla porukasta syttyy lamppu. ``Puhut exästäni\vmq{,}'' he sanovat. Selostan, kuinka opin pärjäämään sellaisten ihmisten kanssa, menneisyydessä ja tulevaisuudessa. ``Sinun pitäisi kirjoittaa kirja\vmq{,}'' he sanovat. ``Teen sitä\vmq{,}'' vastaan.

Neuvoni on, että syleilet menneisyyttäsi. Älä vältä sitä. Kohtaa se ja ymmärrä se. Sitten käytä uutta tietoasi tullaksesi vahvemmaksi, onnellisemmaksi ihmiseksi. Tämä on se syy, miksi kirjoitin tämän kirjan. Halusin selostaa psykopatian positiivisella tavalla. Ei sillä, ettq psykopaatit olisivat kivoja ihmisiä. He ovat kivoja kuin käden läpi lyöty naula. Mutta jos pääset kirjan loppuun saakka, lupaan psykopatiasta kuvan, joka muuttaa kaiken.


\mainmatter

\chapter{Peto}

\begin{quotation}
\noindent Ihmispedot kansoittavat yhteiskuntaamme.\newline ---Stefan Verstappen
\end{quotation}

\section{Ryöstetty pilotti}

\begin{tarina}

Keith puhuu miehelle huoneen nurkassa. Keith on vanha tuttuni, joten menen moikkaamaan häntä. Hän on levoton ja stressaantunut; ei ollenkaan oma itsensä. Hän on aina ollut rauhallinen mies, itsevarma ja hiljainen. Hän omistaa pienen Cessna-lentokoneen. Hän tekee pieniä kaupallisia keikkoja, kuten lentää turisteja Grand Canyonin yli. Kun hänelle jää käteen hiukan ekstraa, hän laittaa sen talteen. ``Vielä joku päivä ostan meille ikioman maatilan\vmq{,}'' hän sanoo minulle. Keith ja Alexis ja heidän eläkepäiviensä unelma.

``Hei, Keith, miten menee?'' kysyn häneltä. Toinen mies istuu vieressä sanomatta mitään. Tummat kiharat, tummennetut lasit. Hieno puku, raskas kultakello. Keith ravistaa päätään ja olkapäitään kiukkuisena. ``Hyvin! Menee ihan hyvin!'' hän sanoo minulle. ``Voisitko nyt mitenkään jättää meidät kahden? Jooko?''

Olen shokissa. En ole koskaan nähnyt häntä tällaisena. ``Totta kai, nähdään myöhemmin\vmq{,}'' sanon ja palaan takaisin pöytääni baarissa. Katselen heitä; he riitelevät. Mies kohauttaa olkiaan ja puhuu. Hän on hiljainen ja kova. Keith rauhoittuu ja lopettaa päänsä pudistelemisen. Nyt hän jo nyökkäilee. Pienoinen draama päättyy heidän kädenpuristukseen. Keith allekirjoittaa jonkinlaisen paperin, jonka mies taittelee, pujottaa takkinsa taskuun. Hän seisoo ja sitten lähtee. Odotan hetken, otan juomani ja istun Keithin viereen.

``Sopiiko istahtaa?'' kysyn häneltä jo istuttuani alas. Meidän oma running joke. Hän katsoo minuun nauramatta ja huokaisee. ``Mitä tuo oikein oli?'' kysyn. Olen edelleen ärsyyntynyt siitä, että hän komensi minut pois aikaisemmin. ``Ei mitään\vmq{,}'' hän sanoo. ``Bisnestä\vmq{,}'' hän tarkentaa ja vaihtaa puheenaihetta. Hän kyselee lapsistani ja juttelemme hetken. Hän on etäinen, laihempi, eikä hän ole ajanut partaansa. Minun tekee mieli kysellä lisää pukumiehestä. Tai ei sittenkään. Ei kannata heittää vettä myllyyn, eihän?

Tämä on viimeinen kerta kun näen Keithin. Kaksi viikkoa myöhemmin saan hautajaiskutsun Alexisilta, hänen leskeltänsä. Soitan hänelle välittömästi. Keith on kuollut. Hänen koneensa syöksyi alas. Ei matkustajia, vain hän. Olen sanaton. ``Olen niin pahoillani\vmq{.}'' En osaa sanoa mitään muuta. Keith?

Tutkija ei löydä koneesta teknistä vikaa. Taivas oli kirkas. Niinpä hän päättelee, että kyseessä oli itsemurha. Keith syöksyi maahan tarkoituksellisesti. Ei henkivakuutusta. Pahinta on, että Keith tyhjensi heidän yhteisen säästötilin aivan pari päivää ennen kuin olin nähnyt hänet. Yli \$180,000 katosi, ilman mitään selitystä.

\end{tarina}

\section{Psykopatia adaptaationa}

Jotkut tutkijat ovat esittäneet, että psykopatia olisi häiriön sijaan adaptaatio.\linkki{sfsd} Sain saman idean vuosia sitten ollessani tekemisissä nuoren hankalan naisen kanssa. Hän kävi terapiassa epävakaan persoonallisuushäiriön takia. Hän liikkui työpaikasta työpaikkaan, syyttäen aina muita. Hänen elämänsä oli pitkä tarina hyväksikäyttäviä vanhempia, ex-puolisoja ja ystäviä.

Hän eli kaaoksen ja emotionaalisen tuskan pilvessä. Kuitenkin ne olivat aina toiset ihmiset, joissa todellinen kärsimys esiintyi. Olipa tilanne mikä tahansa, hän onnistui löytämään tiensä ulos ja vetämään puoleensa uusia ystäviä. Jälkeensä hän jätti vaurioita ja traumoja. Hänellä oli lääkkeensä, joita hän harvoin otti. Hänellä oli terapeutti, jonka hän myöhemmin vietteli. ``Kärsijä'' sai aina sen, mitä hän halusi.

Ennen kaikkea hän oli \emph{petomainen} varmalla ja viattomalla tavalla. Minulla kesti kauan tajuta, että olin maannut psykopaatin kanssa. Hän piti yllään ``epävakaan persoonallisuuden'' maskia ollakseen menestyksekkäämpi uhri. Niihin aikoihin otin muistiinpanoja. Ne dokumentoivat matkani alas toiseen universumiin. Sosiaalisen kanssakäymisen ydinlait loistavat poissaoloaan suhteessa psykopaatin kanssa. Niiden tilalla kasvaa jotakin vierasta ja nälkäistä.

Moni ihminen tutkii psykopatiaa häiriönä. Halusin tutkia psykopatiaa adaptaationa, koska se sopii dataani paremmin. Lisäksi se näytti johtavan positiivisempiin ja hyödyllisempiin päätelmiin.

Psykopatian mallintaminen adaptaationa häiriön sijaan avaa oven uuteen maailmaan. Kysymyksemme vaihtuvat. Kysymme, mikä kamppailu on ajanut tätä kehitystä eteenpäin. Kysymme, millaisia adaptaatioita psykopaateilla tarkalleen ottaen on. Onko heillä pitemmät hampaat? Terävämmät kynnet? Vai onko heidän kykynsä hienovaraisempia?

Kysymme myös, millaisia vasta-adaptaatiota voi olla olemassa sosiaalisissa ihmisissä (ei-psykopaattisessa valtaosassa). Kysymme: ``Voisiko psykopattinen-sosiaalinen -suhde olla itseasiassa peto-saalis -tarina?'' Ja vastauksen havaitsemme olevan painokas ``Kyllä!'' Kysymme: ``Kuinka vanha tämä tarina on?'' ja vastaus on: ``Miljoonia vuosia\vmq{.}''

\section{Petomalli}

\emph{Draculan} kirjoittaja Bram Stoker maalasi psykopaatin pedoksi. Ehkä hänellä oli henkilökohtaista kokemusta. Tarinaa ei ole tarkoitettu kirjaimelliseksi totuudeksi. Se on metafora, ja hyvä sellainen. Dracula tulee paikalle yöllä, pukeutuneena tappamaan. Hän imee sinut kuiviin elämän verestä, vietellen sinut siinä samassa viehätysvoimallansa ja seksuaalisuudellansa.

Dracula ei tapa kokonaan. Sen sijaan hän muuttaa sinut heikoksi kopioksi itsestään. Hän on mahtava ja eläimellinen. Hän kykenee lukemaan ajatuksesi, silloinkin kun kompuroit yrittäessäsi paeta. Paras jännite on aina vampyyrin ja ihmisen välisessä taistossa, vampyyrien välisen konfliktin ollessa vain kirsikka kakun päällä.o

Tohtori Robert Hare alkoi kuvata psykopaatteja sosiaalisina petoina hänen vuoden 1994 \emph{Psycholology Today}-lehden virstanpylväsartikkelissaan.\linkki{sdfds} Aliotsikko kuuluu: \emph{This Charming Psychopath---How to spot social predators before they attack} \emph{(Tämä hurmaava psykopaatti---kuinka huomata sosiaaliset pedot ennen kuin ne hyökkäävät)}. Artikkelissa ja työssään hän keskittyy psykopaattien tunnistamiseen. Moni meistä tuntee hänen ``Psychopathy Checklist''-diagnoosityökalun.

Petojen ydinstrategia on saaliin pettäminen. Ihmispetojen ydinstrategia on uhrien huijaaminen. Kyse on samasta asiasta. Eläinmallit ovat oleellisia ihmiskäytöksen ymmärtämisessä ja ennustamisessa. Meidän on vaikea katsoa itseämme valehtelematta. Itseanalyysimme pysähtyy vapaan tahdon ja tietoisuuden konsepteihin. Emme kykene parantamaan niitä tai heittämään niitä romukoppaan, joten ne riivaavat meitä. Eläimiä taas osaamme analysoida ilman merkkiäkään vastaavista ongelmista.

Olen tässä kirjassa käyttänyt petomallia selkärankana, johon kaikki muut asiat ripustuvat. Aloitamme pedoista, jotka huijaavat tiensä läpi ihmisten sosiaalisen universumin. Kaikki muut asiat lähtevät tästä, ja ne käyvät järkeen siinä kontekstissa.

Ensimmäinen altistumiseni petomallille oli Stefan Verstappenin erinomainen työ \emph{Defence Against the Psychopath}.\linkki{sdfds} Se oli ensimmäinen lukemani teksti, joka esitti strategioita psykopaattien kohtaamiseksi ja päihittämiseksi. Olen valinnut saman polun tässä kirjassa.

\section{Jotakin muuta kuin tuttuja muurahaisia}

\begin{quotation}
\noindent Australiassa on hämähäkkejä, jotka tuoksuvat ja käyttäytyvät kuin muurahaiset: jotkut niistä ovat niin vakuuttavia, että muurahaiset antavat hämähäkin elää pysyvästi yhtenä heistä. Hämähäkki herkuttelee uusilla ystävillänsä, mutta se ei syö kaikkia muurahaisia, tai edes merkittävää osaa niistä; sen sijaan se louhii resurseja hitaasti, kestävästi, ajan kanssa.\newline---Daniel N. Jones, \emph{Snake in the grass}\linkki{sdfds}
\end{quotation}
Kyse ei ole vain jokusesta hämähäkistä. Tuhannet eri hyönteiset ovat hakkeroineet sisään muurahaisyhdyskuntaan tavalla tai toisella. Eräs toukka matkii muurahaiskuningattaren ääntä höynäyttääkseen työläisiä. \emph{Paussus}-kuoriainen syntyy, elää ja kuolee yhdyskunnan sisällä. Se ei pelkästään tuoksu oikealta (muurahaiset käyttävät hajuja ystävien erottamiseksi vihollisista), se myös matkii ääniä, joita muurahaiset päästelevät. Se kirjaimellisesti imitoi kuningatarmuurahaista sanoen ``Kaikki on hyvin'' muurahaisille ja toukille, silloinkin kun se hotkaisee niitä suihinsa.

Muurahaiset kehittyivät työskentelemään yhdessä kerääkseen ruokaa ja suojellakseen sitä varkailta. Ne jakavat työn, pitävät huolta jälkikasvusta yhdessä ja elävät suurissa yhdyskunnissa. Ne kommunkioivat ja ne ajattelevat kollektiivisesti. Muurahaisyhdyskunnan käytöksessä näkyy älykkyys.

Muurahaiset kukoistavat huolimatta niitä, niiden suojelumekanismeja ja niiden ruokaa jahtaavista loisista ja pedoista. Muurahaiset on yksi kaikista menestyneimmistä lajeista. Muurahaisilla on kieliä, heimoidentiteetti, sosiaalinen organisaatio, kyky työskennellä yhdessä. Nämä ovat adaptaatioita. On syytä kysyä, että \emph{minkä ongelmien ratkaisemiseksi ne ovat kehittyneet?} Vastaus näyttää olevan, että täsmälleen tuota huijareiden lauman kanssa pärjäämiseksi.

Muurahaiset aloittivat jakamalla kausittaisen ravinnonlähteen riskiä. Useampi muurahainen kykenee korjaamaan satoa laajemmalta alueelta kuin yksittäinen muurahainen. Onnekas muurahainen voi jakaa löytönsä onnettomien kanssa. Onneton muurahainen selviytyy bad spellistä. Ja niin muurahaisille kehittyi altruismi, mikä on hyvä ratkaisu, jos ravinnonlähde on riskialtis. Muita ratkaisuja ovat muuttaminen, horrostaminen ja synkronoidut parittelukaudet.

Mutta altruismilla on heikkoutensa. Se on huijaava käytös. Muurahaiste ravinnonlähde on auki jokaiselle joka sitä tarvitsee. Jos pääset sisään yhdyskuntaan, voit syödä muurahaisia, toukkia ja ruokaa tekemättä työtä. Joten altruistien täytyi joko kehittää puolustusta huijareiden varalle tai kuolla sukupuuttoon. Muurahaisyhdyskunnalle tämä tarkoittaa tunkeilijoiden tunnistamista ja tappamista. Altruismin geenit voivat selviytyä vain, mikäli ne samalla valvovat vastavuoroisuutta.

Kuten Daniel Jones kirjoittaa:\linkki{sdfds}
\begin{quotation}
\noindent Jotkut pedot ovat nopeita, liikkuvaisia ja wide-ranging, huijataksensa niin montaa uhria kuin mahdollista; ne muistuttavat ihmispsykopaatteja. Toiset ovat hitaita, vaanien petoansa erityisellä, strategisella (melkein Machiavellianisella) tavalla{\ldots} Useimpien elävien asioiden keskuudessa huijarin ja huijatun välillä on kilpavarustelutilanne, joka ei lopu koskaan.
\end{quotation}
Ja niinpä muurahaisille kehittyi hajuin, kosketksin ja äänin toimivia kieliä, jotta he kykenevät tunnistamaan toisensa. Huijareille kehittyi kyky imitoida näitä kieliä. Muurahaisten kielet muuttuivat hienostuneemmiksi. Huijarit kehittyivät paremmiksi. Ja niin edelleen, satojen miljoonien vuosien ajan, antaakseen meille tuon modernin muurahaisen. Yksi muurahaisperhe Argentiinasta peittää suuren osan maailmaa ``mannertenvälisenä superyhdyskuntana\vmq{.}'' Tämä superyhdyskunta on päällekäyvä, hallitseva, ja se ajaa paikalliset muurahaislajit pois. Portugalilainen muurahainen voi astua sisään muurahaispesään Uudessa-Seelannissa ja tulla hyväksytyksi.

Tämän pitäisi olla sinulle tuttua. Yhteistyöhenkistä altruismia esiintyy muissakin lajeissa. Termiitit, mehiläiset ja ampiaislajit kehittyivät pitkin samaa polkua. Samoin kehittyivät vampyyrilepakot, miekkavalaat ja ihmiset. Myös me muodostamme mannertenvälisen superyhdyskunnan, joka on päällekäyvä, hallitseva, ja joka usein käyttäytyy kuin yksi perhe.

Vuoden 2012 kirjassaan \emph{The Social Conquest of the Earth} Edward Wilson kuvasi ihmisiä aitososiaalisina ihmisapinoina. Työnjakomme, päällekäiset sukupolvet ja yhdessä tapahtuva jälkeläisistä huolehtiminen antavat meille ``supervoiman'' johon harva muu laji kykenee.

Ihmisyys ei kehittynyt Eedenin puutarhassa. Useat ilmastonvaihdokset moukaroivat meitä, yhä uudelleen ja uudelleen. Selvisimme monesta sukupuuttoa liippaavasta pullonkaulasta, vain muutaman tuhannen yksilön populaatioista, kerta toisensa jälkeen. Nämä tapahtumat eivät tappaneet meitä. Kuten Argentiinalaiset muurahaiset, mekin polveudumme yhdestä pienestä joukosta geneettisesti samankaltaisia ihmisiä. Tämä antaa meille kyvyn tunnistaa toisemme saman heimon jäseniksi.

Selviydyimme katastrofista katastrofin jälkeen työskentelemällä yhdessä. Kehitimme kyvyn siirtää tietämystä sukupolvelta toiselle. Kehitimme altruismin, kyvyn jakaa riskin heimojen ja sukupolvien kautta.

Aikaiset altruistit kärsivät monista huijareista: haaskansyöjistä, loisista ja ennenkaikkea toisista ihmisistä. Jokaista kehittämäämme sosiaalista vaistoamme kohtaan kehitimme kyvyn huijata toisia. Ja kun huijaukset kehittyivät paremmiksi, sosiaaliset ihmiset kehittyivät paremmiksi niiden tunnistamisessa ja rankaisemisessa.

Ihmiset muodostavat suhteiden verkkoja. Joskus ne ovat hierarkisia. Yleensä luomme siteitä yksilöihin ja ryhmiin. Nuo suhteet eivät ole mielivaltaisia. Ne rakentuvat huolellisen kirjanpidon varaan. Laskemme kauppoja geeneisssä, ruuassa, suojassa, seksissä, kiintymyksessä, informaatiossa, ajassa. Tämä kaikki on yleensä alitajuntaista. Se on myös jatkuvaa ja hallitsevaa.

Meillä on hienostuneet mentaaliset työkalut näiden suhteiden seuraamista varten. Kykenemme muistamaan kasvot koko elämän ajan. Muistamme hyvän ja pahan yksityiskohtaisesti. Voimme arvata minkä tahansa palveluksen tai tavaran suhteellisen arvon annetussa paikassa tai ajassa. Se paahdettu kana jonka jaoit kanssani lounaaksi tarkoittaa kolmea olutta huomenna, tai yhtä kahden viikon aikana. Muistamme huijaukset ikuisuuden, emmekä anna niille anteeksi.

Meillä on mielikuvituksemme, joten voimmme suunnitella, kuinka työskennellä yhdessä. Meillä on kieli, jotta voimme vaihtaa tietämystä. Ilmaisemme tunteitamme kasvoillamme, äänillämme, kehonkielellämme, ja verisellä punastumisela kasvoissamme, korvissamme ja kehoissamme.

Kaikki nämä ovat adaptaatioita huijauksilta puolustautumiseksi. Aivan kuten muurahaisyhdyskunta on kilpavarustelun tulos, niin on ihmisten yhteiskuntakin. Se, mitä olemme, polveutuu loputtomasta sodasta yhtesityön ja lupauksen ``shekki on postissa!'' välillä.

\section{Ikuinen sota}
\begin{quotation}
\noindent Sinä ja minä: me olemme olleet sodassa ajasta ennen kuin kumpikaan meistä oli edes olemassa.\newline---John Connor elokuvassa \emph{Terminator Salvation}
\end{quotation}
Nyt kun puhumme evoluutiosta ja pitkästä kilpavarustelusta, eräs kysymys poksahtaa ilmoilla. Milloin ihmispsykopatia alkoi kehittyä? Mitä aikaväliä meidän tulisi tarkastella? Googlen mukaan kukaan ei ole koskaan kysynyt tätä kysymysta. Yritän vastata siihen.

Ensinnäkin voimme sulkea pois äskettäisen kehittymisen. Psykopatia on konsistentti piirre ympäri maailman. Se on ihmisluonnon universaali ominaisuus. Niinpä se edeltää meidän leviämistämme pois Afrikasta 150 000 vuotta sitten.

Ihmisyyden alkulähteet painuvat kauemmas ja kauemmas taakse ajassa. Homo naledin rituaaliset hautaamiset Etelä-Afrikassa ajoittuvat noin kolmen miljoonan vuoden taakse. Vanhimmat kivityökalut ovat 3.3 miljoonaa vuotta vanhoja.\linkki{sdfs}

Minäpä selitän. Kiven muuttaminen käytännölliseksi työkaluksi vaatii kasautuvaa taitoa ja oppimista, tekniikoiden hidasta evoluutiota seuraavaa adaptoitumista. Tämä tarkoittaa tietämyksen siirtymistä sukupolvelta toiselle, mikä taas tarkoittaa erikoistuneita yksilöitä, tykalujen valmistajien kastia.

Kuten Scientific American kirjoittaa:\linkki{sdfds}
\begin{quotation}
\noindent Lomekwi knapper?s kykenivät toimittamaan riittoisaa tarkoituksellista voimaa irrottaakseen tostuvasti sarjan rinnakkaisia ja kerrostuneita hiutaleita ja jatkamaan knappaamista? pyörittämällä ytimiä. [He] tarkoituksellisesti valitsivat suuria, painavia palikoita hyvin kovaa raakamateriaalia läheisistä lähteistä vaikka pienempiä palikoita olisi ollut saatavilla. He käyttivät monia knappaustekniikoita irroittaakseen teräväreunaisit hiutaleet ytimistä.
\end{quotation}
Raakamateriaalieja ei ole joka paikassa. Työkaluntekijöiden täytyi matkustaa paikkoihin, joista oikeanlaisia kiviä löytyi. Heidän täytyi valmistaa työkalut. Heidän täytyi kuljettaa työkalut takaisin niille, jotka niitä tarvitsivat. Tämä tarkoitti ruoan ja veden, säkkien ja köysin ja niin edelleen kuljettamista??.

Se tarkoitti myös kykyä suunnitella asioita etukäteen ja organisoitua toisten kanssa. Tämä tarkoitti kieltä, tarpeeksi rikasta ilmaistakseen futuureja ja epämääräisiä lupauksia. Tämä kuulostaa kehittyneeltä pieniaivoiselle homidille, kunnes sitä tajuaa, että muurahaiset tekevät pitkälti samaa. Tällaisen käytöksen ei tarvitse olla tietoista. Se voi olla vaistonvaraista.

Kivityökalujen tuotantoketjussa on ytimiä, ketjuja ja alasimia. Se menee paljon yksittäisen ihmisen mentaalista kapasiteettia pidemmälle. Se kertoo, että maailmassa oli sosiaalinen rakenne. Jotkut erikoistuivat työkalujen valmistamiseen. Toiset erikoistuivat työkalujen käyttämiseen metsästyksessä, lihan puhdistamisessa, luiden hajottamisessa, puun leikkaamisessa. Tällainen sosiaalinen rakenne tarkoittaa altruismia eli kykyä jakaa muiden kanssa. Ja aina siellä, missä on altruismia, on huijaamista.

Vastakkaisen näkemyksen mukaan aikaiset ihmiset olivat generalisteja, ja he tekivät itelleen työkaluja sen mukaan, kun he tarvitsivat niitä. Tämän näkemyksen mukaan erikoistumien ja vaihtokauppa tulevat kuvaan vasta paljon myöhemmin. Mutta tämä malli on helppo osoittaa vääräksi. Metsästäminen vaatii omanlaisensa erikoistuneet taidot. Miesten välillä olisi äärimmäistä kilpailua siitä, kuka on paras metsästäjä, tai kuka on paras työkalunvalmistaja. Naiset tarvitsevat työkaluja siinä missä miehetkin, mutta he tuskin ovat yleensä työkalunvalmistajia. Kaksi erityisosaajaa jotka kykenevät vaihtokauppaan pärjäävät \emph{aina} paremmin kuin kaksi generalistia. Joten generalistimalli ei selivä seksuaalisesta valinnasta, ei ekonomiasta eikä misten ja naisten välisestä työnjaosta.

Joten olen sitä mieltä, että voimme ajoittaa ihmispykopatian ainakin kolmen miljoonan vuoden taakse ajassa.

\section{Isojen aivojen ongelma}

Eräs tunnusomaisimmista inhimillisistä piirteistä on ylisuuret aivomme. Fossiilit näyttävät aivojemme kasvaneen yhtäkkiä suuremmiksi ja suuremmiksi, alkaen noin kaksi miljoonaa vuotta sitten. Mikä ajoi tätä laajenemista eteenpäin? Vastauksen huomataan olevan toiset ihmiset. Kuten Davide Geary kirjoittaa:\linkki{sdfds}
\begin{quotation}
\noindent Aivojen koossa tapahtui hyvin vähän muutoksia fossiilikallonäytteissämme ennenkuin osuimme tiettyyn populaatioon. Kun tämä populaatiotiheys oli saavutettu, aivojen koossa tapahtui nopea kasvupyrähdys.
\end{quotation}
Miksi enemmän ihmisiä tarkoittaisi suurempia aivoja? Geary antaa tunnustusta ``sosiaaliselle kilpailulle\vmq{,}'' missä useammat ihmiset kilpailevat samasta ruuasta. Fiksuimmat voittavat, saavat eniten lapsia, ja geenit pienemmille, typerämmille aivoille kuolevat pois, hän argumentoi.

Mutta ihmisten ravinnonlähde ei ole staattinen resurssi. Sen sijaan se on ihmistoiminnan suora seuraus. Enemmän ihmisiä tarkoittaa enemmän ruokaa, ei vähempää. Kalat eivät ryhmity matalaan veteen, odottamaan sitä, että joku kerää ne talteen. Hirvet eivät tule kahdentoista kappaleen paketeissa. Ruoka taistelee vastaan. Kyse on syvästä ja monimutkaisesta pähkinästä, joka ratkeaa teknologian, tietämyksen jakamisen, altruismin ja vaihtokaupan avulla. Sosiaalinen mallimme muuttuu tehokkaammaksi, ei vähemmän tehokkaaksi, kun ihmisiä on enemmän. Joten, mitä enemmän ihmisiä, sitä enemmän ruokaa.

Tämä pitää paikkansa kaikissa muissa tilanteissa, pitsi silloin, kun saavutamme ympäristömme rajat. Tämä tarkoittaa populaation romahdusta, ei räjähdystä. Ainoastaan katasrofaalisissa tilanteissa ihmiset kilpailevat ruuasta.

Voisimme myös kysyä, että miksi älykkyys saisi palkinnoksi ruokaa? Muissa eläimissä sitä ei tapahdu. Suuret aivot ovat kallis ja vaarallinen elin äidille ja lapselle. Miksi laji ei vain kehittäisi suurempia hampaita, tai vahvempia lihaksia, tai pidempiä jalkoja? Vaikuttaa mielivaltaiselta väittää, että älykkyys olisi avain suurien ruokamäärien keräämiseen ilman lisäselostusta.

Saamme Gearyn mallin toimimaan, kun vaihdamme ``sosiaalisen kilpailun'' ``kilpavarusteluun altruistien ja huijarien välillä\vmq{.}'' Kun populaatio koostuu pienistä, eristyksissä elävistä perheistä, huijaaminen on huono strategia. Huijareiden tunnistaminen ja rankaiseminen on helppoa. Pedot tarvitsevat tietyn populaatiotiheyden. Heidän täytyy kyetä liikkumaan tietyn alueen tyhjentämisen jälkeen.

Joten ekonomiset kannustimet huijaamiselle kasvoivat kun muinaiset ihmispopulaatiot kasvoivat. Jossakin pisteessä kilpavarustelu kävi kuumaksi. Yhteistyöhenkisille ihmisille kehittyi sosiaalisia emootioita huijareiden tunnistamista ja rankaisemista varten. Huijareille kehittyi kyky manipuloida ja matkia emootioita, jotta he kykenevät hakkeroimaan emotionaalisia kieliä. Sosiaaliset emootiot muuttuivat monimutkaisemmiksi, kun huijaava matkiminen kehittyi paremmaksi. Siinä missä yhteistyöhenkisten ihmiste sosiaalinen muisti kehittyi paremmaksi, huijarit kehittyivät paremmiksi valehtelijoiksi.

Ja niin edelleen ja niin edelleen. Aivomme ovat pullollaan psykopaattisia kykyjä ja psykopatian tunnistimia. Ei ole kyse siitä, että älykkäämmät ihmiset saivat enemmän lapsia. Kilpavarustaja teki pincer? liikkeen pienille aivoille. Olemme joko loistavia altruisteja, tai olemme loistavia huijareita. Kumpikin vaatii paljon älyä: mitä enemmän, sen parempi. Turvallista välimaastoa ei ole olemassa.

\section{Someone Stole my Lamp. I'm Delighted.}

Tutkailkaamme joitakin näistä psykopaattitunnistimista. Yksi on huumorintajumme. Huumori on ihmisyyden universaali piirre, joka näkyy jo pienissä lapsissa. Vauvat kikattavat ilosta kun he leikkivät vanhempiensa kanssa. Luotamme vaistonvaraisesti ihmisiin, jotka saavat meidät nauramaan. Emme luota niihin, jotka eivät pidä vitseistämme, tai joilla näyttää olevan ongelmia huumorintajussa.

Käytämme huumoria enemmän stressaavissa tilanteissa. Arvostamme omintakeista huumoiria ja palkitsemme ``kertomisen'' paremmin kuin itse vitsin. Kauhuelokuvissamme hirviöt eivät naura muuten kuin karmivalla, pieniä lapsia pelottavalla tavalla. Hirviöillä ei ole huumorintajua.

Vitsi on rakennelma, tarina, jolla on erityinen ja johdonmukainen muoto. Jokainen vitsi, jopa sanaleikit, riippuu mysteeristä. Emme kerro mysteeriä. Se olisi ``vitsin selostamista\vmq{.}'' Sen sijaan kerromme vitsin ja odotamme, että toinen ``tajuaa'' sen. Kun hän tajuaa vitsin, hän nauraa, ja tapahtuma on saatettu loppuun. Tai, riippuen vitsistä, saatamme odottaa huokausta.

Pelkkä nauraminen ei riitä. Molempien osapuolten on naurettavat oikealla hetkellä, ei liian aikaisin eikä liian myöhään. Naurun täytyy kestää tarpeeksi pitkään. Sen ei pidä olla liian kovaäänistä eikä liian hiljaista. Hyvä vitsi saa sekä kertojan että kuulijan iloiseksi. Epäonnistunut vitsi häiritsee ja ärsyttää meitä. Huumori on niin syvästy yhteydessä emootioihimme.

Huumoriprotokolla. Kuinka arvokas asia se onkaan. Tämä ei ole sattumaa.

Se, mitä varten huumori on kehittynyt, on empatian tunnistaminen. Vitsi on kortti, jolla on kaksi puolta. Näytämme toisen, ja pidämme toisen salassa. Jos kuulija on empaattinen tarinan hahmolle, hän näkee kortin piilotetun puolen. Tämä liipaisee naurureaktion. Jos kuulijalla ei ole empatiaa, hän on ymmällään.

Psykopaatti ei kykene nauramaan ``oikealla'' tavalla. Hän ei naura, tai hän nauraa liikaa, tai hän nauraa liian pitkään. Olemme varoisempia sellaisten ihmisten kohdalla jotka nauravat liikaa, kuin sellaisten, jotka eivät naua ollenkaan. Mitä hän piilotteleekaan, pohdiskelemme?

\section{Miltä se tuntuu sinusta?}

Toinen ``ihmisten uniikki kyky'' on taide. Miksi edeltäjämme tykkäsivät maalata kalliolle ja luolien seinämille? Perinteiset selitykset kuuluvat, että maalaukset on tehty taiteellisista syistä, tylsyydestä, huumeiden vaikutuksen alaisina, tai osana metsästysseremonioita.

Mutta esimerkiksi 40 000 vuotta vana norsunluinen Venus ei palvele mitään muuta funktionaalista tarkoitusperää kuin tunteiden saaminen aikaan katselijassa. Kyky luoda on niin laajalle levinnyt, että se soittaa jokaisella kadunkulmalla penneistä. Silti arvostamme sitä, ja vaikuttaa siltä, että lajimme on tehnyt sitä jo kauan. Ylise kaiken, me oletamme, että taide saa meidät ``tuntemaan'' jotakin. Me kysymme tätä toisilta: ``miltä se sinusta tuntuu?'' Ja skannaamme heidän kasvojansa kun he vastaavat.

Psykopaateilla on monia mielenkiintoisia piirteitä, joihin tulen myöhemmin tässä kirjassa. Yksi on heidän kiinnostuksen puute luoviin taiteisiin. He ievät piirrä, maalaa, muotoile tai kaiverra. He eivät ota valokuvia, paitsi heidä itsestään ja heidän tavaroistaan. He eivät kokkaa huvin vuoksi, keksi reseptejä, eivätkä he tee leipää itselleen harrastuksen vuoksi. He eivät luo musiikkia, vaikkakin he voivat olla erinomaisia toisten töiden esittäjiä.

Tämä luovan voiman puute on vinha juttu, kun se ensimmäisen kerran tulee vastaan. Se sopii yhteen heidän yleisellä tasolla tyhjän huumorintajun kanssa. Heidän harrastuksensa ovat matkustaminen, shoppailu, ravintoloissa syöminen, uusien ihmisten tapaaminen. Tämä on kuluttamista, ei luomista.

Taide on kallisarvoinen asia. Se on universaali ihmisten kieli. Aivan kuten komedian kanssa, me palkitsemme omaperäisyyttä enemmän kuin teknistä taitavuutta. Kuten komedian kanssa, nautimme taiteesta mieluummin seurassa kuin yksin. Ja kuten koomikkojen kanssa, ylistämme ja arvostamme taiteilijoita, vaikka kyvyllä ei ole selviytymisen kannalta arvoa. Viimeksi, mittaamme taiteilijoita heidän historiansa mukaan: yksi menestys ei ole tarpeeksi. Yksi menstys voi olla väärennös, varastettu, tai sattumaa. Kun taas urheilijan tai tieteilijän saavutus voi kestää elämän loppuun saakka.

Olen varma, että luovuus on toinen salainen empatian kieli. Se kysyy maailmalta, ``ystävä vai vihollinen? Katso tätä ja kerro minulle, että tunnet jotakin!'' ja katsoja vastaa, tai epäonnistuu testissä. Kyse on hyvin samanlaisesta asiasta kuin vitsin kertomisessa. Kuten loistavat vitsit, loistavan luovan työn täytyy puhua emootioista emootioille. Sen täytyy kertoa puolet tarinasta, jonka ainoastaa sosiaaliset aivot voivat täydentää ja ``tajuta\vmq{.}''

Nuori lapsi oppii piirtämään koulussa ja tuo töitään näytille vanhemille. Nämä lahjat eivät ole hyödyllisiä missää konkreettisessa mielessä. Silti ne ovat tärkeitä ja erityisiä siinä hetkessä. Lapsi katsoo hänen vanhempiensa reaktioita. Kun he näkevät iloa puristuneissa? kasvoissa ja kummallisissa väreissä, lapsikin tuntee iloa. He jakavat hetken, vahvistaen toistensa sosiaalisen ihmisyyden. Lapsi ajattelee, ''Katso, Äiti, olen normaali. Älä torju minua!''

Luomme itsellemme ja toisillemme. Luomme, jotta toiset tuntisivat jotakin. Useimmiten se on onnellisuutta, vaikkakin joskus se on menetystä, surua, tai muita emootioita. Luova taide on empatian viesti. Ja me mittaamme taiteemme laatua aivan samoin kuin mittaamme huumoria: se omaperäisyyden ja siten sen autenttisuuden kautta.

Tämän takia imitatiivinen taide on ``feikkiä\vmq{.}'' Se on syy, miksi insinöörimäinen suunnitteleminen ei ole taidetta, ja miksi pikaruoka tuntuu ``halvalta\vmq{.}'' Sen takia emme selosta vitsejä, ja sen takia taitelija ei voi selittää teoksensa ``pointtia\vmq{.}'' Se on testi, ja jos et tiedä vastausta, se jo itsessään on merkittävää. Sen takia kasa palikoita Tate Galleryssä? on arvoltaan miljoona puntaa. Siinä se vitsi piilee.

\section{Johtopäätökset}

Olemme alkaneet avata psykopaattien mysteeriä käsittelemällä heitä petioina, sen sijana, että käsittelisimme heitä rikkinäisinä ihmisinä. Petomalli saa aikaan muutakin kuin vain selittää psykopatian. Se myöskin selittää ihmismielen evolutiivisen kehityksen lopputuloksena iänikuisen altruistien ja huijareiden välisenä kilpavarusteluna. Kolmen miljoonan vuoden ajan altruistit ovat kehittyneet paremmaksi yhteistyön tekemisessä ja huijarit ovat kehittyneet paremmaksi sen jäljittelemisessä. Seuraavassa luvussa selostan, kuinka psykopaatit metsästävät.

\chapter{Metsästys}\label{the-hunt}

\section{Juhlakalu}

\begin{tarina}

Hän vaeltaa lempipaikkaansa, joka on suuri baari sekä ravintola. Paikka on suosittu kovaäänisen hauskanpitoa etsivän nuorison keskuudessa. Kesäinen lauantai-ilta on alkamassa. Hän juttelee portsarin Miken kanssa. Suuri mies nauttii rutiinin keskeyttävästä hetkestä ja kertoo pieniä palasia elämästään. Ex-tyttöystävä ja odottamaton lapsi. Pomo. Hän heittää portsarin kanssa ylävitoset ja lopettaa keskusteluhetken. ``Törmätään myöhemmin, Mike, käyn hakemassa itselleni kaljan\vmq{.}'' Hän liikkuu baarin toiselle puolelle.

Ulkoa terassilta hän löytää suuren pyöreän pöydän ja istuu siihen kylmän pullonsa kanssa. Ihmisiä valuu sisään. Hän katsoo heitä. Sekalaista sakkia, niinkuin täällä päin Texasia voi olettaa. Maahanmuuttajat tulevat tänne joka puolelta Yhdysvaltoja ja kauempaa. Miehet tsekkailevat naisia. Naiset käyttäytyvät kuin he eivät huomaisi miehiä. Joitakin pareja. Jotkut seisovat yksin selkä seinää vasten, kehonkielen huutaessa, ``Kumpa olisin pitempi\vmq{.}''

Paikka on kohta täynnä. Pienehkö nuorten miesten porukka istuu rappusilla pyöreän pöydän vieressä. Valkoisia, mustia, latinoja. He näyttävät viihtyvän huonosti. Hän kääntyy ja kysyy: ``Mistäs te kaikki olette kotoisin?'' Sotilaita läheisestä armeijan tukikohdasta viettämässä iltaa. Hän viittoo käsillään pöytäänsä. ``Liittykää seuraan\vmq{,}'' hän sanoo, ``pöydän ääressä on parempi meininkin\vmq{,}'' hän hymyilee. Miehet hyväksyvät ehdotuksen ja nousevat ja siirtyvät pöytään. He iloitsevat tuoleista ja siitä, että he olivat tervetulleita pöytään.

Nämä nuoret miehet ovat fiksuja ja uteliaita. Heitä ei ole vielä lähetetty sotaan, ja he ovat optimistisia ja luottavaisia. Hän kertoo heille polveilevia tarinoita ulkomaan seikkailuistaan. He nauravat jännityksestä. Hän pysähtyy, asettaa kätensä pöydälle ja toteaa itsestäänselvyyden. ``Tarvitsemme naisia!'' Yksi sotilaista osoittaa leuallaan: ``Nuo kaksi?'' Hän kääntyy nähdäkseen kaksi nättiä tummahiuksista naista. He näyttävät tylsistyneiltä ja epävarmoilta. ``OK, älkää liikkuko!'' hän sanoo miehille ja nousee ylös.

``Hei, leidit, miten menee?'' hän kysyy heiltä, kuuntelematta vastausta, mikä on aina ``mikäs tässä'' tai ``ihan jees\vmq{.}'' Hän tarkkailee mahdollisia merkkejä kiusallisuudesta. Naiset vaikuttat pitävän miehen kanssa juttelemisesta. ``Odotatteko jotakuta?'' hän kysy, ja vastaus on kieltävä: he ovat liikkeellä kahdestaan. Hän rypistää otsaansa ja tarkastelee heidän piirteitään.

``Mistä olette kotoisin?'' hän kysyy. ``Arvaa\vmq{,}'' vastaa toinen nainen nauraen. Hän yrittää paikallistaa heidät. Tummat kulmakarvat, tummanvihreät silmät, vaalea iho, korkeat poskipäät. Libanon? Georgia (maa, ei osavaltio)? Naiset nauravat ja pudistelevat päitään. ``Ei\vmq{.}''

``Haluaisitteko liittyä seuraamme? Pöydässämme on tilaa\vmq{,}'' hän sanoo, pyyhkäisten kädellänsä ilmaan laajan kaaren. He katsovat komeaa, pystytukkaista miestä kasvoihin, kohauttavat olkiaan, ja hyväksyvät tarjouksen. ``Miksipä ei!''

Naiset istuvat hänen viereensä. Mies puhuu heille, arvaillen lisää väärin. Venäjä? Armenia? He nauravat. Sotilaat ostavat heille juomia. Kaikki ovat iloisia, juhlat ovat parhaimmillaan. Lopulta mies tunnustaa tappiansa, jolloin naiset kertovat olevansa kotoisin Intiasta. Hän on shokissa, vaikuttunut ja lumoissaan.

``Kaunein nainen, jonka olen koskaan tavannut, oli kotoisin Georgiasta. Juttelimme viitisen minuuttia, ja halusin hänen kanssaan naimisiin siinä paikassa. Teillä molemmilla on samat piirteet. Olin varma, että olette Georgiasta! Mutta että Intiasta\ldots vau, Intiasta!''

``Kyllä, Intiasta!'' he nauravat imarreltuina. He viihtyvät tilanteessa. He juttelevat läpi yön. Baari menee kiinni ja asiakkaat siirtyvät parkkipaikalle. He lähtevät viimeisenä. Sotilaat hyvästelevät ja menevät matkoihinsa, ja nämä kaksi naista jäävät miehen seuraan. ``Haluaisitteko mennä jonnekin muualle?'' hän kysyy. ``Autoni on tuolla\vmq{.}'' Hän painaa avaimen nappulaa ja uuden avo-Mustanging valot välähtävät.

Myöhemmin toinen naisista kysyy: ``Kuinka kauan oletkaan tuntenut nuo kaverit?'' Hän vastaa: ``Aa ne, en pitkään, tapasin heidät vasta tänä iltana\vmq{.}'' ``Mitä?!'' hän huudahtaa shokeeraantuneena. ``Luulimme, että olet tuntenut heidät vuosia! Olitte kuin parhaat ystävät!''

\end{tarina}

\section{Sosiaalinen peto}

Psykopaatit kykenevät kohdistamaan lumoavan määrän voimaa toisiin ihmisiin. Se on kuin kahden kultti. Kun tapaamme sellaisia ihmisiä ja elämämme alkaa kietoutua heidän elämäänsä, meistä tuntuu siltä, kuin kohtalo kuljettaisi meitä. Tunne on kummallinen sekoitus varmuutta ja kontrollin menetystä. Aivan kuin putoaisi kovassa tuulessa. Uskonnollisen fanaattisuuden kuuma tuli. Ja se näyttää aina päättyvän kyyneliin.

Kysymys, jota muut usen kysyvät, on ``Miksi?'' Suhteet psykopaatin ja sosiaalisen ihmisen välillä ovat niin tuhoisia ja karvaita. ``Miksi'' on hyvä ruutu aloittaa. Kun kykenemme vastaamaan siihen, voimme alkaa kysyä ``Kuinka?'' ja ``Kuka?'' ja muita syvempiä kysymyksiä.

Tarinani jokaisen luvun alussa kertovat aina jonkunlaisesta pedosta. Jokainen psykopaatti toimii tällä tavoin. Psykopaatit metsästävät toisia ihmisiä. He hyökkäävät ja sieppaavat uhrinsa. He elävät uhriensa ajasta, resursseista, vallasta ja energiasta. He hankkiutuvat eroon jäänteistä. Ja he liikkuvat eteenpäin.

Väkivalta on piilevää. Joskus se päättyy uhrin itsetuhoiseen käytökseen, jopa itsemurhaan. Yleensä se päättyy masennukseen. Jokainen suhde sosiaalisen ihmisen ja psykopaatin välillä noudattaa samaa kaavaa. Vaikuttaa siltä, että poikkeuksia ei ole; ``kivoja'' psykopaatteja ei ole olemassa. Psykopaattius tarkoittaa sitä, että on peto.

Tämä ei ole metafora. Tämä on avain psykopatian dekoodaamiseen. \emph{He ovat petoja tai loisia, jotka elävät toisten ihmisten kustannuksella.} Ilman tätä avainta psykopatia on mystinen ja hämmentävä ilmiö. Kuin muinainen kirjoitus täynnä symboleja ja glyphejä. Teksti, joka vaikuttaa niin moneen meistä, ja jonka avaaminen on kuitenkin mahdotonta. Avaimen avulla voimme lukea tarinat ja voimmme ymmärtää.

Siinä, missä kuvailuni koskevat usein yksittäisiä suhteita, kaavat toimivat monissa tilanteissa. Näemme ne kulteissa, hyväksikäyttävässä yritystoiminnassa ja muissa petomaisissa organisaatioissa.

\section{Mallory, Alisa ja Bob}

Tietoturva-alalla vihamielistä hyökkääjää kutsutaan silloin tällöin Malloryksi. Samalla viattomia uhreja kutsutaan Alisaksi ja Bobiksi. Käytän näitä nimiä tässä kirjassa, jotta teksti olisi helpommin luettavaa ja pureskeltavaa.

Mallory voi olla mies tai nainen. Kirjoitan naisesta tai miehestä sen mukaan, mikä sattuu sopimaan tilanteeseen. Hän on aikuinen, vähintäänkin 14--16-vuotias, ja alle 70-vuotias. Mallory on psykopaatti.

Alisa ja Bob ovat altruistisia, sosiaalisia ihmisiä. He ovat Malloryn huomion kohteita.

\section{Kävele näin}

Ensitapaamisesi Malloryn kanssa on intensiivinen, henkilökohtainen ja syvä kokemus. Siis Alisalle ja Bobille. Mallorylle se on merkityksetön, kasuaali refleksi. Kun hän sanoo ``Moi'' sadalle ihmiselle, 96\% vaikuttuu. Hymy, silmät, tuon yksinkertaisen tervehdyksen \emph{syvyys}. Mallorylle tervehdys ei tunnu missään, eikä se vaadi häneltä minkäänlaista emotionaalista panostusta.

Tämä on se ``karisma'' josta puhutaan. Se ilon heijastus kun tapaamme jonkun, josta välitämme kovasti. Alisa ja Bob eivät voi väärentää sitä. He näyttävät sen ainoastaan niille, joista he välittävät. Se, että välittää \emph{jokaisesta} tapaamastaan ihmisestä on äärimmäinen perspektiivi, jonka löytäminen vie vuosikymmeniä. Mallory jäljittelee tätä refleksinomaisesti nuoresta iästä lähtinen. Se ei vaadi opettelua. Se on hänen ensimmäinen selviytymisen sääntö: \emph{muiden on ihailtava sinua.}

Vaikutus on niin voimakas, että voit käyttää sitä psykopaattien tunnistamiseen luonnossa. Palaan tähän luvussa \ref{hunting-mallory}. Useimmat, jotka törmäävät Malloryyn, ovat taipuvaisia kokemaan pikkuruisia mielihyvän purkauksia. Jos henkilö on edes pikkuruisen yksinäinen, houkuttelee se hänet takaisin keskusteluun. Samalla Mallory skannailee uusia kohteita. Siihen ei tarvita keskustelua. Hän näkee ihmisten haavoittuvuudet kehonkielestä.

Sosiaaliset ihmiset voivat oppia tämän kyvyn vuosien harjoittelun myötä. Mallory ei tarvitse harjoittelua. Se on yksi hänen monista sisäsyntyisistä kyvyistä. \emph{Forbes}-lehti kirjoittaa:\linkki{sdfsd} ``Vaikuttaa siltä, että psykopaatit eivät tarvitse meditatiivista harjoitusta ollakseen suhteettoman tarkkaavaisia{\ldots} toisten heikkouksien suhteen\vmq{.}''

Heikkoutemme näkyy erityisesti kahdessa asiassa. Ensinnäkin yksinäisyys ja yksinäisyydestä kertova kehonkieli ovat osoituksia heikkoudesta. Toiseksi, pelon ja epävarmuuden näyttäminen ja hyväksikäytön kokemusten näyttäminen kertovat heikkoudesta.

Monet uskovat, että hyväksikäytön uhreista tulee itsekin hyväksikäyttäjiä. Prosenttiosuus on kuitenkin vain kymmenen luokkaa,\linkki{sdfds} paitsi jos hyväksikäyttäjä ja hyväksikäytetty ovat samasta perheestä. Siinä tapauksessa prosentti nousee merkittävästi. Ja joka kolmas näistä aikuisistä hyväksikäyttäjistä oli lapsena julma eläimiä kohtaan.

En epäile, etteikö psykopaatit hyväksikäyttäisi, laiminlöisi ja kiduttaisi henkisesti omia lapsiaan, ja etteikö moni näistä kasvaisi psykopaatiksi. Kyse on mekanismista, jonka olen havainnut, ja jonka selostan myöhemmin. Mutta tämä ``hyväksikäyttö aiheuttaa hyväksikäyttöä''-malli jättää huomiotta ne 90\% lapsista, jotka kokevat seksuaalista hyväksikäyttöä, ja joista kuitenkin kasvaa aikuisia, jotka eivät vahingoita toisia. Mielestäni tämä johtuu sosiaalityöntekijöistä, joita nuoret psykopaatit ovat onnistuneet huijaamaan. ``Isäni hyväksikäytti minua, ja siksi satutan muita\vmq{.}'' Psykopaatit eivät koskaan ota vastuuta teoistaan.

Todellisuudessa hyväksikäytön uhrit ovat taipuvaisia todistamaan oman elämänsä traumoja sanomatta sanaakaan. Tai, kuten Joanna Moore kirjoittaa kirjassa? \emph{The Faces of Narcissism},\linkki{sdsf} ``on helppoa syyttää vihaista uhria ja tukea rauhallista hyväksikäyttäjää\vmq{.}''

Menneisyydessä koettu hyväksikäyttö on ensisijainen indikaattori tulevaisuudessa koettavalle hyväksikäytölle. Hyväksikäyttävästä perheestä tulemin leimaa meidät pelolla ja epävarmuudella. Hyväksikäyttävä työnantaja tai puoliso saa aikaan saman. Pelkomme ja epävarmuutemme on kuin neonvaloa hohtava ``Syö minut!''-kyltti kaikille ohikulkevilli psykopaateille.

Toisten ihmisten pelko näkyy kehonkielessämme.\linkki{sdfds} Hyväksikäytön uhrit nostavat jalkojansa korkeammalle kävellessään. He ottavat keskimääräistä pitempiä tai lyhyempiä askelia. He sätkyttelevät käsiänsä ja jalkojansa. He välttelevät katsekontaktia. He näyttäväþ alistuvia ja puolustavia kehon asentoja.

Kaikki nämä vihjeet ovat helppoja luettavia, jos lukijalla on sopivanlainen mieli. Rikollisista psykopaateista tehdyt tutkimukset osoittavat, kuinka psykopaatit poimivat tällaisia vihjeitä.

Minulla ei ole numeroita siitä, kuinka nopeasti tämä tapahtuu, ainoastaan anekdoottisia kertomuksia. Voisin arvata, että Mallory kykenee käymään läpi sata ihmistä suurin piirtein kymmenessä minuutissa.

\section{Suuret, siniset munat}

Todennäköisiä kohteita etsivä Mallory kävelee läpi väkijoukon. Hän projisoi seksuaalisuuttaan vain hiukan muita naisia kovemnin. Hän etsi yksinäisiä, menestyviä miehiä. Hän tarkkailee, kuinka miehet katsovat häntä, ja näkee yhden hermostuneen reaktion, joka kiinnittää hänen huomionsa. Hän nousee ylös ja heilauttaa hiuksiansa, hengittää sisään, hymyilee hänelle. Hän katsoo miehen kasvoja. Hän katsoo hiukan liian pitkään. Hän hymyilee itselleen.

Valittuaan Bobin kaikkien potentiaalisten kohteiden joukosta, Mallory tekee siirron. Mitään näkyvää takaa-ajoa, juoksemista tai kirkumista ei tapahdu. Siirrot eivät kerro totuutta. Mallory liukuu sisään Bobin elämään kuin kauan kadoksissa ollut ystävä. Hän vaikuttaa niin kivalta, harmittomalta ja vilpittömältä.

Hän avaa pelin laajalla paletilla taktiikoita, joka riippuu kontekstista. Nämä hyökkäykset toimivat sekä hyökkääjässä että uhrissa vaistojen tasolla. Hän aloittaa laajalla, keskittämättömällä luotaamisella. Kun Bob vastaa vaistonvaraisesti, Mallory siirtää ja säätää peliään ja painaa kaasua.

Vaistonvaraisen käytöksen liipaisijat ovat yleensä yksinkertaisia karikatyyrejä. Evoluutio on tässä mielessä laiska. Esimerkiksi monet ihmiset pelkäävät hämähäkkejä niin kovasti, että ne saavat heidät kirkumaan. Liipaisin sijaitsee geeneissämme monella jalalla varustettuna mustana pisteenä. Pelkäämme tiettyä tapaa, jolla jalat liikkuvat. Piirretty hämähäkki, joka kävelee oikealla tavalla, on yhtä pelottava kuin oikea hämähäkki. Laita se kävelemään kuin ihminen?, ja se näyttää harmittomalta. Liioittele hämähäkkikävelyä ja piirretty hämähäkki on oikeaa pelottavampi.

Kun liipaisimen eristää ja sen voimakkuutta kasvattaa, kasvaa myös reaktio. Vain taivas on rajana. Otetaan esimerkiksi lajimme makeanhimo. Reagoimme vaistonvaraisesti fruktoosiin, jota kasvit pumppaavat hedelmiin pieninä annoksina. Makeus iskee samoihin aivojemme alueisiin kuin kokaiinin kaltainen huume. Luonnossa tämä ajoi kaksijalkaiset edeltäjämme syömään niin paljon hedelmiä kuin he vain kykenivät löytämään. Sitten jalostimme aina vain makeampia hedelmiä. Sitten opimme jalostamaan sokeria ja aloimme lataamaan sitä ruokavalioomme. Syömme satoja paunoja sokeria vuodessa, itsetuhoon asti.

Vaistonvaraisen fruktoosihimomme rajat eivät tulleet missään vaiheessa tätä tarinaa vastaan. Sen sijaan mitä enemmän kulutamme, sitä onnellisemmaksi näytämme tulevan.

Tämä eskaloituva reaktio tiivistettyyn liipaisimeen on tunnettu ilmiö nimeltä ``supernormaali stimuli\vmq{.}''

Pedot ja loiset ovat erikoistuneet käyttämään supernormaalia stimulia saaliiseensa. Se pakottaa itseään rankaisevaan ja epäloogiseen käytökseen, kunnes liipaisumekanismin ymmärtää. Joten parasiittinen lintulaji saattaa kaapata ne liipaisimet, joita nuoret poikaset käyttävät ruoan kinuamiseen. Esimerkiksi avoin punainen suu. Parasiitti imitoi ja liioittelee tätä liipaisinta, joka saa lintuemon ruokkimaan loispoikasta ennen sen omia poikasia.

Merien syvimmissä vesissä krotti roikuttaa pimeässä loistavaa kirkasta syöttiä. Tämä liipaisee saaliskalan uimaan kohti sen hampaista suuksi kutsuttua ansaa.

Tai mietitäänpä puussa pesivän laululinnun munia. Ne ovat usein vaaleansinisiä munia, joissa on tummanharmaita tai ruskeita pilkkuja. Tämä nimenomainen värikaava liipaisee naaraan tai koiraan istumaan munan päällä. Ehkä siksi, etteivät ne istuisi satunnaisten kivien tai toisen lintulajin munien päälle. Parasiittinen käki munii pesään suurempia, sinisempiä munia, joissa on tummempia pisteitä. Tämä saa laululinnun suosimaan käen munia omiensa sijaan.

Kilpavarustelu loisen ja isäntälajin välillä luo luonnollisen tasapainon. Liipaisimen liiallinen käyttäminen kääntyy parasiittia itseään vastaan. Jos käki tekisi munistaan liian houkuttelevia, haavoittuvaiset laululinnut eivät lisääntyisi ollenkaan. Laululinnut, jotka eivät reagoi liipaisimeen, saisivat etulyöntiaseman ja dominoisivat. Loiselle isäntälajin tappaminen on häviävä strategia. Tämä tarkoittaa sitä, että ainoastaan vastustuskykyiset isäntälajine edustajat lisääntyvät.

Niko Tinbergen, biologi joka löysi ja nimesi supernormaalin stimulin, rakensi muovisia munia. Hän havaitsi, että linnut suosivat muovimunia omiensa sijaan. Ne suosivat omiaan suurempia munia. Ne suosivat normaalia kylläisempiä värejä. Ja ne suosivat niiden omien munien kuvioita rajumpia kuvioita.

Joten pienen vaaleansinisen harmaantäplikkään munan sijaan hän tarjosi laululinnulle väärennöksen. Hänen munansa oli valtava, kirkkaan sininen, ja siinä oli suuria mustia täpliä. Lintu yritti istua munalle, yhä uudelleen ja uudelleen, ja putoili sen päältä pois.

Tämä saattaa naurattaa, mutta linnulle tämä on järjetöntä käytöstä. Näemme, että supernormaali stimuli voi tuottaa järjetöntä käytöstä järkevästi kehittyneistä vaistoista. Kyse on evolutiivisesta porsaanreiästä, jota monet pedot ja loiset käyttävät hyväkseen. Ihmispsykopaatit käyttävät sitä usein manipuloidakseen kohteitaan haluamaansa suuntaan.

\section{Avaussiirrot}

Vuonna 1989 Clatk ja Hatfield Floridan Yliopistosta? tekivät kuuluisan tutkimuksen.\linkki{dsfs} Heidän viehättävä tutkimusassistenttinsa käveli ympäri kampusta ja ehdotti treffejä ihmisille.

Tulokset ovat tunnettuja. Yli puolet miehistä hyväksyivät treffiehdotuksen, ja vielä useampi oli valmis menemään kotiin vieraan naisen kanssa. Kolme neljäsosaa hyväksyi suoran seksiehdotuksen. Naiset taas yleensä sanoivat ``ei\vmq{.}'' Opiskelijat eivät ehkä tyypillisiä esimerkkejä koko populaatiosta, mutta muut ovat saaneet samanlaisia tuloksia.

Tulen sukupuolien eroihin hetken päästä. Alkuperäinen tutkimus esitti, että naiset eivät harrasta satunnaista seksiä. Tiedämme, että tämä ei ole totta, ainakaan kaikissa konteksteissa. Ensimmäinen kysymykseni on: ``Kuinka niin moni mies on niin helposti koukutettavissa?'' Sitä voisi sanoa, että riski, jolle mies altistuu satunnaisessa seksissä on matala, mutta asia ei ole niin. Ilmiselviä riskejä ovat taudit ja yllätysvanhemmuus. Ja sitten on paljon suurempi vaara, että koko homma on lavastettu jonkinlaista huijausta varten.

Ja kuitenkin useimmat miehet sanova ``Ehdottomasti!'' Kuinka naisen viehätysvoima voi olla niin tehokas syötti? Ovatko miehet epätoivoisia, kiimaisia ja typeriä? Ovatko naiset fiksumpia? Ehkäpä vastaus on jotakin hieman hienovaraisempaa. Käy myös ilmi, että naiset eivät ole sen vastustuskykyisempiä kuin miehetkään. Kyse on vain siitä, käytetäänkö oikeanlaista syöttiä.

Kun vastaamme näihin kysymyksiin, näemme, kuinka merkittävä asia sukupuoli on. Sillä on oleellinen rooli psykopaatin avaussiirroissa. On olemassa neljä erillistä kaavaa: nainen miehelle, mies naiselle, mies miehelle ja nainen naiselle. Monet sosiaaliset vaistomme taipuvat kohti maskuliinista ja feminiinistä napaa, aivan kuten kehommekin.

Kehomme ja mielemme ovat lähtökohtaisesti feminiiniset. Kun mies kehittyy, ajoitetut testosteronipurkaukset siirtävät kehoa ja mieltä kohti miehuutta. Miehet ja naiset eroavat kehoiltansa ja mieliltänsä evoluutiota ajavien voimien vuoksi.

Joten kun sanon ``mies\vmq{,}'' se sisältää naiset, joilla on miehelle tyypilliset vaistot. Ja kun sanon ``nainen\vmq{,}'' se sisältää miehet, joilla on naiselle tyypilliset vaistot. Nämä avaussiirrot eivät oleta heteronormatiivisuutta. Psykopaateilla on usein nestemäinen seksuaali-identiteetti.\linkki{sdfds} He ovat yhtä itsevarmoja ja petomaisia homoseksuaaleina kuin heteroseksuaaleina.

\section{Naiset metsästämässä miehiä}

Psykopatian ``antisosiaalinen'' osa ei tarkoita sitä, etteikö henkilö haluaisi toisten seuraa.\linkki{sdfds} Se tarkoittaa haluttomuutta kunnioittaa sosiaalisia normeja ja käytäntöjä. Psykopaatit ovat taipuvaisia hypersosiaalisuuteen ja he etsivät pakkomielteisesti uusia ystäviä. Kyse on reviiristä?.

He voivat esiintyä henkilökohtaisesti ja hienotunteisesti, mikä yleensä piilottaa intensiivisen taustatyön. Web on tehnyt tästä paljon helpompaa, tarjotessaan monia tapoja puhua toisille yksityisesti.

Haluaisin louhia Facebook-dataa yksityisten keskustelujan, julkisten postauksien ja selfieiden takia. Arvaukseni on, että löytäisimme erillisen ryhmän käyttäjiä, joilla on paljon keskimääräistä enemmän yksityisiä keskusteluja, paljon keskimääräistä useamman eri ihmisen kanssa. Ennustan, että näkisimme kaksi päällekäistä kellokäyrää, emme vain yhtä.

Pedot tykkäävät mesästää sellaisissa paikoissa ja tapahtumissa, missä heillä on etulyöntiasema. Tilanteen täytyy tarjota tuore uusien kohteiden lähde. Kohteiden täytyy haluta jotakin, mitä peto voi hyödyntää. Kohteiden pitää tarjota potentiaalista hyötyä metsästäjälle. Tilanteen tulee tarjota suojaa ennen ja jälkeen minkä tahansa hyökkäyksen sekä sen aikana. Tilanteen pitää tehdä asioista jälkeenpäin puhuminen uhreille hankalaksi.

Deittailuskene on ilmiselvä mahdollisuus. Baarit, yökerhot ja deittisivustot ovat ideaalista aluetta sekä nais- että miespsykopaateille. Deittailun popkulttuuri on käsitellyt psykopatiaa jo tovin. He käyttävät kiertoilmaisua ``narsisti\vmq{.}''

Eräällä webbisivulla Susan Walsh käsittelee naispuolista narsismia\linkki{dfgf} ja listaa sellaisen henkilön piirteitä. Ensiksi, fyysinen olemus:
\begin{quotation}
\noindent Pukeutuu provokatiivisesti, rehennellen seksuaalisesti viihjaavilla kehonosilla; keskittää huomion meikkiin ja hiuksiin, jopa kaikista arkipäiväisimmissä tilanteissa; yli-itsevarma ulkonäöstään; pitää brändejä arvossa, ja kokee olevansa oikeutettu pukeutumaan ``parhaaseen;'' osta usein uusia vaatteita, eikä tee eroa haluille ja tarpeille; käyttää tavanomaista todennäköisemmin plastiikkakirurgiaa, tyypillisimmin rintojen suurennusta; tykkää olla valokuvauksen kohteena, ja usein pyytää muita ottamaan itsestään kuvan; jakaa innoissaan itsestään parhaat kuvat sosiaaliseen mediaan.
\end{quotation}
Toiseksi, persoonallisuus ja luonne:
\begin{quotation}
\noindent Vaatii päästä olemaan huomion keskipisteenä, monesti huoneen hurmaavin henkilö; hakee usein myötämielistä kohtelua, ja automaattista myöntymistä; uskoo olevansa erityinen; on erittäin materialistinen; on altis kateudelle, vaikka esittää itsensä äärimmäisen itsevarmana; etsii mahdollisuuksia toisten torpedoimiseen; on varma, että muut ovat kateellisia ja mustasukkaisia hänelle; häneltä puuttuu empatiakyky, ja jopa yleinen kohteliaisuus silloin tällöin; mollaa muita, mukaanlukien sinua; ei epäröi käyttää muita hyväksi; on kilpailuhenkinen; uskoo olevansa älyllisesti ylivertainen; syyttää muita ongelmista; ilmaisee ylimielistä asennetta silloin, kun hänen puolustus on alhaalla tai joku haastaa hänet; on epärehellinen ja usein valehtelee saadakseen mitä hän haluaa; on ``psyko\vmq{,}'' käyttäytyy riskialttiisti, ja omaa addiktoivan persoonallisuuden, ja on taipuvainen agressiiviseen käytökseen torjuttaessa; hänen mielialat ja teot ovat mahdottomia ennustaa.
\end{quotation}
Tämä sopii 95-prosenttisesti moniin naispsykopaatteihin joita tunnen tai olen tuntenut. Kirjoittaja sanoo: ``Pohjautuen kaikenikäisiin naisiin jotka olen tuntenut elämässäni, mielestäni 10\% on tarkka arvio narsistien osuudessa naispopulaatiossa\vmq{.}'' Numero on suuri, mutta se sopii arvoihin, joiden mukaan 10\% populaatiosta on sub-kliinisiä psykopaatteja. Olen vakuuttunut, että tämä petomainen ja tuhoisa ``narsismi\vmq{,}'' jota Walsh kuvailee, on yksi psykopatian maskeista.

Fyysinen olemus on kuin suuri ase, joka tähtää kohti miehen biologiaa. Sen vaikutus voi olla tuhoisa. Floridan yliopiston tutkimuksen mittauksen mukaan 75\% miehistä tarttuu sellaiseen syöttiin. Ehkä osuus on pienempi koko populaatiossa, kuin mitä se on yliopistokampuksella. Mutta deittailutilanteessa useimmat miehet etsivät satunnaista seksiä. Luku liukuu kohti sataa prosenttia.

Kun Mallory metsästää miehiä, hän ei vain kysy jokaista miestä drinkille. Se olisi turhan yksinkertaista. Hän tietää mitä hän etsii. Joten hän voi valita parhaat kohteet, jopa ennen kuin he näkevät hänen olevan paikalla.

Ihmiset reagoivat kuten mikä tahansa muukin elämänmuoto liipaisimiin ja supernormaaliin stimulin. Naiset, jotka pyrkivät vetämään miehiä puoleensa, investoivat oleellisten liipaisimien vahvistamiseen. Tässä on lista liipaisimista, jotka olen onnistunut tunnistamaan ja keräämään:
\begin{description}
\item[Vyöntärön suhde lantioon (WHR).] Tämä on ensisijaine signaali ihmisnaisen seksuaalisuudesta. Ideaalinen WHR liikkuu välillä 0,6--0,8 riippuen kulttuurista. Kapea vyötärö indikoi nuoruutta ja leveät lanteet hedelmällisyyttä. Yksinkertaisin WHR-huijaus on topata lantio. Sitten voi käyttää korsettia, ja sitten kauneuskirurgiaa. Rehellinen vastaliike on pitää tiukempia vaatteita ja näyttää enemmän paljasta pintaa.
\item[Rintojen koko ja muoto.] Ihmisnaisen rintojen kehityksestä käydään paljon väittelyä. Niiden koko ja muoto ei tarkoita suurempaa määrää tai parempilaatuista maitoa. Jotkut ajattelevat että rinnat kehittyivät vauvan tyynyiksi. Jotkut ovat sitä mieltq, että ne imitoivat pakaroita. Minun nähdäkseni ne kertovat naisellisesta nuoruudesta ja saatavuudesta, jotka molemmat ovat liipaisimia miehille. Ennen moderneja aikoja vauvoja imetettiin usein kahdesta kolmeen vuoteen. Rintaruokinta muuttaa rasvakudosta ja koossa pitävää kudosta? (Coopers's Ligaments). Joten rinnat näyttävät välittömästi, onko nainen jo saanut vauvoja vai ei. Vauvat tarkoittavat, että nainen on huonommin saatavilla, ja osoittaa kohti suojelevaa aviomiestä. Kuten WHR:n tapauksessa, huijari voi käyttää toppausta tai kirurgiaa. Ja rehellinen reaktio on taas kerran näyttää enemmän paljasta pintaa.
\item[Muut luotettavat nuoruuden indikaattorit.] Ensinnäkin sileä iho käsissä ja kasvoissa. Mitä sileämmät kasvot, sitä vahvemmin silmät, kulmakarvat ja huulet loistavat. Naiset voivat piilottaa kauneusvirheitä meikillä. He voivat liioitella silmien muotoa, huulia ja kulmakarvoja. Silmiinpistävät piirteet sileällä, virheettömällä iholla ovat liipaisin. Ja täydet huulet, pieni nenä ja korvat. Huulemme ohenevat iän myötä, ja nenämme ja korvamme kasvavat. Nainen voi vaikuttaa näihin ainoastaan plastiikkakirurgialla.
\item[Muut luotettavat hedelmällisyyden indikaattorit,] joita estrogeenihormooni ilmaisee. Tärkeimpiä ovat korkeat poskipäät ja ääni. Korkea, melodinen ääni kertoo hedelmällisyydestä ja nuoruudesta. Havaitse, kuinka jotkut naiset vaihtavat äänensä hiukan korkeammaks pyytäessään palvelusta. Sitä on lähes mahdoton tehdä kuulostamatta epäuskottavalta.
\item[Jalkojen suhde kehoon (LBR).] Mikä tahansa, mikä keskeyttää kasvamisen nuoruudesta aikuisuuteen vaikuttaa LBR:n. Tämä on hyvä geenien, dieetin ja terveyshistorian indikaattori. LBR ennustaa vastustuskykyä monille taudeille diabeteksesta moniin eri syöpiin. Pitkät jalat tarkoittavat terveyttä, ja terveys on seksikästä. Tämä on yksi harvoista liipaisimista, jotka toimivat molemmissa sukupuolissa. Naiset voivat väärentää LBR:ää käyttämällä korkeita korkoja ja lyhyitä hameita.
\item[Muut geneettisen resistanssin ja terveyshistorian indikaattorit.] Nämä ovat symmetriset kasvot, pitkä, puhdas tukka, kirkkaat, säkenöivät silmät, ja terveet kynnet. Tukka ja kynnet on nykyään helppo väärentää, eikä mitään todellista puollusta ole, lukuunottamatta tulitaukoa. Ehkä huivit ja muut ovat kehittyneet tämän takia??. Mascara voi saada silmät näyttämään valkoisemmilta ja säihkyvämmiltä.
\item[Haavoittuvuuden ja alistuvuuden indikaattorit.] Neito pulassa liipaisee petomaisen suojelevan reaktion miehissä.\linkki{sfdfs} Tekstitys kuuluu: ``Pelasta minut ja palkitsen sinut seksillä\vmq{.}'' Tummempi versi on: ``Olen yksin enkä voisi estää sinua vaikka haluaisin\vmq{.}'' Tähän liittyy useampia kehonkielisiä liipaisimia. Jalat yhdessä, ranteet näkyvillä tai velttoina. Pää alhaalla, katsekontaktin välttely. Tai pää alhaalla ja katse ylöspäin, näyttääkseen nuorelta. Ja viimeisenä humaltuneen näytteleminen. Palaan myöhemmin psykopaatteihin ja alkoholiin.
\item[Seksuaalisen saatavuuden ja himon merkit.] Toisin sanoen se, kun nainen kertoo miehelle: ``Haluan sinua, ja haluan seksiä kanssasi juuri nyt\vmq{.}'' Tähän liittyy ainakin kaksi liipaisinjoukkoa. Yksi on kosmeettinen huulien ja poskien punaaminen. Tämä matkii naisen kiihottumisen merkkejä (loistavat huulet ja kasvot). Toinen liipaisin on kehonkieli. Nainen pitää yllä katsekontaktia. Hän liikkuu lähemmäksi miestä. Hän käyttää paljastavaa vaatetusta. Hän vaihtaa asentoaan ja vaatetusta näyttääkseen enemmän paljasta pintaa. Hän koskee miehen käsivartta, leikkii hiuksillaan ja avaa huulensa. Hän nostaa kulmakarvojaan ja sulkee silmänsä puoliksi. Yleensäkin hän käyttäytyy kuin he olisiva sängyssä ja hän nauttisi siitä. Tämä on sosiaaliselle naiselle mahdotonta, lukuunottamatta turvallisissa olosuhteissa tapahtuvaa pelitilannetta. Torjutuksi tulemisen pelko on liian suuri. Psykopaateilla ei ole sellaista pelkoa, joten he voivat viedä tämän näytelmän äärimmäisyyksiin.
\item[Uutuudenviehätys.] Kutsutaan myös Coolidge-ilmiöksi. Siinä missä monogamiset parit ovat normi, monet ihmiset ovat opportunistisia pettäjiä. Useimmat miehet suosivat uusia avoimia seksipartnereita olemassaolevien ohi. Tämä ilmiö on yksi syy sille, miksi pornoteollisuus kaipaa niin kovasti uusia nuoria tähtiä. Kuinka nainen, joka ei pelkää joutuvansa naurunalaiseksi, voi näyttää joka viikko \emph{erilaiselta?} Se on yksinkertaista: hän vaihtaa hiustyyliä ja vaatteita. Uudet hiukset tarkoittavat uusia kasvoja. Uudet vaatteet ovat uusi keho. Molemmat saavat aikaan voimakkaampia reatkioita miehissä, jotka tuntevat naisen jo etukäteen.
\end{description}
Tämä joukko liipaisimia ja rajoittamattomat miesten reaktiot niihin selittää pornografian. Monille miehille se on lähes addiktoivaa. Se on kuin sokeri: jalostetun liipaisimen lähde. Porno näyttää nuorien, saatavilla olevien naisten virran, joka osuu kaikkiin kohtiin ylläolevassa listassa. Pornosivustotilastoissa maailmanlaajuisesti suosituin kategoria on ``teini\vmq{.}''\linkki{sdfds}

Miksi pakkomielle nuoruuteen? Deittailusivusto OKCupid havaitsi, että kaikenikäiset naiset suosivat omanikäistä partneiria. Kaikenikäiset miehet suosivat---silloin kun kukaan ei ole näkemässä---22-vuotiaita tai nuorempia naisia. Deittisivustoprofiilit ja haut voivat vaikuttaa heikolta pohjalta tieteelle. Turiun yliopiston Åbo Akademian? 12 000 suomalaista koskenut tutkimus havaitsi pitkälti saman asian: naiset suosivat omanikäisiään miehiä ja miehet suosivat noin 25-vuotiaita naisia.

Vastaus löytyy lajimme pitkäaikasen monogamisen suhteen mallista. Naisten hedelmällisyys on huipussaan nuorena, joten miehet ovat kehittyneet näkemään 16--22 vuoden iän (18--25, kun muut ovat näkemässä) ``seksikkyyden'' huippuna. Simpansseilla, läheisillä sukulaisillamme, on erillainen perhemalli. Ne elävät laajennetuissa perheissä, ilman monogamisia pareja. Niinpä urossimpanssit eivät liipaistu naisellisesta nuoruudesta. Ja simpanssinaaraat eivät esitä nuoruutta.

Jos olet koskaan pohtinut, miksi miehet ovat niin lumoutuneita naiskauneuteen, tiedät nyt vastauksen. Kuten myöhemmin selostan, naiset ovat aivan yhtä lumoutuneita miehisiin liipaisimiin.

Aikuiset miehet reagoivat oletusarvoisesti näihin supernormaaleihin stimuleihin. He ovat kuin laululintu, joka hoippuu jalkapallokentän kokoisen supersinisen munan pällä. He yrittävät uudelleen ja uudelleen käynnistää seksuaalisia suhteita naisen kanssa. Kohtelipa nainen heitä tai muita kuinka kaltoin, he kiipeävät takaisin munan päälle ja yrittävät uudelleen. Se näyttää järjettömältä. Se voi johtaa itsetuhoon.

Useimmat naiset jotka pukeutuvat vietelläkseen eivät ole psykopaatteja. Useimmissa tapauksissa se on aitoa ja terveellistä. Ero on siinä, kuinka syvällistä harhautus on. On niitä, jotka hiukan vääristävät totuutta, ja sitten on ammattilaisvalehtelijoita. Naispsykopaatit lähettävät seksuaalisuuttaan viekoitellakseen ja laajemmin kuin mihin sosiaalinen nainen kykenee. He käyttävät sitä hallitakseen narratiivia. He provosoivat off-the-charts?? reaktion, ja sitten he pidättelevät. Kaikki menee paljon peliä pidemmälle. Lupaus on: ``Minä ole täydellinen nainen ja minä olen sinun\vmq{.}'' Totuus on: ``Sinä kärsit ja maksat, etkä tule koskaan saamaan sitä ensimmäistä huumaa takaisin, ikinä\vmq{.}''

Haluaisin tarjota lähteitä tälle ilmiölle, mutta se on käsittääkseni dokumentoimaton asia. Olen kokenut sen ja havainnut sen tarpeeksi usein todetakseni, että se on todellinen ja tarkoituksellinen. Ja mekanismi vaikuttaa tutulta. Miehen reaktio naisellisiin seksuaalisiin signaaleihin asuu aivan riippuvuuden naapurissa.

Meidät on johdotettu kokemaan nautintoa näiden liipaisimien painamisesta. Biologiamme toimii sillä tavoin. Dopamiini osuu aivon emotionaalisiin keskuksiin. Ne kokevat iloa. Tämä vahvistaa sitä käytöstä mikä saikaan meidät alunperin liipaisimen äärelle. Kun joku vahvistaa liipaisinta, syntyy suurempi dopamiiniryöppy. Mieli kompensoi muuttumalla vähemmän herkemmäksi. Joten niinpä tarvitsemme lisää liipaisemista kokeaksemme saman ilon tunteen.

Addiktio ei tässä ole metafora. Se on psykopaatin kanssa tapahtuvan seksuaalisen suhteen ydin. Ja sellainen suhde on yhtä mahtava ja terveellinen kuin kokaiiniin tai raakaan viinaan pohjautuva elämä.

Psykopaatit suosivat sellaisia liipaisimia, joita he voivat jäljitellä tai vahvistaa keskitetyllä vaivannäöllä. Joten Mallory voi olla jokseenkin koruton, ja kuitenkin haltioiva silloin kuin hän niin haluaa. Selostin tapoja huijata useampia eri liipaisimia. Jokaiseta huijausta kohti---sanotaan vaikka, että nainen valehtelee ikänsä---on olemassa rehellinen pelaaja. Vaikkapa nuorempi nainen joka kilpailee samoista miehistä. Tämä on hidas, muinainen kilpavarustelu eri strategioiden välillä.

Naispsykopaatit näkevät enemmän vaivaa ulkonäön eteen, ja vähemmän vaivaa ystävien ja perheen eteen. He jäljittelevät liipaisimia, joita sosiaaliset naiset eivät voi tai halua jäljitellä. Sosiaaliset naiset sen sijaan kilpailevat heidän todellisilla antimillaan. Tämä kilpavarustelu kiittää autenttisia naisten painelemia miesten liipaisimia. Eli täydet rinnat, leveät lanteet, pitkät hiukset ja sileä iho. Ne ovat seksuaalivalinnan ja huijareiden ja rehellisten pelaajien välisen kilpavarustelun hedelmä.

\section{Miehet metsästämässä naisia}

Naiset ovat tottakai erilaisia, ja he ovat immuuneja halvalle imartelulle ja jäljitellyille pullistumille. Naisreaktiot vierailta miehiltä tuleviin seksiehdotuksiin ovat lähellä nollaa. Nainen käsittelee sellaista tarjousta yleensä vihamielisenä tekona. Hän on taipuvainen soittamaan apua poliisilta tai miespuolisilta ystäviltä.

Kyse on kuitenkin vain kontekstista. Kun liipaisimet ovat kohdallaan, useimmat naiset reagoivat. He kävelevät kohti heikkoja liipaisimia. He hölkkäävät kohti vahvoja liipaisimia. Ja he harppovat itsetuhoisen draaman saattelemana kohti supernormaaleita liipaisimia. Aivan kuten miehet.

Joten mitkä ovatkaan näitä liipaisimia? Mikä tekee miehistä viehättäviä naisten silmissä? Kyse on jossain mielessä iänikuisesta mysteeristä. Mutta vastaus in ilmeinen, kun sen vain näkee.

Biologia ja empiirinen tutkimus sulkee pois monia ilmiselviä vaihtoehtoja. Saatavuus ja haluukkuu seksiin eivät ole liipaisimia. Nuoruus ei ole liipaisin. Miehet näyttävät geeninsä, terveyshistoriansa ja hedelmällisyytensä aivan kuten naisetkin, joten jonkinlaista päällekkäisyyttä on. Pitkät jalat, hyvänmuotoiset pakarat, symmetriset kasvot, kiva tukka, vahva leka, korkeat poskipäät. Nämä toimivat samoin molemmissa sukupuolissa.

Mutta ulkonäkö toimii ainoastaan yhdessä muiden liipaisimien, kuten itsevarmuuden ja hurmaavuuden, kanssa. Yksin miehien ulkonäkö ei ole liipaisin naisille. Sosiaaliset naiset eivät koe hyvännäköistä mutta epävarmaa miestä viehättäväksi. Molemman sukupuoliset psykopaatit vaikuttavan pitävän sellaisilla miehillä leikkimisestä.

Mitä piirteitä naiset suosivat miehissä?

Useimmat naiset kilpailevat ollakseen kaikista kauneimman ja nuorimman näköinen. Kilpailu voi olla hunnutettua ja tiedostamatonta. Mutta se on läsnä kaikkialla, koska siihen miehet reagoivat.

Mistä miehet kilpailevat? Mikä on se asia, jonka kerryttämiseksi ja kiinnipitämiseksi miehet taistelevat elämänsä ajan? Se ei ole nuori ulkonäkö, lukuunottamatta joitakin kummallisia miehiä. Se ei ole pitkät hiukset. Eikä suorat piirteet eivätkä pitemmät jalat.

Miehet kilpailevat vallasta ja sen edustajasta, rahasta. Aivan kuten naisten ulkonäkökilpailussa, kilpailu voi ola hunnutettua ja jopa alitajuntaista. Mutta se on kaikkalla. Työssä, urheilussa, sosiaalisessa toiminnassa. Miesvalta ottaa monia eri muotoja. Se voi olla fyysistä, älyllistä, taloudellista. Jopa miehet, jotka eivät eksplisiittisesti kilpaile, ottavat kantaa.

Miesviehättävyydessä näyttää olevan kaksi pääteemaa. Yksi on hallitsevuus: pituus, matala ääni, itsevarmuus, näkyvät karvat kasvoissa, ja kilpailullisuus. Ainakin yksi tutkimus esittää,\linkki{sdfds} että tämä nousee ja laskee naisen hedelmällisyyssyklin mukaan. Toisin sanoen, kun nainen on korkeimmillaan hedelmällisyydessä, hän hakee todennäköisemmin opportunisista seksiä. Sitten naiset suosivat ``miehekkäitä'' miehiä, ja sosiaalisesti hallitsevia miehiä hyvien partnerien tai isien sijaan.

Toinen teema on persoonallisuus, mikä näyttäytyy sellaisina asioina kuin ``älykkyys\vmq{,}'' ``hyvä huumorintaju\vmq{,}'' ja ``on kiltti\vmq{.}'' Kun naiset etsivät pitkäaikaisia kumppaneita, he keskittyvät tähän teemaan. Toisin sanoen, potentiaalisesti hyviin kumppaneihin ja isiin. Jos muotoilet tämän uusiksi muotoon ``on empaattinen ja herkkä\vmq{,}'' se on koodi sille että mies ei ole Mallory.

Miesvalta on liipaisin naisille, ainakin osan ajasta, aivan kuten naishedelmällisyys on liipaisin miehille.

Asian evolutionaarinen järki on yksinkertainen. Vaikutusvaltaisten miesten kumppanit saavat enemmän lapsenlapsia kuin heikkojen miestin vaimot. Hyvät geenit ovat hyvä lähtökohta. Niiden työntäminen tuleville sukupolville ottaa voimaa, kun vastassa on loppumaton kilpailu. Ihmiskäsittein tämä tarkoittaa valtaa.

Korjataanpa Clarkin ja Hatfieldin tutkimus. Näyttelijät olivat nuoria, viehättäviä miehiä ja naisia. Tämä itsessään luo jo valtavan vinoutuman. Parempi tutkimus käyttäisi valikoimaa erilaisia näyttelijöitä, sekä miehiä että naisia. Näyttelijät poikkeaisivat toisistaan iän, viehättävyyden ja vaikutusvaltaisuuden osalta. Sitten mittaisimme heidän suhteellisen menestyneisyyden sekailaisessa joukossa kohteita.

Kerron arvaukseni siitä, mitä tapahtuisi. Havaitsisimme, että miehet reagoivat pääasiassa nuoruuteen, sitten ulkonäköön, sitten saatavuuteen, ja sitten luonteeseen. Saatavuus on kriittistä. Mies-miestä-vastaan -väkivalta naisten vuoksi on niin merkittävää, että ``ei saatavilla'':n luulisi olevan turn-off useimmille miehille. Tämän lisäksi havaitsisimme naisten reagoivan valtaan, sitten luonteeseen, sitten saatavuuteen, ja viimeisenä ulkonäköön. Hedelmällisyyden huippuaikana naiset reagoisivat pääasiassa valtaan ja ulkonäköön.

Kysymys kuuluu: kuinka mies näyttää ``Olen vaikutusvaltainen'' -liipaisimensa? Kuinka nainen tietää, milloin mies valehtelee omasta tärkeydestänsä? Kuinka psykopaatit liioittelevat näitä liipaisimia?


Taas kerran vastaus piilottelee päivänvalossa. Jos kysyt yksinäiseltä mieheltä hänen statuksestaan, hän luultavasti yksinkertaisesti valehtelee. Miehet liioittelevat ansioitansa. He valehtelevat saavutuksistaan. He piilottavat epäonnistumisensa. Ja niin edelleen. Oletamme tätä, ja mitä ikinä mies itsestään kertookin, suhtaudumme siihen harmittomana fantasiana. \problem{kuinka suomentaa plain sight}

Miehet näyttävät vaikutusvaltansa kehonkielessänsä ja siinä, kuinka he käyttäytyvät toisia---esimerkiksi tilannetta havainnoivaa naista, yhtä tai useampaa toista miesta tai toisia naisia---kohtaan.

Kun tapaamme tuntemattomia, arvioimme heitä vaistonvaraisesti, silloinkin kun kyse on vain jalkakäytävällä vastaantulevasta satunnaisesta ohikulkijasta. Kuinka yksi tekee tietä toiselle? Huone täynnä toisilleen tuntemattomia ihmisiä järjestyy itsestään. Miehet siirtyvät ryppäisiin, hallitsevimmasta vähiten hallitsevaan, ja joka ryppäässä on yksi mies johdossa. Naiset liikkuvat erilaisiin kuvioihin miestin ympärille, tai heistä erilleen. Miehet saattavat kerääntyä yksittäisten naisten ympärille. Miehet tsekkaavat naiset. Naiset tsekkaavat miehet. Ryhmä muodostaa mielipiteen. Kaikki tapahtuu minuuteissa.

Miesvalta tarkoittaa kykyä hallita muita miehiä (ja naisia; mutta pääasiassa miehiä). Tämä voi olla hienovaraista ja epäsuoraa, ja ylettyy kauas fyysisen dominoinnin tuolle puolen. Kirjoittajana voin hallita yksinkertaisesti naputtamalla näppäimistöä. Kehonkieli ja naamatusten dominointi ovat kuitenkin hyvä paikka aloittaa.

Me ymmärrämme dominoivaa kehonkieltä hyvin. Kyse on osittain suuremmalta näyttämisestä. Tämä tulee ihmisapinaedeltäjistämme, joiden tapauksessa dominointi tarkoitti sitä, että on vahvempi. Suurin, vahvin mies johti ryhmää. Kyse on myös osittain muihin nähden ylempiarvoisesti käyttäytymisestä.

Näyttääkseen suuremmalta mies seisoo suorassa ja nostaa leukaansa. Hän seisoo jalat kauempana toisistaan. Hän ottaa enemmän tilaa, kuin olisi tarpeen. Hän käyttää käsieleitä ja työntää kyynärpäitänsä ulospäin.

Vaikuttaakseen ylempiarvoiselta, mies kontrolloi keskustelua. Hän ylläpitää katsekontaktia räpäyttämättä silmiään. Hän jättää huomiotta ne, joista hän ei ole kiinnostunut. Hän hymyilee vähemmän ja liikuttaa päätänsä vähemmän. Hän puhuu vähemmän, matalammalla ja hiljaisemmalla äänellä. Yleisesti ottaen hän jättää sosiaaliset vihjeet huomiotta. Hän tulee myöhässä, keskeyttää muita, ja on ankara, olematta kuitenkaan tyly.

Meillä on paradoksi käsillä. Nämä piirteet sekä vetävät puoleensa naisia että karkottavat heitä. Ne kertovat miehestä, joka on kykenevä hallitsemaan muita miehiä, mutta myös miehestä, joka on epäherkkä ja mahdollisesti raaka. Paradoksista on olemassa tie ulos. Naiset olettavat, että kaikki miehet valehtelevat, kunnes toisin todistetaan. Hallitsevuuden piirteet vaativat toisen liipaisimen toimiakseen. Ilman tätä toista liipaisinta mies näyttää naisen silmissä vain öykkäriltä.

Toiset liipaisimet tulevat toisilta miehiltä. Kuten kaikki liipaisimet, mitä olemme tarkastelleet, ne ovat minimalistisia ja elegantteja. Näen kaksi erityistä liipaisinta. Ensimmäinen on toiset miehet, jotka hyväksyvät dominoinnin, ja käyttäytyvät oikealla tavalla, eli näyttävät alistuvaa kehonkieltä. He ovat hiljaa, kun dominoiva mies puhuu. He hyväksyvät miehen tunkeilevan asenteen aivan kuin heidän mielestään mies olisi ansainnut oikeuden siihen.

Toinen liipaisin on dramaattisempi. Siihen tarvitaan ryhmän ulkopuolinen mies haastajaksi. Dominoiva mies joko pitää paikkansa tai häviää pelin. Naiset kokevat tällaisen voittaja-vie-kaikein draaman houkuttelevana, oli kyse sitten todellisesta elämästä, urheilusta tai viihteestä. Naiset saavat siitä saman dopamiinipotkun kuin miehet saavat katsellessaan kauniin naisen riisuutumista.

Vihätys on varmasti suhteellista. Mutta siinä missä naisen viehättävyys on yksilöllinen asia, miesvalta virtaa toisten miesten kautta. Olipa kyse kuinka rikkaasta tai itsevarmasta miehestä, yksin hän on pelkkä nolla. Miesvallassa on kyse toisista miehistä.

Samankaltainen ilmiö on nähtävissä siinä, kun naiset liipaisevat miehiä. Naisen ulkoasi sekä vetää miehiä puoleensa että hylkii heitä. Nainen, joka käyttää liikaa meikkiä ja parfyymiä, flirttaa turhan aggressiivisesti ja näyttää liikaa paljasta pintaa, sammuttaa useimmat miehet. Mutta jos hän on muiden naisten seurassa, ja he vaikuttavat pitävän hänestä, toimii tämä toisena liipaisinta. Ja useimmat miehet viihtyvät taas.

Nämä sekundääriset liipaisimet vaikuttavat kiteytyvän ajatukseen ``ei psykopaatti\vmq{.}'' Empatia ja herkkyys ovat suhteellisen helppoja väärentää, joten pinnalliset empatian merkit eivät ole liipaisimia. Tulen empatiatesteihin myöhemmin.

Nyt, kun olemme selvittäneet signaalit, katsotaan, kuinka niitä on mahdollista väärentää. Vaelteleva miespsykopaatti on yksinäinen. Hänellä ei ole todellisia ystäviä. Hän ei voi vain näyttää narsistista ``minä olen tärkeä'' -maskiansa naisille. Sen sijaan hänen täytyy vakuuttaa muille miehille, että hän on tärkeä. Hänen täytyy löytää tai rakentaa ryhmä, jota dominoida, ja hänen täytyy näyttää tulokset.

Pinnan alla toimivat mekanismit ovat hienovaraisia. En tiedä aiheesta mitään muuta tutkimusta kuin omani. Joten tämä kaikki on spekulointia ja hypoteeseja, jotka perustuvat pitkäaikaiseen havainnointiin ja analyysiin.

Sanoin miesten ja naisten eroavan toisistaan merkittävillä tavoilla. Yksi näistä tavoista on se, kuinka kommunikoimme. Kyse on syvemmästä asiasta, kuin siitä, mistä puhumme. Kyse on siitä, miksi meille alunperin ylipäänsä kehittyi kielipää. 

Miehet ja naiset molemmat vaihtavat valtaa ja tietämystä, ja rakentavat rakenteita. Mutta tässä on selvä sukupuoliero. Miehet puhuvat vaihtaakseen teknistä tietämystä ja he organisoituvat keskenään valtarakenteisiin. Naiset puhuvat vaihtaakseen sosiaalista tietämystä ja organisoituvat keskenään aivan toisenlaisiksi rakenteiksi. Kielemme heijastelee näitä kahta mallia.

Olen kutsunut näitä puhetapoja ihmisprotokolliksi.\linkki{dsfds} On kaksi pääprotokollaa: miesprotokolla ja maisprotokolla. On pienempiäkin protokollia, kuten aikuinen-lapselle. Useimmat miehet kykenevät vähintäänkin imitoimaan naisprotokollaa ja toisinpäin. Mutta se on kovaa työtä.

Mallory on miesprotokollan ekspertti. Hän voi dominoida miesryhmiä käyttäen valheiden, lupauksien ja itsevarmuuden sekoitusta. Suurta petomaista bisnestä ei voi erottaa kultista. Siinä, missä Alisa on taipuvainen epäilemään yksinäistä miestä, Bob ja hänen kaverinsa eivät ole. Tämä on erityisen totta silloin, kun yksinäinen mies lähestyy ryhmää. Niinpä Mallory tykkää tehdä taikatemppunsa ensin Bobille ja hänen kavereilleen. Tämä antaa hänelle statusta ja valtaa, jota hän voi heijastaa kohti naisia.

Tämä voi tapahtua minuuteissa. Miesprotokolla mahdollistaa silmänräpäyksessä syntyvät suhteet, jotka perustuvat pelkästään tulevaisuuden mahdollisuuksiin. ``Seuraa minua! Lupaan sinulle kultaa!'' Naisprotokolla taas on kyyninen ja varovainen. Naistenvälisten suhteiden kehittyminen kestää usein vuosia. Naisten on investoitava suhteisiinsa. Miesten täytyy vain pitää futuurit avoinna.

Tämä on valtava ero. Molemmat sukupuolet heijastavat heidän omia arvojaan ja mittauksiaan toisista. Alisa olettaa, että suhteen rakentaminen kestää kauan. Hän olettaa, että suhteet ilmaisevat kyynistä kijanpitoa menneistä faktoista. Joten hän yliarvioi suhdetta, jonka hän näkee Bobin ja Malloryn välillä. Ja samalla tavoin Bob aliarvioi Alisan suhteita.

Tästä saamme erään Malloryn klassisen metsästyskaavan. Hän ensin valloittaa Bobin ja jotkin muut miehet hymyllä ja äärimmäisellä itsevarmuudella. Hän vihjailee jonkinlaisia lupauksia tai tulevaisuuden tuomaa hyötyä. Tämä voi kestää vain minuutteja. Sitten hän esittelee tämän väliaikaisen rakenteen ohikulkevalle Alisalle. Hän näyttää, kuinka hän dominoi kokoamiansa miehiä. Alisa kokee tämän hurmaavan miehen haluttavana. Hän alkaa vastata miehelle. Kun mies esittää, että he voisivat jatkaa juttua jossakin yksityisessä paikassa, nainen hyväksyy ajatuksen. Reaktio on aivan sama, kuin miten miehet reagoivat kohotettuihin rintoihin ja terveeksi toivottaviin hymyihin.

\section{Valtapyramidit}

Miespsykopaatit kykenevät toimimaan ja toimivat henkilökohtaisella tasolla. Se on kuitenkin vain pölyä verrattuna monien organisaatioiden teollisen mittakaavan miespsykopatialle. Ennen kuin katsomme, kuinka Mallory metsästää Bobia, meidän on syytä käydä pienellä kiertoajelulla. Selostan, kuinka ihmiset organisoituvat laajassa skaalassa.

Ihmiset vaikuttavat organisoituvan kahdella erityisellä, vastakkaisella tavalla. Nämä ovat ``elävä järjestelmä'' ja ``valtapyramidi\vmq{.}''

Elävä järjestelmä on löysä ekonominen verkko sisäisesti riippumattomia toimijoita tai palasia. Nämä palaset vaihtavat resursseja kuten tietämystä, työtä tai rahaa. Nämä resurssit virtaavat järjestelmän läpi eri suuntiin omalla tahdillansa. Elävissä järjestelmissä ei ole ilmiselvää valtarakennetta. Niillä ei ole tunnistettavia omistajia eikä keskitettyä auktoriteettia. Ne eivät harrasta keskitettyä päätöksentekoa tai suunnittelua.

Valtapyramidi on eksplisiittinen rakenne, jolla on nimi, tarkoitus ja johtohierarkia. Päätökset ja suunnitelmat liikkuvat alas ja voitot liikkuvat ylös. Koko pyramidi on yläosassa istuvien omaisuutta. Kyse on selvästä hierarkiasta, jossa paikka määrittää statuksen ja status määrittää paikan.

Elävät järjestelmä ovat rajoittamattomia ja lähes näkymättömiä. Niillä ei ole markkinointijaostoa, paneeleita tai toimitusjohtajia. Ne koostuvat tuhansista, jopa miljoonista riippumattomista toimijoista. Toimijat organisoituvat itsestään mielenkiintoisimmille aloille. Elävät järjestelmät ovat paljon tuottoisampia kuin valtapyramidit. Mutta tuotot leviävät laajalle ja niiden mittaaminen on vaikeaa.

Elävät järjestelmät ovat tehokkaita. Ne ovat elämäntapamme ytimessä. Ne ruokkivat kaupunkimme ja täyttävät kaupat tuotteilla. Ne tuovat meille vaatteet, tietämyksemme, Internetin. Jokainen kaupunki on elävä järjestelmä tai vankila. Jokainen talous on elävä järjestelmä, tai sitten se on suunniteltu epäonnistuminen.

Elävissä järjestelmissä tapahtuvat vaihtokaupat riippuvat avoimista sopimuksista. Elävät järjestelmät tunnistavat ja rankaisevat huijareita, käyttäen yksinkertaista vapaan valinnan menetelmää. Ne tarvitsevat jonkinlaista sääntelyä, eli lakia ja sen toimeenpanoa. Luonnolliset elävät järjestelmät käyttävät fysiikan, kemian ja biologian lakeja. Keinotekoiset elävät järjestelmät tuottavat omat auktoriteettinsa, yleensä evoluution ja kilpailun kautta.

Elävät järjestelmät ovat reiluja (tai eettisiä) jokaista osallistujaa kohtaan puhtaasti tehokkuuden tarpeen johdosta. Ne siis ovat taipuvaisia kohtelemaan syrjintää ja huijaamista epätehokkuuksina, ongelmina, jotka on syytä ratkaista. Elävät järjestelmät tekevät kokeita jatkuvasti pienillä ratkaisuilla uusiin ongelmiin. Ne hautaavat epäonnistumiset ja edistät menestystä. Tämän takia ne ovat hyviä sopeutumaan muutokseen. Ne ovat sitkeitä ja selviytyvät, kunnes joku katastrofaalinen ulkopuolinen tapahtuma rikkoo ne. Kaupungit selviytyvät imperiumeista, jos niitä vain ei tuhota maan tasalle.

Valtapyramidit ovat erikoistuneet hankkimaan asiakkaita, toimittajia ja työntekijöitä antamaan enemmän vähemmällä. Ne toimivat loismaisesti. Jos käännät valtapyramidin ylösalaisin, se näyttää suppilolta. Äärimmäisiä esimerkkejä ovat suuret vähittäiskaupat, jotka maksavat toimittajillensa niin vähän kuin mahdollista ja tekevät miljardien edestä voittoja.

Massojen pitäminen paikallaan samalla kun niistä imee resursseja ottaa työtä. Valtapyramidit tekevät sen käyttäen voiman, lahjuksien ja uhkauksien sekoitusta. Ne vaativat fyysistä läsnäoloa. Ne lupaavat kuukausipalkkoja ja bonuksia. Kyvyttömyys myötäillä tarkoittaa potkuja. Kyse on jatkuvan matalan tason sisäisen väkivallan muodosta. Valtapyramidit heijastavat väkivaltaa myös ulkopuolisia uhkia kohti. Ne käyttävät voimaa poistaakseen kilpailijoita ja saavuttaakseen päämääriänsä. Ne ovat pragmaattisia ja säälimättömiä.

Eivät kaikki isot bisnekset ole puhtaita valtapyramideja; useimmat ovat sekoituksia, joissa on joitakin elävän järjestelmän aspekteja. En ole kuitenkaan ikinä kuullut bisneksen pahoittelevan kilpailijansa voittamista. Enkä ole koskaan kuullut valtion pahoittelevan sodan voittamista. Selviytyminen määrittelee moraalin, lyhyellä tähtäimellä. Pitkällä tähtäimellä moraali määrittelee selviytymisen. Niinpä elävät järjestelmät tuppaavat päihittämään valtapyramidit.

Näemme valtapyramideja useimmiten yritysmaailmassa, hallituksissa ja organisoituneissa uskonnoissa. Kun nämä kolme sekoittuvat, meillä on käsissä fasismia ja kansanmurhia. Tätä tapahtuu, kun yhteiskunta on liian heikko tai naiivi pistääkseen kampoihin. Useammin valtapyramidit ovat piittaamattomia, sen sijaan että ne olisivat suoraviivaisen tuhoisia.

Kun valtapyramidi tuottaa tuotteita, se tähtää surkeimpaan mahdolliseen laatuun ja korkeimpaan hintaan. Joskus se saattaa myrkyttää ja addiktoida asiakkaansa voittojen vuoksi. Esimerkkinä toimii elintarvikeala, joka keskittyy sokeriin. Valtapyramidit eivät kuuntele markkinoita. Sen sijaan ne yrittävät pakottaa ihmiset vastaanottamaan tuotteensa, käyttäen rajua markkinointia. Voisi sanoa, että valtapyramideilla ei ole empatiakykyä.

Valtapyramidit ovat mestareita harhaanjohtamisessa ja todellisen luonteen peittämisessä. Ne markkinoivat itseään ``eettisinä\vmq{,}'' ``positiivisina\vmq{,}'' ``hyvinä'' ja ``hauskoina\vmq{.}'' Ne käyttävät miljardeja brändäykseen ja imagoon, luoden narratiiveja myydäkseen tuotteitaan. Ihmiset uskovat näihin narratiiveihin ja investoivat niihin raskaasti.

Mitä surkeampi tuote, sitä raskaampaa on mainonta. Coca-Cola. Microsoft. Kraft. Heinz, USA! Valtapyramidit kommunikoivat käyttäen totuuden elementtien ympärille kiedottuja valheita. Niiden ydinarvot ovat voitot ja selviytyminen, ei enempää, ei vähempää.

Huolimatta niiden keskittymisestä selviytymiseen, valtapyramidit ovat huonoja sopeutumaan muutokseen. Ajan kanssa ne tulevat yhä riippuvaisemmiksi valheista ja voimasta, kun maailma niiden ympärillä muuttuu. Niistä tulee haurauta ja alttiita nopealle, katastrofaaliselle romahdukselle. Nokia, Blackberry, Neuvostoliitto.

Empatian puute, sydämmettömyys ja petomaisuus, taipumus käyttää hyväksi{\ldots} Monet suuret yritykset, uskonnot, ja tietyt hallituksen maut ovat psykopaattisia valtapyramideja.

Hyvän työkokemuksen omaava kaverini kävi suuren teknologiafirman työhaastattelussa. Prosessi jätti hänet hämmentyneeseen ja nöyryytettyyn mielentilaan. ``Miksi minun täytyy todistella itseäni nuorelle työhaastattelijalle?'' hän kysyi minulta. ``Miksi he eivät vain katsoisi työtäni. Se on kaikki Internetissä\vmq{.}'' Pohdimme asiaa. Sanoin, että ehkä nöyryytys on sellaisen työhaastattelun päämäärä. Jos hyväksyt sen, tulet hyväksymään paljon pahempaakin, vastikkeena mehukkaalle palkallesi. Myötäily on koe.

Edustavatko nämä kaksi mallia miesten ja naisten tapaa työskennellä? Se on houkutteleva päähänpisto. Miesprotokolla toimii miesvallan ympärillä, ja naisprotokolla sosiaalisen tietämyksen ympärillä. Mutta olisi typerää luonnehtia ``miehekkyyttä'' pahuudeksi. Lajimme ei kehittänyt sukupuolieroja jakaakseen meidät. Teimme sen, jotta voisimme työskennellä yhdessä tehokkaammin. Sekä miehet että naiset ovat onnellisempia ja tehokkaampia elävissä järjestelmissä. Ja valtapyramidit hyväksikäyttävät sekä miehiä että naisia.

Mielestäni totuus on hienostuneempi. Elävät järjestelmät tarvitsevat sekä tietämyksen että toiminnan virtaa. Mies- ja naisprotokolla toimivat yhdessä, ratkaisten eri osia suuremmasta pulmasta. Valtapyramidit ovat vääristymä, miespsykopaattien rakennelma. Ne ovat massateollisuuden ja urbanisaation hetkellinen hedelmä. Ne ovat niin laajalle levinneitä, että pidämme niitä itsestäänselvyyksinä. Uskon, että ne ovat kuitenkin vahingollisia ja antisosiaalisia, ja niiden kohtalo on hiipua hitaasti pois.

Mitä miespsykopaatit yli kaiken haluavat on toisten ihmisten hallinta. Kyse on harvoin rahasta. Kuten Frank Underwood toteaa Netflix-sarjassa \emph{House of Cards}:
\begin{quotation}
\noindent Hän valitsi vallan sijaan rahan. Mitä ajan haaskausta. Virhe, jonka melkein jokainen tässä kaupungissa tekee. Raha on Sarasotan McMansion, joka alkaa hajota kappaleiksi kymmenessä vuodessa. Valta on vanha kivirakennus, joka pysyy pystyssä vuosisatoja. En voi arvostamaan ihmistä, joka ei näe eroa.
\end{quotation}

\section{Miehet metsästämässä miehiä}

Valtapyramidi rekrytoi nuoria miehiä helposti. Kuten kaikkien vaistojemme kohdalla, näemme syyn evolutiivisessa historiassamme. Ihmismiehet muodostivat ryhmiä metsästäkseen suuria villieläimiä. Kyse on korkean riskin ja korkean hyödyn ryhmätyöskentelystä. Vanhemmat miehet jakavat tietämyksensä, ja nuoremmat miehet jakavat fyysiset kykynsä ja aikansa. Nuoret miehet, jotka reagoivat vanhempien miesten ``seuraa minua''-viestiin tulevat todennäköisemmin takaisin mukanaan lihaa. Tämä näkyy suoraan lisääntymismenestyksessä.

Tulevaisuuden palkintojen ``Seuraa minua''-lupaus vanhemmalta mieheltä on liipaisin. Mitä suurempi lupaus, sitä suurempi reaktio. Sen ei tarvitse olla looginen tai järkevä. Itseasiassa järjettömät lupaukset ovat usein \emph{houkuttelevampia} kuin järjelliset lupaukse. Järkevä ehdotus vaatii kovaa työtä ja kärsivällisyyttä. Järjetön ohdotus ei vaadi muuta kuin epäuskon suyrjään siirtämistä. ``Tiedän jotakin riskisijoittajia ja he investoivat miljoonia ideaasi!'' Tähän on vaikea sanoa ei.

On olemassa joukku ``seuraa minua''-liipaisimia, jotka antavat yhdelle miehelle mahdollisuuden ottaa toisia, jopa kokonaisen ryhmän, hallintaansa. Liipaisimet toimivat parhaiten nuoriin, alle nelikymppisiin miehiin, joilla ei ole lapsia. Tässä vat ne, jotka tiedän:
\begin{description}
\item[Yksinäinen lähestyminen.] Tämä kertoo itsevarmuudesta ja purka ryhmän luonnollisen puolustusjärjestelmän. Yksittäinen mies ei voi ola fyysinen uhka ryhmälle.
\item[Dominoivan kehonkielen näyttäminen,] erityisesti kohti senhetkistä dominoivaa miestä. Jos dominoiva mies ei tastele vastaan, hän on astunut alas, ainakin hetkeksi.
\item[Vanhemmalta ja viisaammalta vaikuttaminen.] Tämä liipaisee ``viisas vanha mies''-reaktion nuorissa miehissä. Viisaus on kallisarvoista, kunhan se vain on relevanttia.
\item[Keskustelun kontrolloiminen.] Dominoivat miehet ylläpitävät katsekontaktia, keskittyvät korkeamman statuksen miehiin, ja hymyilevät vähemmän. He puhuvat vähemmän, matalammalla ja hiljaisemmalla äänellä. Tämä pakottaa toiset kiinnittämään lähestä huomiota heihin. Tämä liipaisee ``tuo kaveri on dominoiva''-reaktion.
\item[Lupausten tekeminen potentiaalisista voitoista.] Nämä voivat olla suuria ja niin vaikeita kuin mahdollista. Mitä hullumpi, sen parempi. Nuoren miehen biologia tekee hänestä luonnollisen uhkapelaajan. Valtava potentiaalinen voitto liipaisee uhkapelireaktion, riippumatta siitä, kuinka pieni voiton mahdollisuus on.
\item[Yhteiseen viholliseen vetoaminen.] Tämä liipaisee puolustusreaktion. Se antaa ryhmälle keskittymistä ja energiaa, jonka ulkopuolinen voi omistaa ja jota se voi ohjata.
\item[Toiminnan vaatiminen ja suunnitelman ehdottaminen.] Tämä liipaisee ``seuraa minua''-reaktion. Jos valtaosa ryhmästä reagoi, pääsee ulkopuolinen johtoasemaan.
\item[Sisäisiä vihollisia vastaan hyökkääminen,] erityisesti vastaan vanhaa johtoa. Tämä liipaisee paranoia- ja kostoreaktiot. Jos uudella johtajalla on matkassa onnea, kykenee hän puhdistamaan hierarkian kaikista potentiaalisista uhista.
\end{description}
Kaikki kauniit, flirttailevat naiset eivät ole psykopaatteja, ja kaikki miehet, jotka käyttävät näitä tekniikoita, eivät ole psykopaatteja. Ero piilee lopputuloksissa. Näemmekö harhautusta ja hyväksikäyttöä vai rehellisyyttä ja yhteistä hyötymistä? Palavatko ihmiset loppuun ja masentuvat, vai tulevatko he onnellisemmiksi ja itsenäisemmiksi? Psykopatia piilottelee taidokkaasti, mutta silloin kun se organisoi ihmisiä omia tarkoitusperiänsä varten, vauriot kyllä jossakin vaiheessa tulevat esille.

Nämä liipaisimet kehittyivät pätevistä syistä. Kyky organisoitua karismaattisten johtajien ympärille pelasti edeltäjämme monta kertaa. Ja me harppaamme reagoidessamme. Jos liipaisimet ylipäänsä vaikuttavat meihin, biologinen määräys on olla ensimmäinen????.

Kun reaktio iskee, se kasvaa sopiakseen liipaisimeen. Luonnollinen johtaja kykenee kasvattaman stimulia johonkin pisteeseen, mutta ei siitä yli. Mallory kasvattaa ja kasvattaa, paljon yli normaalin ja tarpeellisen. Vaikutus rauhoittuu jossain vaiheessa. Kuitenkin tämä supernormaali stimulishokki jättää jäljen, joka kestää vuosia.

Mallory ohjaa ryhmää kohti itsetuhoa ja tyhjentää samalla kaapit. Kun hän sanoo: ``Seuraa minua!'' koukuttuu Bob tilanteeseen, jota Bob ei voi kontrolloida. Bob kokee, että hän ei voi lähteä pettämättä hänen ihanteitansa, ystäviänsä ja hänen omia investointejansa.

Olen nähnyt tämän satoja kertoja, monesti katastrofisten vaikutusten kanssa. Se saa aikaan loppuunpalamista: äärimmäistä uupumusta, inhotusta ja masennusta. Tänään tunnistamme tämän klassisena psykopaattisen suhteen lopputuloksena.

Mikä on selvin merkki siitä, kumpaa tyyppiä organisaatio on? Oman kokemukseni mukaan se on riippumattoman tiimiin koko. ``Riippumattomalla'' tarkoitan vapautta organisoitua ja työskennellä oman halun mukaan. Tusina tai vähemmän indikoi elävää järjestelmää. Enemmän kuin kaksitoista on todennäköisesti valtapyramidi tai sen osa.

Hyvä teoria mahdollistaa uusien deduktioiden ja päätelmien tekemisen. Kokeillaanpa muutamaa:
\begin{itemize}
\item Miksi valtapyramideissa on niin vähän naisia? Johtuuko se seksismistä ja syrjinnästä? Vaikka seksismi ja syrjintä rehottavat, en usko että ne selittävät tätä asiaa. Miehet tykkäävät työskennellä naisten kanssa ainakin yleensä. Syy on osittain siinä, että valtapyramidit eivät sovi yhteen kokopäiväisen vanhemmuuden kanssa, eivätkä varsinkaan äitiyden kanssa. Syy on osittain myös siinä, että naiset tuppaavat jättämään huomiotta ``seuraa minua''-liipaisimen, joka saa miehet uhraamaan perheaikansa.
\item Kykenevätkö psykopaattiset naiset nousemaan valtapyramideissa? Tämä vaikuttaa epätodennäköiseltä. Useimmat naispsykopaatit halveksivat miesten massavallan konseptia eivätkä näytä puhuvan miesprotokollan kieltä. Sekä mies- että naispsykopaateilta puuttuvat kyvyt valtapyramidien rakentamiseksi, ja niissä menestymiseksi, lukuun ottamatta ylöspäin suuntautuvaa valloitusta. Naispsykopaatit ovat taipuvaisia sihtaamaan vaikutusvaltaisia miehiä. Useimmat valtapyramideissa menestyvät miehet eivät ole psykopaatteja, ja ovat siten haavoittuvaisia.
\item Useimmat meistä pelkäävät valtapyramideja eivätkä luota niihin hyvästä syystä. Sellaiset organisaatiot tekevät miljardit ihmiset onnettomiksi, siitäkin huolimatta, että nämä vaikutukset ovat piilossa paljon suurempien elävien järjestelmien menestystarinoiden alla. Se ei tarkoita sitä, että kaikki bisnekset olisivat myrkyllisiä, ei sinne päinkään. Se ei tarkoita sitä, että kaikesta pitäisi syyttää vapaata markkinataloutta. Todelliset vapaat markkinat toimivat elävien järjestelmien konepeltien alla. Todelliset vapaat markkinat ovat valtapyramidien vihollinen.
\item Meillä on nyt käsillä evolutiivinen selitys ennenaikaiselle harmaantumiselle kaljuuntumiselle ja kaljuuntumiselle??. Harmaantuminen ja kaljuuntuminen kertovat kypsyydestä ja iästä. Ne liipaisevat ``viisas vanha mies''-reaktion nuorissa miehissä ja pyrkavat kilpailulliset vaistot. Kun mies harmaantuu tai kaljuuntuu ennen muita, hän jäljittelee vanhan iän signaaleja. Tämä voi antaa hänelle etulyöntiaseman, jos hän on tarpeeksi fiksu käyttääkseen sitä.
\end{itemize}

\section{Naiset metsästävät naisia}

Naispsykopaatit metsästävät muita naisia. Tämä on selvää. Kysymys on, millä tavoin, ei, että tapahtuuko sitä vai ei. Kuten mies-mies-kuviota, voi tätäkin ola vaikea nähdä. Se voi olla lähes kryptistä. Sokeutemme sukupuolen vinottamalle käytökselle saattaa tehdä tästä tutkimuksesta vaikeampaa kuin mitä sen pitäisi olla. Joka kerta, kuin kirjoitan ``miehet tekevät \(X\)'' tai ``naiset tekevät \(Y\)\vmq{,}`` se tekee kipeää. Ja silti nämä yleistykset ovat elintärkeä työkalu totuuden lähestymisessä.

Miesten suhteet tuppaavat olemaan halpoja, kovaäänisiä ja julkisia. Valtapyramidit näyttävät heijastavan miesvaltaa yli kokonaisten teollisuusalojen, talouksien ja maiden.

Naisten suhteet ovat puolestaan salailevia ja syvällisiä. Ne kuljettavat elintärkeää tietämystä ihmisistä ja tapahtumista. Ennenkaikkea ne ovat tärkeä puolustus Mallorya vastaan, olipa hänen sukupuolensa mikä hyvänsä. Jotta oppisimme, kuinka nais-Mallory metsästää Alisaa, meidän täytyy dekoodata naisprotokolla. Sitten meidän täytyy selvittää, kuinka sitä voi huijata.

Naisprotokolla esittää keskustelun kahden jo ennestään toisensa tuntevan naisen välillä. Nämä kaksi keskustelevat ihmisistä ja tapahtumista. Keskustelu ei ole mitä sattuu. Se on vaihtokauppaa. Dialogi jatkuu, kunnes kummallakin naisella on se, mitä he haluavat, ja sitten se päättyy.

Tämä on jokseenkin helppo nähdä. Kaksi naista, jotka tuntevat toisensa, ja jotka ovat olleet jonkin aikaa erossa toisistaan, istuvat alas ja juttelevat. He puhuvat ja kuuntelevat vuorottain, eikä kumpikaan ole dominoiva tai alistuva. Jutteluhetken jälkeen he siirtävät keskittymisensä toisistaan takaisin muuhun maailmaan. Olisi houkuttelevaa kutsua tätä ``juoruprotokollaksi\vmq{.}'' Mutta on tarkempaa kutsua sitä ``sukimisprotokollaksi'' (engl. \emph{grooming protocol}). Se on intiimi, mutta se ei ole seksuaalinen. Oman näkemykseni mukaan kyse on ihmisten versiosta siitä sukimiskäytöksestä, mitä muut kädelliset harjoittavat.

Myös miehet käyttävät sukimisprotokollaa suhteidensa syventämiseen. Se on kuitenkin ohutta verrattuna naisten versioon. Saippuaoopperioiden tuottajat tietävät tämän. Se on naismieli, joka on pakkomielteisen kiinostunut sosiaalisista juonista kertovista tarinoista. Miesten sukimisprotokolla ei ole paljoa enempää, kuin ``Moi, sulla menee varmaan ihan jees?'' jonka päälle juodaan pari kaljaa neutraaleissa olosuhteissa. Miehet vaihtavat palveluksia, mutta se on marginaalista.

Naisten sukimisprotokolla on keskeinen naisen identiteetin ja vallan kannalta. Vaikutusvaltaisella naisella on monia suhteita toisten naisten kanssa ja hän saa arvokasta tietoa aikaisessa vaiheessa. Arvokas tieto on oikea-aikaista, tarkkaa, salaista ja yksityiskohtaista. Heikolla naisella on vähän suhteita ja hänen tietämyksensä sosiaalisesta maailmasta on huonolla tolalla---siis epätarkkaa, hyvin tunnettua ja epätäydellistä.

Sukimisprotokollalla on kolme päätarkoitusta, mitkä kaikki toimivat samaan aikaan. Ensinnäkin, kuten minkä tahansa lajin tapauksessa, sukiminen luo luottamusta kahden yksilön välille. Toiseksi, protokolla levittää tarkkaa tietoa ihmisistä ja tapahtumista läpi ihmisten yhteiskunnan. Viimeiseksi, se tunnistaa ja rankaisee huijareita.

Kaikista aiheista mistä naiset tykkäävät puhua on seksuaalinen uskottomuus ykkösenä. Kyse ei ole niin yksinkertaisesta asiasta, kuin että ``huonot uutiset liikkuvat nopeasti\vmq{.}'' Seksuaalinen uskottomuus ei ole dataa. Pettäminen on, koska toistuva huono seksuaalinen käytös on psykopatian ykköstuntomerkkejä.

Mallory valehtelee ja liioittelee sukiessa. Se antaa hänen pysyä dominoivana suhteessa. Hänen tarpeensa kontrolloida narratiivia on punainen lippu, jos sen huomaa. Se on paremmin näkyvillä, kuin hänen tarjoamansa datan huono laatu. Mallory on uhri, satutettu ja välittämisen tarpeessa. Hän raportoi hänen mieskumppaninsa viimeisimmät kauheat teot. Hän anelee apua ja tukea. Hän imartelee ja hurmaa.

Vastapainoksi Mallory saa arvokasta tietoa muista ihmisistä. Alisa on eksyksissä. Hän saa jotakin, joka aluksi tuntuu arvokkaalta ja syvälliseltä ystävyydeltä. Se on kuitenkin tyhjä, ja ajan kanssa yhä hyväksikäyttävämpi.

Nämä näyttävät olevan pääliipaisimet:
\begin{description}
\item[Dramaattinen tarinankerronta.] Brasilialaisen saippuaoopperan draama. Hahmot ovat kauniita tai pahoja tai molempia. He ovat väkivaltaisia ja emotionaalisia, ylpeitä ja kovaäänisiä. Tarinat ovat epätosia ja loputtomia. Alisasta tuntuu kuin hän olisi viisivuotias ja hän olisi kuulemassa uskomatonta iltasatua.
\item[Avuttoman uhrin näytteleminen.] Syyllinen on kumppani, työnantaja tai viranomaiset. Rikokset ovat uskottomuus, väkivalta ja varkaus. ``Hän hakkasi minut ja lapseni, vei perheen rahat ja käytti sen huoriin\vmq{.}'' Alisa kokee olevansa vanhempi sisko, pakotettu tarjoamaan neuvoja ja apua.
\item[Kuulijan imarteleminen.] Tämä tarkoittaa kohteliaisuuksia, huomiota syntymäpäiviin ja henkilökohtaisiin tapahtumiin ja ylettömiä määriä huomiota. Tämä on yksi muoto ``rakkauspommitusta\vmq{,}'' jota tutkailen myöhemmin seuraavassa luvussa. Alisa kokee olevansa tärkeä, arvokas ja rakastettu.
\item[Toisten salaisuuksien paljastaminen.] Nämä ovat negatiivisia, intiimejä, yksityiskohtaisia ja usein keksittyjä. Kuulijasta tuntuu, kuin hän saisi harvinaista ja arvokasta tietämystä. Hän kokee olevansa vaikutusvaltainen. Mallory käyttää tätä erottamaan Alisan hänen ystävistään ja kollegoistaan.
\item[Äärimmäinen, kouriintuntuva vilpittömyys.] Mallory valehtelee usein mistä tahansa. Mutta hän ei näytä minkäänlaista stressireaktiota tai epäröintiä valehdellessaan. Hän vikuttaa olevan syvällisen vilpitön hänen äänensä, ilmeidensä ja kehonkielensä perusteella. Alisan reaktio on yliarvostaa kaikkea, mitä Mallory sanoo. Se ei ole pelkästään totta, se on hypertotta. Mitä kummallisempi Malloryn vale on, sitä todemmalta se tuntuu Alisasta.
\end{description}
Kuinka Alice vastaa tällaisiin liipaisimiin, jos hän ei erää ja lähde karkuun vihastuneena? Yleensä hän avautuu ja kertoo kaikki salaisuutensa. Hän kohtele Malloryä luotettavana BFF:nä. Alisa esittelee Malloryn muille kavereilleen ja kytkee hänet sosiaalisiin aktiviteetteihin. Todellisuus iskee vasta vuosia myöhemmin. Näiden lävistävien valheiden peruuttaminen vaatii monta vastaääntä. Sitten, kun Alisa alkaa kyseenalaistaa suhdetta, voivat vauriot olla syviä. Jos hän kykenee, hän postuu suhteesta häpeissään, eikä hän puhu Mallorystä enää koskaan.

\section{Katso, olen isäsi}

Olen käynyt läpi, kuinka Mallory metsästää toisia aikuisia. Yleisesti ottaen aikuisten metsästäminen on reilua peliä, ja aikuisten voi olettaa kykenevän puolustamaan itseään. Viranomaisilla ja suurella yleisöllä ei ole juurikaan sympatiaa aikuisia kohtaan. Kun lakia rikotaan, saattavat poliisi ja poikeusistuimet astua esiin. On olemassa kaksi yleistä tapausta, jotka herättävät enemmän vihaa ja inhoa. Ne ovat, kun Mallory vaanii nuoria tai vanhoja ihmisiä.

Tarkastaan ensin nuorten ihmisten tapausta. Näemme selviä, toistuvia haavoittuvuuden kuvioita. Uskon, että nämä sekä houkuttelevat että synnyttävät psykopaatteja.

Homma lähtee lapsista, jotka ovat leikkautuneet irti perheistänsä etäisyyden, eristämisen tai hylkäämisen johdosta. Vakaissa yhteiskunnissa orvot ja nuoret rikolliste kasvatetaan taloissa. Talousromahduksen murjomassa yhteiskunnassa nuoret juoksevat pakoon ja ryhtyvät katulapsiksi. Sodassa perheet saattavat jakautua ja nuoret päätyä pakolaisleireille.

Sitten näemme, kun Mallory astuu sisään ja alkaa rakentamaan hyväksikäyttäviä verkostoja. Hän saattaa esiintyä avustustyöntekijänä, uskonnollisena organisaationa tai nuorisotyöntekijänä. Tai hän saattaa odottaa kulkupisteillä odottamassa uusia saapujia, poimien kandidaatteja. ``Hei, näytät nälkäiseltä, haluaisitko jotakin purtavaa?''

Liipaisin on kuuluminen perheeseen. Nuoret ihmiset kaukana sukulaisistaan kokevat yksinäisyyttä ja turvattomuutta. He reagoivat aikuisiin, jotka käyttäytyvät vanhemman tavoin. Kuten muidenkin liipaisimien kanssa, Mallory voi liioitella käytöstänsä ja saada aikaan vahvempia reaktioita. Aikuinen näyttää itsevarmuutta ja antaa lisää väärennettyä huomiota. Nuori reagoi luottamalla voimakkaammin. Mallory voi venyttää tätä pidemmälle ja nopeammin kuin sosiaalinen ihminen.

Nuorista ihmisistä käydään valtavan suurta maailmanlaajuista kauppaa. Joskus se kutsuu itseään ``kulttuurilliseksi'' tai ``urheilulliseksi'' vaihdoksi. Nuoret tytöt Guatemalasta, jotka kuvittelevat ryhtyvänsä tanssijoiksi. Nuoret miehet Etelä-Afrikasta, jotka unelmoivat tulevaisuudesta eurooppalaisessa jalkapallossa.

Joskus kyse on epätoivoisista vanhemmista, jotka lähettävät lapsensa kohti ``parempaa tulevaisuutta\vmq{.}'' He maksavat välittäjille, jotta he veisivät heidän lapset Eurooppaan tai Amerikkan. Numeroista emme tiedä. Tätä kaupankäyntiä ei dokumentoida. Miljoona vuodessa? Kymmenen miljoonaa? Kukaan ei tiedä. Lapset vain katoavat.

Joskus se on räikeää orjakauppaa.\linkki{dsfds} Välittäjät matkustavat köyhiin kyliin, ostaen tai kidnapaten nuoria poikia ja tyttöjä. He siirtävät nämä lapset kauas pois ja laittavat heidät työskentelemään kodeissa, tehtaissa ja bordelleissa.

Mikä ikinä onkaan syynä etäisyyteen rakastavasta ja suojelevasta suvusta, tarkoittaa se haavoittuvuutta. Haavoittuvaiset lapset houkuttelevat aina Malloryä. Hän näkee raakamateriaalia jota hän voisi omistaa, muokata, käyttää ja myydä. Bobin ei pidä kuolla, hänen ei pidä juosta karkuun, ja hänestä pitää ola hyötyä Mallorylle. Tämä rajoittaa Malloryn mielenkiintoa.

Sitten Mallory rakentaa kauppaverkoston kaltaistensa kanssa. Hän aloittaa liikuttamaan nuoria ihmisiä ylös ja alas tässä verkossa. 
Hän erikoistuu ostamiseen. Tai kenties valitsemaan ja kouluttamaan lapsia erilaisia rooleja varten. Tai siirtämään heitä yli rajoijen, Eurooppaan ja Amerikkaan, missä heidän arvonsa on korkeampi.

Lasten salakuljettaminen rajojen yli on edelleen helppoa ja halpaa. Täytyy vain tietää, kuinka se tehdään. Väärennetty passi maksaa 500--2000 euroa riippuen maasta. Se on oikea lapsen passi, jossa on uhrin kuva. Tummat lapset näyttävät kaikki samalta, eikö? \emph{Päivitys: ainakin Belgiassa tämä porsaanreikä suljettiin vuonna 2015, kun alaikäisten passinhaltijoiden tarkastamista parannettiin.}

Ja sitten Mallory suodattaa porukasta nuoria potentiaalisia psykopaatteja. Hän valmentaa heitä, käyttäen muita lapsia harjoitusmateriaalina. Hän ylentää heitä ja tekee heistä oikeita käsiänsä.

Tämä lapsikauppa on vanha, häpeän vartioima ongelma, jonka ratkaiseminen on ollut vaikeaa. Ensin halveksitaan hikipajaa ja sen jälkeen pidetään päällä t-paitaa. Hyvin usein se piilottelee perheissä, perinteen ja rasismin kerrosten alla.

Kuinka ratkaista tämä onelma? Voimme toivoa vähentävämme sotia, vakauttavamme talouksia ja vahvistavamme perheitä. Voimme kasvattaa tietämystä hyväksikäytön mekanismeista. Voimme yrittää pitää pedot poissa haavoittuvaisista lapsista. Mutta emme voi poistaa psykopaatteja, tai kytkeä heidän petomaista luonnettaan pois päältä. Kuka vahtisi vahtijoita?

Näyttää siltä, että on olemassa toinenkin vastas, minkä uskon ilmaantuvan pikku hiljaa. Malloryn strategiassa on yksi heikkous: nuoren Alisan ja Bobin täytyy olla yksin, kaukana avusta. Ongelman ydin on eristys, mikä tarkoittaa sitä, että Malloryn väärennetylle vanhemmuuden maskille ei ole tarjolla vaihtoehtoa. Jos sinulla ei ole ketään, kelle puhua, Mallory näyttää ystävältä.

Meidän täytyy antaa lapsille työkaluja luoda heidän omia sosiaalisia verkostoja, kasvotusten tai Internetissä. Opettaa lapsille, kuinka pyytää apua, vertaisilta ja muilta.

Kyky pitää yhteyttä toisiin ihmisiin Internetin välityksellä, käyttäen henkilökohtaista laitetta, on elintärkeää. Se on yhtä tärkeää kuin puhdas vesi, koulutus ja pääsy terveydenhuoltoon. Jonain päivän teknologia on melkein ilmaista ja saatavilla jokaiselle planeetan lapselle.

\section{Hyvät kuuntelijat}

Elämän toissessa päässä on vanhuus. Tarkastellaan vanhuksia, joilla on omaisuutta. Voisi kuvitella, että mitä vanhemmiksi tulemme, sitä enemmän kykenemme vastustamaan huijareita. Mutta ei se ole niin. Vanhusten riisuminen omaisuudesta on melkein teollisuudenala. Se ei johdu vanhusten dementiasta, eikä siitä, että he olisivat keskimääräistä typerämpiä. Se johtuu siitä, että psykopaatit ovat hyviä tässä. Yksinäisyys tarkoittaa haavoittuvuutta.

Suurperheitä ei ole enää ainakaan useimmissa länsimaissa. Tämän takia moni vanhus jää yksin\linkki{sdfds} kotiinsa tai vanhainkotiin. Heidän lapsensa ovat aikuisia, joilla on omat perheet jossain kaukana. Kymmenet vuodet talouskasvua tarkoittaa sitä, että monella vanhuksella on omaisuutta. Tämä sukupolvi tarjoaa tuottoisan kohteen psykopaateille. Ja Mallory tähtää sitä huolellisesti ja tarkasti.

Tähän liittyy useamman tyyppisiä hyökkäyksiä, joita olen havainnut, ja jotka selostan:
\begin{description}
\item[Avulias neuvonantaja.] Mallory houkuttelee vanhemman Alisan uhkapelaamaan. Hän ottaa perheelle työskentelevän taloudellisen neuvonantajan roolin. Hän käyttää lasten ahneutta heidän äitiänsä vastaan. Ehkä hän ehdottaa, että Alisa ottaa lisää velkaa, käyttäen hänen taloaan panttina. Mallory saa hyvän komission. Lapset saavat rahaa. Alise huomaa myöhemmin olevansa kykenemätön maksamaan velkaa. Perhe menee vararikkoon. Liipaisimet ovat perheen paine ja ahneus.
\item[Avuton muukalainen.] Mallory kysyy vanhemmalta Bobilta rahaa. Mikä tahansa tekosyy päästää hänet olohuoneeseen. Pikkuinen juttuhetki luottamuksen saamiseksi. Sitten jotakin, vaikkapa: ``Nyky-yhteiskunnan ongelma on ihmisten itsekkyys. Kukaan ei enää välitä mistään. Äitinikin on kuolemassa syöpään, ja pankkimme yrittää potkia meidät ulos talostamme!'' Bob saataa kysyä, josko hän voisi auttaa. Mallory kieltäytyy suoraan, kyynelten valuessa hänen silmistään. Bob vaatii! Mallory kieltäytyy taas, sanoen, että hän kyllä löytää jonkun tavan. Bobin ehdotus loukkaa häntä. Mutta Bob on peräänantamaton, ja Mallory poistuu paikalta mukanaan kirjekuori täynnä rahaa. Tässä syyllisyydentunto on liipaisin.
\item[Vanhemman auktoriteetti.] Mallory saa täyden kontrollin vanhempaan Alisaan. Kontrolli on emotionaalista, fyysistä ja lopulta taloudellista. En sano, että kaikki yksityiset hoivakodit olisivat tällaisia. Vain tietty osa niistä. Painostavat Alisan toimimaan ja tuntemaan kuin lapsi. Hän vastaa auktoriteettiin, joka sanoo: ``Allekirjoita tämä dokumentti, kiitos\vmq{.}''
\item[Korvikelapsi.] Mallory käyttäytyy kuin vanhemman Bobin lapsi. Hän löytää tavan viettää aikaa Bobin kanssa. Hän kuuntelee, kysyy viattomia kysymyksiä. Vain olemalla huomaavainen ja alistuva hän liipaisee vanhempimaisen reaktion Bobissa. Sitten hän tulee huoliensa ja ongelmiensa kanssa. Bob yrittää ratkaista niitä. Mitä suurempi ongelma, sitä enemmän Bob auttaa.
\end{description}

\section{Arvaa, kuka tulee päivälliselle?}

On vielä yksi areena, jossa Mallory metsästää. Se on itse perhe. Perhe-elämä ei tuo usein sisään muukalaisia. Se kuitenkin luo hyvää suojaa. Perheet sietävät merkittävän määrän vallan epätasapainoa. Mallory voi saada aikaan erikoislaautisia vaurioita perheeseen sisältä päin. Kaikki tämä voi olla näkymättömissä ulkopuolisille, jopa aktiivisille havaitsijoille.

Voimme rikkoa tämän kahteen päätapaukseen. Joissakin tapauksissa Mallory hallitsee perhettä jo valmiiksi. Hän työskentelee säilyttääkseen ja laajentaakseen valtaansa. Useimmissa tapauksissa perhe ei ole\ldots\emph{saanut tartuntaa}\ldots ja Mallory yrittää astua sisään ja saada valtaa. Ensimmäinen tapaus koskee jokaisen uuden tulokkaan resurssien varastamista. Toinen taas koskee koko ryhmän resurssien varastamista.

Vauriot, joita Mallory voi saada aikaan perheelle, voivat olla äärimmäisiä. On henkilökohtaiset vauriot, trauma ja vallan menetys. On kollektiiviset vauriot, omaisuuden ja säästöjen menetys. Katsahda perhettä, joka on murtunut kiistojen takia. Katso tarkemmin, ja näet Malloryn työssään.

Kuinka Mallory astuukaan isäntäperheeseen? Selvin vaihtoehto on mennä naimisiin. Vanhempien reaktio uuteen kumppaniin on usein niin äärimmäinen, että siitä on tullut suosittu karikatyyri. Ottaen huomioon psykopaattien vaaran, epäilys ja vihamielisyys on normaalia. Kaikki muu olisi piittaamattomuutta.

Miesten on vaikea ymmärtää naisten todellisia motivaatioita ja luonnetta. Tiellä on liian paljon signaaleja ja liipaisimia. Joten uusien tyttöystävien arviointi jää äidin tehtäväksi. Vastaavasti naiset ovat liian huonoja arvioimaan miehiä. Niinpä uusien poikaystävien on voitettava puolellensa isän luottamus.

Klassinen kuvio alkaa kuulustelulla ja kredentiaalien?? tarkastuksella. Näitä seuraa joko ehdollinen hyväksyminen tai hylkäys. Sitten hetken koeaika. Sitten juhlat ja ehkä vauvoja, ainakin vanhempien mielessä.

Tämä draama toistuu uudelleen ja uudelleen, sekä todellisessa elämässä että populaarikulttuurissa. Meillä on vanhemmat ja heidän halunsa nähdä heidän tyttärensä ja poikansa saamassa lapsia. Meillä on heidän epäluottamuksensa uutta tulokasta kohtaan.

Meillä on nuoruus ja sen vatimukset itsenäisyyteen ja itsemääräämiseen. Nämä seikat mahdollistavat laajan skaalan hahmoja ja juonenkäänteitä.

Otetaan esimerkiksi paljon parjattu anoppi, joka on vitsien peruskohde joka ihmisyhteiskunnassa. Harva naimisissaoleva mies pitää anopistaan. On vaikea antaa anteeksi ihmiselle, joka lähtee siitä ajatuksesta, että olet psykopaatti. Ironia on rikasta. Anoppi joka ei kyseenalaista tyttärensä valintoja usein tarkoittaa ongelmia edessä.

Ero jakaa perheen ja paljastaa sen pedoille. Jos isä on poissa, tilanne on helpompi miespsykopaateille. Ero yleensä levittää omaisuutta. Joten siinä, missä eronjälkeiset perheet ovat helpompia kohteita, ne ovat yleensä myös vähemmän kannattavia kohteita.

Kun perhe on varakas maassa, jossa on heikko valtio, se on tuottoisa kohde. Vahvat perheet kehittyvissä maissa kehittävät järjestettyjen avioliittojen kulttuurin. Ne arvioivat kandidaatteja vainoharhaisesti. Yksi huono valinta voi tuhota sukupolvien verran kasautunutta varallisuutta.

Tämä malli antaa meidän tehdä ennusteen missä tahansa yhteiskunnassa: \emph{järjestettyjen avioliittojen määrä korreloi parin sosiaalisen statuksen kanssa.} Mitä korkeampi status, sitä vähemmän vapautta valita. Tämä näyttää pitävän paikkansa kaikissa yhteiskunnissa. Yhteiskuntien välillä heikompi valtio tarkoittaa suurempaa määrää järjestettyjä avioliittoja. Tämä johtuu siitä, että heikot valtiot eivät kykene suojelemaan perheiden omaisuutta pedoilta.

Liiton jälkeen seuraava tie perheen kaapistoihin käy viettelyksen kautta. Tällaisista suhteista tuppaa tulemaan yleistä tietoa. Kun nuori mies lounastaa lesken kanssa, tai kun nainen käy treffeillä kaksi kertaa hänen ikäisensä miehen kanssa, kysymme saman kysymyksen. ``Kuinka paljon rahaa on pöydällä?'' Jos vastaus on ``paljon\vmq{,}'' ajattelemme pahinta. Vain siinä tapauksessa, että rahaa ei ole pelissä, saatamme uskoa, että kyse on rakkaudesta.

Mallory kykenee viettelemään naimisissa olevan miehen saadakseen lahjoja ja rahaa. Hän käyttää seksuaalisia liipaisimiaan luodakseen kaiken kuluttavan riippuvuuden miehelle. Tämä toimii tehokkaammin, jos mies on liitossa, josta romantiikka on haihtunut aikapäivää sitten. Se on banaalia. Joskus Mallory ystävystyy vaimon kanssa. Hän vakuuttaa vaimon siitä, että huoleen ei ole syytä. Mallorystä tulee vaimon paras ystävä, ilman pienintäkään pelon pistosta tai katumusta.

Mies-mallorylle on vaikeampaa päästä tällä tavalla perheeseen. Hän voi vetää Bobin mukaan rahantekojuoniin, jotka päättyvät katastrofeihin ja häviöihi. Tällä tavoin Wall Streetin Malloryt tyhjentävät monia säästötilejä. Tai hän voi vietellä Bobin vaimon Alisan, ja usutaa Alisan varastamaan hänelle.

\section{Johtopäätökset}

Malory on kyltymätön, aina nälkäinen, ja ajattelee seuraavaa ateriaansa. Kun hän löytää uskottavia mahdolisuuksia, hän lähestyy ja vaihtaa tapojaan ja käytöstään. Nyt hän käyttäytyy arasti ja hillitysti. Sitten hän vaikuttaa dominoivalta ja ylimieliseltä. Hän näyttää hämmästyttävän skaalan käytösmalleja. Hän muuttaa ääntään ja kehonkieltään, asentoaan ja aksenttiaan. Hän muuttuu keneksi ikinä hänen tarvitseekaan muuttua päästäkseen lähemmäs saalistaan.

Muotoaan muuttavien maskien alla Malloryllä on todellinen ja johdonmukainen persoonallisuus. Se ei ole näkyvillä juuri koskaan. Jos se sattuu vilahtamaan, on sitä vaikea tunnistaa ihmiseksi. Psykologi Kathleen Vohs\linkki{sdfds} on näyttänyt kuinka ihmisten pohjustaminen ajattelemaan rahaa tekee heistä antisosiaalisia, epäempaattisia ja taipuvaisempia huijaamaan. Toisin sanoen enemmän Malloryn kaltaisia.

Mallory näkee maailman tällä tavoin koko ajan. Hän ei ole niinkuin muut ihmiset omassa mielessään. Ja muut eivät ole niinkuin hän. Muut ovat paperinohuita muotoja, joilla on yksinkertainen joukko ominaisuuksia. Jotkut ovat mehukkaista, jotkut ovat tylsiä, jotkut ovat hyödyllisiä, jotkut vaarallisia. Hän näkee maailman saaliina ja itsensä synnynnäisenä petona.














\chapter{Hyökkäys ja vangitseminen}\label{attack-and-capture}

\section{Huolellinen hoitaja}

\begin{tarina}%Sofia on suomentajan keksimä nimi
Sofia kertoo pitävänsä vanhuksista, koska he puhuvat niin paljon. Ja hoitokodin vanhat miehet ja naiset näyttävät pitävän hänestä. ``Hän on aina niin hyvällä tuulella\vmq{,}'' he hehkuttavat toisilleen. ``Niin hyvä kuuntelija!''

Hän on tehnyt paljon töitä hoitajan paperien eteen. Loputtomasti kirjoja, opiskelua, kirjoittamista, tenttejä. Opiskelijakaverit saavat parempia arvosanoja. Sofian arvaus on, että he kaikki huijaavat. Heillä on apureita, ja he lahjovat opettajia. No, se on helppoa. Sofiakin kykenee siihen. Hän ei osaa käyttää tietokonetta, se on niin hankalaa! Voisitko tarkastaa tämän jutun minulle? Jooko?

Viimeinkin kidutus on ohi ja Sofia saa tuon maagisen paperin käteensä. ``Pätevä hoitaja\vmq{,}'' hän toistelee itselleen. ``Pätevä hoitaja!'' Jo iltapäivällä hän lähettelee sähköposteja mahdollisiin työpaikkoihin. Pian hänellä on jo keikka kalenterissa. Sofia tulee liittymään syöpää sairastavasta varakkaasta miehestä huolehtivaan tiimiin.

Sofia pukeutuu työpäivää varten. Hän sitoo hiuksensa taakse ja pukee ylleen tyylikkään asukokonaisuuden, jonka hoitokoti on sanellut. Mustaa ja sinistä, valkoinen hattu, pitkä hame, tummat kengät. Kalliit tummat kengät. Hänellä on kolme kollegaa, joiden kanssa hän vuorottelee. Potilas on seitsemissä kymmenissä, ja hän viettää suurimman osan ajastaan vuoteessa. Ennen lounasta he nostavat hänet ylös, pukevat hänet ja vievät hänet puutarhaan kävelylle. Hän palaa väsyneenä ja nukahtaa. Kollegat suosivat ilta- ja yövuoroja, joina tekemistä on vähemmän. Huolimatta ylimääräisestä työstä suosii Sofia aamuvuoroa, jolloin vanhus on hereillä ja puheliaalla tuulella.

Vanhus on mielenkiintoinen mies. Hän on rakentanut useamman ison bisneksen. He ovat lähekkäin, aina juttelemassa. Sofia kysyy vanhukselta kerran, nauraen: ``Minkäs arvoinen omaisuutesi sitten onkaan?'' Mies nauraa takaisin: ``Se on asia mitä kadun. En koskaan päässyt miljardiin asti\vmq{.}'' Sofia nostaa toista kulmakarvaa ja paheksuu??. ``Hupsu mies, varmasti on muitakin asioita, mitä kadut!''

Ja hänellä on. Kova työ on hyväksi, hän kertoo Sofialle. Mutta se ei korvaa perhettä. Vanhus oli naimisissa, mutta hänen vaimonsa kuoli auto-onnettomuudessa kaksikymmentä vuotta sitten. Hänellä on 45-vuotias poikka. Poika vihaa isäänsä ja käy velvollisuudentunnosta vierailulla kerran viikossa. He eivät puhu mistään. Poika lähtee mahdollisimman pian mustalla Mercedeksellään.Sofia saa tietää, että poika on eronnut vaimostaan ja kieltäytynyt auttamaan isäänsä bisneksessä.

Sofialta kestää melkein kuusi kuukautta sovitella isän ja pojan välit. Lopulta he halaavat ja Sofia hymyilee itselleen. Vanha mies vahvistuu. Hän on ylentänyt Sofian hoitotiimin johtajaksi. Sofia vaihtaa tiimin muut jäsenet omiin tuttuihinsa. Nyt hän on tiimin ainoa nainen. Eräänä iltana poika lähtee ja Sofia menee mukaan. He illastavat läheisessä ravintolassa, ja Sofia menee pojan luo yöksi.

Kun he pian menevät naimisiin, molemmat tuntevat, että aika on kohdallaan. Miksi odottaa pidempään? Kohtalo tekee omat suunnitelmansa. He ostavat maatilan korkealta mäeltä ja suunnittelevat unelmakotiaan. Isä kuolee nukkuessaan jokusen kuukauden kuluttua. He nimeävä lapsensa isän mukaan. Se on poika.
\end{tarina}

\section{Hidasta väkivaltaa}

Olemme nähneet hiukan sitä, kuinka Mallory operoi väijyessään uhrejaan. Hän vaanii kaikista luonnon pedoista häikälemättömimpiä: toisia ihmisiä. Richard Connellin novellissa \emph{The Most Dangerous Game} on päätynyt osaksi populaarikulttuuria.\linkki{sfds} Mutta valikoimalla heikoimmat, haavoittuvaisimmat ihmiset, Mallory tuppaa selviytymään vahingoittumattomana. Suurin osa vaurioista kohdistuu uhreihin. Joskus Mallory valitsee uhrikseen vastaan taistelevan kohteen ja vahingoittuu.

Näin asia on klassisessa peto-saalis-suhteessa. Peto poimii kaikista haavoittuvaisimmat yksilöt. Tämä ajaa saalislajin evoluutiota kohti vahvempaa vastustuskykyä. Se ole ainut evoluutiota ajava tekijä, mutta merkittävä se voi olla molemmille genomeille.

Havaittuaan Alisan Mallory hyökkää ja yrittää vangita hänet. Voimme kuvitella leijonan jahtaavan ja sitten kaatavan seepran. Seepra väistelee vasemmalle ja oikealle. Leijona harppaa ja iskee. Se, mitä tosiasiassa näemme, on paljon puhetta ja sitten yhtäkkisiä ja epätavallisia päätöksiä. Väkivalta on harvoin avointa, mutta se on aina läsnä muodossa tai toisessa. Kun Mallory hyökkää, Alisa taistelee vastaan kaikin voimin.

Psykopaattisen suhteen ydin on ``idealize-devalue-discard'' (IDD) -sykli. Moni kirjoittaja kuvailee tätä sykliä.\linkki{sdfds} Jokainen, joka on ollut tekemisissä psykopaatin kanssa, tunnistaa sen. IDD-sykli alkaa Mallory asettaessa Alisan jalustalle. Hän pirskottaa ylistystä ja kiintymyksen ilmaisuja Alisan päälle. Hetken päästä hän tekee täyskäännöksen ja muuttuu välinpitämättömäksi ja kylmäksi. Ja sitten hän katkaisee kaiken ja menee matkoihinsa.

IDD-sykli voi pyörähtää ympäri muutamassa tunnissa tai vuosikymmenessä. Useimmat kirjoittajat näkvät sen epäonnistumisen merkkinä. Mallory ei kykene luomaan todellisia suhteita. Hän tarvitsee ihailua narsisminsa ruokkimiseen, jonka jälkeen hän tylsistyy, tarina kertoo. Tämä selitys vaikutta virheelliseltä useammasta syystä. Se olettaa, että on olemassa normatiivinen suhteen malli, jota Mallory ei kykene toteuttamaan. Se olettaa, että narsismi on todellinen asia, jolla on halunsa ja tarpeensa. Ja ennen kaikkea se jättää huomiotta ne monet psykopaatit, jotka rakentavat vuosien mittaisia suhteita.

On käytännöllisempää nähdä IDD-sykli yhtenä Malloryn monista työkaluista. Hän ei ole rikkoutunut ihminen. Keskimäärin hän on yhtä menestyksekäs kuin Alisa. Hän suhteensa ovat yhtä ``normaalieja'' kuin Alisanki nsuhteen. Kyse on vain erilaisesta normaaliudesta, joka perustuu pedon ja saaliin dynamiikkaan.

Saalistamisen Wikipedia-artikkeli sanoo:\linkki{sdfd}
\begin{quotation}
\noindent Saalistamisen akti voidaan jakaa neljään vaiheeseen: saaliin havaitsemiseen, hyökkäykseen, nappaamiseen ja lopuksi kuluttaminen.
\end{quotation}
Tunnistusta ennen on vielä yksi vaihe, \emph{piilottelu.} Malloryn on pysyttävä saaliinsa näkymättömissä niin pitkään kuin mahdollista. Jos Bob näkee hänen todelliset värinsä, Bob kävelee tai juoksee karkuun. Paljastuminen rampauttaa. Vain uudessa seurassa, kaupungissa tai maassa voi Mallory taas metsästää.

Piilotteluvaihe on kriittinen. Mutta IDD-sykli jättää sen huomiotta ja kohtelee ``idealize''-vaihetta kaiken alkuna. Sykli jättää huomiotta myös Malloryn päämäärän, mikä on ``kuluttaminen\vmq{.}'' Hän ei syö uhriensa lihaa. Hän vain tyhjentä heidät rahasta, vallasta ja energiasta kunnes kuolema voi kokea olevansa tervetullut.

Viimeiseksi IDD-sykli jättää huomiotta yksityiskohtaisen tavan, jolla Mallory rakentaa kontrollinsa ajan kuluessa. Kyse ei ole vain yhdestä syklistä. Sen sijaan syklejä on pienempiä ja suurempia, jotka toistuvat ja toimivat päällekkäin. Tämän takia aikaskaalat näyttävät niin vaihtuvilta. Mallory saattaa pyöräyttää IDD:n ensitapaamisella. Ja sitten astua vuosikymmenen mittaiseen suhteeseen saman henkilön kanssa. Kuten näytän myöhemmin, tämä ei ole ristiriitaista. Se on osa mekanismia.

Joten IDD-sykli on tarkka, johdonmukainen ja ennustettava. Se on myös epätäydellinen ja pinnallinen. Se ei mallinna suhdetta Malloryn ja hänen lapsiensa välillä. Se ei mallinna suhdetta Malloryn ja hänen apulaistensa (sekundääristen psykopaattien) välillä. Se ei selitä, miksi Mallory työskentelee rikkoakseen ihmiset ennen heidän hylkäämistään.

Itse näen asian niin, että IDD-sykli kertoo tarinan uhrin puolelta. Tarina pitää paikkansa, mutta se on vinoutunut ja epätäydellinen. Meidän täytyy katsoa laajemmin ja syvemmälle ymmärtääksemme kokonaiskuvan. ``Hylkäämiset'' ovat valheita, joiden tarkoitus on upottaa koukkua syvemmälle. Ainoastaan niistä viimeinen on totta. Ja kun Mallory lähtee, suhdetta ei ole enää. Se ei ole pelkästään ohi: Malloryn mielessä sitä ei ollut olemassa \emph{missään vaiheessa.}

Kelataanpa taaksepäin. Mallory on havainnut kohteen. Ennen ensimmäistä IDD-sykliä hän valmistelee hyökkäystään. Valmistelu on niin johdonmukaista, että voit tietää, missä olet Malloryn silmissä katsoen hänen maskiansa. Siinä vaihessa, kun hän idealisoi sinua, olet jo kietoutunut Malloryyn. Selostan, kuinka kietoutuminen alkaa.

\section{Piilottelu}

Mallory ei voi selvitä paljastumisesta. Jo nuoresta iästä lähtien hänen on opittava käyttäytymään ``normaalisti\vmq{.}'' Tämä on vaikea haaste, sillä ``normaali'' on katalan liikkuva tavoite. aivan kuin sosiaaliset ihmiset liikuttaisivat tavoitetta joka päivä häiritäkseen psykopaatteja. Mallory välttää ongelman usein piiottelemalla plain sightissä. Yksi tapa on olla niin kovaääninen, että ihmiset eivät vilkaise toista kertaa. Voimme kutsua tätä huomionhakuiseksi ja narsistiseksi käytökseksi.

En väitä, että jokainen kovaääninen, koreileva, yliampuva henkilö on psykopaatti. Psykopatialle ei ole tarkkaa näkyvää testiä, eikä sellaista tule koskaan olemaan. Tätä evoluution tasapaino on. Mutta jos ajattelet, että Mallory on kylmä, harmaa persoona, olet  väärässä. Hän on usein dramaattinen, mahdoton ennustaa, mystinen, sanoinkuvaamaton. Hän on intohimoinen ja emotionaalinen. Kyse on kuitenkin näyttelemisestä, joka toimii useimpiin, lukuun ottamatta toisia psykopaatteja.

Otetaan esimerkkitapaus, joka selventää asiaa. Olet juhlissa kavereiden kanssa. Paikalla on nainen, joka ei naura, eikä näytä minkäänlaisia muitakaan tunteita. Hän katsoo sinua kuin teurastaja, joka valitsee sikaa tapettavaksi. Välillä hän katsoo muita, ja sitten hän katsoo takaisin sinuun. Hän hymyilee, eikä hymy ole ystävällinen. Tämä olisi Mallory ilman maskia. Kyse on pelottavasta näystä, joka harvoin tulee vastaan.

Ja sitten on mies, jolla on hölmö oranssi hattu päässään, ja joka on pukeutunut hulluihin väreihin. Hän nauraa kovaäänisesti typerä virne naamallaan. Kaikki pitävät hänestä. Hän on juhlien elämä ja sielu, joka metelöi iloisesti ja äänekkäästi, kasvot ja kädet liikkuen. Hän vaikuttaa harmittomalta hupsulta. Jos kohtaat hänen katseensa, koet lämpöä ja liikutusta. Katse on niin syvä, että voisit pudota siihen. Tämä on Mallory pukeutuneena väreihinsä.

Mallory oppii naamionsa ystäviltä ja perheeltä. Hän matkii aksentteja ja äänenpainoja, puheen kuvioita, kasvojen ilmeitä, kehonkieltä. Hän on ammattilaisnäyttelijä, joka uppoutuu rooleihin. Hän pitää näitä naamioita hyllyssä koko elämänsä ajan, säätää niitä, ja pukeutuu niihin tarvittaessa. Naamiot ovat karikatyyrejä, mutta ne ovat uskottavampia kuin niiden esikuvat.

Tarkoitus on harhauttaa ja kontrolloida. Kyse on lavataikurin tekniikasta. Draama, musiikki ja sujuvat sanat saavat yleisön katsomaan tiettyyn suuntaan. Ja niin he eivät huomaa Malloryn todelisia siirtoja. Homma toimii sekä yksittäisen ihmisen että huoneellisen ihmisiä kanssa.

\section{Haastattelu}

Joskus nuorena opiskelijana Yorkissa päädyin kavereideni kanssa pieneen taloon. Siellä ystävälliset ihmiset tarjosivat meille ``ilmaisia persoonallisuustestejä\vmq{.}'' He antoivat meidän leikkiä skifivekottimilla jotka mittasivat stressitasojamme. He istuttivat meidät alas kahdenkeskisiin keskusteluihin. Nuori nainen jutteli kanssani hetkisen siitä, miksi olin paikassa ja mitä halusin elämältäni. Hän kirjoitti ylös nimeni ja osoitteeni ja alkoi tehdä muistiinpanoja. ``Mikä on pahinta, mitä olet koskaan tehnyt elämässäsi?'' hän kysyi minulta. Sävy oli samalla tavalla kasuaali, kuin kysyessäsi joltakulta: ``Mitä söit aamiaiseksi?''

Kysymys oli odottamaton kylmä sormi joka tökki yksityistä mieltäni. Löin sen pois mielestäni. Kiitin naista teekupista, keräsin kaverini ja lähdimme pois. Yksi tyttö jäi hiukan pidempään. Puoli vuotta myöhemmin olimme melkein menettäneet hänet. Kovan vakuuttelun myötä hän lopetti ryhmässä käymisen. Hän lopetti rahan käyttämisen heidän outoon kirjallisuuteen ja kursseihin. Hän jatkoi opiskelujaan.

Nuo ystävälliset ihmiset kyttäsivät häntä ja meitä. He menivät hänen soluasuntoonsa ja seurasivat meitä kaduilla. ``Miksi et käy sessioissa?'' he kysyivät häneltä, eivätkä he enää olleet niin ystävällisiä. ''Miksi satutatte ystäväänne?'' he kysyivät meiltä. ``Hän tarvitsee kursseja\vmq{,}'' he huusivat joskus, kun emme huomioineet heitä. He pyörivät ympärillä yli vuoden ennen kuin he luovuttivat.

En mainitse nimiä, sillä Skientologia saattaisi loukkaantua. Joitakin vuosia myöhemmin tuo psykopaattinen organisaatio vei serkkuni. Hän palasi vasta vuosien päästä eri ihmisenä. Ilo ja nauru olivat mennyttä.

Ihmiset kysyvät minulta joskus, mistä mielenkiintoni Mallorya kohtaan tulee. Ystäväni ja perheeni, sekä minä itse{\ldots} olemme vuotaneet verta ja kyyneliä kerta toisensa jälkeen. Kuinka voisin olla huomaamatta meitä vaanivaa petojen paraatia? En tee siitä henkilökohtaista. En suutu siitä. Sen sijaan dekoodaan, ymmärrän ja puran ne valheiden runkorakenteet joista Mallory on riippuvainen.

Haastattelu on yksi noista valheista. Se alkaa näin: ``Minä välitän sinusta, ja me jaamme intiimin hetken\vmq{.}'' Se päättyy kiristykseen ja kiskontaan. Se on harvoin niin avointa kuin henkilö kirjoittamassa muistioon. Yleensä se tapahtuu baarissa tai yökerhossa tai jossakin sosiaalisessa tilanteessa. Nämä ovat konteksteja, joissa odotamme small talkia ja juttelemme iloisesti. Usein mukana on alkoholia, mikä saa meidät laskemaan kilpemme.

Mallory haluaa tietää, kuinka hyvä kohde Bob on, ja mihinkä suuntaan työstää tilannetta. Aivan kuten autokauppiaskin kysyy: ``Mitä teet työksesi?'' ja sitten: ``Oletko naimisissa?'' Luotaaminen voi olla hellävaraista, mutta se on hellittämätöntä. Hän leikkaa kohti Bobin heikkouksia ja mahdollisuuksia ja riskejä, joita Bobiin liittyy.

Mallory haluaa sulkea pois kohteita, jotka näyttävät hyviltä mahdollisuuksilta, mutta eivät ole sitä. Joten kysymykset keskittyvät Bobiin. Ilmoilla voi olla teatteria ja draamaa. Mutta keskustelu zoomaa sisään ja Mallory lukee joka ikisen reaktion ja nyanssin.

Haastattelu on osa kasvavaa lupausta jostakin asiasta: seksistä, rahasta tai pelastuksesta. Uhka ilmestyy samaan tahtiin lupauksen kanssa. Jos et vastaa, menen pois, ja lupaus katoaa mukanani.

Voit nähdä, kun Mallory haastattelee sinua, jos hän on kiireinen. Jos hän on huolellinen, et voi huomata sitä, sillä se tapahtuu päivien ja viikkojen aikavälillä, jopa pidemmällä ajanjaksolla. Ja haastattelu voi tapahtua selkäsi takana, tuntemasi ihmisten kautta.

Yleensä se on kuitenkin näkyvillä. Hän on kivempi, kuin mitä hänen tarvitsisi olla. Hän hymyilee paljon, ja käyttäytyy kasuaalin dominoivasti. Hän lähestyy sinua, ei toisin päin. Hän kysyy kysymyksiä taustastasi, perheestäsi, suhteistasi. Kysymykset ovat melkoisen intiimejä ottaen huomioon, että kyseessä on satunnainen, rento keskustelu. Intuitiosi kiemurtelee. Mutta vuorovaikutus osuu liipaisimiisi ja saat dopamiinisia mielihyvän potkuja. Joten jatkat keskustelua.

Jos kuuntelet intuitiotasi, saattaa sinusta tuntua hankalalta tämän hymyilevän ihmisen seurassa. Voit tietysti mennä matkoihisi hetkellä millä hyvänsä. Mutta onhan se \emph{mahdollista,} että hän on vilpittömän kiinnostunut sinusta. Kaikkia kiinnostuneita ihmisiä pakeneminen on huono strategia. Joten on olemassa toinenkin puolustus, jota usein käytämme. Se on jatkaa keskustelua, mutta vaihtaa neutraaliin puheenaiheeseen tai toiseen keskustelukumppaniin.

Ihminen, joka nauttii keskustelusta kanssasi, menee virran mukana, olipa keskustelunaihe mikä tahansa. Mahdollisuus ohjata keskustelua satunnaisiin suuntiin on valkoinen lippu. Mallory kääntää keskustelun vaivatta takaisin mieleisekseen. Hän väistää kysymykset omasta taustastaan, tai sitten hän valehtelee ja liioittelee. Hän voi puhua itsestään ja jakaa ``intiimejä'' yksityiskohtia, mutta kaikki tähtää aina siihen, että sinä paljastat itsestäsi lisää. Hänen valehteluaan on lähes mahdotonta huomata, paitsi jos saat hänet kiinni jostakin tietystä epätotuudesta.

\section{Kylmälukeminen ja haulikointi}

Haasttelun aikana Mallory kylmälukee ja haulikoi. Tämä tarkoittaa Alisan elämän merkittävien yksitysikohtien arvailua lyhyessä ajassa. Kyse on mentalistien, huijareiden, autokauppiaiden ja sekalaisten mystikkojen perustyökalusta. Mallory kykenee siihen ilman vaivannäköä tai harjoittelua.

Useimmat meistä ovat eksperttejä toisten ihmisten lukemisessa, toisin tiedostamattaan. Aivomme tulkitsevat kaikkea näkemäämme emotionaalisiksi signaaleiksi ja empaattisiksi reaktioiksi. Emme näe stressiä, vaan tunnemme sen. Mallory näkee kaiken ympärillä ilman emotionaalista linssiä.

Klassinen kylmälukeminen on biletemppu, jota mystikot käyttävät tehdäkseen vaikutuksen ihmisiin. ``Menetit isäsi hiljattain\ldots hän lähettää sinulle viestin\vmq{.}'' Kenttäolosuhteissa kylmälukeminen on tunkeilevampaa.

Useimmat ihmiset ovat luettavissa suurimman osan ajasta. Lukija aloittaa keskustelun ja kysyy oikeat kysymykset. Hän voi sitten tehdä hyviä arvauksia seuraavista asioista:
\begin{itemize}
\item Missä luettavat on kasvanut. Tästä kertoo aksentti.
\item Onko luettava ensimmäinen lapsi vai ei. Tästä kertoo se, kuinka paljon hän kokee stressiä epäjärjestyksestä. Häiritseekö luettavaa myöhästymiset ja sotku vai ei?
\item Onko luettavalla nuorempia sisaruksia, tai tuleeko hän suurperheestä. Tästä kertoo se, kuinka hän kohtelee pieniä lapsia.
\item Ovatko luettavan vanhemmat riidelleet rajusti tai eronnet. Tästä kertoo yleinen itsevarmuus.
\item Mitä asioita luettava on opiskellut. Tästä kertoo se, kuinka hän puhuu ja toimii.
\item Kunka paljon luettava tienaa. Tästä kertoo konteksti ja käytös.
\item Miksi luettava on tilanteessa ja mitä hän odottaa tapahtuvan.
\item Kuinka luettava kokee tilanteen ja lukijan.
\item Mitä luettava haluaa sillä hetkellä eniten.
\end{itemize}
Ja niin edelleen. Useimmat ihmiset voivat opetella kylmälukemisen taidon johonkin pisteeseen saakka. Sinun täytyy opetella maadoittamaan tunteesi, jonka jälkeen tarvitset harjoitusta. Suuri osa tästä on vain avoimuutta sille, mitä ihmiset tuovat ilmi sanoillaan tai käytöksellään. Ihmiset, jotka eivät ole luettavissa, salailevat jotakin syystä tai toisesta.

Malloryn kylmälukutaito on neron luokkaa. Hän yhdistää täydellisen lukemisen haulikointiin. Hän tekee pikaisia, karkeita arvauksia monista eri asioista. Arvaukset joko menevät päin seiniä tai liipaisevat pienen reaktion. Heitä ilmoille viisi vaihtoehtoa, näe reaktio numeroon neljä, ja olet päässyt kotipesälle. Lopputulos näyttää tyrmistyttävältä kyvyltä lukea ajatuksia.

Voit huomata, milloin Mallory haulikoi sinua. Se tuntuu haastattelulta, paitsi on pahempaa. Hän tekee teräviä arvauksia yksityiskohdista, joita hänen ei pitäisi tietää, ja joista hänen ei pitäisi kysellä. Hän tekee näitä arvauksia aivan kuin hän toteaisi absoluuttisia totuuksia. Ei kysymyksiä, hän vain täräyttelee väitteen väitteen perään, kunnes hän osuu oikeaan.

Haulikoiminen liikkuu hienovaraisen skannailun ja avoimen aggression välillä. Mallory voi ammuskella syytöksiä, joiden on määrä tuottaa suurta tuskaa. Kyse on erikoisentyyppisestä keskustelusta. Sillä näyttää olevan kolme tavoitetta. Ensinnäkin satuttaa ja epävakauttaa vastustajaa. Toisekseen, saada reaktioita sitä kautta löytää haavoittuvaisuuksia ja piilotetuja totuuksia. Kolmanneksi, vakuuttaa ulkopuoliset havaitsija Malloryn viattomuudesta ja hyveellisyydestä.

Internet-trollit omaavat moinia psykopaattisia piirteitä. Heissä ei näy empaattisuutta eikä emotionaalisuuta. He tuppaavat olemaan yksinäisiä ja petomaisia. He myös haulikoivat vastustajiaan. He ovat usein niin väkivaltaisia, kuin näppäimistön kautta voi vain olla.\linkki{sdfds}

On merkittävää nähdä Malloryn kuuluttavan suoria valheita horjuttaakseen toisten mainetta. Hän voi olla rauhallinen ja surullinen, tai palaa oikeamielistä suuttumusta. Kaikki on teatteria, joka on suunnattu yleisöön. Parhaat valheet ovat uskottavia ja värikkäitä. Ne tuppaavat olemaan helposti osoitettavissa vääräksi---harvempi vain vaivautuu tekemään sitä. Mielemme ovat kehittyneet olemaan samaa mieltä auktoriteettien kanssa. Hyväksymme ilmiselviä valheita, jos ne tulevat meihin nähden ylempiarvoisesti käytäytyvän henkilön suusta. Ilmiötä kutsutaan Asch-efektiksi.\linkki{sdfsd}

\section{Imitaatiopeli}

Kädellisillä ja linnuilla on sosiaaliset vaistot toisten käytöksen kopioimista varten. Päämekanismeja näyttää olevan kolme kappaletta: konvergenssi (convergence), peilaaminen (mirroring) ja imitoiminen (mimicking). Jokaisella mekanismilla on evolutiivinen järkensä. Jokainen on työkalu psykopaatin käsissä.

Olemme sosiaalinen laji ja identiteettimme elää ryhmissä, joihin kuulumme. ``Ryhmän'' käsitteemme skaalautuu kahdesta hengestä miljooniin. Monet mekanismit toimivat näillä kaikilla tasoilla. Konvergenssi on yksi niistä.

Osa ryhmän identiteettiä on enemmän tai vähemmän johdonmukainen kulttuuri. Ennen kaikkea tämä tarkoittaa ulkoasua, käytöstä ja kieltä. Ryhmät eivät kuitenkaan tähtää vain johdonmukaisuuteen. Ne tähtäävät myös omaleimaisuuteen, erityisesti verrattuna läheisiin kilpailijoihin.

Sille, miksi ryhmät pyrkivät johdonmukaisuuteen ja omaleimaisuuteen on olemassa useampia syitä. Jo valmiiksi ryhmään kuuluvilla on intressi ryhmän laajentamiseen. Koko on valtaa. Kulttuuri toimii brändinä. ``Tätä me olemme!'' on rekrytointimainos. Ja ``Emme ole kuin he!'' estää jäsenten loikkaamista toisiin ryhmiin samalla alueella.

Niillä, jotka liittyvät ryhmään, tai jotka ovat syntyneet siihen, on suuri motivaatio mukautua siihen. ``Erilaisuus'' paljastaa yksilön hylkäämiselle ja ulkopuolisten väkivallalle. Kun ryhmien välillä on konflikti, ensimmäisiä kohteita ovat syrjäytyneet.

Ryhmässä piilottelun evolutiiviset edut ovat vanha kulttuurillinen moottori. Ihmiskielen ja -käytöksen monimuotoisuus ei ole satunnaista kaaosta. Se nousee jokaisen ryhmän tarpeesta löytää oma ekologinen rakonsa. Se näkyy aksenteissa, murteessa ja meemeissä.

Se näkyy ruokatabuissa, mikä ovat usein absurdin monimutkaisia. Mielivaltaiset, monimutkaiset säännöt ovat kontrollityökalu, niinkuin selitän luvussa \ref{the-feeding}. Ruokatabut kehittyivät mitä luultavimmin keräilijä-metsästäjä-menneisyydessämme. ``Tämä on myrkyllistä'' on helpompi selittää, kuin ``Tämä on kiellettyä\vmq{.}'' Voimme oppia inhoamaan kerran ja se pelastaa henkemme sata kertaa. Sama vaisto antaa heimoille mahdollisuuden kieltää ruokia, joita naapuriheimot syövät. Se estää ihmisiä lipumasta pois.

Konvergenssi tapahtuu neuovottelemalla ja imitoimalla. Dominoivat yksilöt luovat kuvion, jota vähemmän dominoivat seuraavat. Kun ihmiset jakavat valtaa, he neuvottelevat painotetun keskiarvon. Siitä tulee termi ``slavish conformity\vmq{.}'' Jopa lapsi yrittää neuvotella vanhempiensa kanssa. Lopputulos on ryhmäkonsistenssi lyhyellä tähtäimellä ja evoluutio pitkällä tähtäimellä.

Konvergenssi ottaa aikaa ja energiaa ja on yksilöiden välinen neuvottelu. Tämä tarkoittaa, että voit nähdä, kuinka läheisiä ihmiset ovat siitä, kuinka he esiintyvät, toimivat ja puhuvat.

Sekä miehet että naiset konvergoivat, omilla tavoillansa. Miehet tuppaavat konvergoimaan ryhmän kieltä käytöstä ja ilmiasua kohti. Naiset tuppaavat konvergoimaan toisia naisyksilöitä kohti. Harvoin näkee kahden miehen pukeutuvan samalla tavalla, paitsi jos he ovat osa jotakin laajempaa ryhmää. Mutta kahden naisen voi nähdä usein konvergoituvan kohti toisiaan. Havaitse kaksi naista yhdessä, ja voit usein tietää, kuinka hyvin he tuntevat toisensa. Se näkyy hiuksissa, vaatteissa, kengissä, tarvikkeissa ja kehonkielessä.

On ihmisiä, jotka eivät konvergoi. Tässä mielessä erilaisia ihmisiä on ainakin kolmea eri tyyppiä. On heitä, joilla on jonkinasteista autismia. On synnynnäisiä johtajia. Ja on psykopaatteja. Selostan nämä tyypit, jotta näet eron.

Autistiset ihmiset eivät kykene lukemaan sosiaalisia vihjeitä. Tämä tarkoittaa sitä, että he eivät konvergoi, olipa konteksti mikä hyvänsä. He näyttävät yksinäisiltä, epäsosiaalisilta ja ``oudoilta'' monilla eri tavoilla. Populaari on demonisoinut yksinäiset ihmiset epävakaiksi ja vaarallisiksi, mutta kyse on myytistä. Sellaisilla yksilöillä on keskimääräistä suurempi riski joutua syrjityksi.

Synnynnäiset johtajat konvergoivat silloin, kun he liittyvät korkea-arvoisempien seuraan. He eivät konvergoi silloin, kun he tapaavat potentiaalisia seuraajia. Tämä pakottaa toiset lisätyöhön konvergoinnin eteen. Arvostamme suhteitamme sen mukaan kuinka paljon investoimme niihin. Joten kovempi työ konvergenssin eteen luo syvemmän sitoumuksen johtajaan. Ja tästä ryhmä rakentuu, kun osallistuja konvergoituvat yhteen henkilöön.

Tässä vaiheessa Mallory käyttäytyy pitkälti synnynnäisen johtajan tavoin. Mutta hän alkaa hyväksikäyttää ja kaltoinkohdella jäseniä melkein välittömästi. Synnynnäinen johtaja kohtelee ja suojelee ryhmää kuin perhettään. Mallory kohtelee ryhmää kuin omaisuuttaan tai leluaan. Tämä on narsismi, yksi hänen maskeistaan.

Äärimmäisissä tapauksissa hän pakottaa muut äärimmäisiin tekoihin kovergoinnin vuoksi. Johdonmukaisen asun, kielen ja käytöksen pakottaminen on hyväksikäytön muoto. Se rikkoo yksilön identiteetiin ja minäkuvan. Tämä on puhdas psykopaattinen piirre, yksilöissä ja organisaatioissa.
































%\input{ch04}
%\input{ch05}
%\input{ch06}
%\input{ch07}

\chapter{Loppupuhe}\label{postface}

\emph{Rakas Mallory.} Haluaisin kiittää henkilöitä F, S, H, J, M, F, B, K ja M. Opetitte minulle psykopatiasta enemmän kuin koskaan halusin oppia. En usko, että koskaan ymmärsitte, millaisia vaurioita aiheutitte toisille. Hehän eivät koskaan olleet todellisia. Ihmiset, joita käytitte ja hylkäsitte kuin rikkoutuneen lelun. Jos luette tämän kirjan ja huomaatte olevanne sen väärällä puolella, voi se olla shokeeraava kokemus. Jollakin tasolla olemme kaikki viattomia sisimmässämme, jopa meistä pahimmat. Olkaa rauhassa. En aio sanoa nimiänne ääneen.

Asiat, jotka opetitte minulle, eivät ole menneet hukkaan. Olen käyttänyt jokaisen vuorovaikutuksen joka meillä on koskaan ollut ymmärtääkseni kuinka teidän mielenne toimii. Olen tehnyt pitkiä, monimutkaisia kokeita teillä. Monesti en edes tarkoittanut sitä. \emph{Minun} mieleni vain toimii sillä tavalla. Haluan ratkaista mysteerejä ja te olitte syvä mysteeri. Annoitte minulle hyvää, tiukkaa dataa, joten kiitos siitä. Jos teissä on jotakin, se on johdonmukaisuus. Tulokset ovat työssäni ja tässä kirjassa. Omistaisin sen teille, paitsi että en kuitenkaan.

\vspace{0.1in}

\noindent ---Pieter Hintjens, Brysselissä lokakuussa 2015


\appendix

\chapter{Kreetalainen menetelmä}\label{the-cretan-method}

Totuuden etsiminen on ikivanha ja vaikea matka. Moderni yhteiskunta kylpee edelleen valheiden meressä politiikassa, markkinoilla ja etenkin ohjelmistoalalla, missä suuret valheet jalostuvat kuin kultit. Minun työtäni on tien leikkaaminen valheiden läpi, tavoitteena kehitettää parempia teorioita totuudesta. Käy ilmi, että valheet aiheutavat kipua, ja kun lähestymme totuutta, muutumme onnellisemmiksi. Tässä esseessä selostan, kuinka teen sen, ja annan sarjan työkaluja ja opetuksia avuksesi.

\section{Totuuden lähestyminen}

Ihmisyys on voimakas aitososiaalinen eläinlaji. Eräs supervoimistamme on kokoelma todellisesta maailmasta kertovaa tietämystä. Parannamme sitä rakentamalla malleja tai teorioita, jalostamme niitä käytännön kautta ja opettamalla niitä jälkeläisillemme. Se, mikä kaikista mallinnettavissa olevista todellisuuden kerroksista vaikuttaa meihin eniten, joka ikinen päivä, on toisten ihmisten todellisuus. Karl Popper kirjoitti: ``Kaikista syvimmällä olemassaolon tasolla olemme sosiaalisia olentoja\vmq{.}''

Teorioiden muotoilu, testaaminen ja soveltaminen on taiteenlaji. Totuus on merkillinen asia. Aivan kuin irrationaaliluku, totuus on olemassa, siitä ei voida neuvotella, se ei ole subjektiivinen, eikä sitä voida koskaan saavuttaa. Totuus on neliulotteisen avaruusajan absoluuttinen ominaisuus. Emme voi koskaan saavuttaa totuutta; voimme vain rakentaa teorioita, jotka approksimoivat sitä paremmin ja paremmin.

Rakennamme teorioita näennäisesti tyhjästä. Otamme havaintoja ja tuntemuksia ja sen loputtoman teorioiden perinnön, jonka aikaisemmat sukupolvet ovat meille jättäneet. Suunnittelemme uusia tai paranneltuja teorioita ja enkoodaamme ne kieleen ja sanoihin argumentointia, muistamista ja jakamista varten.

Popper järkeili että teorioita on oleellisesti kahta sorttia. On tieteellisiä teorioita, jotka voidaan todistaa vääräksi datalla tai havainnoilla, ja on mystisiä teorioita, joita ei voida falsifioida. Toisin sanoen teoriaa ei voida koskaan todistaa oikeaksi, sillä totuutta ei voida koskaan saavuttaa. Voimme kuitenkin yrittää todistaa teoria vääräksi ja epäonnistua. Kun poistat kaiken, mikä on todistettavasti väärin, se, mikä jää jäljelle, lähestyy totuutta.

Kaikki teoriat ovat jossakin määrin epätarkkoja, mutta falsifioitavissa olevilla tieteellisillä teorioilla on mielenkiintoinen ominaisuus, joka mystisiltä teorioilta puuttuu: tieteellisiä teorioita voidaan kehittää kohti hienompia ja hienompia totuuden approksimointeja. Tässä on joukko teorioita, jotka tunnistat:
%\begin{itemize}
%\item Ympyrän ympärysmitan suhde sen halkaisijaan, \(\pi\), on vakio.
%\item \(\pi=3\).
%\item \(\pi=\frac{22}{7}\).
%\item \(\pi={3,}141592\).
%\end{itemize}
\begin{itemize}
\item Ympyrän ympärysmitan suhde sen halkaisijaan, Pii, on vakio.
\item Piin arvo on kolme.
\item Piin arvo on \(\frac{22}{7}\).
\item Piin arvo on {3,}141592.
\end{itemize}
Nämä ovat kaikki falsifioitavissa, eikä mitään niistä voi todistaa oikeaksi. Otetaan listan ensimmäin teoria. Voimme mitata niin monta ympyrää kuin haluamme, ja saamme joka kerta suurin piirtein saman vastauksen. Emme saa mitään dataa, mikä todistaisi teorian vääräksi. Emme myöskään saa dataa, mikä todistaisi sen oikeaksi. Toistaiseksi kaikki hyvin. Toinen teoria vaikuttaa hyvältä, ja sen mitätöiminen on triviaalia. Väärässä oleminen on tieteellistä! Kolmas teoria on parempi; joskin mittaustyökalujemme parantaminen paljastaa, että Pii näyttää enemmän luvulta {3,}1416 kuin luvulta {3,}1429. Neljäs teoria on vielä paljon parempi ja sen mitätöiminen vaatii paljon työtä. Ei ole olemassa Piin teoriaa, jonka voisimme absoluuttisesti sanoa olevan totta. Meillä on ainoastaan tarkempia ja tarkempia malleja.

Tässä taas tulee maagisia teorioita, jotka saatat tunnistaa: ``C++ on mahtava, koska siinä on Standardimallikirjasto\vmq{,}'' ja ``Java on mahtava, koska se estää koodareita tekemästä pahoja virheitä\vmq{.}'' Kuinka falsifioida nämä teoriat? Mahdotonta. Ne eivät ole pelkästään epätotta: ne eivät ole edes väärässä,\linkki{https://en.wikipedia.org/wiki/Not_even_wrong} kuten Wolfgang Pauli sanoisi.

Tieteellinen teoria on aina jonkin verran väärässä, ja sitä voidaan parantaa vähentämällä sen virheellisyyttä testaamisen, havainnoimisen ja mittaamisen avulla. Mystinen teoria on epätotta, eikä sitä voida parantaa.

\section{Valheiden teoria}

Jos haluamme etsiä totuutta, kannattaa meidän ymmärtää harhautuksen luonnetta. Kreikkalaiset ajattelivat (tai teeskentelivät), että harhautus oli paha henki nimeltään Apate, joka pakeni Pandoran avatessa pahamaineisen lippaansa. Sun Tzu kirjoitti, että ``kaikki sodankäynti perustuu harhautukseen\vmq{,}`` millä hän tarkoitti, että harhautus on menestyksekäs strategia konfliktitilanteessa.

Useimmat mielet ovat rehellisiä valehtelijoita. Voimme olettaa, että valehtelu on ihmismielen sisäänrakennettu toiminto. Kyky valehdella, tai vähintäänkin bluffata, on välttämätöntä maailmaa kuvaavien teorioiden rakentamisessa, sillä kaikki teoriat ovat jossakin määrin valheellisia. Ainoastaan toimintahäiriöinen mieli on kyvytön valehtelemaan. Kun mieli muotoilee teoriaa, se remiksaa tietämiään olemassaolevia teorioita uusiin havaintoihin, lisää sen approksimaatiota, arvauksia, otaksumia ja uskomuksia, yrittäen yhdistää ne johdonmukaiseksi tarinaksi. Useimmat mielet tekevät tätä hitaasti, viikkojen tai vuosien kuluessa. Tällaisissa teorioissa valheet ovat väliaikaisia rakennustelineitä, jotka voidaan poistaa ajan myötä.

Jotkut mielet jatkuvassa sodassaan muuta ihmisyyttä vastaan valehtelevat strategisesti manipuloidakseen ja aiheuttaakseen sekaannusta. Tällaiset mielet kykenevät rakentamaan täysin mystisiä teorioita niin nopeasti, että kaikki tapahtuu reaaliajassa, ja kertomaan ne kuuntelijalle äärimmäisen uskottavina valheina. Kyse on hyökkäyksestä, sodankäynnin aseesta. Voimme kutsua tällaisia mieliä ``psykopaattisiksi\vmq{,}'' sillä heidän tavoitteena on toisten saalistaminen. Heidän teorioissaan totuus on väliaikainen rakennusteline, joka vaihdetaan myöhemmin sepityksiin.

Valheen arvo aseena on selvä. Peto-saalis-suhteessa pedon on pidettävä saalis sekaantuneena ja liikkumiskyvyttömänä, jotta se voi ruokailla turvallisesti. Ihmisten tapauksessa emotionaaliset siteet ovat tehokkaampia ja vähemmän riskialttiita kuin fyysiset siteet. Injektoimalla mystisiä teorioita saaliin mieleen peto rampauttaa saaliin teoriaprosessin. Tämä on yksi tapa, jolla kultit pyydystäväþ uhrejansa: injektoimalla suuria mystisiä teorioita, jotka häiritsevät loogista ajattelua.

Ensimmäinen käytännön opetus kuuluu näin: \emph{jokainen valehtelee,} kuten Tohtori House (tai pikemminkin hänen käsikirjoittajansa) selostaa. Oletan, että on rehellisiä valehtelijoita (kuten minä, vakuutan), ja psykopaatteja. Olen niin skeptinen, että pilkkaan koko universumia, ja lopetan vasta, kun se näyttää minulle dataa. Ja silloinkin paras, mitä se voi saada aikaan, on ``väärin'' ``epätoden'' sijaan.

\section{Mieli evolutiivisesti kehittyneenä strategiana}

Kun seurasin kolmevuotista poikaani opettelemassa ja siten pelaamassa Minecraftia, valaistuin. Olen aina kohdellut lapsiani protoaikuisina: keskeneräisinä, mutta toimintavalmiina. Jos et ole nähnyt Minecraftia, kyse on rakennussimulaattorista, joka yksinkertainen, syvällinen ja sosiaalinen. Se on paras malli idealiselle tietokoneohjelman kehitysprosessille, mitä olen tähän mennessä nähnyt.

Vanha ``mieli on savea\vmq{,}'' jonka yhteiskunta muovailee -teoria on mitätöity kaksostutkimuksilla. Tiedämme, että mielemme on määritelty pitkälle jo ennen syntymää. ``Mieli evoluutiivisesti kehittyneenä strategiana, jota yhteiskunta muovaa ja kalibroi'' vaikuttaa paljon tarkemmalta. Haluaisin nähdä tutkimuksia kaksosista, jotka ovat kasvaneet dramaattisen erilaisissa kulttuureissa. Kalibroiko vaikkaka Kinshasan kaduilla kasvaminen paranoiaa suuremmalle kuin esimerkiksi Lontoossa kasvaminen?

Mentaaliset työkalumme ovat selvästi teräviä ja funktionaalisia hyvin varhaisesta iästä lähtien. Noin vuoden vanha lapsi alkaa vuorovaikuttaa äidin lisäksi muiden ihmisten kanssa, ja useimmiten hän tekee sen täsmällisesti. Jos haluat ymmärtää sellaista käsittettä kuin ``luottamus\vmq{,}'' havainnoi nuoria lapsia. Lapset aloittavat luottamalta ainoastaan vanhempiinsa ja sisaruksiinsa. He laajentavat luottamustansa muihin aikuisiin, kuin heidän vanhempansa kertovat, että se on OK. He luottavat toisiin lapsiin implisiittisesti, paitsi jos ikäero on liian suuri. He eivät luota tuntemattomiin eläimiin, paitsi silloin, kun he ovat vanhempiensa kanssa.

Samalla tava ``vapauden'' käsite, joka inspiroinut miljardeja kirjoitettuja sanoja, ja jonka kuitenkin olen itse summannut ``mahdollisuudeksi tehdä mielenkiintoisia asioita toisten ihmisten kanssa\vmq{.}'' Tämä käy selvästi ilmi tyypillisen lapsen pikaisesta havainnoimisesta, ja sillä on syvällisestä arvoa toimivana teoriana. Olen puhunuttästä paljon enemmän kirjassani \emph{Culture and Empire: Digital Revolution.}

Lapsen maailma on pakosta yksinkertainen, ja lapsen mieltä eivät sotke mystiset teoriat. Me olemme tehokkaita teoreetikkoja jo nuorena. Lapset ovat luonnonltansa tieteilijöitä.

Tämä tuo meidät toiseen käytännön opetukseen: \emph{etsi lapsenmielistä intuitiota.} Työkalut, joita tarvitsemme maailman ymmärtämiseen ovat meissä sisäänrakennettuna. Ne ovat kehittyneet pitkän ja heltymättömän tietämysmarkkinoiden kilpailun seurauksena. Olemme hyviä kehittämään teorioitamme, kunnes mystinen ajattelu myrkyttää mielemme.

\section{Kipu on pätevää dataa}

Yksi juttu lapsissa on se, kuinka suodattamattomia he ovat. Kun he eivät pidä jostakin, he kertovat sen. Kohtaamme synkeitä kasvoja, kovaäänistä valitusta ja jopa kirkumista ja itkua. Vanhemmalle nämä voivat tuntua turhauttavilta. On kuitenkin kiehtovaa verrata sitä siihen, kuinka paljon aikuiset hyväksyvät ärsytystä ja kipua aikuiset hyväksyvät ilman silmänräpäytystäkään.

Kaikki teoriat ovat savua niin kauan, kun niitä ei testata, mikä tarkoittaa niiden soveltamista todellisuuteen ja tulosten havainnointia. Hyvä teoria toimii sulavasti ja lähes ääneti. Huono teoria luo jotain, mitä kutsun ``kitkaksi\vmq{,}'' joka näyttäytyy ärsytyksenä, kustannuksina, viiveinä, stressinä. Esimerkiksi kotitaloudessani on kaksi keskenään ristiriitaista teoriaa siitä, mikä on paras tapa mennä kouluun:
\begin{itemize}
\item Paras tapa mennä kouluun on autolla (kolmen minuutin ajomatka).
\item Paras tapa mennä kouluun on kävellä (kymmenen minuutin kävelymatka).
\end{itemize}
Tänä aamuna olimme myöhässä ja meistä tuntui laiskalta. Menimme autolla. Matkassa meni kaksikymmentä minuutti, josta suurimman osan vietin kihisten liikenteelle ja huonoille kuskeille. Sitten tajusin, että jokainen hetki on koe, ja että kipu on pätevää dataa, ja olin taas tyytyväinen. Autoteoria on mitätöity ja voimme taas kävellä.

Kun käytät teoriaa---ja muista, kaikki teoriat ovat jossakin suhteessa väärässä---koet aina jonkinlaista ärsytystä. Joskus ärsytyksen pyyhkii pois suuremmat huolenaiheet. Esimerkiksi ``OK, olen jumissa liikenteessä, mutta ainakin olen suojassa rankkasateelta\vmq{.}'' Tai ``OK, pomoni persereikä, mutta onpahan minulla ainakin joku työpaikka\vmq{.}''

Varsinainen henkinen kipu (vertaa rikkoutuneen lasin päällä istumiseen) on merkki vakavasta kitkasa, kuten selvistä valheista. Pieni ärsytys indikoi kevyttä kitkaa, kuten epätarkkoja oletuksia. Jokainen teoria on joko tieteellinen, ja määritelmän mukaan väärässä ja paranneltavissa, tai se on mystinen ja mahdoton korjata. Mystinen teoria aiheuttaa kipua aina, kun yrität käyttää sitä mihinkään vakavaan tarkoitukseen.

Tämä vie meidät opetukseen numero kolme: \emph{kehitä teorioitasi tai heitä ne menemään.} Mystisen teorian avainindikaattori on se, että et voi poistaa kitkaa parantamalla teoriaa. Joko hyväksyt teorian täysin ja kokonaan, tai et ollenkaan. Mystisiä teorioita voi pitää tarttuvina henkisinä sairauksina.

\section{Psykopaattinen side}

Lienet huomannut, että ihmiset pysyvät psykopaattisissa suhteissa paljon pidempään, kuin mitä olettaisit. Kun aloitimme ZeroMQ-projektin, pääkehittäjät halusivat käyttää C++:aa oman mieltymykseni C:n sijaan. He sanoivat minulle: ``On totta, että kielen opettelemiseen menee kymmenen vuotta, mutta se on paljon tehokkaampi\vmq{.}'' Niihin aikoihin minulla ei ollut antaa vasta-argumentteja. He tarvitsivat viisi vuotta ja yhden oikean open surce -projektin, ennen kuin he tekivät U-käännöksen ja hylkäsivät C++:n.

Psykopaattinen riippuvuus on intuition vastainen mekanismi, joka kannattaa ymmärtää. Arvostamme suhteitamme sen mukaan, kuinka paljon olemme investoineet niihin aikaa, vaivaa, rahaa, kiintumystä, resursseja. Normaalissa sihteessa tämä tapahtuu kahdella tavalla: Aleksandra ja Bob vaihtavat vaihtelevan hienovaraisia lahjoja ja tekevät päänsisäisiä laskuja pitääkseen kirjaa saldoistaan. ``Terveessä'' suhteessa tase on lähellä nollaa, ja laskennon vaiva merkitsee suhteen syvyyttä. ``Sairaassa'' suhteessa toisella puolella on rankasti velkaa (ja laskennon vaiva lähentelee 100\% CPU-aikaa, toimien DoS-hyökkäyksenä).

On olemassa huijaava strategia, joka perustuu tulevaisuuden lupauksiin. Jos Mallory lupaa suuren voiton tulevaisuudessa, Bob investoi sen mukaan. Moni huijaus, muun muassa Espanjalainen Vanki,\linkki{https://en.wikipedia.org/wiki/Spanish_Prisoner} toiselta nimeltään Nigerialaiskirjeet, perustuu tähän ilmiöön.

Homma toimii tällä tavalla: Mallory tekee Bobille suuren lupauksen ja antaa samalla pienen lahjan todisteena hyvistä aikomuksista. Bob vastaa lahjoilla ja Mallory hyväksyy ne yllyttääkseen Bobia. Mallory epäonnistuu lupauksensa pitämisessä joka kerta johtuen aina traagisista ulkopuolisista olosuhteista. Mallory imartelee Bobia, leikkii uhria ja pyytää Bobilta poletti-investointeja.

Bobin mielessä suhde syvenee ja sen arvo nousee. Tilikirjassa on Malloryn suuri lupaus sekä kaikki Bobin investoinnit. Se tuntuu oikealta ja syvältä. Bobille kehittyy yhä vain suureneva kiintymys teoriaan Malloryn lupauksista ja sen sijaan, että hän perääntyisi, hän tekee suurempia ja suurempia investointeja.

Mallory siirtyy kolmansien osapuolien syyttämisestä Bobin syyttämiseen. Kaikki alkaa olla hänen syytänsä. Mallory uudelleenkirjoittaa historiaa selittääkseen, kuinka Bob on kaikkien Malloryn ongelmien aiheuttaja. Bob hyväksyy nämä velat ja tilikirja alkaa kääntyä massiivisesti häntä vastaan. Lupaukset hukkuvat ja unohtuvat. Bob tekee ylitöitä maksaakseen ``velkansa'' ja normalisoidakseen suhteen, mikä mahdollistaa yhä vain pahemmaksi kehittyvän Malloryn käytöksen. Mitä kovemmin Mallory pahoinpitelee Bobia, sitä enemmän Bob investoi, ja mitä enemmän hän investoi, sitä arvokkaampana hän pitää suhdetta ja sitä kovemmin hän sitoutuu Malloryyn. Malloryä ei tietysti kiinnosta pätkääkään. Hänellä on negatiivinen side Bobiin, joka koostuu pääasiasa halveksunnasta ja inhosta.

Tämä side voi kestää vuosia, jopa elämän loppuun saakka. Ulkopuolelle se näyttäytyy käsittämättömänä ja moraalittomana. Mutta kyse on vain siitä hinnasta, jonka maksamme aitososiaalista voimistamme: ne mahdollistavat petojen luokan, Malloryt. Mallory ei ole aina henkilö: se voi olla organisaatio tai joukko mystisiä teorioita.

Psykopaattisesta siteestä on vain kaksi tietä ulos. Ensimmäinen vaihtoehto on, että Bobilla ei ole enää mitään tarjota Mallorylle, joka hylkää Bobin kuin roskan selostaen samalla muille yksityiskohtaisesti, kuinka kaikki oli Bobin syytä. Toinen vaihtoehto on, että Mallory vaatii Bobilta jotakin liian suurta. Tässä pisteessä Bob saattaa herätä tai tuhota itsensä.

Mieltemme Espanjalaisen Vangin hyökkäykseen kehittynyt vastaus on pitää tulevaisuutta massiivisen paljon vähempiarvoisena. Sen piekseminen on kuitenkin triviaalin helppoa: lisäät vain tarpeeksi nollia. Niinpä modernit huijaukset lupaavat aina naurettavan suuria määriä rahaa tai valtaa.

Teoriat, jotka ovat ylen monimutkaisia ja jotka lupaavat tulevaisuudessa palkintoja ovat eräs Nigerialaiskirjeiden muoto. Jos sinun täytyy opetella kymmenen vuotta kieltä, joka lupaa sinulle ``tehokkuutta\vmq{,}'' sinulle valkenee, ettq C++-käyttäjät ovat psykopaattisessa suhteessa kielensä kanssa. C++ on ohjelmointikielten Skientologia. Java, rehottava massauskonto.

Merkillistä on, että kun haastat jonkun, joka on Espanjalaisen Vangin syleilyssä, hän taistelee sinua vastaan. Massiivisen, elämän kokoisen investoinnin kyseenalaistaminen tuntuu äärimmäisen vihamieliseltä teolta. Vain sitten, kun syleily on rikkoutumassa, Bobit nyökkäävät ja hyväksyvät, että he saattavat olla hukassa.

Tästä pääsemme opetukseen numero neljä: \emph{ihmiset puolustavat maagisia teorioita kaikista koviten.} Tietokoneohjelmien maailmassa on hyvin vähän teorioita, jotka eivät ole petollisia tällä tavalla. Suurin osa koodista perustuu mystisiin teorioihin, ja useimmat koodarit ovat Bobeja. Tieteellinen menetelmä puuttuu ohjelmoinnin maailmasta melkein kokonaan (ZeroMQ C4.1 -prosessi on yksi harvinaisista yrityksistä).

\section{Emootioiden rooli}

Lapset saattavat vaikuttaa hyvin emotionaalisilta, mutta tarkempi katsaus paljastaa, että useimmat lapset voivat kytkeä emootionsa päälle ja pois halunsa mukaan. On sanottu, että kaikki lapset ovat psykopaatteja. (Tässä on parempi teoria: kaikki psykopaatit ovat lapsellisia.) Lapset menettävät tämän kyvyn, kuin heille kehittyy empatia. Emootiot ovat sosiaalisia kommunikaatiotyökaluja, tapa manipuloida toiset käyttäytymään haluamallamme tavalla. Ne ovat alkuperäinen, ikivanha kielemme, joka näkyy kasvoillamme ja kehoissamme.

Tätä on helppo demonstroida. Jos joku kävelee eteesi jalkakäytävällä, astut sivuun, hymyilet tai nyökkäät, ja aluillaan oleva ärsytys (se pienenpieni ripe, joka näkyy esimerkiksi kulmakarvoijen nousemisena) muuttuu pikkuruiseksi miellyttäväksi vuorovaikutukseksi. Mutta sama vuorovaikutus kahden auton välillä voi usein johtaa molempien kuskien kokemaan intensiiviseen raivoon. Ero on siinä, että auto muodostaa kuskin ympärille häkin, joka leikkaa verbaalisen ja nonverbaalisen kommunikaation poikki. Autokuskin ja jalankulkijan on helpompi ymmärtää toisiaan kuin kahden autokuskin.

Kun kasvava ärsytys ei saa mitään vastausta, aivot menevä ``taistele-tai-pakene'' -tilaan, johon saattaisit mennä, jos joku tarkoituksella astuisi eteesi jalkakäytävällä. Rattiraivo on perustavanlaatuinen selviytymisvaisto toimimassa väärässä kontekstissa. Jyrkkä reaktio uhkaan on turvallisempi vaihtoehto kuin ei reaktiota ollenkaan. Paits tietysti silloin, kun uhkaa ei ole olemassa.

Jotkut ihmiset (ne Malloryt joista puhuin aikaisemmin) säilyttävät lapsenomaisen kykynsä kääntää tunteensa aikuiseksi kasvaessaan. He projisoivat väärennettyjä tunteita---kateutta, vihaa, pelkoa, raivoa, itsesääliä, surullisuutta---ajaakseen toisia Espanjalaisen Vangin syleilyyn. \emph{Katso, olen hullun mustasukkainen! Se todistaa, että rakastan sinua, joten anna minulle lisää huomiota!}

Jopa vieläkin mielenkiintoisempaa on se, kuinka itse käytämme emootioitamme kohdatessamme huonosti toimivia teorioita. Ilmaisemme usein kivun ja ärsytyksen vihana toisia kohtaan. Kuten ``paras tapa mennä kouluun''-esimerkkini näytti, emootiot ovat pätevää dataa, mutta prosessina ne ovat kelvottomia. Toiselle kuskille huutaminen ei ole Aristotelinen dialogi.

Emootiot saavat aikaan loistavaa taidetta ja karmeaa tiedettä. Voimme itseasiassa mitata mystiikan ja tieteen suhdetta emootioiden voimakkuudella. Huomasimme tämän selvästi ZeroMQ-yhteisössä: kun siirryimme mystisestä prosessista tieteelliseen alkuvuodesta 2011, kaikki emotionaaliset argumentit katosivat.

Selitän myöhemmin, kuinka säännellä tunteita, aktiviteetti jota kutsun ``maadoittamiseksi\vmq{.}'' Kyse on vaikeasta tekniikasta, joka kuitenkin auttaa äärimmäisen paljon etäisyyden ottamisessa. Ja etäisyyden ottaminen omista kokemuksista on ironisesti paras tapa ymmärtää ne kunnolla. Kipu on pätevää dataa vain silloin, kun emootiot ovat hiljaa.

Nähdäksesi kitkaa voit joko havainnoida kipua ja stressiä toisissa ihmisissä tai itsessäsi. Toisten seuraaminen reagoimatta heidän tunteisiinsa on ei-psykopaateille jo valmiiksi vaikeaa. Ja psykopaatit eivät voi ymmärtää tunteita; he kykyenevät ainoastaan lukemaan ja matkimaan niitä. (Arvaan, ilman kovaa dataa, että he ovat surkeita tieteessä.) Omien kokemusten havainnoiminen ilman niihin tunteella mukaan lähtemistä on äärimmäisen vaikeaa. Mutta siellä sinä olet, 24/7 sisällä omassa päässäsi. Jos onnistut kehittämään tämän kyvyn, voit nähdä kitkan käytännössä missä tilanteessa tahansa yksinkertaisesti osallistumalla siihen.

Tämä tuo meidät opetukseen numero viisi: \emph{olet itse paras instrumenttisi.}

\section{Ydinprosessi}

Olemme keränneet tarpeeksi aksioomia ja teorioita teorioista puhuaksemme itse ydinprosessista. Jos olet säännöllinen lukijani, tiedät tämän prosessin jo etukäteen. Kuten moni hyvä teoria, se rakentuu menestyksekkään harjoituksen varaan. Prosessi vaikuttaa yksinkertaiselta:
\begin{itemize}
\item Havaitse kitkaa sosiaalisessa tilanteessa.
\item Löydä tai arvaa taustalla vaikuttavat teoriat.
\item Identifioi teorioiden viat.
\item Parantele teorioita, tai hylkää niitä.
\item Kokeile paranneltuja teorioita.
\item Toista, kunnes tylsistyt.
\end{itemize}
Olemme nähneet, että jokaista tieteellistä teoriaa on mahdollista parannella. Ei ole olemassa ``lokaalia maksimia\vmq{,}'' vain loputtomiin kasvavaa tarkkuutta jonka voit saavuttaa käyttämällä aikaa ja näkemällä vaivaa. Se, mikä on yllättävämpää monille (ehkä heidän äitinsä kertoi heille usein, kuinka erityisiä he ovat) on, että tämä prosessi on täysin mekaaninen. Toinen tapa sanoa tämä on, että yksilöllinen älykkyys on jokseenkin yliarvostettua.

Yksilöllisen älykkyyden arvostaminen on kuin osoittaisi yhtä muurahaista muurahaispesässä ja sanoisi, ``Katsokaa, tuo muurahainen on superfiksu!'' Kyse on mystisestä ajattelusta. Kuten muurahaisetkin, me olemme kollektiivinen laji ja ajattelemme ryhmissä, emme yksilöllisesti. Fiksuin koskaan elänyt muurahainen työskentelemässä yksin ei ole mitään verrattuna muutamaan ``tavalliseen'' muurahaiseen, jotka työskentelevät yhdessä.

Itseasiassa tilanne on vielä tätäkin pahempi. Fiksut ihmiset käyttävät usein älykkyyttään kitkan kompensoimiseen. Ihmisten kyky rationalisoida pahimmatkin tilanteet voi olla hämmästyttävä. Jos on olemassa jokin yksilöllisyyden älykkyyden laji, jota arvostan, on se kyky havaita kitkaa ja kiusaantua siitä. Jopa sekin riippuu muista ihmisistä: yksinäisyydessä ei esiinny sosiaalista kitkaa.

Joten prosessi, jota juuri kuvailin, ei olekaan niin yksinkertainen. Se toimii ainoastaan ryhmän harjoittamana. Olen havainnut melko tyypillisen, lähes ideaalisen syklin ajatusprosessissamme: opi toisilta, kokeile uutta tietämystä (leiki), sovella sitä todellisiin ongelmiin (työskentele), ja sitten opeta sitä toisille. Kutsun tätä lyhyesti nimellä opi-leiki-työskentele-opeta (OLTO).

OLTOlla on mielenkiintoisia ominaisuuksia. Ensinnäkin se on synnynnäinen. Näet sen ilmaantuvan pienten lasten käytökseen ilman opastusta. Jos otetaan taas Minecraft esimerkiksi, prosessi menee näin: opi katselemalla YouTube-videoita, joissa toiset lapset pelaavat, sitten harjoittele yksin varmistuaksesi siitä, että kerätty tietämys toimii, sitten pelaa peliä toisten lasten kanssa, ja sitten opeta toisille lapsille se, mitä olet oppinut. Lapset oppivat pelaamaan Minecraftia toisilta lapsilta. Ei kirjoja, ei kouluja. Ja silti kyse on poikkeuksellisen rikkaasta ja syvästä tietämyksestä.

Toiseksi, se toimii erittäin hyvin. Avoimen koodin projektini rakentavat tietokoneohjelmia sen avulla. Vuosikymmenet prosessin hiomista, ja päädyimme sinne, missä lapseni ovat. Kirjoitamme koodia tiukoissa OLTO-sykleissä. Koodi ei ole ikinä täydellistä, mutta sitä voidaan täydellistettää: tieteellinen teoria, joka on aina väärässä, mutta ei koskaan epätotta.

OLTO on kertaluokkia tehokkaampi kuin klassinen yksisuuntainen oppiminen. Sen päälle rakentuu kokonainen teoria itseorganisaatiosta ongelmien ympärille. Olen puhunut siitä useita kertoja kirjoituksissani.

Ja OLTU hädin tuskin tuntuu työltä. Se on nautinnollista, lähes koukuttavaa. On erikoista nähdä ammattielämäni kaareutuvan takaisin lapselliseen maailmankuvaan. Luonnollisen matkan sijaan tämä tuntuu pakotetulta. Todellakin: pohdiskelen, mitä minulle tapahtuikaan kaikkina noina vuosina.

Ah, kyllä, massakoulutus ja massatyö, nuo teollisuus- ja massamedia-aikakauden kiroukset. Jossakin kohtaa 1800-luvulla vanhat OLTO-rakenteet rikkoutuivat ja ne korvattiin tarkasti eritellyillä kouluilla, yliopistoilla ja työpaikoilla. Leikimme lapsina, opimme nuorina, työskentelemme aikuisina, ja eläköidymme ja kuolemme vanhoina. Opettaminen on jonkin sortin työ joillekin harvoille.

Leiki, opi, tyskentele, kuole. Elämän neljä vaihetta. Tällä neljä-elämää-teorialla on monia ongelmia sen lisäksi, että se vain tekee ihmisistä onnettomia. Listaan muutamia:
\begin{itemize}
\item Se sallii meidän kontribuoida noin 45 vuotta, 60\% elämästämme. Kuulostaako tämä paljolta? ``Videopelin'' asun kautta seitsenvuotiaani opettaa nelivuotiaalleni arkkitehtuuria, fysiikkaa, kemiaa. Minun silmääni näyttää siltä, että OLTO antaa meidän kontribuoida kolme- tai neljävuotiaasta kuolemaan saakka.
\item Se sulkee pois suuria osia populaatiosta. Koulutuksen kallis hinta (ajassa mitattuna) suosii niitä, joilla ei ole mitään parempaa tekemistä, mikä tarkoittaa nuoria miehiä. Monessa kulttuurissa naisten voi olla vaikea investoida korkeampaan koulutukseen.
\item Se suosii mystisiä teorioita tieteellisten sijaan. Jos meinaat viettää vuosia opiskellen, ilman mitään tapaa testata oppimaasi oikeassa elämässä, olet jo pelaamassa Espanjalaista Vankia. Mystiset teoriat tulevat aina olemaan houkuttelevampia, sillä ne voivat yksinkertaisesti valehdella sinulle (``Kyllä, opettele minut ja tulet paljon rikkaammaksi!'').
\item Se kohtelee vanhoja ihmisiä jätteenä. ``Eläköitymisen'' käsite on merkki viha-viha-suhteesta työntekijän ja työnantajan välillä. OLTOn kanssa iällä ei ole merkitystä. Jos kykenet oppimaan, leikkimään, työskentelemäqn ja opettamaan nelivuotiaana, voit tehdä sitä myös 84-vuotiaana (olettaen, että kykenet keskustelemaan).
\item Se ei kykene käsittelemään muutosta ja mahdollisuuksia. Jos koulutus on pääsyvaatimus työhön, ei jo työelämään päätynyt voi enää oppia uusia asioita hänen työnsä ulkopuolelta maksamatta kovaa hintaa.
\item Se jakaa sukupolvet ja ikäluokat. Työssäkäyvät eivät voi opettaa nuoria. Sen sijaan opetus kulkee ``koulutusjärjestelmäksi'' kutsutun kerroksen kautta, joka päättää yhdessä poliittisen eliitin kanssa mitä tietämystä se opettaa. Luotan siihen, että näet mahdollisuudet vallan monopoleille ja kulttien luomiselle.
\end{itemize}
Yhteiskunnan 4-elämää-teoria on näkyvästi epätarkka ja se tuotta loputtomasti kitkaa, jota moni meistä kokee suurena osana elämästämme. Luulen, että jos eivät kaikki, niin monet tähän järjestelmään jumiutuneet haaveilevat paosta, paluusta siihen vapauteen, jota he kokivat lapsena. Jotkut määrittävät oman elämänsä. Se on kuitenkin harvinaista ja vaikeaa. Yhteiskunta tuppaa paheksua kulkureita ja opportunisteja.

Tästä pääsemme opetukseen numero kuusi: \emph{yhteiskunta on kylläinen mystisistä teorioista.} Se nalkuttava ääni, joka mielesi perukoilla sanoo jonkin olevan syvällisesti pielessä elämässäsi, on asian ytimessä. Moderni teknologia ja hintagravitaatio kompensoivat jonkin verran, mutta elämä voisi kuitenkin olla niin paljon parempaa.

OLTO on hyvissä voimissa muodollisten koulutus- ja työjärjestelmien ulkopuolella. Opimme sen avulla Internet-maailmassa. Massiiviset Internet-foorumit toimivat sen kautta. Kun kaikki voivat kontribuoida täydellä nopeudella, on OLTO erittäin tehokas kollektiivinen oppimismenetelmä. Kun ihmiset käyttävät sitä vihaisena, niinkuin Anonymous teki, he muuttuvat poliittiseksi voimaksi ja vakavaksi uhaksi vakiintuneille valtarakenteille.

\section{Iltasatuja}
\begin{quotation}
---Olipa kerran taikalinna korkeilla vuorilla\ldots

---Minkä värinen se oli?

---Onko sillä väliä? OK, sanotaan, että se oli valkoinen. Joka tapauksessa, tässä linnassa asusti prinsessa ypöyksin\ldots

---Ypöyksin? Kuinka prinsessa voisi asua yksin? Kuka teki hänelle ruokaa? Kuka siivosi linnan?

---Hei, haluatko kuulla tarinan vai et?

---OK, OK, olen pidän suuni kiinni.
\end{quotation}
Puhun monissa konferensseissa. Tyylini on muuttunut vuosien myötä. Käytin aikaisemmin kalvoja, kuten useimmat muutkin puhujat. Tyylikkäät kalvot, jotka korostavat avainasioita, herättäen tunteita, kertoen huolella rakentamani tarinan. Nykyään tulen ilman kalvoja ja käsikirjoitusta, ja sen sijaan, että kertoisin tarinani yleisölle monologina, improvisoin dialogin heidän kanssaan.

Dialogi on kovaa työtä, ja se uuvuttaa. Mutta se on helpompaa, kuin kalvojen tekeminen. Olen vaihtanut dialogiformaattiin yksinkertaisesta syystä. Kun käytin monologeja, ehkä jokunen ihminen sadasta ``tajusi'' juttuni. Jokainen näki, mitä selostin. Hyvin harva uskoi minua. Dialogin avulla saan huoneesta puolet tai yli tajuamaan ja reagoimaan.

Dialogi on muinainen muoto, joka on lähes kadonnut nykypäivänä. Olemme niin rakastuneita teknologiaan, että olemme unohtaneet sen. Katso videoni. Klikkaa linkkejäni. Ryhdy kaverikseni Linked-inissä. Mutta luoja varjelkoon jos pyydän sinua kumoamaan hypoteesiani livenä, yleisön edessä. 

Syy, miksi käyn konferensseissa, ei ole myydä tai evankelisoida. Syy on oppia. Dialogi on oleellisesti se tapa, millä suoritamme ydinprosessin testausvaiheen. Tässä on teoriani, anna kun selitän, ja sinä kerrot minulle, jos sinulla on dataa tai havaintoja, jotka falsifoivat sen. Jos kykenen kertomaan tarinani viidelle asiantuntijayleisölle, eikä sitä kumota, se seisoo. Jotta tämä prosessi toimisi, on minun ehdottoman pakko käyttää dialogia.

Mutta---ja tämä selittää, miksi useimmat puhujat rakastavat kalvojaan---jos haluan vakuuttaa sinut mystisestä teoriasta, silloin dialogi on viimeinen asia, mitä haluan. Sen sijaan turvaudun toiseen lapsuusajan skenaarioon, nimittäin iltasatuun. Nukkumaanmenoaikaan hyväksymme minkä tahansa sadun kysymättä. Emme välitä siitä, minkä värinen linna on, sillä olemme jo puolinukuksissa, unelmoiden korkeista muureista.

Tämä on monologin voima: se lähettää yleisön jonkinlaiseen nukkumistilaan missä yleisö hyväksyy mitä vain. Poliittiset puhuja, saarnaajat, luennoitsijat ja halvat myyjät tietävät tämän hyvin. Kuten tavallista, ongelma on siinä, että olemme jollakin tasolla immuuneja tällaiselle yksinkertaiselle manipuloinnille. Heräämme, ja uni on kadonnut. Parhaat kauppiaat kuuntelevat tarkasti yleisöään ennen kuin he askartelevat valheensa jotka vetoavat siihen.

Konferenssikalvot halventavat satua entisestään supistaen sen kaksiulotteiseksi visuaaliseksi pikaruoaksi. Tässä on iskusana. Tuolla on söpö kuva. Tässä on jokunen mieleenpainuva sana! Viihteenä tämä on oivallista, jopa eleganttia. Opetus- ja oppimistyökaluna se on traagisen surkea. Maailmanluokan puhujat taustallaan vuosikymmenten kokemus matkustavat kaukaisiin kohteisiin vain leikkiäkseen teksti-ääneksi-syntetisaattoria 45 minuutiksi. Jos menet konferenssiin, veikkaan, että muista satunnaisia kuvia, mutta et tarinoita. ``Parhaat keskustelut käydään käytävillä\vmq{.}'' Tämä lausahdus kertoo epäonnistumisesta.

Ja niinpä tulemme opetukseen numero seitsemän: \emph{paras oppimis- ja opetuskaava on dialogi.} Tarkemmin sanottuna kaava, jossa yksi henkilö esittää teorian ja toiset yrittävät kumota sen, ja missä sivustaseuraajat aploodeeraavat ja kannustavat. Tämä kuulostaa hyökkäävältä, mutta mikäli osallistuminen on tarpeeksi vapaata, ei ensimmäisellä kerralla ole pakko onnistua, ja niinpä väärässäolemisessa ei ole mitään hävettävää. Itseasiassa, kun joku ottaa työsi ja parantaa sitä, tunnet tyydytystä, et häpeää.

Tällä tavalla tykkään järjestää ohjelmistoprojektini. Sillä tavoin organisoin työpajani. Ja sillä tavoin organisoisin myös konferenssin (ja tulenkin organisoimaan vuonna 2015, jos kaikki menee hyvin): monituntisia sessioita, joissa puhujat esittelevät teoriansa (teknologiasta) ja pyytävät yleisöä kumoamaan ne. Moni muukin asia on rikki totutussa konferenssimallissa. Lihaa toista esseetä varten.

\section{Yksinkertainen ei ole helppoa}

Olen kuullut monimutkaisia malleja kutsuttavan ``ylisuunnitelluiksi\vmq{.}'' Olisi tarkempaa kutsua niitä ``alisuunnitelluiksi\vmq{,}'' sillä monimutkaisuuden muuttaminen yksinkertaisuudeksi vaatii oikeaa työtä. Voin kuvailla tämän prosessin, sillä kirjoitan ohjelmakoodia sitä käyttäen.

Otat olemassa olevan teorian ja sovellat sitä uusiin ongelmiin. Tämä luo kitkaa, minkä voit ratkaista laajentamalla teoriaa. Voit tehdä tämän kerta toisensa jälkeen, ja teoria muutta leveämmäksi ja monimutkaisemmaksi. Aivan kuin lisäisit pizzaan lisää täytteitä.

Jos olet tottunut kitkaan, kuten jotkut ihmiset näyttävät olevan, ei sinua haittaa, vaikka pizzasta tulisi kuinka suuri ja sotkuinen. Teet vain aina tarvittaessa hiukan lisää tilaa mentaaliselle pöydällesi. Mutta jos ärsyynnyt kitkasta helposti, kuten minä ärsyynnyn, kesyttämättömästä pizzakompleksisuudesta tulee ongelma, joka pitää ratkaista. Täytteiden lisääminen täytyy lopettaa, ja sen sijaan on luotava päälle uusi teoria, abstraktio, joka saa aikaan saman asian, mutta on paljon yksinkertaisempi.

Sen sijaan, että meillä olisi ``\emph{pizza, jonka päällä on tomaattia, mozzarellaa, anjovisfileitä ja kaprista}\vmq{,}'' saamme ``\emph{pizza Napoletana}\vmq{.}'' Nyt voimme sekoittaa kokojen teorian (``pieni\vmq{,}''  ``keskikikoinen\vmq{,}'' ``suuri\vmq{,}'' ``Americano'') reseptien teoriaan. Tällaisten abstraktioiden käyttäminen, todellisia pizzoja vastaan testaaminen (``En tilannut sipulia!'') ja parantaminen on helpompaa sekä asiakkaille että ravintolalle.

Tämä vie meidät opetukseen numero kahdeksan: \emph{yksinkertaisuus voittaa aina monimutkaisuuden.}

\section{Sukupuolien välisen kuilun selvittäminen}

Ei ole uusi havainto, että naisista tulee miehiä parempia tieteilijöitä. Niinpä sukupuolten välinen kuilu aiheuttaa joillakin aloilla, kuten vaikkapa ohjelmoinnissa, sekaannusta monissa ihmisissä. Olen alkanut uskoa, että vastaus piilee siinä, että suurin osa ohjelmoinnista ei yksinkertaisesti ole tiedettä. Asia on ollut näin siitä lähtien, kun aloin ohjelmoimaan: se valtava ohjelmistojen bulkki, jota kirjoitetaan ja käytetään, perustuu mystisiin teorioihin.

Useimmat ohjelmistoprojektit epäonnistuvat. Tämä ei ole suunnittelua. Se on käsien heiluttamista ankan sisäelimien yllä. Ooh, kyllä, me kaikki teeskentelemme tietävämme, mitä teemme. Ala on aina itsevarma, erityisesti silloin kun sillä ei ole hajuakaan. Totuus on, että me hädin tuskin saamme kiinni paikkansapitävyyden lautasta epäonnistumisen merellä.

Kuten jo selitin, mystiset teoriat tuppaavat syrjimään. Niiden oppiminen ottaa aikaa, vuosien opiskelua ja harjoittelua, ja ne levittäytyvät yhden tai useamman vuosikymmenen ajalle nuoren aikuisen elämässä. Kaikki eivät voi tehdä tällaista investointia. Eivätkä kaikki ole tarpeeksi tyhmiä hyväksyäkseen mystisiä teorioita ilman ankaraa ``Mitä helvettiä tämä on?''-tyyppistä henkistä oksennusrefleksiä. Erityisesti useimmat nuoret naiset rakentavat todellisia sosiaalisia verkostojaan liian kiireissään omistaakseen vuosia elämästään hölynpölyn opiskeluun.

Sille, miksi teknologian temppelit ovat täynnä nuoria, ei-niin-työllistyneitä miehiä, on syynsä. Mystiset teoriat houkuttelevat nuoria miehiä yhtä voimakkaasti, kuin ne yököttävät nuoria naisia. Kilpailkaamme siitä, kuka tietää salaisimmat faktat. Katsotaan, kuka voittaa yleisen äänestyksen. Hakataan toisemme väittelyssä, ei käyttämällä tieteellistä menetelmää, vaan lainaamalla mystiikkaa toisillemme. Minun näkymättömyysviittani voittaa tulimiekkasi!

Ja papit rakentavat sekaannuttavia pyramidihuijauksia, jotka lupaavat ``hyväksy vain tämä mystinen teoria ja opeta sitä muille, niin saat itsekin valtaa\vmq{.}'' Näin kultit toimivat. Sitä voisi argumentoida, että ainakin teknokultismi absorboi tämän typerän yksinkertaisen yhteiskunnan siivun, jota nuoret miehet edustavat, energian. Jokatapauksessa se tuntuu haaskaukselta.

Joten tulemme opetukseen numero yhdeksän: \emph{todellinen tiede toivottaa jokaisen tervetulleeks,} olipa henkilön ikä, sukupuoli ja syntyperä mikä tahansa.

\section{Tunnista itsesi, tunkeilija!}

Ärsyttäviä mystisiä teorioita on loputtomasti. Eräs sellainen teoria, joka iski minuun, oli Internet-identiteetin teoria. Internetistä on tullut henkilökohtaisten profiiliemme seuraaja. Tämä perustuu valheellisuuteen, jonka selostan. Kyse on hyvästä esimerkistä sellaisesta tilanteesta, missä muutama ihminen käyttää mystisiä teorioita hyötyäkseen kaikkien muiden kustannuksella.

Tosielämässä identiteetillä on syvyyttä. Aloitamme muukalaisina väkijoukossa, ja paljastamme identiteettimme vähitellen. Se on osa suhteiden vakiinnuttamisen prosessia. Näytän sinulle omani, jos sinä näytät minulle omasi. Nimilapun kanssa kulkeminen outoa. Teemme sitä konferensseissa ja kummallisissa juhlissa. Se tuntuu pakotetulta ja epäluonnolliselta, koska se on sitä. Kuka tahansa, joka tulee kadulla vastaan nimilapun kanssa, on{\ldots} joku, jota tulisi välttää.

Kun aloitamme identiteetistä ja sen jälkeen rakennamme sosiaalisen verkoston sen ympärille---niinkuin Internetissä tapahtuu---lopputulos on outo ja ärsyttävä. Se toimisi, jos olisimme kaikki narsisteja, joiden tavoite on kerätä seuraajia. Se toimii esimerkiksi minulle, sillä joitakin vuosia sitten päätin tehdä ``Pieter Hintjensistä'' päätuotteeni.

Kuitenkin toimiakseni tehokkaasti ylläpidän useita muitakin identiteettejä. Eri joukot ovat eri maailmoja. Identiteetti-ensin -teoriassa on useita vikoja:
\begin{itemize}
\item Meidän on pakko kohdata ``Käytänkö oikeaa nimeäni?'' -valnta, jonka kohtaamme harvoin tosielämässä.
\item Se ei anna meille minkäänlaista mahdollisuutta paljastaa identiteettiämme pikku hiljaa yhdessä.
\item Se pakottaa meidät käyttämään nimimerkkejä eri konteksteissa.
\item Se halventaa suhteitamme niin pitkälle, että ``ystävyys'' ei tarkoita enää mitään.
\item Se antaa keskitetyille välittäjille mahdollisuuden käytännössä omistaa ja verottaa suhteitamme.
\end{itemize}
Nämä ongelmat menevät ärsytyksen tuolle puolen ja kumoavat koko teorian. Termi ``Facebook divorce'' saa 145M osumaa Googlessa.

Kuten kaikki mystiset teoriat, identiteetin litistyminen aiheuttaa syrjintää. Se edistää trollien ja huomionhakijoiden asiaa ja häivyttää maltillisia, rauhallisia ja järkeviä ihmisiä. Se on syypää siihen, miksi näemme tulehtuneita ja energiaa tuhlaavia argumentteja sukupuolesta ja rodusta foorumeilla. Pseudonyymeillä ei ole merkitystä. Karmahuora haisee aina yhtä pahalle, käyttipä hänestä mitä nimitystä tahansa.

Joten saavumme opetukseen numero kymmenen: \emph{Webin ydinkonseptit ovat maagisia teorioita,} joita ei voida korjata.

\section{Johtopäätökset}

Tämä esseen tarkoitus on selostaa henkilökohtainen tapani oppia asioita. Kutsun sitä jokseenkin mielikuvituksettomasti Kreetalaiseksi menetelmäksi. Kaikki, mitä olen kirjoittanut, on väärin.

Olen selostanut, kuinka yhteiskunta kerää ja lähettää tietämystä maailmasta teorioiden muodossa. Jotkut teoriat ovat väärässä, ja loput ovat täysin epätotta. Voimme aina parannella virheellistä teoriaa (joita kutsutaan myös tieteellisiksi teorioiksi) testaamalla sitä, tunnustelemalla sen virheellisyyden aiheuttamaa kitkaa ja korjaamalla teoriaa havaintojemme mukaan. Epätosia teorioita (joita kutsutaan myös mystisiksi teorioiksi) emme voi parantaa.

Olemme synnynnäisiä tieteilijöitä. Yhteiskunta taivuttaa useimmat meistä uskomaan magiaan. Jos olemme onnekkaita, onnistumme pakenemaan sitä ja palaamme tieteen pariin. Suuret osat yhteiskuntaa perustuvat edelleen mystisiin teorioihin. Ajan myötä ne kuolevat ja korvautuvat tarkentuvilla tieteellisillä teorioilla. Eräs esimerkki on modernin yhteiskunnan opi/työskentele/leiki-trikotomia. Intuitiivinen ja paljon tehokkaampi lähestymistapa on sekoittaa oppiminen, työskentely, leikkiminen ja opettaminen sulavaksi, jatkuvaksi sykliksi.

Olen antanut ymmärtää, että paras tapa testata sosiaalisen käytöksen teorioita on, että käytät itseäsi instrumenttina. Tämä tarkoitta valmiutta mitata omia kokemuksiasi ilman emotionaalista reaktiota. Selostan omien tunteiden tunteiden sääntelytekniikan---maadoittamisen---viimeisessä osiossa.

Ja olen antanut ymmärtää, että siinä missä monologi (myyntipuhe tai luentokalvot) on hyvä tapa myydä mystisiä teorioita, paras tapa opettaa tieteellinen teoria on dialogi, jossa opiskelijat etsivät kitkapisteitä teoriasta ja jopa kumoavat sen, jos mahdollista. Vapaa ja avoin dialogi voi nopeasti kumota epätosia teorioita, parantaa virheellisiä teorioita ja luoda kokonaan uusia teoreettisia abstraktioita. Minä uskon, että kaikki tietämystyö olisi parasta jäsennellä dialogiksi.

Lopuksi tulemme prosessiin, jota voitaisiin kuvailla ``kivun ajamaksi kehittämiseksi\vmq{.}'' Otetaan teoria, testataan sitä todellista maailmaa vastaan, havainnoidaan omaa ja muiden kipua ja käytetään sitä opastamaan teorian parantelua. Kyse on mekaanisesta sosiaalisesta prosessista, johon ei tarvita erityistä älykkyyttä. Jos haluat väkisellä imarrella yksilöä, ehkä toiset ihmiset ovat herkempiä havaitsemaan kitkaa ja nopeampia kehittämään uusia teoreettisia malleja. Ja toiset ovat parempia jättämään kitkan huomiotta ja selviytymään mystisessä maisemassa.

\section{Maadoittaminen}

Maadoittaminen tarkoittaa kaikkien negatiivisten emootioiden poistamista mielestä, jolloin jäljelle jää onnellisuus ja selkeys. Se on kaikista arvokkaimpia asioita, mitä olen oppinut elämässäni. Ajan kanssa opin, kuinka maadoittaa itseni tietoisesti ja nopeasti. Tosin kuin meditaatio, johon tarvitaan rauhallisia olosuhteita, maadoittaminen toimii jopa silloin, kun ihmiset tarkoituksellisesti provosoivat sinua.

Maadoittamisessa kaikuu kognitiivinen käyttäytymisterapia (KKT), joka on ``epävakaaksi'' luonnehfittujen emotionaalisesti äärimmäisten mielten kanssa työskentelevien terapeutikkojen kehittämä tekniikka. Epäilen, että KKT on hyödyllisempi terapeutikoille kuin heidän potilailleen. Maadoittaminen on yksinkertaisempaa ja se on tarkoitettu itsensä kanssa työskentelemiseen olohuoneissa ja työpaikoilla.

Päämäärä on pysyä rauhallisena ja onnellisena, olipa tilanne kuinka vaikea tahansa, ja tehdä tarkkoja havaintoja itsestä ja muista saavutetussa tilassa. Teorioiden parantelu ja kumoaminen onnistuu vain tarkkojen havaintojen avulla. Emotionaalinen kaaos voi helposti muovautua häkiksi ympärille. Joten maadoittaminen on polku vapauteen ja mahdollisuuksiin. Se mahdollistaa pahimpien tilanteiden kokemisen ``käyttökelpoisena datana'' ``tuhoisan kivun'' sijaan.

Kuvailen ensin maadoittamisen prosessin ja annan sen jälkeen enemmän yksityiskohtia. Aloitamme tunnistamalla vahvimman negatiivisen tunteen ja nimeämällä sen. Sitten takaisinmallinnamme (\emph{reverse engineer}) sen käyttäen teoriaa, jonka mukaan emootiot ovat sosiaalisen kommunikaation muoto. Löydämme juuriteorian, johon emootio perustuu. Kumoamme teorian mielessämme. Emootio katoaa.

Tämä ei ole helppoa. Voit olettaa, että vuosien päästä näiden sanojen lukemisesta et vieläkään onnistu siinä. Joidenkin emootioiden debuggaaminen ja korjaaminen on triviaalia. Jotkut ovat äärimmäisen vaikeita. Kaikki riippuu pinnan alla olevien teorioiden voimasta. Niiden rakentaminen ja purkaminen voivat molemmat ottaa vuosia.

Käyn läpi ``Hello, World''-tyyppisen emotionaalisen hetken. Alex on treffeillä Britan kanssa, ja Brita kertoo, että hänen on lähdettävä kotiin. Itsesääli ja kipu täyttävät Alexin nopeasti. Hän kokee, että Brita satuttaa häntä tarkoituksellisest. Hän on salaisesti vihainen Britalle, ja kun Brita sanoo: ``\ldots koska kissani on kipeä ja minun on tarkistettava tilanne\vmq{,}'' hän kokee syyllisyyttä vahvasta reaktiostaan Britan sanoihin.

Klassisra Alexia. Debugataanpa tämä pikku tarina. Syyllisyys tulee siitä pelosta, että Brita on saattanut huomata vihaisuuden ja reagoida siihen. Vihaisuus on ``pakene tai taistele'' -reaktio kipuun. Kipu tulee hylkäämisen ja yksinäisyyden pelosta. Itssääli on avunhuuto, tunne, jota vauva kokee itkiessään äitiään.

Selvästi Alex, joka ajattelee etupäässä seksin mahdollisuutta Britan kanssa, ei juurikaan kykene säännöstelemään tunteitaan. Onneksi Brita valehteli kissaastansa (itseasiassa hän on naimisissa ja hänen miehensä odottaa häntä) ja hän on itsekin täynnä epäilystä, syyllisyyttä ja pelkoa. Hän ei lue Alexia käytännössä ollenkaan.

Vuosia myöhemmin viisaampi ja lisääntymisestä vähemmän stressaantunut Alex tapaa Britan uudelleen. He juttelevat kahvikupin äärellä. Myöhemmin Brita sanoo taas, että hänen täytyy lähteä. Alex kokee pienepienen itsesäälin potkaisun ja käy läpi pienen sisäisen dialogin. ``OK, itsesääli, mikä on teoriasi?'' hän kysyy. Itsesääli vastaa: ``Jos itket, hän saattaa antaa sinulle halauksen. Ja sitten seksiä!'' Alex pilkkaa itsesäälin teoriaa. ``Onko tuo koskaan toiminut?'' Itsesääli ravistelee vastahakoisesti päätään ja menee matkoihinsa.

Tämä on peruslähtökohta. Listaan seuraavaksi tärkeimmät negatiiviset emootiot ja niiden klassiset kaavat:
\begin{description}
\item[Itsesääli] eli vanhemman huolenpidon itkuhuuto. Itsesääli muuttuu nopeasti vihaisuudeksi kun kuvittelemamme aikuinen ei lohdutakaan meitä. Teoria on: ``Olen alle setsemän vuotta vanha\vmq{.}'' Tämä emootio on kaikista helpoin nimeämitä ja debunkata.
\item[Mustasukkaisuus] eli itkuhuuto huomille kilpaillessa toisen henkilön kanssa. Myös mustasukkaisuus muuttuu nopeasti vihaisuudeksi, kun kohde jättää reagoimatta. Teoria on: ``voin taivutella tämän henkilön viettämään enemmän aikaa kanssani\vmq{.}'' Tämä on vaikeampi debunkata, sillä se usein pitää hetken aikaa paikkansa. Jokatapauksessa, hyvä kumoaminen voisi olla: ``Ei ole viehättävää käyttäytyä lapsellisesti\vmq{.}''
\item[Viha] eli taistelutahdon näyttäminen. Teoria on: ``Vihan näyttäminen taivuttaa toisen henkilön tahtooni\vmq{.}'' Tämän teorian voi rikkoa monella tavalla riippuen kontekstista. Esimerkiksi: ``Se oli vahinko'' tai ``Tämän ihmissuhteen pelastaminen ei ole mitenkään mahdollista'' tai ``Hän haluaa nähdä minut vihaisena'' tai ``Voin ratkaista tämän tuottoisamminkin ilman fyysisen välkivallan riskiä\vmq{.}''
\item[Pelko] eli jossakin häämöttävän vaaran aiheuttaman stressin näyttäminen. Vaara voisi olla hylkääminen ja yksinjääminen tai loukkaantuminen. Teoria on: ``Jos näytän pelkoni, äiti tai isä tulee ja pelastaa minut\vmq{.}'' Yksinkertaisin kumoaminen on: ``Ei, he eivät tule, sinun on selvittävä tästä yksin\vmq{.}'' Hienovaraisempi kumoaminen on: ``Vaara, jonka koet, ei ole todelinen koska \(X\) ja \(Y\)\vmq{,}'' mikä pätee useimpiin sosiaalisiin pelkoihin.
\item[Suru] eli tärkeän asian menetyksestä aiheutuneen stressin näyttäminen. Aivan, kuten pelonkin kohdalla, teoria on: ``Äiti tai isä tulee ja hoitaa asian kuntoon\vmq{.}'' Teorian voi debunkata suhteellisen helposti useimmissa tapauksissa.``Tuoko surullisuus takaisin asian \(X\)? Millaisista mahdollisuuksista oikeasti puhutaan?''
\item[Häpeä] eli toisten ihmisten väheksynnästä aiheutuneen stressin näyttäminen. Teoria on: ``Minut eristetään, jonka jälkeen kuolen yksin kivisessä ahteessa\vmq{.}'' Teorian voi helposti debunkata näin: ``Jos ollaan rehellisiä, ketään ei kiinnosta eristää sinua\vmq{,}'' joskin tämä johtaa itsesääliin. Ehkä diplomaattisempi vaihtoehto on ``Jokainen on niin kiireinen omien ongelmiensa parissa, ettei kukaan kerkeä huomata, että housusi ovat kintuissa\vmq{.}''
\item[Syyllisyys] eli häpäisyn pelon näyttäminen. Teoria on: ``Jos näytän kuinka huonolta virheiden tekeminen tuntuu, ehkä ihmiset antavat minulle anteeksi\vmq{.}'' Tämä on yllättävän helppo debunkata: ``Jokainen tekee virheitä. Pyydät vain vilpittömästi anteeksi ja jatkat eteenpäin\vmq{,}'' tai joskus: ``Itseasiassa et tehnyt mitään väärää; sen sijaan toinen henkilö manipuloi sinua\vmq{.}''
\item[Yksinäisyys] eli perheeseen tai ryhmään kuulumisen kaipuun näyttäminen. Teoria on: ``Ihmiset kaipaavat minua, jos näytän surulliselta\vmq{.} Ja kumoaminen menee näin: ``Se toimii vain vauvoille. Aikuisen tapauksessa hymy vie paljon pidemmälle\vmq{.}''
\end{description}
Omien tunteiden kanssa sisäisen dialogin käyminen voi kuulostaa oudolta. Se kuitenkin on tämä maadoittamisen tekniikka: emootion debunkaaminen näyttäen itselle eksplisiittisesti havainnot ja data, jotka kumoavat emootion pinnan alla piilevän teorian.

Kun aloitat, käytä kynää ja paperia. Mielesi ei kykene käymään sisäistä dialogia kunnolla. Joten etsi tästä listasta vahvinta emootiotasi, kirjoita se paperille, ja sitten kysy itseltäsi: ``Mistä tämä tulee?'' Emootiot usein liipaisevat toisia emootioita, joten saatat joutua purkamaa nuseita kerroksia, ennen kuin pääset juuriteoriaan. Kun teet tätä useammin, huomaat maadoittavasi itseäsi lennosta, joten kerrokset eivät kerkeä kasaantua.

Tätä voi olla hyödyllistä harjoittaa toisten ihmisten kanssa, jos he luottavat sinuun riittävästi. ``Mitä tunnet juuri nyt?'' voi olla mielenkiintoinen lähtökohta. Jos ikinä löydät itsesi toimimassa terapeuttina emotionaalisesti hyväksikäyttävälle ihmiselle, saattavat nämä tekniikat säästää sinut tuhoisalta loppuunpalamiselta, joka tulee emotionaalisesti paatuneoiden myrskyjen riepoteltavana olemisesta.

Laajemmassa mielessä maadoittuminen antaa sinun tavata ja keskustella kenen kanssa tahansa lähtemättä mukaan emotionaalisiin peleihin, joita useimmat meistä pelaavat alitajuntaisesti. Tämä sisältää psykopaatit, jotka yökkäävät kohteidensa emotionaaliseen pinta-alaan. Maadoittuneena et tarjoa mitään mahdollisuuksia manipulointiin, ja sinusta tulee ``harmaa kivi'' jonka psykopaatit vain sivuuttavat.

\section{Kiitokset}

Olen kirjoittanut tätä esseetä vuosien ajan. Se on loputtomien kavereiden sekä kavereiksi pian muuttuvien muukalaisten kanssa käytyjen dialogien tulos. Haluan kiittää kaikkia, jotka ovat auttaneet minua tajuamaan näitä asioita, erityisesti Schikoa, Tuulia ja äitiäni.


\end{document}
