\chapter{Hyökkäys ja vangitseminen}\label{attack-and-capture}

\section{Huolellinen hoitaja}

\begin{tarina}%Sofia on suomentajan keksimä nimi
Sofia kertoo pitävänsä vanhuksista, koska he puhuvat niin paljon. Ja hoitokodin vanhat miehet ja naiset näyttävät pitävän hänestä. ``Hän on aina niin hyvällä tuulella\vmq{,}'' he hehkuttavat toisilleen. ``Niin hyvä kuuntelija!''

Hän on tehnyt paljon töitä hoitajan paperien eteen. Loputtomasti kirjoja, opiskelua, kirjoittamista, tenttejä. Opiskelijakaverit saavat parempia arvosanoja. Sofian arvaus on, että he kaikki huijaavat. Heillä on apureita, ja he lahjovat opettajia. No, se on helppoa. Sofiakin kykenee siihen. Hän ei osaa käyttää tietokonetta, se on niin hankalaa! Voisitko tarkastaa tämän jutun minulle? Jooko?

Viimeinkin kidutus on ohi ja Sofia saa tuon maagisen paperin käteensä. ``Pätevä hoitaja\vmq{,}'' hän toistelee itselleen. ``Pätevä hoitaja!'' Jo iltapäivällä hän lähettelee sähköposteja mahdollisiin työpaikkoihin. Pian hänellä on jo keikka kalenterissa. Sofia tulee liittymään syöpää sairastavasta varakkaasta miehestä huolehtivaan tiimiin.

Sofia pukeutuu työpäivää varten. Hän sitoo hiuksensa taakse ja pukee ylleen tyylikkään asukokonaisuuden, jonka hoitokoti on sanellut. Mustaa ja sinistä, valkoinen hattu, pitkä hame, tummat kengät. Kalliit tummat kengät. Hänellä on kolme kollegaa, joiden kanssa hän vuorottelee. Potilas on seitsemissä kymmenissä, ja hän viettää suurimman osan ajastaan vuoteessa. Ennen lounasta he nostavat hänet ylös, pukevat hänet ja vievät hänet puutarhaan kävelylle. Hän palaa väsyneenä ja nukahtaa. Kollegat suosivat ilta- ja yövuoroja, joina tekemistä on vähemmän. Huolimatta ylimääräisestä työstä suosii Sofia aamuvuoroa, jolloin vanhus on hereillä ja puheliaalla tuulella.

Vanhus on mielenkiintoinen mies. Hän on rakentanut useamman ison bisneksen. He ovat lähekkäin, aina juttelemassa. Sofia kysyy vanhukselta kerran, nauraen: ``Minkäs arvoinen omaisuutesi sitten onkaan?'' Mies nauraa takaisin: ``Se on asia mitä kadun. En koskaan päässyt miljardiin asti\vmq{.}'' Sofia nostaa toista kulmakarvaa ja paheksuu??. ``Hupsu mies, varmasti on muitakin asioita, mitä kadut!''

Ja hänellä on. Kova työ on hyväksi, hän kertoo Sofialle. Mutta se ei korvaa perhettä. Vanhus oli naimisissa, mutta hänen vaimonsa kuoli auto-onnettomuudessa kaksikymmentä vuotta sitten. Hänellä on 45-vuotias poikka. Poika vihaa isäänsä ja käy velvollisuudentunnosta vierailulla kerran viikossa. He eivät puhu mistään. Poika lähtee mahdollisimman pian mustalla Mercedeksellään.Sofia saa tietää, että poika on eronnut vaimostaan ja kieltäytynyt auttamaan isäänsä bisneksessä.

Sofialta kestää melkein kuusi kuukautta sovitella isän ja pojan välit. Lopulta he halaavat ja Sofia hymyilee itselleen. Vanha mies vahvistuu. Hän on ylentänyt Sofian hoitotiimin johtajaksi. Sofia vaihtaa tiimin muut jäsenet omiin tuttuihinsa. Nyt hän on tiimin ainoa nainen. Eräänä iltana poika lähtee ja Sofia menee mukaan. He illastavat läheisessä ravintolassa, ja Sofia menee pojan luo yöksi.

Kun he pian menevät naimisiin, molemmat tuntevat, että aika on kohdallaan. Miksi odottaa pidempään? Kohtalo tekee omat suunnitelmansa. He ostavat maatilan korkealta mäeltä ja suunnittelevat unelmakotiaan. Isä kuolee nukkuessaan jokusen kuukauden kuluttua. He nimeävä lapsensa isän mukaan. Se on poika.
\end{tarina}

\section{Hidasta väkivaltaa}

Olemme nähneet hiukan sitä, kuinka Mallory operoi väijyessään uhrejaan. Hän vaanii kaikista luonnon pedoista häikälemättömimpiä: toisia ihmisiä. Richard Connellin novellissa \emph{The Most Dangerous Game} on päätynyt osaksi populaarikulttuuria.\linkki{sfds} Mutta valikoimalla heikoimmat, haavoittuvaisimmat ihmiset, Mallory tuppaa selviytymään vahingoittumattomana. Suurin osa vaurioista kohdistuu uhreihin. Joskus Mallory valitsee uhrikseen vastaan taistelevan kohteen ja vahingoittuu.

Näin asia on klassisessa peto-saalis-suhteessa. Peto poimii kaikista haavoittuvaisimmat yksilöt. Tämä ajaa saalislajin evoluutiota kohti vahvempaa vastustuskykyä. Se ole ainut evoluutiota ajava tekijä, mutta merkittävä se voi olla molemmille genomeille.

Havaittuaan Alisan Mallory hyökkää ja yrittää vangita hänet. Voimme kuvitella leijonan jahtaavan ja sitten kaatavan seepran. Seepra väistelee vasemmalle ja oikealle. Leijona harppaa ja iskee. Se, mitä tosiasiassa näemme, on paljon puhetta ja sitten yhtäkkisiä ja epätavallisia päätöksiä. Väkivalta on harvoin avointa, mutta se on aina läsnä muodossa tai toisessa. Kun Mallory hyökkää, Alisa taistelee vastaan kaikin voimin.

Psykopaattisen suhteen ydin on ``idealize-devalue-discard'' (IDD) -sykli. Moni kirjoittaja kuvailee tätä sykliä.\linkki{sdfds} Jokainen, joka on ollut tekemisissä psykopaatin kanssa, tunnistaa sen. IDD-sykli alkaa Mallory asettaessa Alisan jalustalle. Hän pirskottaa ylistystä ja kiintymyksen ilmaisuja Alisan päälle. Hetken päästä hän tekee täyskäännöksen ja muuttuu välinpitämättömäksi ja kylmäksi. Ja sitten hän katkaisee kaiken ja menee matkoihinsa.

IDD-sykli voi pyörähtää ympäri muutamassa tunnissa tai vuosikymmenessä. Useimmat kirjoittajat näkvät sen epäonnistumisen merkkinä. Mallory ei kykene luomaan todellisia suhteita. Hän tarvitsee ihailua narsisminsa ruokkimiseen, jonka jälkeen hän tylsistyy, tarina kertoo. Tämä selitys vaikutta virheelliseltä useammasta syystä. Se olettaa, että on olemassa normatiivinen suhteen malli, jota Mallory ei kykene toteuttamaan. Se olettaa, että narsismi on todellinen asia, jolla on halunsa ja tarpeensa. Ja ennen kaikkea se jättää huomiotta ne monet psykopaatit, jotka rakentavat vuosien mittaisia suhteita.

On käytännöllisempää nähdä IDD-sykli yhtenä Malloryn monista työkaluista. Hän ei ole rikkoutunut ihminen. Keskimäärin hän on yhtä menestyksekäs kuin Alisa. Hän suhteensa ovat yhtä ``normaalieja'' kuin Alisanki nsuhteen. Kyse on vain erilaisesta normaaliudesta, joka perustuu pedon ja saaliin dynamiikkaan.

Saalistamisen Wikipedia-artikkeli sanoo:\linkki{sdfd}
\begin{quotation}
\noindent Saalistamisen akti voidaan jakaa neljään vaiheeseen: saaliin havaitsemiseen, hyökkäykseen, nappaamiseen ja lopuksi kuluttaminen.
\end{quotation}
Tunnistusta ennen on vielä yksi vaihe, \emph{piilottelu.} Malloryn on pysyttävä saaliinsa näkymättömissä niin pitkään kuin mahdollista. Jos Bob näkee hänen todelliset värinsä, Bob kävelee tai juoksee karkuun. Paljastuminen rampauttaa. Vain uudessa seurassa, kaupungissa tai maassa voi Mallory taas metsästää.

Piilotteluvaihe on kriittinen. Mutta IDD-sykli jättää sen huomiotta ja kohtelee ``idealize''-vaihetta kaiken alkuna. Sykli jättää huomiotta myös Malloryn päämäärän, mikä on ``kuluttaminen\vmq{.}'' Hän ei syö uhriensa lihaa. Hän vain tyhjentä heidät rahasta, vallasta ja energiasta kunnes kuolema voi kokea olevansa tervetullut.

Viimeiseksi IDD-sykli jättää huomiotta yksityiskohtaisen tavan, jolla Mallory rakentaa kontrollinsa ajan kuluessa. Kyse ei ole vain yhdestä syklistä. Sen sijaan syklejä on pienempiä ja suurempia, jotka toistuvat ja toimivat päällekkäin. Tämän takia aikaskaalat näyttävät niin vaihtuvilta. Mallory saattaa pyöräyttää IDD:n ensitapaamisella. Ja sitten astua vuosikymmenen mittaiseen suhteeseen saman henkilön kanssa. Kuten näytän myöhemmin, tämä ei ole ristiriitaista. Se on osa mekanismia.

Joten IDD-sykli on tarkka, johdonmukainen ja ennustettava. Se on myös epätäydellinen ja pinnallinen. Se ei mallinna suhdetta Malloryn ja hänen lapsiensa välillä. Se ei mallinna suhdetta Malloryn ja hänen apulaistensa (sekundääristen psykopaattien) välillä. Se ei selitä, miksi Mallory työskentelee rikkoakseen ihmiset ennen heidän hylkäämistään.

Itse näen asian niin, että IDD-sykli kertoo tarinan uhrin puolelta. Tarina pitää paikkansa, mutta se on vinoutunut ja epätäydellinen. Meidän täytyy katsoa laajemmin ja syvemmälle ymmärtääksemme kokonaiskuvan. ``Hylkäämiset'' ovat valheita, joiden tarkoitus on upottaa koukkua syvemmälle. Ainoastaan niistä viimeinen on totta. Ja kun Mallory lähtee, suhdetta ei ole enää. Se ei ole pelkästään ohi: Malloryn mielessä sitä ei ollut olemassa \emph{missään vaiheessa.}

Kelataanpa taaksepäin. Mallory on havainnut kohteen. Ennen ensimmäistä IDD-sykliä hän valmistelee hyökkäystään. Valmistelu on niin johdonmukaista, että voit tietää, missä olet Malloryn silmissä katsoen hänen maskiansa. Siinä vaihessa, kun hän idealisoi sinua, olet jo kietoutunut Malloryyn. Selostan, kuinka kietoutuminen alkaa.

\section{Piilottelu}

Mallory ei voi selvitä paljastumisesta. Jo nuoresta iästä lähtien hänen on opittava käyttäytymään ``normaalisti\vmq{.}'' Tämä on vaikea haaste, sillä ``normaali'' on katalan liikkuva tavoite. aivan kuin sosiaaliset ihmiset liikuttaisivat tavoitetta joka päivä häiritäkseen psykopaatteja. Mallory välttää ongelman usein piiottelemalla plain sightissä. Yksi tapa on olla niin kovaääninen, että ihmiset eivät vilkaise toista kertaa. Voimme kutsua tätä huomionhakuiseksi ja narsistiseksi käytökseksi.

En väitä, että jokainen kovaääninen, koreileva, yliampuva henkilö on psykopaatti. Psykopatialle ei ole tarkkaa näkyvää testiä, eikä sellaista tule koskaan olemaan. Tätä evoluution tasapaino on. Mutta jos ajattelet, että Mallory on kylmä, harmaa persoona, olet  väärässä. Hän on usein dramaattinen, mahdoton ennustaa, mystinen, sanoinkuvaamaton. Hän on intohimoinen ja emotionaalinen. Kyse on kuitenkin näyttelemisestä, joka toimii useimpiin, lukuun ottamatta toisia psykopaatteja.

Otetaan esimerkkitapaus, joka selventää asiaa. Olet juhlissa kavereiden kanssa. Paikalla on nainen, joka ei naura, eikä näytä minkäänlaisia muitakaan tunteita. Hän katsoo sinua kuin teurastaja, joka valitsee sikaa tapettavaksi. Välillä hän katsoo muita, ja sitten hän katsoo takaisin sinuun. Hän hymyilee, eikä hymy ole ystävällinen. Tämä olisi Mallory ilman maskia. Kyse on pelottavasta näystä, joka harvoin tulee vastaan.

Ja sitten on mies, jolla on hölmö oranssi hattu päässään, ja joka on pukeutunut hulluihin väreihin. Hän nauraa kovaäänisesti typerä virne naamallaan. Kaikki pitävät hänestä. Hän on juhlien elämä ja sielu, joka metelöi iloisesti ja äänekkäästi, kasvot ja kädet liikkuen. Hän vaikuttaa harmittomalta hupsulta. Jos kohtaat hänen katseensa, koet lämpöä ja liikutusta. Katse on niin syvä, että voisit pudota siihen. Tämä on Mallory pukeutuneena väreihinsä.

Mallory oppii naamionsa ystäviltä ja perheeltä. Hän matkii aksentteja ja äänenpainoja, puheen kuvioita, kasvojen ilmeitä, kehonkieltä. Hän on ammattilaisnäyttelijä, joka uppoutuu rooleihin. Hän pitää näitä naamioita hyllyssä koko elämänsä ajan, säätää niitä, ja pukeutuu niihin tarvittaessa. Naamiot ovat karikatyyrejä, mutta ne ovat uskottavampia kuin niiden esikuvat.

Tarkoitus on harhauttaa ja kontrolloida. Kyse on lavataikurin tekniikasta. Draama, musiikki ja sujuvat sanat saavat yleisön katsomaan tiettyyn suuntaan. Ja niin he eivät huomaa Malloryn todelisia siirtoja. Homma toimii sekä yksittäisen ihmisen että huoneellisen ihmisiä kanssa.

\section{Haastattelu}

Joskus nuorena opiskelijana Yorkissa päädyin kavereideni kanssa pieneen taloon. Siellä ystävälliset ihmiset tarjosivat meille ``ilmaisia persoonallisuustestejä\vmq{.}'' He antoivat meidän leikkiä skifivekottimilla jotka mittasivat stressitasojamme. He istuttivat meidät alas kahdenkeskisiin keskusteluihin. Nuori nainen jutteli kanssani hetkisen siitä, miksi olin paikassa ja mitä halusin elämältäni. Hän kirjoitti ylös nimeni ja osoitteeni ja alkoi tehdä muistiinpanoja. ``Mikä on pahinta, mitä olet koskaan tehnyt elämässäsi?'' hän kysyi minulta. Sävy oli samalla tavalla kasuaali, kuin kysyessäsi joltakulta: ``Mitä söit aamiaiseksi?''

Kysymys oli odottamaton kylmä sormi joka tökki yksityistä mieltäni. Löin sen pois mielestäni. Kiitin naista teekupista, keräsin kaverini ja lähdimme pois. Yksi tyttö jäi hiukan pidempään. Puoli vuotta myöhemmin olimme melkein menettäneet hänet. Kovan vakuuttelun myötä hän lopetti ryhmässä käymisen. Hän lopetti rahan käyttämisen heidän outoon kirjallisuuteen ja kursseihin. Hän jatkoi opiskelujaan.

Nuo ystävälliset ihmiset kyttäsivät häntä ja meitä. He menivät hänen soluasuntoonsa ja seurasivat meitä kaduilla. ``Miksi et käy sessioissa?'' he kysyivät häneltä, eivätkä he enää olleet niin ystävällisiä. ''Miksi satutatte ystäväänne?'' he kysyivät meiltä. ``Hän tarvitsee kursseja\vmq{,}'' he huusivat joskus, kun emme huomioineet heitä. He pyörivät ympärillä yli vuoden ennen kuin he luovuttivat.

En mainitse nimiä, sillä Skientologia saattaisi loukkaantua. Joitakin vuosia myöhemmin tuo psykopaattinen organisaatio vei serkkuni. Hän palasi vasta vuosien päästä eri ihmisenä. Ilo ja nauru olivat mennyttä.

Ihmiset kysyvät minulta joskus, mistä mielenkiintoni Mallorya kohtaan tulee. Ystäväni ja perheeni, sekä minä itse{\ldots} olemme vuotaneet verta ja kyyneliä kerta toisensa jälkeen. Kuinka voisin olla huomaamatta meitä vaanivaa petojen paraatia? En tee siitä henkilökohtaista. En suutu siitä. Sen sijaan dekoodaan, ymmärrän ja puran ne valheiden runkorakenteet joista Mallory on riippuvainen.

Haastattelu on yksi noista valheista. Se alkaa näin: ``Minä välitän sinusta, ja me jaamme intiimin hetken\vmq{.}'' Se päättyy kiristykseen ja kiskontaan. Se on harvoin niin avointa kuin henkilö kirjoittamassa muistioon. Yleensä se tapahtuu baarissa tai yökerhossa tai jossakin sosiaalisessa tilanteessa. Nämä ovat konteksteja, joissa odotamme small talkia ja juttelemme iloisesti. Usein mukana on alkoholia, mikä saa meidät laskemaan kilpemme.

Mallory haluaa tietää, kuinka hyvä kohde Bob on, ja mihinkä suuntaan työstää tilannetta. Aivan kuten autokauppiaskin kysyy: ``Mitä teet työksesi?'' ja sitten: ``Oletko naimisissa?'' Luotaaminen voi olla hellävaraista, mutta se on hellittämätöntä. Hän leikkaa kohti Bobin heikkouksia ja mahdollisuuksia ja riskejä, joita Bobiin liittyy.

Mallory haluaa sulkea pois kohteita, jotka näyttävät hyviltä mahdollisuuksilta, mutta eivät ole sitä. Joten kysymykset keskittyvät Bobiin. Ilmoilla voi olla teatteria ja draamaa. Mutta keskustelu zoomaa sisään ja Mallory lukee joka ikisen reaktion ja nyanssin.

Haastattelu on osa kasvavaa lupausta jostakin asiasta: seksistä, rahasta tai pelastuksesta. Uhka ilmestyy samaan tahtiin lupauksen kanssa. Jos et vastaa, menen pois, ja lupaus katoaa mukanani.

Voit nähdä, kun Mallory haastattelee sinua, jos hän on kiireinen. Jos hän on huolellinen, et voi huomata sitä, sillä se tapahtuu päivien ja viikkojen aikavälillä, jopa pidemmällä ajanjaksolla. Ja haastattelu voi tapahtua selkäsi takana, tuntemasi ihmisten kautta.

Yleensä se on kuitenkin näkyvillä. Hän on kivempi, kuin mitä hänen tarvitsisi olla. Hän hymyilee paljon, ja käyttäytyy kasuaalin dominoivasti. Hän lähestyy sinua, ei toisin päin. Hän kysyy kysymyksiä taustastasi, perheestäsi, suhteistasi. Kysymykset ovat melkoisen intiimejä ottaen huomioon, että kyseessä on satunnainen, rento keskustelu. Intuitiosi kiemurtelee. Mutta vuorovaikutus osuu liipaisimiisi ja saat dopamiinisia mielihyvän potkuja. Joten jatkat keskustelua.

Jos kuuntelet intuitiotasi, saattaa sinusta tuntua hankalalta tämän hymyilevän ihmisen seurassa. Voit tietysti mennä matkoihisi hetkellä millä hyvänsä. Mutta onhan se \emph{mahdollista,} että hän on vilpittömän kiinnostunut sinusta. Kaikkia kiinnostuneita ihmisiä pakeneminen on huono strategia. Joten on olemassa toinenkin puolustus, jota usein käytämme. Se on jatkaa keskustelua, mutta vaihtaa neutraaliin puheenaiheeseen tai toiseen keskustelukumppaniin.

Ihminen, joka nauttii keskustelusta kanssasi, menee virran mukana, olipa keskustelunaihe mikä tahansa. Mahdollisuus ohjata keskustelua satunnaisiin suuntiin on valkoinen lippu. Mallory kääntää keskustelun vaivatta takaisin mieleisekseen. Hän väistää kysymykset omasta taustastaan, tai sitten hän valehtelee ja liioittelee. Hän voi puhua itsestään ja jakaa ``intiimejä'' yksityiskohtia, mutta kaikki tähtää aina siihen, että sinä paljastat itsestäsi lisää. Hänen valehteluaan on lähes mahdotonta huomata, paitsi jos saat hänet kiinni jostakin tietystä epätotuudesta.

\section{Kylmälukeminen ja haulikointi}

Haasttelun aikana Mallory kylmälukee ja haulikoi. Tämä tarkoittaa Alisan elämän merkittävien yksitysikohtien arvailua lyhyessä ajassa. Kyse on mentalistien, huijareiden, autokauppiaiden ja sekalaisten mystikkojen perustyökalusta. Mallory kykenee siihen ilman vaivannäköä tai harjoittelua.

Useimmat meistä ovat eksperttejä toisten ihmisten lukemisessa, toisin tiedostamattaan. Aivomme tulkitsevat kaikkea näkemäämme emotionaalisiksi signaaleiksi ja empaattisiksi reaktioiksi. Emme näe stressiä, vaan tunnemme sen. Mallory näkee kaiken ympärillä ilman emotionaalista linssiä.

Klassinen kylmälukeminen on biletemppu, jota mystikot käyttävät tehdäkseen vaikutuksen ihmisiin. ``Menetit isäsi hiljattain\ldots hän lähettää sinulle viestin\vmq{.}'' Kenttäolosuhteissa kylmälukeminen on tunkeilevampaa.

Useimmat ihmiset ovat luettavissa suurimman osan ajasta. Lukija aloittaa keskustelun ja kysyy oikeat kysymykset. Hän voi sitten tehdä hyviä arvauksia seuraavista asioista:
\begin{itemize}
\item Missä luettavat on kasvanut. Tästä kertoo aksentti.
\item Onko luettava ensimmäinen lapsi vai ei. Tästä kertoo se, kuinka paljon hän kokee stressiä epäjärjestyksestä. Häiritseekö luettavaa myöhästymiset ja sotku vai ei?
\item Onko luettavalla nuorempia sisaruksia, tai tuleeko hän suurperheestä. Tästä kertoo se, kuinka hän kohtelee pieniä lapsia.
\item Ovatko luettavan vanhemmat riidelleet rajusti tai eronnet. Tästä kertoo yleinen itsevarmuus.
\item Mitä asioita luettava on opiskellut. Tästä kertoo se, kuinka hän puhuu ja toimii.
\item Kunka paljon luettava tienaa. Tästä kertoo konteksti ja käytös.
\item Miksi luettava on tilanteessa ja mitä hän odottaa tapahtuvan.
\item Kuinka luettava kokee tilanteen ja lukijan.
\item Mitä luettava haluaa sillä hetkellä eniten.
\end{itemize}
Ja niin edelleen. Useimmat ihmiset voivat opetella kylmälukemisen taidon johonkin pisteeseen saakka. Sinun täytyy opetella maadoittamaan tunteesi, jonka jälkeen tarvitset harjoitusta. Suuri osa tästä on vain avoimuutta sille, mitä ihmiset tuovat ilmi sanoillaan tai käytöksellään. Ihmiset, jotka eivät ole luettavissa, salailevat jotakin syystä tai toisesta.

Malloryn kylmälukutaito on neron luokkaa. Hän yhdistää täydellisen lukemisen haulikointiin. Hän tekee pikaisia, karkeita arvauksia monista eri asioista. Arvaukset joko menevät päin seiniä tai liipaisevat pienen reaktion. Heitä ilmoille viisi vaihtoehtoa, näe reaktio numeroon neljä, ja olet päässyt kotipesälle. Lopputulos näyttää tyrmistyttävältä kyvyltä lukea ajatuksia.

Voit huomata, milloin Mallory haulikoi sinua. Se tuntuu haastattelulta, paitsi on pahempaa. Hän tekee teräviä arvauksia yksityiskohdista, joita hänen ei pitäisi tietää, ja joista hänen ei pitäisi kysellä. Hän tekee näitä arvauksia aivan kuin hän toteaisi absoluuttisia totuuksia. Ei kysymyksiä, hän vain täräyttelee väitteen väitteen perään, kunnes hän osuu oikeaan.

Haulikoiminen liikkuu hienovaraisen skannailun ja avoimen aggression välillä. Mallory voi ammuskella syytöksiä, joiden on määrä tuottaa suurta tuskaa. Kyse on erikoisentyyppisestä keskustelusta. Sillä näyttää olevan kolme tavoitetta. Ensinnäkin satuttaa ja epävakauttaa vastustajaa. Toisekseen, saada reaktioita sitä kautta löytää haavoittuvaisuuksia ja piilotetuja totuuksia. Kolmanneksi, vakuuttaa ulkopuoliset havaitsija Malloryn viattomuudesta ja hyveellisyydestä.

Internet-trollit omaavat moinia psykopaattisia piirteitä. Heissä ei näy empaattisuutta eikä emotionaalisuuta. He tuppaavat olemaan yksinäisiä ja petomaisia. He myös haulikoivat vastustajiaan. He ovat usein niin väkivaltaisia, kuin näppäimistön kautta voi vain olla.\linkki{sdfds}

On merkittävää nähdä Malloryn kuuluttavan suoria valheita horjuttaakseen toisten mainetta. Hän voi olla rauhallinen ja surullinen, tai palaa oikeamielistä suuttumusta. Kaikki on teatteria, joka on suunnattu yleisöön. Parhaat valheet ovat uskottavia ja värikkäitä. Ne tuppaavat olemaan helposti osoitettavissa vääräksi---harvempi vain vaivautuu tekemään sitä. Mielemme ovat kehittyneet olemaan samaa mieltä auktoriteettien kanssa. Hyväksymme ilmiselviä valheita, jos ne tulevat meihin nähden ylempiarvoisesti käytäytyvän henkilön suusta. Ilmiötä kutsutaan Asch-efektiksi.\linkki{sdfsd}

\section{Imitaatiopeli}

Kädellisillä ja linnuilla on sosiaaliset vaistot toisten käytöksen kopioimista varten. Päämekanismeja näyttää olevan kolme kappaletta: konvergenssi (convergence), peilaaminen (mirroring) ja imitoiminen (mimicking). Jokaisella mekanismilla on evolutiivinen järkensä. Jokainen on työkalu psykopaatin käsissä.

Olemme sosiaalinen laji ja identiteettimme elää ryhmissä, joihin kuulumme. ``Ryhmän'' käsitteemme skaalautuu kahdesta hengestä miljooniin. Monet mekanismit toimivat näillä kaikilla tasoilla. Konvergenssi on yksi niistä.

Osa ryhmän identiteettiä on enemmän tai vähemmän johdonmukainen kulttuuri. Ennen kaikkea tämä tarkoittaa ulkoasua, käytöstä ja kieltä. Ryhmät eivät kuitenkaan tähtää vain johdonmukaisuuteen. Ne tähtäävät myös omaleimaisuuteen, erityisesti verrattuna läheisiin kilpailijoihin.

Sille, miksi ryhmät pyrkivät johdonmukaisuuteen ja omaleimaisuuteen on olemassa useampia syitä. Jo valmiiksi ryhmään kuuluvilla on intressi ryhmän laajentamiseen. Koko on valtaa. Kulttuuri toimii brändinä. ``Tätä me olemme!'' on rekrytointimainos. Ja ``Emme ole kuin he!'' estää jäsenten loikkaamista toisiin ryhmiin samalla alueella.

Niillä, jotka liittyvät ryhmään, tai jotka ovat syntyneet siihen, on suuri motivaatio mukautua siihen. ``Erilaisuus'' paljastaa yksilön hylkäämiselle ja ulkopuolisten väkivallalle. Kun ryhmien välillä on konflikti, ensimmäisiä kohteita ovat syrjäytyneet.

Ryhmässä piilottelun evolutiiviset edut ovat vanha kulttuurillinen moottori. Ihmiskielen ja -käytöksen monimuotoisuus ei ole satunnaista kaaosta. Se nousee jokaisen ryhmän tarpeesta löytää oma ekologinen rakonsa. Se näkyy aksenteissa, murteessa ja meemeissä.

Se näkyy ruokatabuissa, mikä ovat usein absurdin monimutkaisia. Mielivaltaiset, monimutkaiset säännöt ovat kontrollityökalu, niinkuin selitän luvussa \ref{the-feeding}. Ruokatabut kehittyivät mitä luultavimmin keräilijä-metsästäjä-menneisyydessämme. ``Tämä on myrkyllistä'' on helpompi selittää, kuin ``Tämä on kiellettyä\vmq{.}'' Voimme oppia inhoamaan kerran ja se pelastaa henkemme sata kertaa. Sama vaisto antaa heimoille mahdollisuuden kieltää ruokia, joita naapuriheimot syövät. Se estää ihmisiä lipumasta pois.

Konvergenssi tapahtuu neuovottelemalla ja imitoimalla. Dominoivat yksilöt luovat kuvion, jota vähemmän dominoivat seuraavat. Kun ihmiset jakavat valtaa, he neuvottelevat painotetun keskiarvon. Siitä tulee termi ``slavish conformity\vmq{.}'' Jopa lapsi yrittää neuvotella vanhempiensa kanssa. Lopputulos on ryhmäkonsistenssi lyhyellä tähtäimellä ja evoluutio pitkällä tähtäimellä.

Konvergenssi ottaa aikaa ja energiaa ja on yksilöiden välinen neuvottelu. Tämä tarkoittaa, että voit nähdä, kuinka läheisiä ihmiset ovat siitä, kuinka he esiintyvät, toimivat ja puhuvat.

Sekä miehet että naiset konvergoivat, omilla tavoillansa. Miehet tuppaavat konvergoimaan ryhmän kieltä käytöstä ja ilmiasua kohti. Naiset tuppaavat konvergoimaan toisia naisyksilöitä kohti. Harvoin näkee kahden miehen pukeutuvan samalla tavalla, paitsi jos he ovat osa jotakin laajempaa ryhmää. Mutta kahden naisen voi nähdä usein konvergoituvan kohti toisiaan. Havaitse kaksi naista yhdessä, ja voit usein tietää, kuinka hyvin he tuntevat toisensa. Se näkyy hiuksissa, vaatteissa, kengissä, tarvikkeissa ja kehonkielessä.

On ihmisiä, jotka eivät konvergoi. Tässä mielessä erilaisia ihmisiä on ainakin kolmea eri tyyppiä. On heitä, joilla on jonkinasteista autismia. On synnynnäisiä johtajia. Ja on psykopaatteja. Selostan nämä tyypit, jotta näet eron.

Autistiset ihmiset eivät kykene lukemaan sosiaalisia vihjeitä. Tämä tarkoittaa sitä, että he eivät konvergoi, olipa konteksti mikä hyvänsä. He näyttävät yksinäisiltä, epäsosiaalisilta ja ``oudoilta'' monilla eri tavoilla. Populaari on demonisoinut yksinäiset ihmiset epävakaiksi ja vaarallisiksi, mutta kyse on myytistä. Sellaisilla yksilöillä on keskimääräistä suurempi riski joutua syrjityksi.

Synnynnäiset johtajat konvergoivat silloin, kun he liittyvät korkea-arvoisempien seuraan. He eivät konvergoi silloin, kun he tapaavat potentiaalisia seuraajia. Tämä pakottaa toiset lisätyöhön konvergoinnin eteen. Arvostamme suhteitamme sen mukaan kuinka paljon investoimme niihin. Joten kovempi työ konvergenssin eteen luo syvemmän sitoumuksen johtajaan. Ja tästä ryhmä rakentuu, kun osallistuja konvergoituvat yhteen henkilöön.

Tässä vaiheessa Mallory käyttäytyy pitkälti synnynnäisen johtajan tavoin. Mutta hän alkaa hyväksikäyttää ja kaltoinkohdella jäseniä melkein välittömästi. Synnynnäinen johtaja kohtelee ja suojelee ryhmää kuin perhettään. Mallory kohtelee ryhmää kuin omaisuuttaan tai leluaan. Tämä on narsismi, yksi hänen maskeistaan.

Äärimmäisissä tapauksissa hän pakottaa muut äärimmäisiin tekoihin kovergoinnin vuoksi. Johdonmukaisen asun, kielen ja käytöksen pakottaminen on hyväksikäytön muoto. Se rikkoo yksilön identiteetiin ja minäkuvan. Tämä on puhdas psykopaattinen piirre, yksilöissä ja organisaatioissa.































