\chapter{Metsästys}\label{the-hunt}

\section{Juhlakalu}

\begin{tarina}

Hän vaeltaa lempipaikkaansa, joka on suuri baari sekä ravintola. Paikka on suosittu kovaäänisen hauskanpitoa etsivän nuorison keskuudessa. Kesäinen lauantai-ilta on alkamassa. Hän juttelee portsarin Miken kanssa. Suuri mies nauttii rutiinin keskeyttävästä hetkestä ja kertoo pieniä palasia elämästään. Ex-tyttöystävä ja odottamaton lapsi. Pomo. Hän heittää portsarin kanssa ylävitoset ja lopettaa keskusteluhetken. ``Törmätään myöhemmin, Mike, käyn hakemassa itselleni kaljan\vmq{.}'' Hän liikkuu baarin toiselle puolelle.

Ulkoa terassilta hän löytää suuren pyöreän pöydän ja istuu siihen kylmän pullonsa kanssa. Ihmisiä valuu sisään. Hän katsoo heitä. Sekalaista sakkia, niinkuin täällä päin Texasia voi olettaa. Maahanmuuttajat tulevat tänne joka puolelta Yhdysvaltoja ja kauempaa. Miehet tsekkailevat naisia. Naiset käyttäytyvät kuin he eivät huomaisi miehiä. Joitakin pareja. Jotkut seisovat yksin selkä seinää vasten, kehonkielen huutaessa, ``Kumpa olisin pitempi\vmq{.}''

Paikka on kohta täynnä. Pienehkö nuorten miesten porukka istuu rappusilla pyöreän pöydän vieressä. Valkoisia, mustia, latinoja. He näyttävät viihtyvän huonosti. Hän kääntyy ja kysyy: ``Mistäs te kaikki olette kotoisin?'' Sotilaita läheisestä armeijan tukikohdasta viettämässä iltaa. Hän viittoo käsillään pöytäänsä. ``Liittykää seuraan\vmq{,}'' hän sanoo, ``pöydän ääressä on parempi meininkin\vmq{,}'' hän hymyilee. Miehet hyväksyvät ehdotuksen ja nousevat ja siirtyvät pöytään. He iloitsevat tuoleista ja siitä, että he olivat tervetulleita pöytään.

Nämä nuoret miehet ovat fiksuja ja uteliaita. Heitä ei ole vielä lähetetty sotaan, ja he ovat optimistisia ja luottavaisia. Hän kertoo heille polveilevia tarinoita ulkomaan seikkailuistaan. He nauravat jännityksestä. Hän pysähtyy, asettaa kätensä pöydälle ja toteaa itsestäänselvyyden. ``Tarvitsemme naisia!'' Yksi sotilaista osoittaa leuallaan: ``Nuo kaksi?'' Hän kääntyy nähdäkseen kaksi nättiä tummahiuksista naista. He näyttävät tylsistyneiltä ja epävarmoilta. ``OK, älkää liikkuko!'' hän sanoo miehille ja nousee ylös.

``Hei, leidit, miten menee?'' hän kysyy heiltä, kuuntelematta vastausta, mikä on aina ``mikäs tässä'' tai ``ihan jees\vmq{.}'' Hän tarkkailee mahdollisia merkkejä kiusallisuudesta. Naiset vaikuttat pitävän miehen kanssa juttelemisesta. ``Odotatteko jotakuta?'' hän kysy, ja vastaus on kieltävä: he ovat liikkeellä kahdestaan. Hän rypistää otsaansa ja tarkastelee heidän piirteitään.

``Mistä olette kotoisin?'' hän kysyy. ``Arvaa\vmq{,}'' vastaa toinen nainen nauraen. Hän yrittää paikallistaa heidät. Tummat kulmakarvat, tummanvihreät silmät, vaalea iho, korkeat poskipäät. Libanon? Georgia (maa, ei osavaltio)? Naiset nauravat ja pudistelevat päitään. ``Ei\vmq{.}''

``Haluaisitteko liittyä seuraamme? Pöydässämme on tilaa\vmq{,}'' hän sanoo, pyyhkäisten kädellänsä ilmaan laajan kaaren. He katsovat komeaa, pystytukkaista miestä kasvoihin, kohauttavat olkiaan, ja hyväksyvät tarjouksen. ``Miksipä ei!''

Naiset istuvat hänen viereensä. Mies puhuu heille, arvaillen lisää väärin. Venäjä? Armenia? He nauravat. Sotilaat ostavat heille juomia. Kaikki ovat iloisia, juhlat ovat parhaimmillaan. Lopulta mies tunnustaa tappiansa, jolloin naiset kertovat olevansa kotoisin Intiasta. Hän on shokissa, vaikuttunut ja lumoissaan.

``Kaunein nainen, jonka olen koskaan tavannut, oli kotoisin Georgiasta. Juttelimme viitisen minuuttia, ja halusin hänen kanssaan naimisiin siinä paikassa. Teillä molemmilla on samat piirteet. Olin varma, että olette Georgiasta! Mutta että Intiasta\ldots vau, Intiasta!''

``Kyllä, Intiasta!'' he nauravat imarreltuina. He viihtyvät tilanteessa. He juttelevat läpi yön. Baari menee kiinni ja asiakkaat siirtyvät parkkipaikalle. He lähtevät viimeisenä. Sotilaat hyvästelevät ja menevät matkoihinsa, ja nämä kaksi naista jäävät miehen seuraan. ``Haluaisitteko mennä jonnekin muualle?'' hän kysyy. ``Autoni on tuolla\vmq{.}'' Hän painaa avaimen nappulaa ja uuden avo-Mustanging valot välähtävät.

Myöhemmin toinen naisista kysyy: ``Kuinka kauan oletkaan tuntenut nuo kaverit?'' Hän vastaa: ``Aa ne, en pitkään, tapasin heidät vasta tänä iltana\vmq{.}'' ``Mitä?!'' hän huudahtaa shokeeraantuneena. ``Luulimme, että olet tuntenut heidät vuosia! Olitte kuin parhaat ystävät!''

\end{tarina}

\section{Sosiaalinen peto}

Psykopaatit kykenevät kohdistamaan lumoavan määrän voimaa toisiin ihmisiin. Se on kuin kahden kultti. Kun tapaamme sellaisia ihmisiä ja elämämme alkaa kietoutua heidän elämäänsä, meistä tuntuu siltä, kuin kohtalo kuljettaisi meitä. Tunne on kummallinen sekoitus varmuutta ja kontrollin menetystä. Aivan kuin putoaisi kovassa tuulessa. Uskonnollisen fanaattisuuden kuuma tuli. Ja se näyttää aina päättyvän kyyneliin.

Kysymys, jota muut usen kysyvät, on ``Miksi?'' Suhteet psykopaatin ja sosiaalisen ihmisen välillä ovat niin tuhoisia ja karvaita. ``Miksi'' on hyvä ruutu aloittaa. Kun kykenemme vastaamaan siihen, voimme alkaa kysyä ``Kuinka?'' ja ``Kuka?'' ja muita syvempiä kysymyksiä.

Tarinani jokaisen luvun alussa kertovat aina jonkunlaisesta pedosta. Jokainen psykopaatti toimii tällä tavoin. Psykopaatit metsästävät toisia ihmisiä. He hyökkäävät ja sieppaavat uhrinsa. He elävät uhriensa ajasta, resursseista, vallasta ja energiasta. He hankkiutuvat eroon jäänteistä. Ja he liikkuvat eteenpäin.

Väkivalta on piilevää. Joskus se päättyy uhrin itsetuhoiseen käytökseen, jopa itsemurhaan. Yleensä se päättyy masennukseen. Jokainen suhde sosiaalisen ihmisen ja psykopaatin välillä noudattaa samaa kaavaa. Vaikuttaa siltä, että poikkeuksia ei ole; ``kivoja'' psykopaatteja ei ole olemassa. Psykopaattius tarkoittaa sitä, että on peto.

Tämä ei ole metafora. Tämä on avain psykopatian dekoodaamiseen. \emph{He ovat petoja tai loisia, jotka elävät toisten ihmisten kustannuksella.} Ilman tätä avainta psykopatia on mystinen ja hämmentävä ilmiö. Kuin muinainen kirjoitus täynnä symboleja ja glyphejä. Teksti, joka vaikuttaa niin moneen meistä, ja jonka avaaminen on kuitenkin mahdotonta. Avaimen avulla voimme lukea tarinat ja voimmme ymmärtää.

Siinä, missä kuvailuni koskevat usein yksittäisiä suhteita, kaavat toimivat monissa tilanteissa. Näemme ne kulteissa, hyväksikäyttävässä yritystoiminnassa ja muissa petomaisissa organisaatioissa.

\section{Mallory, Alisa ja Bob}

Tietoturva-alalla vihamielistä hyökkääjää kutsutaan silloin tällöin Malloryksi. Samalla viattomia uhreja kutsutaan Alisaksi ja Bobiksi. Käytän näitä nimiä tässä kirjassa, jotta teksti olisi helpommin luettavaa ja pureskeltavaa.

Mallory voi olla mies tai nainen. Kirjoitan naisesta tai miehestä sen mukaan, mikä sattuu sopimaan tilanteeseen. Hän on aikuinen, vähintäänkin 14--16-vuotias, ja alle 70-vuotias. Mallory on psykopaatti.

Alisa ja Bob ovat altruistisia, sosiaalisia ihmisiä. He ovat Malloryn huomion kohteita.

\section{Kävele näin}

Ensitapaamisesi Malloryn kanssa on intensiivinen, henkilökohtainen ja syvä kokemus. Siis Alisalle ja Bobille. Mallorylle se on merkityksetön, kasuaali refleksi. Kun hän sanoo ``Moi'' sadalle ihmiselle, 96\% vaikuttuu. Hymy, silmät, tuon yksinkertaisen tervehdyksen \emph{syvyys}. Mallorylle tervehdys ei tunnu missään, eikä se vaadi häneltä minkäänlaista emotionaalista panostusta.

Tämä on se ``karisma'' josta puhutaan. Se ilon heijastus kun tapaamme jonkun, josta välitämme kovasti. Alisa ja Bob eivät voi väärentää sitä. He näyttävät sen ainoastaan niille, joista he välittävät. Se, että välittää \emph{jokaisesta} tapaamastaan ihmisestä on äärimmäinen perspektiivi, jonka löytäminen vie vuosikymmeniä. Mallory jäljittelee tätä refleksinomaisesti nuoresta iästä lähtinen. Se ei vaadi opettelua. Se on hänen ensimmäinen selviytymisen sääntö: \emph{muiden on ihailtava sinua.}

Vaikutus on niin voimakas, että voit käyttää sitä psykopaattien tunnistamiseen luonnossa. Palaan tähän luvussa \ref{hunting-mallory}. Useimmat, jotka törmäävät Malloryyn, ovat taipuvaisia kokemaan pikkuruisia mielihyvän purkauksia. Jos henkilö on edes pikkuruisen yksinäinen, houkuttelee se hänet takaisin keskusteluun. Samalla Mallory skannailee uusia kohteita. Siihen ei tarvita keskustelua. Hän näkee ihmisten haavoittuvuudet kehonkielestä.

Sosiaaliset ihmiset voivat oppia tämän kyvyn vuosien harjoittelun myötä. Mallory ei tarvitse harjoittelua. Se on yksi hänen monista sisäsyntyisistä kyvyistä. \emph{Forbes}-lehti kirjoittaa:\linkki{sdfsd} ``Vaikuttaa siltä, että psykopaatit eivät tarvitse meditatiivista harjoitusta ollakseen suhteettoman tarkkaavaisia{\ldots} toisten heikkouksien suhteen\vmq{.}''

Heikkoutemme näkyy erityisesti kahdessa asiassa. Ensinnäkin yksinäisyys ja yksinäisyydestä kertova kehonkieli ovat osoituksia heikkoudesta. Toiseksi, pelon ja epävarmuuden näyttäminen ja hyväksikäytön kokemusten näyttäminen kertovat heikkoudesta.

Monet uskovat, että hyväksikäytön uhreista tulee itsekin hyväksikäyttäjiä. Prosenttiosuus on kuitenkin vain kymmenen luokkaa,\linkki{sdfds} paitsi jos hyväksikäyttäjä ja hyväksikäytetty ovat samasta perheestä. Siinä tapauksessa prosentti nousee merkittävästi. Ja joka kolmas näistä aikuisistä hyväksikäyttäjistä oli lapsena julma eläimiä kohtaan.

En epäile, etteikö psykopaatit hyväksikäyttäisi, laiminlöisi ja kiduttaisi henkisesti omia lapsiaan, ja etteikö moni näistä kasvaisi psykopaatiksi. Kyse on mekanismista, jonka olen havainnut, ja jonka selostan myöhemmin. Mutta tämä ``hyväksikäyttö aiheuttaa hyväksikäyttöä''-malli jättää huomiotta ne 90\% lapsista, jotka kokevat seksuaalista hyväksikäyttöä, ja joista kuitenkin kasvaa aikuisia, jotka eivät vahingoita toisia. Mielestäni tämä johtuu sosiaalityöntekijöistä, joita nuoret psykopaatit ovat onnistuneet huijaamaan. ``Isäni hyväksikäytti minua, ja siksi satutan muita\vmq{.}'' Psykopaatit eivät koskaan ota vastuuta teoistaan.

Todellisuudessa hyväksikäytön uhrit ovat taipuvaisia todistamaan oman elämänsä traumoja sanomatta sanaakaan. Tai, kuten Joanna Moore kirjoittaa kirjassa? \emph{The Faces of Narcissism},\linkki{sdsf} ``on helppoa syyttää vihaista uhria ja tukea rauhallista hyväksikäyttäjää\vmq{.}''

Menneisyydessä koettu hyväksikäyttö on ensisijainen indikaattori tulevaisuudessa koettavalle hyväksikäytölle. Hyväksikäyttävästä perheestä tulemin leimaa meidät pelolla ja epävarmuudella. Hyväksikäyttävä työnantaja tai puoliso saa aikaan saman. Pelkomme ja epävarmuutemme on kuin neonvaloa hohtava ``Syö minut!''-kyltti kaikille ohikulkevilli psykopaateille.

Toisten ihmisten pelko näkyy kehonkielessämme.\linkki{sdfds} Hyväksikäytön uhrit nostavat jalkojansa korkeammalle kävellessään. He ottavat keskimääräistä pitempiä tai lyhyempiä askelia. He sätkyttelevät käsiänsä ja jalkojansa. He välttelevät katsekontaktia. He näyttäväþ alistuvia ja puolustavia kehon asentoja.

Kaikki nämä vihjeet ovat helppoja luettavia, jos lukijalla on sopivanlainen mieli. Rikollisista psykopaateista tehdyt tutkimukset osoittavat, kuinka psykopaatit poimivat tällaisia vihjeitä.

Minulla ei ole numeroita siitä, kuinka nopeasti tämä tapahtuu, ainoastaan anekdoottisia kertomuksia. Voisin arvata, että Mallory kykenee käymään läpi sata ihmistä suurin piirtein kymmenessä minuutissa.

\section{Suuret, siniset munat}

Todennäköisiä kohteita etsivä Mallory kävelee läpi väkijoukon. Hän projisoi seksuaalisuuttaan vain hiukan muita naisia kovemnin. Hän etsi yksinäisiä, menestyviä miehiä. Hän tarkkailee, kuinka miehet katsovat häntä, ja näkee yhden hermostuneen reaktion, joka kiinnittää hänen huomionsa. Hän nousee ylös ja heilauttaa hiuksiansa, hengittää sisään, hymyilee hänelle. Hän katsoo miehen kasvoja. Hän katsoo hiukan liian pitkään. Hän hymyilee itselleen.

Valittuaan Bobin kaikkien potentiaalisten kohteiden joukosta, Mallory tekee siirron. Mitään näkyvää takaa-ajoa, juoksemista tai kirkumista ei tapahdu. Siirrot eivät kerro totuutta. Mallory liukuu sisään Bobin elämään kuin kauan kadoksissa ollut ystävä. Hän vaikuttaa niin kivalta, harmittomalta ja vilpittömältä.

Hän avaa pelin laajalla paletilla taktiikoita, joka riippuu kontekstista. Nämä hyökkäykset toimivat sekä hyökkääjässä että uhrissa vaistojen tasolla. Hän aloittaa laajalla, keskittämättömällä luotaamisella. Kun Bob vastaa vaistonvaraisesti, Mallory siirtää ja säätää peliään ja painaa kaasua.

Vaistonvaraisen käytöksen liipaisijat ovat yleensä yksinkertaisia karikatyyrejä. Evoluutio on tässä mielessä laiska. Esimerkiksi monet ihmiset pelkäävät hämähäkkejä niin kovasti, että ne saavat heidät kirkumaan. Liipaisin sijaitsee geeneissämme monella jalalla varustettuna mustana pisteenä. Pelkäämme tiettyä tapaa, jolla jalat liikkuvat. Piirretty hämähäkki, joka kävelee oikealla tavalla, on yhtä pelottava kuin oikea hämähäkki. Laita se kävelemään kuin ihminen?, ja se näyttää harmittomalta. Liioittele hämähäkkikävelyä ja piirretty hämähäkki on oikeaa pelottavampi.

Kun liipaisimen eristää ja sen voimakkuutta kasvattaa, kasvaa myös reaktio. Vain taivas on rajana. Otetaan esimerkiksi lajimme makeanhimo. Reagoimme vaistonvaraisesti fruktoosiin, jota kasvit pumppaavat hedelmiin pieninä annoksina. Makeus iskee samoihin aivojemme alueisiin kuin kokaiinin kaltainen huume. Luonnossa tämä ajoi kaksijalkaiset edeltäjämme syömään niin paljon hedelmiä kuin he vain kykenivät löytämään. Sitten jalostimme aina vain makeampia hedelmiä. Sitten opimme jalostamaan sokeria ja aloimme lataamaan sitä ruokavalioomme. Syömme satoja paunoja sokeria vuodessa, itsetuhoon asti.

Vaistonvaraisen fruktoosihimomme rajat eivät tulleet missään vaiheessa tätä tarinaa vastaan. Sen sijaan mitä enemmän kulutamme, sitä onnellisemmaksi näytämme tulevan.

Tämä eskaloituva reaktio tiivistettyyn liipaisimeen on tunnettu ilmiö nimeltä ``supernormaali stimuli\vmq{.}''

Pedot ja loiset ovat erikoistuneet käyttämään supernormaalia stimulia saaliiseensa. Se pakottaa itseään rankaisevaan ja epäloogiseen käytökseen, kunnes liipaisumekanismin ymmärtää. Joten parasiittinen lintulaji saattaa kaapata ne liipaisimet, joita nuoret poikaset käyttävät ruoan kinuamiseen. Esimerkiksi avoin punainen suu. Parasiitti imitoi ja liioittelee tätä liipaisinta, joka saa lintuemon ruokkimaan loispoikasta ennen sen omia poikasia.

Merien syvimmissä vesissä krotti roikuttaa pimeässä loistavaa kirkasta syöttiä. Tämä liipaisee saaliskalan uimaan kohti sen hampaista suuksi kutsuttua ansaa.

Tai mietitäänpä puussa pesivän laululinnun munia. Ne ovat usein vaaleansinisiä munia, joissa on tummanharmaita tai ruskeita pilkkuja. Tämä nimenomainen värikaava liipaisee naaraan tai koiraan istumaan munan päällä. Ehkä siksi, etteivät ne istuisi satunnaisten kivien tai toisen lintulajin munien päälle. Parasiittinen käki munii pesään suurempia, sinisempiä munia, joissa on tummempia pisteitä. Tämä saa laululinnun suosimaan käen munia omiensa sijaan.

Kilpavarustelu loisen ja isäntälajin välillä luo luonnollisen tasapainon. Liipaisimen liiallinen käyttäminen kääntyy parasiittia itseään vastaan. Jos käki tekisi munistaan liian houkuttelevia, haavoittuvaiset laululinnut eivät lisääntyisi ollenkaan. Laululinnut, jotka eivät reagoi liipaisimeen, saisivat etulyöntiaseman ja dominoisivat. Loiselle isäntälajin tappaminen on häviävä strategia. Tämä tarkoittaa sitä, että ainoastaan vastustuskykyiset isäntälajine edustajat lisääntyvät.

Niko Tinbergen, biologi joka löysi ja nimesi supernormaalin stimulin, rakensi muovisia munia. Hän havaitsi, että linnut suosivat muovimunia omiensa sijaan. Ne suosivat omiaan suurempia munia. Ne suosivat normaalia kylläisempiä värejä. Ja ne suosivat niiden omien munien kuvioita rajumpia kuvioita.

Joten pienen vaaleansinisen harmaantäplikkään munan sijaan hän tarjosi laululinnulle väärennöksen. Hänen munansa oli valtava, kirkkaan sininen, ja siinä oli suuria mustia täpliä. Lintu yritti istua munalle, yhä uudelleen ja uudelleen, ja putoili sen päältä pois.

Tämä saattaa naurattaa, mutta linnulle tämä on järjetöntä käytöstä. Näemme, että supernormaali stimuli voi tuottaa järjetöntä käytöstä järkevästi kehittyneistä vaistoista. Kyse on evolutiivisesta porsaanreiästä, jota monet pedot ja loiset käyttävät hyväkseen. Ihmispsykopaatit käyttävät sitä usein manipuloidakseen kohteitaan haluamaansa suuntaan.

\section{Avaussiirrot}

Vuonna 1989 Clatk ja Hatfield Floridan Yliopistosta? tekivät kuuluisan tutkimuksen.\linkki{dsfs} Heidän viehättävä tutkimusassistenttinsa käveli ympäri kampusta ja ehdotti treffejä ihmisille.

Tulokset ovat tunnettuja. Yli puolet miehistä hyväksyivät treffiehdotuksen, ja vielä useampi oli valmis menemään kotiin vieraan naisen kanssa. Kolme neljäsosaa hyväksyi suoran seksiehdotuksen. Naiset taas yleensä sanoivat ``ei\vmq{.}'' Opiskelijat eivät ehkä tyypillisiä esimerkkejä koko populaatiosta, mutta muut ovat saaneet samanlaisia tuloksia.

Tulen sukupuolien eroihin hetken päästä. Alkuperäinen tutkimus esitti, että naiset eivät harrasta satunnaista seksiä. Tiedämme, että tämä ei ole totta, ainakaan kaikissa konteksteissa. Ensimmäinen kysymykseni on: ``Kuinka niin moni mies on niin helposti koukutettavissa?'' Sitä voisi sanoa, että riski, jolle mies altistuu satunnaisessa seksissä on matala, mutta asia ei ole niin. Ilmiselviä riskejä ovat taudit ja yllätysvanhemmuus. Ja sitten on paljon suurempi vaara, että koko homma on lavastettu jonkinlaista huijausta varten.

Ja kuitenkin useimmat miehet sanova ``Ehdottomasti!'' Kuinka naisen viehätysvoima voi olla niin tehokas syötti? Ovatko miehet epätoivoisia, kiimaisia ja typeriä? Ovatko naiset fiksumpia? Ehkäpä vastaus on jotakin hieman hienovaraisempaa. Käy myös ilmi, että naiset eivät ole sen vastustuskykyisempiä kuin miehetkään. Kyse on vain siitä, käytetäänkö oikeanlaista syöttiä.

Kun vastaamme näihin kysymyksiin, näemme, kuinka merkittävä asia sukupuoli on. Sillä on oleellinen rooli psykopaatin avaussiirroissa. On olemassa neljä erillistä kaavaa: nainen miehelle, mies naiselle, mies miehelle ja nainen naiselle. Monet sosiaaliset vaistomme taipuvat kohti maskuliinista ja feminiinistä napaa, aivan kuten kehommekin.

Kehomme ja mielemme ovat lähtökohtaisesti feminiiniset. Kun mies kehittyy, ajoitetut testosteronipurkaukset siirtävät kehoa ja mieltä kohti miehuutta. Miehet ja naiset eroavat kehoiltansa ja mieliltänsä evoluutiota ajavien voimien vuoksi.

Joten kun sanon ``mies\vmq{,}'' se sisältää naiset, joilla on miehelle tyypilliset vaistot. Ja kun sanon ``nainen\vmq{,}'' se sisältää miehet, joilla on naiselle tyypilliset vaistot. Nämä avaussiirrot eivät oleta heteronormatiivisuutta. Psykopaateilla on usein nestemäinen seksuaali-identiteetti.\linkki{sdfds} He ovat yhtä itsevarmoja ja petomaisia homoseksuaaleina kuin heteroseksuaaleina.

\section{Naiset metsästämässä miehiä}

Psykopatian ``antisosiaalinen'' osa ei tarkoita sitä, etteikö henkilö haluaisi toisten seuraa.\linkki{sdfds} Se tarkoittaa haluttomuutta kunnioittaa sosiaalisia normeja ja käytäntöjä. Psykopaatit ovat taipuvaisia hypersosiaalisuuteen ja he etsivät pakkomielteisesti uusia ystäviä. Kyse on reviiristä?.

He voivat esiintyä henkilökohtaisesti ja hienotunteisesti, mikä yleensä piilottaa intensiivisen taustatyön. Web on tehnyt tästä paljon helpompaa, tarjotessaan monia tapoja puhua toisille yksityisesti.

Haluaisin louhia Facebook-dataa yksityisten keskustelujan, julkisten postauksien ja selfieiden takia. Arvaukseni on, että löytäisimme erillisen ryhmän käyttäjiä, joilla on paljon keskimääräistä enemmän yksityisiä keskusteluja, paljon keskimääräistä useamman eri ihmisen kanssa. Ennustan, että näkisimme kaksi päällekäistä kellokäyrää, emme vain yhtä.

Pedot tykkäävät mesästää sellaisissa paikoissa ja tapahtumissa, missä heillä on etulyöntiasema. Tilanteen täytyy tarjota tuore uusien kohteiden lähde. Kohteiden täytyy haluta jotakin, mitä peto voi hyödyntää. Kohteiden pitää tarjota potentiaalista hyötyä metsästäjälle. Tilanteen tulee tarjota suojaa ennen ja jälkeen minkä tahansa hyökkäyksen sekä sen aikana. Tilanteen pitää tehdä asioista jälkeenpäin puhuminen uhreille hankalaksi.

Deittailuskene on ilmiselvä mahdollisuus. Baarit, yökerhot ja deittisivustot ovat ideaalista aluetta sekä nais- että miespsykopaateille. Deittailun popkulttuuri on käsitellyt psykopatiaa jo tovin. He käyttävät kiertoilmaisua ``narsisti\vmq{.}''

Eräällä webbisivulla Susan Walsh käsittelee naispuolista narsismia\linkki{dfgf} ja listaa sellaisen henkilön piirteitä. Ensiksi, fyysinen olemus:
\begin{quotation}
\noindent Pukeutuu provokatiivisesti, rehennellen seksuaalisesti viihjaavilla kehonosilla; keskittää huomion meikkiin ja hiuksiin, jopa kaikista arkipäiväisimmissä tilanteissa; yli-itsevarma ulkonäöstään; pitää brändejä arvossa, ja kokee olevansa oikeutettu pukeutumaan ``parhaaseen;'' osta usein uusia vaatteita, eikä tee eroa haluille ja tarpeille; käyttää tavanomaista todennäköisemmin plastiikkakirurgiaa, tyypillisimmin rintojen suurennusta; tykkää olla valokuvauksen kohteena, ja usein pyytää muita ottamaan itsestään kuvan; jakaa innoissaan itsestään parhaat kuvat sosiaaliseen mediaan.
\end{quotation}
Toiseksi, persoonallisuus ja luonne:
\begin{quotation}
\noindent Vaatii päästä olemaan huomion keskipisteenä, monesti huoneen hurmaavin henkilö; hakee usein myötämielistä kohtelua, ja automaattista myöntymistä; uskoo olevansa erityinen; on erittäin materialistinen; on altis kateudelle, vaikka esittää itsensä äärimmäisen itsevarmana; etsii mahdollisuuksia toisten torpedoimiseen; on varma, että muut ovat kateellisia ja mustasukkaisia hänelle; häneltä puuttuu empatiakyky, ja jopa yleinen kohteliaisuus silloin tällöin; mollaa muita, mukaanlukien sinua; ei epäröi käyttää muita hyväksi; on kilpailuhenkinen; uskoo olevansa älyllisesti ylivertainen; syyttää muita ongelmista; ilmaisee ylimielistä asennetta silloin, kun hänen puolustus on alhaalla tai joku haastaa hänet; on epärehellinen ja usein valehtelee saadakseen mitä hän haluaa; on ``psyko\vmq{,}'' käyttäytyy riskialttiisti, ja omaa addiktoivan persoonallisuuden, ja on taipuvainen agressiiviseen käytökseen torjuttaessa; hänen mielialat ja teot ovat mahdottomia ennustaa.
\end{quotation}
Tämä sopii 95-prosenttisesti moniin naispsykopaatteihin joita tunnen tai olen tuntenut. Kirjoittaja sanoo: ``Pohjautuen kaikenikäisiin naisiin jotka olen tuntenut elämässäni, mielestäni 10\% on tarkka arvio narsistien osuudessa naispopulaatiossa\vmq{.}'' Numero on suuri, mutta se sopii arvoihin, joiden mukaan 10\% populaatiosta on sub-kliinisiä psykopaatteja. Olen vakuuttunut, että tämä petomainen ja tuhoisa ``narsismi\vmq{,}'' jota Walsh kuvailee, on yksi psykopatian maskeista.

Fyysinen olemus on kuin suuri ase, joka tähtää kohti miehen biologiaa. Sen vaikutus voi olla tuhoisa. Floridan yliopiston tutkimuksen mittauksen mukaan 75\% miehistä tarttuu sellaiseen syöttiin. Ehkä osuus on pienempi koko populaatiossa, kuin mitä se on yliopistokampuksella. Mutta deittailutilanteessa useimmat miehet etsivät satunnaista seksiä. Luku liukuu kohti sataa prosenttia.

Kun Mallory metsästää miehiä, hän ei vain kysy jokaista miestä drinkille. Se olisi turhan yksinkertaista. Hän tietää mitä hän etsii. Joten hän voi valita parhaat kohteet, jopa ennen kuin he näkevät hänen olevan paikalla.

Ihmiset reagoivat kuten mikä tahansa muukin elämänmuoto liipaisimiin ja supernormaaliin stimulin. Naiset, jotka pyrkivät vetämään miehiä puoleensa, investoivat oleellisten liipaisimien vahvistamiseen. Tässä on lista liipaisimista, jotka olen onnistunut tunnistamaan ja keräämään:
\begin{description}
\item[Vyöntärön suhde lantioon (WHR).] Tämä on ensisijaine signaali ihmisnaisen seksuaalisuudesta. Ideaalinen WHR liikkuu välillä 0,6--0,8 riippuen kulttuurista. Kapea vyötärö indikoi nuoruutta ja leveät lanteet hedelmällisyyttä. Yksinkertaisin WHR-huijaus on topata lantio. Sitten voi käyttää korsettia, ja sitten kauneuskirurgiaa. Rehellinen vastaliike on pitää tiukempia vaatteita ja näyttää enemmän paljasta pintaa.
\item[Rintojen koko ja muoto.] Ihmisnaisen rintojen kehityksestä käydään paljon väittelyä. Niiden koko ja muoto ei tarkoita suurempaa määrää tai parempilaatuista maitoa. Jotkut ajattelevat että rinnat kehittyivät vauvan tyynyiksi. Jotkut ovat sitä mieltq, että ne imitoivat pakaroita. Minun nähdäkseni ne kertovat naisellisesta nuoruudesta ja saatavuudesta, jotka molemmat ovat liipaisimia miehille. Ennen moderneja aikoja vauvoja imetettiin usein kahdesta kolmeen vuoteen. Rintaruokinta muuttaa rasvakudosta ja koossa pitävää kudosta? (Coopers's Ligaments). Joten rinnat näyttävät välittömästi, onko nainen jo saanut vauvoja vai ei. Vauvat tarkoittavat, että nainen on huonommin saatavilla, ja osoittaa kohti suojelevaa aviomiestä. Kuten WHR:n tapauksessa, huijari voi käyttää toppausta tai kirurgiaa. Ja rehellinen reaktio on taas kerran näyttää enemmän paljasta pintaa.
\item[Muut luotettavat nuoruuden indikaattorit.] Ensinnäkin sileä iho käsissä ja kasvoissa. Mitä sileämmät kasvot, sitä vahvemmin silmät, kulmakarvat ja huulet loistavat. Naiset voivat piilottaa kauneusvirheitä meikillä. He voivat liioitella silmien muotoa, huulia ja kulmakarvoja. Silmiinpistävät piirteet sileällä, virheettömällä iholla ovat liipaisin. Ja täydet huulet, pieni nenä ja korvat. Huulemme ohenevat iän myötä, ja nenämme ja korvamme kasvavat. Nainen voi vaikuttaa näihin ainoastaan plastiikkakirurgialla.
\item[Muut luotettavat hedelmällisyyden indikaattorit,] joita estrogeenihormooni ilmaisee. Tärkeimpiä ovat korkeat poskipäät ja ääni. Korkea, melodinen ääni kertoo hedelmällisyydestä ja nuoruudesta. Havaitse, kuinka jotkut naiset vaihtavat äänensä hiukan korkeammaks pyytäessään palvelusta. Sitä on lähes mahdoton tehdä kuulostamatta epäuskottavalta.
\item[Jalkojen suhde kehoon (LBR).] Mikä tahansa, mikä keskeyttää kasvamisen nuoruudesta aikuisuuteen vaikuttaa LBR:n. Tämä on hyvä geenien, dieetin ja terveyshistorian indikaattori. LBR ennustaa vastustuskykyä monille taudeille diabeteksesta moniin eri syöpiin. Pitkät jalat tarkoittavat terveyttä, ja terveys on seksikästä. Tämä on yksi harvoista liipaisimista, jotka toimivat molemmissa sukupuolissa. Naiset voivat väärentää LBR:ää käyttämällä korkeita korkoja ja lyhyitä hameita.
\item[Muut geneettisen resistanssin ja terveyshistorian indikaattorit.] Nämä ovat symmetriset kasvot, pitkä, puhdas tukka, kirkkaat, säkenöivät silmät, ja terveet kynnet. Tukka ja kynnet on nykyään helppo väärentää, eikä mitään todellista puollusta ole, lukuunottamatta tulitaukoa. Ehkä huivit ja muut ovat kehittyneet tämän takia??. Mascara voi saada silmät näyttämään valkoisemmilta ja säihkyvämmiltä.
\item[Haavoittuvuuden ja alistuvuuden indikaattorit.] Neito pulassa liipaisee petomaisen suojelevan reaktion miehissä.\linkki{sfdfs} Tekstitys kuuluu: ``Pelasta minut ja palkitsen sinut seksillä\vmq{.}'' Tummempi versi on: ``Olen yksin enkä voisi estää sinua vaikka haluaisin\vmq{.}'' Tähän liittyy useampia kehonkielisiä liipaisimia. Jalat yhdessä, ranteet näkyvillä tai velttoina. Pää alhaalla, katsekontaktin välttely. Tai pää alhaalla ja katse ylöspäin, näyttääkseen nuorelta. Ja viimeisenä humaltuneen näytteleminen. Palaan myöhemmin psykopaatteihin ja alkoholiin.
\item[Seksuaalisen saatavuuden ja himon merkit.] Toisin sanoen se, kun nainen kertoo miehelle: ``Haluan sinua, ja haluan seksiä kanssasi juuri nyt\vmq{.}'' Tähän liittyy ainakin kaksi liipaisinjoukkoa. Yksi on kosmeettinen huulien ja poskien punaaminen. Tämä matkii naisen kiihottumisen merkkejä (loistavat huulet ja kasvot). Toinen liipaisin on kehonkieli. Nainen pitää yllä katsekontaktia. Hän liikkuu lähemmäksi miestä. Hän käyttää paljastavaa vaatetusta. Hän vaihtaa asentoaan ja vaatetusta näyttääkseen enemmän paljasta pintaa. Hän koskee miehen käsivartta, leikkii hiuksillaan ja avaa huulensa. Hän nostaa kulmakarvojaan ja sulkee silmänsä puoliksi. Yleensäkin hän käyttäytyy kuin he olisiva sängyssä ja hän nauttisi siitä. Tämä on sosiaaliselle naiselle mahdotonta, lukuunottamatta turvallisissa olosuhteissa tapahtuvaa pelitilannetta. Torjutuksi tulemisen pelko on liian suuri. Psykopaateilla ei ole sellaista pelkoa, joten he voivat viedä tämän näytelmän äärimmäisyyksiin.
\item[Uutuudenviehätys.] Kutsutaan myös Coolidge-ilmiöksi. Siinä missä monogamiset parit ovat normi, monet ihmiset ovat opportunistisia pettäjiä. Useimmat miehet suosivat uusia avoimia seksipartnereita olemassaolevien ohi. Tämä ilmiö on yksi syy sille, miksi pornoteollisuus kaipaa niin kovasti uusia nuoria tähtiä. Kuinka nainen, joka ei pelkää joutuvansa naurunalaiseksi, voi näyttää joka viikko \emph{erilaiselta?} Se on yksinkertaista: hän vaihtaa hiustyyliä ja vaatteita. Uudet hiukset tarkoittavat uusia kasvoja. Uudet vaatteet ovat uusi keho. Molemmat saavat aikaan voimakkaampia reatkioita miehissä, jotka tuntevat naisen jo etukäteen.
\end{description}
Tämä joukko liipaisimia ja rajoittamattomat miesten reaktiot niihin selittää pornografian. Monille miehille se on lähes addiktoivaa. Se on kuin sokeri: jalostetun liipaisimen lähde. Porno näyttää nuorien, saatavilla olevien naisten virran, joka osuu kaikkiin kohtiin ylläolevassa listassa. Pornosivustotilastoissa maailmanlaajuisesti suosituin kategoria on ``teini\vmq{.}''\linkki{sdfds}

Miksi pakkomielle nuoruuteen? Deittailusivusto OKCupid havaitsi, että kaikenikäiset naiset suosivat omanikäistä partneiria. Kaikenikäiset miehet suosivat---silloin kun kukaan ei ole näkemässä---22-vuotiaita tai nuorempia naisia. Deittisivustoprofiilit ja haut voivat vaikuttaa heikolta pohjalta tieteelle. Turiun yliopiston Åbo Akademian? 12 000 suomalaista koskenut tutkimus havaitsi pitkälti saman asian: naiset suosivat omanikäisiään miehiä ja miehet suosivat noin 25-vuotiaita naisia.

Vastaus löytyy lajimme pitkäaikasen monogamisen suhteen mallista. Naisten hedelmällisyys on huipussaan nuorena, joten miehet ovat kehittyneet näkemään 16--22 vuoden iän (18--25, kun muut ovat näkemässä) ``seksikkyyden'' huippuna. Simpansseilla, läheisillä sukulaisillamme, on erillainen perhemalli. Ne elävät laajennetuissa perheissä, ilman monogamisia pareja. Niinpä urossimpanssit eivät liipaistu naisellisesta nuoruudesta. Ja simpanssinaaraat eivät esitä nuoruutta.

Jos olet koskaan pohtinut, miksi miehet ovat niin lumoutuneita naiskauneuteen, tiedät nyt vastauksen. Kuten myöhemmin selostan, naiset ovat aivan yhtä lumoutuneita miehisiin liipaisimiin.

Aikuiset miehet reagoivat oletusarvoisesti näihin supernormaaleihin stimuleihin. He ovat kuin laululintu, joka hoippuu jalkapallokentän kokoisen supersinisen munan pällä. He yrittävät uudelleen ja uudelleen käynnistää seksuaalisia suhteita naisen kanssa. Kohtelipa nainen heitä tai muita kuinka kaltoin, he kiipeävät takaisin munan päälle ja yrittävät uudelleen. Se näyttää järjettömältä. Se voi johtaa itsetuhoon.

Useimmat naiset jotka pukeutuvat vietelläkseen eivät ole psykopaatteja. Useimmissa tapauksissa se on aitoa ja terveellistä. Ero on siinä, kuinka syvällistä harhautus on. On niitä, jotka hiukan vääristävät totuutta, ja sitten on ammattilaisvalehtelijoita. Naispsykopaatit lähettävät seksuaalisuuttaan viekoitellakseen ja laajemmin kuin mihin sosiaalinen nainen kykenee. He käyttävät sitä hallitakseen narratiivia. He provosoivat off-the-charts?? reaktion, ja sitten he pidättelevät. Kaikki menee paljon peliä pidemmälle. Lupaus on: ``Minä ole täydellinen nainen ja minä olen sinun\vmq{.}'' Totuus on: ``Sinä kärsit ja maksat, etkä tule koskaan saamaan sitä ensimmäistä huumaa takaisin, ikinä\vmq{.}''

Haluaisin tarjota lähteitä tälle ilmiölle, mutta se on käsittääkseni dokumentoimaton asia. Olen kokenut sen ja havainnut sen tarpeeksi usein todetakseni, että se on todellinen ja tarkoituksellinen. Ja mekanismi vaikuttaa tutulta. Miehen reaktio naisellisiin seksuaalisiin signaaleihin asuu aivan riippuvuuden naapurissa.

Meidät on johdotettu kokemaan nautintoa näiden liipaisimien painamisesta. Biologiamme toimii sillä tavoin. Dopamiini osuu aivon emotionaalisiin keskuksiin. Ne kokevat iloa. Tämä vahvistaa sitä käytöstä mikä saikaan meidät alunperin liipaisimen äärelle. Kun joku vahvistaa liipaisinta, syntyy suurempi dopamiiniryöppy. Mieli kompensoi muuttumalla vähemmän herkemmäksi. Joten niinpä tarvitsemme lisää liipaisemista kokeaksemme saman ilon tunteen.

Addiktio ei tässä ole metafora. Se on psykopaatin kanssa tapahtuvan seksuaalisen suhteen ydin. Ja sellainen suhde on yhtä mahtava ja terveellinen kuin kokaiiniin tai raakaan viinaan pohjautuva elämä.

Psykopaatit suosivat sellaisia liipaisimia, joita he voivat jäljitellä tai vahvistaa keskitetyllä vaivannäöllä. Joten Mallory voi olla jokseenkin koruton, ja kuitenkin haltioiva silloin kuin hän niin haluaa. Selostin tapoja huijata useampia eri liipaisimia. Jokaiseta huijausta kohti---sanotaan vaikka, että nainen valehtelee ikänsä---on olemassa rehellinen pelaaja. Vaikkapa nuorempi nainen joka kilpailee samoista miehistä. Tämä on hidas, muinainen kilpavarustelu eri strategioiden välillä.

Naispsykopaatit näkevät enemmän vaivaa ulkonäön eteen, ja vähemmän vaivaa ystävien ja perheen eteen. He jäljittelevät liipaisimia, joita sosiaaliset naiset eivät voi tai halua jäljitellä. Sosiaaliset naiset sen sijaan kilpailevat heidän todellisilla antimillaan. Tämä kilpavarustelu kiittää autenttisia naisten painelemia miesten liipaisimia. Eli täydet rinnat, leveät lanteet, pitkät hiukset ja sileä iho. Ne ovat seksuaalivalinnan ja huijareiden ja rehellisten pelaajien välisen kilpavarustelun hedelmä.

\section{Miehet metsästämässä naisia}

Naiset ovat tottakai erilaisia, ja he ovat immuuneja halvalle imartelulle ja jäljitellyille pullistumille. Naisreaktiot vierailta miehiltä tuleviin seksiehdotuksiin ovat lähellä nollaa. Nainen käsittelee sellaista tarjousta yleensä vihamielisenä tekona. Hän on taipuvainen soittamaan apua poliisilta tai miespuolisilta ystäviltä.

Kyse on kuitenkin vain kontekstista. Kun liipaisimet ovat kohdallaan, useimmat naiset reagoivat. He kävelevät kohti heikkoja liipaisimia. He hölkkäävät kohti vahvoja liipaisimia. Ja he harppovat itsetuhoisen draaman saattelemana kohti supernormaaleita liipaisimia. Aivan kuten miehet.

Joten mitkä ovatkaan näitä liipaisimia? Mikä tekee miehistä viehättäviä naisten silmissä? Kyse on jossain mielessä iänikuisesta mysteeristä. Mutta vastaus in ilmeinen, kun sen vain näkee.

Biologia ja empiirinen tutkimus sulkee pois monia ilmiselviä vaihtoehtoja. Saatavuus ja haluukkuu seksiin eivät ole liipaisimia. Nuoruus ei ole liipaisin. Miehet näyttävät geeninsä, terveyshistoriansa ja hedelmällisyytensä aivan kuten naisetkin, joten jonkinlaista päällekkäisyyttä on. Pitkät jalat, hyvänmuotoiset pakarat, symmetriset kasvot, kiva tukka, vahva leka, korkeat poskipäät. Nämä toimivat samoin molemmissa sukupuolissa.

Mutta ulkonäkö toimii ainoastaan yhdessä muiden liipaisimien, kuten itsevarmuuden ja hurmaavuuden, kanssa. Yksin miehien ulkonäkö ei ole liipaisin naisille. Sosiaaliset naiset eivät koe hyvännäköistä mutta epävarmaa miestä viehättäväksi. Molemman sukupuoliset psykopaatit vaikuttavan pitävän sellaisilla miehillä leikkimisestä.

Mitä piirteitä naiset suosivat miehissä?

Useimmat naiset kilpailevat ollakseen kaikista kauneimman ja nuorimman näköinen. Kilpailu voi olla hunnutettua ja tiedostamatonta. Mutta se on läsnä kaikkialla, koska siihen miehet reagoivat.

Mistä miehet kilpailevat? Mikä on se asia, jonka kerryttämiseksi ja kiinnipitämiseksi miehet taistelevat elämänsä ajan? Se ei ole nuori ulkonäkö, lukuunottamatta joitakin kummallisia miehiä. Se ei ole pitkät hiukset. Eikä suorat piirteet eivätkä pitemmät jalat.

Miehet kilpailevat vallasta ja sen edustajasta, rahasta. Aivan kuten naisten ulkonäkökilpailussa, kilpailu voi ola hunnutettua ja jopa alitajuntaista. Mutta se on kaikkalla. Työssä, urheilussa, sosiaalisessa toiminnassa. Miesvalta ottaa monia eri muotoja. Se voi olla fyysistä, älyllistä, taloudellista. Jopa miehet, jotka eivät eksplisiittisesti kilpaile, ottavat kantaa.

Miesviehättävyydessä näyttää olevan kaksi pääteemaa. Yksi on hallitsevuus: pituus, matala ääni, itsevarmuus, näkyvät karvat kasvoissa, ja kilpailullisuus. Ainakin yksi tutkimus esittää,\linkki{sdfds} että tämä nousee ja laskee naisen hedelmällisyyssyklin mukaan. Toisin sanoen, kun nainen on korkeimmillaan hedelmällisyydessä, hän hakee todennäköisemmin opportunisista seksiä. Sitten naiset suosivat ``miehekkäitä'' miehiä, ja sosiaalisesti hallitsevia miehiä hyvien partnerien tai isien sijaan.

Toinen teema on persoonallisuus, mikä näyttäytyy sellaisina asioina kuin ``älykkyys\vmq{,}'' ``hyvä huumorintaju\vmq{,}'' ja ``on kiltti\vmq{.}'' Kun naiset etsivät pitkäaikaisia kumppaneita, he keskittyvät tähän teemaan. Toisin sanoen, potentiaalisesti hyviin kumppaneihin ja isiin. Jos muotoilet tämän uusiksi muotoon ``on empaattinen ja herkkä\vmq{,}'' se on koodi sille että mies ei ole Mallory.

Miesvalta on liipaisin naisille, ainakin osan ajasta, aivan kuten naishedelmällisyys on liipaisin miehille.

Asian evolutionaarinen järki on yksinkertainen. Vaikutusvaltaisten miesten kumppanit saavat enemmän lapsenlapsia kuin heikkojen miestin vaimot. Hyvät geenit ovat hyvä lähtökohta. Niiden työntäminen tuleville sukupolville ottaa voimaa, kun vastassa on loppumaton kilpailu. Ihmiskäsittein tämä tarkoittaa valtaa.

Korjataanpa Clarkin ja Hatfieldin tutkimus. Näyttelijät olivat nuoria, viehättäviä miehiä ja naisia. Tämä itsessään luo jo valtavan vinoutuman. Parempi tutkimus käyttäisi valikoimaa erilaisia näyttelijöitä, sekä miehiä että naisia. Näyttelijät poikkeaisivat toisistaan iän, viehättävyyden ja vaikutusvaltaisuuden osalta. Sitten mittaisimme heidän suhteellisen menestyneisyyden sekailaisessa joukossa kohteita.

Kerron arvaukseni siitä, mitä tapahtuisi. Havaitsisimme, että miehet reagoivat pääasiassa nuoruuteen, sitten ulkonäköön, sitten saatavuuteen, ja sitten luonteeseen. Saatavuus on kriittistä. Mies-miestä-vastaan -väkivalta naisten vuoksi on niin merkittävää, että ``ei saatavilla'':n luulisi olevan turn-off useimmille miehille. Tämän lisäksi havaitsisimme naisten reagoivan valtaan, sitten luonteeseen, sitten saatavuuteen, ja viimeisenä ulkonäköön. Hedelmällisyyden huippuaikana naiset reagoisivat pääasiassa valtaan ja ulkonäköön.

Kysymys kuuluu: kuinka mies näyttää ``Olen vaikutusvaltainen'' -liipaisimensa? Kuinka nainen tietää, milloin mies valehtelee omasta tärkeydestänsä? Kuinka psykopaatit liioittelevat näitä liipaisimia?


Taas kerran vastaus piilottelee päivänvalossa. Jos kysyt yksinäiseltä mieheltä hänen statuksestaan, hän luultavasti yksinkertaisesti valehtelee. Miehet liioittelevat ansioitansa. He valehtelevat saavutuksistaan. He piilottavat epäonnistumisensa. Ja niin edelleen. Oletamme tätä, ja mitä ikinä mies itsestään kertookin, suhtaudumme siihen harmittomana fantasiana. \problem{kuinka suomentaa plain sight}

Miehet näyttävät vaikutusvaltansa kehonkielessänsä ja siinä, kuinka he käyttäytyvät toisia---esimerkiksi tilannetta havainnoivaa naista, yhtä tai useampaa toista miesta tai toisia naisia---kohtaan.

Kun tapaamme tuntemattomia, arvioimme heitä vaistonvaraisesti, silloinkin kun kyse on vain jalkakäytävällä vastaantulevasta satunnaisesta ohikulkijasta. Kuinka yksi tekee tietä toiselle? Huone täynnä toisilleen tuntemattomia ihmisiä järjestyy itsestään. Miehet siirtyvät ryppäisiin, hallitsevimmasta vähiten hallitsevaan, ja joka ryppäässä on yksi mies johdossa. Naiset liikkuvat erilaisiin kuvioihin miestin ympärille, tai heistä erilleen. Miehet saattavat kerääntyä yksittäisten naisten ympärille. Miehet tsekkaavat naiset. Naiset tsekkaavat miehet. Ryhmä muodostaa mielipiteen. Kaikki tapahtuu minuuteissa.

Miesvalta tarkoittaa kykyä hallita muita miehiä (ja naisia; mutta pääasiassa miehiä). Tämä voi olla hienovaraista ja epäsuoraa, ja ylettyy kauas fyysisen dominoinnin tuolle puolen. Kirjoittajana voin hallita yksinkertaisesti naputtamalla näppäimistöä. Kehonkieli ja naamatusten dominointi ovat kuitenkin hyvä paikka aloittaa.

Me ymmärrämme dominoivaa kehonkieltä hyvin. Kyse on osittain suuremmalta näyttämisestä. Tämä tulee ihmisapinaedeltäjistämme, joiden tapauksessa dominointi tarkoitti sitä, että on vahvempi. Suurin, vahvin mies johti ryhmää. Kyse on myös osittain muihin nähden ylempiarvoisesti käyttäytymisestä.

Näyttääkseen suuremmalta mies seisoo suorassa ja nostaa leukaansa. Hän seisoo jalat kauempana toisistaan. Hän ottaa enemmän tilaa, kuin olisi tarpeen. Hän käyttää käsieleitä ja työntää kyynärpäitänsä ulospäin.

Vaikuttaakseen ylempiarvoiselta, mies kontrolloi keskustelua. Hän ylläpitää katsekontaktia räpäyttämättä silmiään. Hän jättää huomiotta ne, joista hän ei ole kiinnostunut. Hän hymyilee vähemmän ja liikuttaa päätänsä vähemmän. Hän puhuu vähemmän, matalammalla ja hiljaisemmalla äänellä. Yleisesti ottaen hän jättää sosiaaliset vihjeet huomiotta. Hän tulee myöhässä, keskeyttää muita, ja on ankara, olematta kuitenkaan tyly.

Meillä on paradoksi käsillä. Nämä piirteet sekä vetävät puoleensa naisia että karkottavat heitä. Ne kertovat miehestä, joka on kykenevä hallitsemaan muita miehiä, mutta myös miehestä, joka on epäherkkä ja mahdollisesti raaka. Paradoksista on olemassa tie ulos. Naiset olettavat, että kaikki miehet valehtelevat, kunnes toisin todistetaan. Hallitsevuuden piirteet vaativat toisen liipaisimen toimiakseen. Ilman tätä toista liipaisinta mies näyttää naisen silmissä vain öykkäriltä.

Toiset liipaisimet tulevat toisilta miehiltä. Kuten kaikki liipaisimet, mitä olemme tarkastelleet, ne ovat minimalistisia ja elegantteja. Näen kaksi erityistä liipaisinta. Ensimmäinen on toiset miehet, jotka hyväksyvät dominoinnin, ja käyttäytyvät oikealla tavalla, eli näyttävät alistuvaa kehonkieltä. He ovat hiljaa, kun dominoiva mies puhuu. He hyväksyvät miehen tunkeilevan asenteen aivan kuin heidän mielestään mies olisi ansainnut oikeuden siihen.

Toinen liipaisin on dramaattisempi. Siihen tarvitaan ryhmän ulkopuolinen mies haastajaksi. Dominoiva mies joko pitää paikkansa tai häviää pelin. Naiset kokevat tällaisen voittaja-vie-kaikein draaman houkuttelevana, oli kyse sitten todellisesta elämästä, urheilusta tai viihteestä. Naiset saavat siitä saman dopamiinipotkun kuin miehet saavat katsellessaan kauniin naisen riisuutumista.

Vihätys on varmasti suhteellista. Mutta siinä missä naisen viehättävyys on yksilöllinen asia, miesvalta virtaa toisten miesten kautta. Olipa kyse kuinka rikkaasta tai itsevarmasta miehestä, yksin hän on pelkkä nolla. Miesvallassa on kyse toisista miehistä.

Samankaltainen ilmiö on nähtävissä siinä, kun naiset liipaisevat miehiä. Naisen ulkoasi sekä vetää miehiä puoleensa että hylkii heitä. Nainen, joka käyttää liikaa meikkiä ja parfyymiä, flirttaa turhan aggressiivisesti ja näyttää liikaa paljasta pintaa, sammuttaa useimmat miehet. Mutta jos hän on muiden naisten seurassa, ja he vaikuttavat pitävän hänestä, toimii tämä toisena liipaisinta. Ja useimmat miehet viihtyvät taas.

Nämä sekundääriset liipaisimet vaikuttavat kiteytyvän ajatukseen ``ei psykopaatti\vmq{.}'' Empatia ja herkkyys ovat suhteellisen helppoja väärentää, joten pinnalliset empatian merkit eivät ole liipaisimia. Tulen empatiatesteihin myöhemmin.

Nyt, kun olemme selvittäneet signaalit, katsotaan, kuinka niitä on mahdollista väärentää. Vaelteleva miespsykopaatti on yksinäinen. Hänellä ei ole todellisia ystäviä. Hän ei voi vain näyttää narsistista ``minä olen tärkeä'' -maskiansa naisille. Sen sijaan hänen täytyy vakuuttaa muille miehille, että hän on tärkeä. Hänen täytyy löytää tai rakentaa ryhmä, jota dominoida, ja hänen täytyy näyttää tulokset.

Pinnan alla toimivat mekanismit ovat hienovaraisia. En tiedä aiheesta mitään muuta tutkimusta kuin omani. Joten tämä kaikki on spekulointia ja hypoteeseja, jotka perustuvat pitkäaikaiseen havainnointiin ja analyysiin.

Sanoin miesten ja naisten eroavan toisistaan merkittävillä tavoilla. Yksi näistä tavoista on se, kuinka kommunikoimme. Kyse on syvemmästä asiasta, kuin siitä, mistä puhumme. Kyse on siitä, miksi meille alunperin ylipäänsä kehittyi kielipää. 

Miehet ja naiset molemmat vaihtavat valtaa ja tietämystä, ja rakentavat rakenteita. Mutta tässä on selvä sukupuoliero. Miehet puhuvat vaihtaakseen teknistä tietämystä ja he organisoituvat keskenään valtarakenteisiin. Naiset puhuvat vaihtaakseen sosiaalista tietämystä ja organisoituvat keskenään aivan toisenlaisiksi rakenteiksi. Kielemme heijastelee näitä kahta mallia.

Olen kutsunut näitä puhetapoja ihmisprotokolliksi.\linkki{dsfds} On kaksi pääprotokollaa: miesprotokolla ja maisprotokolla. On pienempiäkin protokollia, kuten aikuinen-lapselle. Useimmat miehet kykenevät vähintäänkin imitoimaan naisprotokollaa ja toisinpäin. Mutta se on kovaa työtä.

Mallory on miesprotokollan ekspertti. Hän voi dominoida miesryhmiä käyttäen valheiden, lupauksien ja itsevarmuuden sekoitusta. Suurta petomaista bisnestä ei voi erottaa kultista. Siinä, missä Alisa on taipuvainen epäilemään yksinäistä miestä, Bob ja hänen kaverinsa eivät ole. Tämä on erityisen totta silloin, kun yksinäinen mies lähestyy ryhmää. Niinpä Mallory tykkää tehdä taikatemppunsa ensin Bobille ja hänen kavereilleen. Tämä antaa hänelle statusta ja valtaa, jota hän voi heijastaa kohti naisia.

Tämä voi tapahtua minuuteissa. Miesprotokolla mahdollistaa silmänräpäyksessä syntyvät suhteet, jotka perustuvat pelkästään tulevaisuuden mahdollisuuksiin. ``Seuraa minua! Lupaan sinulle kultaa!'' Naisprotokolla taas on kyyninen ja varovainen. Naistenvälisten suhteiden kehittyminen kestää usein vuosia. Naisten on investoitava suhteisiinsa. Miesten täytyy vain pitää futuurit avoinna.

Tämä on valtava ero. Molemmat sukupuolet heijastavat heidän omia arvojaan ja mittauksiaan toisista. Alisa olettaa, että suhteen rakentaminen kestää kauan. Hän olettaa, että suhteet ilmaisevat kyynistä kijanpitoa menneistä faktoista. Joten hän yliarvioi suhdetta, jonka hän näkee Bobin ja Malloryn välillä. Ja samalla tavoin Bob aliarvioi Alisan suhteita.

Tästä saamme erään Malloryn klassisen metsästyskaavan. Hän ensin valloittaa Bobin ja jotkin muut miehet hymyllä ja äärimmäisellä itsevarmuudella. Hän vihjailee jonkinlaisia lupauksia tai tulevaisuuden tuomaa hyötyä. Tämä voi kestää vain minuutteja. Sitten hän esittelee tämän väliaikaisen rakenteen ohikulkevalle Alisalle. Hän näyttää, kuinka hän dominoi kokoamiansa miehiä. Alisa kokee tämän hurmaavan miehen haluttavana. Hän alkaa vastata miehelle. Kun mies esittää, että he voisivat jatkaa juttua jossakin yksityisessä paikassa, nainen hyväksyy ajatuksen. Reaktio on aivan sama, kuin miten miehet reagoivat kohotettuihin rintoihin ja terveeksi toivottaviin hymyihin.

\section{Valtapyramidit}

Miespsykopaatit kykenevät toimimaan ja toimivat henkilökohtaisella tasolla. Se on kuitenkin vain pölyä verrattuna monien organisaatioiden teollisen mittakaavan miespsykopatialle. Ennen kuin katsomme, kuinka Mallory metsästää Bobia, meidän on syytä käydä pienellä kiertoajelulla. Selostan, kuinka ihmiset organisoituvat laajassa skaalassa.

Ihmiset vaikuttavat organisoituvan kahdella erityisellä, vastakkaisella tavalla. Nämä ovat ``elävä järjestelmä'' ja ``valtapyramidi\vmq{.}''

Elävä järjestelmä on löysä ekonominen verkko sisäisesti riippumattomia toimijoita tai palasia. Nämä palaset vaihtavat resursseja kuten tietämystä, työtä tai rahaa. Nämä resurssit virtaavat järjestelmän läpi eri suuntiin omalla tahdillansa. Elävissä järjestelmissä ei ole ilmiselvää valtarakennetta. Niillä ei ole tunnistettavia omistajia eikä keskitettyä auktoriteettia. Ne eivät harrasta keskitettyä päätöksentekoa tai suunnittelua.

Valtapyramidi on eksplisiittinen rakenne, jolla on nimi, tarkoitus ja johtohierarkia. Päätökset ja suunnitelmat liikkuvat alas ja voitot liikkuvat ylös. Koko pyramidi on yläosassa istuvien omaisuutta. Kyse on selvästä hierarkiasta, jossa paikka määrittää statuksen ja status määrittää paikan.

Elävät järjestelmä ovat rajoittamattomia ja lähes näkymättömiä. Niillä ei ole markkinointijaostoa, paneeleita tai toimitusjohtajia. Ne koostuvat tuhansista, jopa miljoonista riippumattomista toimijoista. Toimijat organisoituvat itsestään mielenkiintoisimmille aloille. Elävät järjestelmät ovat paljon tuottoisampia kuin valtapyramidit. Mutta tuotot leviävät laajalle ja niiden mittaaminen on vaikeaa.

Elävät järjestelmät ovat tehokkaita. Ne ovat elämäntapamme ytimessä. Ne ruokkivat kaupunkimme ja täyttävät kaupat tuotteilla. Ne tuovat meille vaatteet, tietämyksemme, Internetin. Jokainen kaupunki on elävä järjestelmä tai vankila. Jokainen talous on elävä järjestelmä, tai sitten se on suunniteltu epäonnistuminen.

Elävissä järjestelmissä tapahtuvat vaihtokaupat riippuvat avoimista sopimuksista. Elävät järjestelmät tunnistavat ja rankaisevat huijareita, käyttäen yksinkertaista vapaan valinnan menetelmää. Ne tarvitsevat jonkinlaista sääntelyä, eli lakia ja sen toimeenpanoa. Luonnolliset elävät järjestelmät käyttävät fysiikan, kemian ja biologian lakeja. Keinotekoiset elävät järjestelmät tuottavat omat auktoriteettinsa, yleensä evoluution ja kilpailun kautta.

Elävät järjestelmät ovat reiluja (tai eettisiä) jokaista osallistujaa kohtaan puhtaasti tehokkuuden tarpeen johdosta. Ne siis ovat taipuvaisia kohtelemaan syrjintää ja huijaamista epätehokkuuksina, ongelmina, jotka on syytä ratkaista. Elävät järjestelmät tekevät kokeita jatkuvasti pienillä ratkaisuilla uusiin ongelmiin. Ne hautaavat epäonnistumiset ja edistät menestystä. Tämän takia ne ovat hyviä sopeutumaan muutokseen. Ne ovat sitkeitä ja selviytyvät, kunnes joku katastrofaalinen ulkopuolinen tapahtuma rikkoo ne. Kaupungit selviytyvät imperiumeista, jos niitä vain ei tuhota maan tasalle.

Valtapyramidit ovat erikoistuneet hankkimaan asiakkaita, toimittajia ja työntekijöitä antamaan enemmän vähemmällä. Ne toimivat loismaisesti. Jos käännät valtapyramidin ylösalaisin, se näyttää suppilolta. Äärimmäisiä esimerkkejä ovat suuret vähittäiskaupat, jotka maksavat toimittajillensa niin vähän kuin mahdollista ja tekevät miljardien edestä voittoja.

Massojen pitäminen paikallaan samalla kun niistä imee resursseja ottaa työtä. Valtapyramidit tekevät sen käyttäen voiman, lahjuksien ja uhkauksien sekoitusta. Ne vaativat fyysistä läsnäoloa. Ne lupaavat kuukausipalkkoja ja bonuksia. Kyvyttömyys myötäillä tarkoittaa potkuja. Kyse on jatkuvan matalan tason sisäisen väkivallan muodosta. Valtapyramidit heijastavat väkivaltaa myös ulkopuolisia uhkia kohti. Ne käyttävät voimaa poistaakseen kilpailijoita ja saavuttaakseen päämääriänsä. Ne ovat pragmaattisia ja säälimättömiä.

Eivät kaikki isot bisnekset ole puhtaita valtapyramideja; useimmat ovat sekoituksia, joissa on joitakin elävän järjestelmän aspekteja. En ole kuitenkaan ikinä kuullut bisneksen pahoittelevan kilpailijansa voittamista. Enkä ole koskaan kuullut valtion pahoittelevan sodan voittamista. Selviytyminen määrittelee moraalin, lyhyellä tähtäimellä. Pitkällä tähtäimellä moraali määrittelee selviytymisen. Niinpä elävät järjestelmät tuppaavat päihittämään valtapyramidit.

Näemme valtapyramideja useimmiten yritysmaailmassa, hallituksissa ja organisoituneissa uskonnoissa. Kun nämä kolme sekoittuvat, meillä on käsissä fasismia ja kansanmurhia. Tätä tapahtuu, kun yhteiskunta on liian heikko tai naiivi pistääkseen kampoihin. Useammin valtapyramidit ovat piittaamattomia, sen sijaan että ne olisivat suoraviivaisen tuhoisia.

Kun valtapyramidi tuottaa tuotteita, se tähtää surkeimpaan mahdolliseen laatuun ja korkeimpaan hintaan. Joskus se saattaa myrkyttää ja addiktoida asiakkaansa voittojen vuoksi. Esimerkkinä toimii elintarvikeala, joka keskittyy sokeriin. Valtapyramidit eivät kuuntele markkinoita. Sen sijaan ne yrittävät pakottaa ihmiset vastaanottamaan tuotteensa, käyttäen rajua markkinointia. Voisi sanoa, että valtapyramideilla ei ole empatiakykyä.

Valtapyramidit ovat mestareita harhaanjohtamisessa ja todellisen luonteen peittämisessä. Ne markkinoivat itseään ``eettisinä\vmq{,}'' ``positiivisina\vmq{,}'' ``hyvinä'' ja ``hauskoina\vmq{.}'' Ne käyttävät miljardeja brändäykseen ja imagoon, luoden narratiiveja myydäkseen tuotteitaan. Ihmiset uskovat näihin narratiiveihin ja investoivat niihin raskaasti.

Mitä surkeampi tuote, sitä raskaampaa on mainonta. Coca-Cola. Microsoft. Kraft. Heinz, USA! Valtapyramidit kommunikoivat käyttäen totuuden elementtien ympärille kiedottuja valheita. Niiden ydinarvot ovat voitot ja selviytyminen, ei enempää, ei vähempää.

Huolimatta niiden keskittymisestä selviytymiseen, valtapyramidit ovat huonoja sopeutumaan muutokseen. Ajan kanssa ne tulevat yhä riippuvaisemmiksi valheista ja voimasta, kun maailma niiden ympärillä muuttuu. Niistä tulee haurauta ja alttiita nopealle, katastrofaaliselle romahdukselle. Nokia, Blackberry, Neuvostoliitto.

Empatian puute, sydämmettömyys ja petomaisuus, taipumus käyttää hyväksi{\ldots} Monet suuret yritykset, uskonnot, ja tietyt hallituksen maut ovat psykopaattisia valtapyramideja.

Hyvän työkokemuksen omaava kaverini kävi suuren teknologiafirman työhaastattelussa. Prosessi jätti hänet hämmentyneeseen ja nöyryytettyyn mielentilaan. ``Miksi minun täytyy todistella itseäni nuorelle työhaastattelijalle?'' hän kysyi minulta. ``Miksi he eivät vain katsoisi työtäni. Se on kaikki Internetissä\vmq{.}'' Pohdimme asiaa. Sanoin, että ehkä nöyryytys on sellaisen työhaastattelun päämäärä. Jos hyväksyt sen, tulet hyväksymään paljon pahempaakin, vastikkeena mehukkaalle palkallesi. Myötäily on koe.

Edustavatko nämä kaksi mallia miesten ja naisten tapaa työskennellä? Se on houkutteleva päähänpisto. Miesprotokolla toimii miesvallan ympärillä, ja naisprotokolla sosiaalisen tietämyksen ympärillä. Mutta olisi typerää luonnehtia ``miehekkyyttä'' pahuudeksi. Lajimme ei kehittänyt sukupuolieroja jakaakseen meidät. Teimme sen, jotta voisimme työskennellä yhdessä tehokkaammin. Sekä miehet että naiset ovat onnellisempia ja tehokkaampia elävissä järjestelmissä. Ja valtapyramidit hyväksikäyttävät sekä miehiä että naisia.

Mielestäni totuus on hienostuneempi. Elävät järjestelmät tarvitsevat sekä tietämyksen että toiminnan virtaa. Mies- ja naisprotokolla toimivat yhdessä, ratkaisten eri osia suuremmasta pulmasta. Valtapyramidit ovat vääristymä, miespsykopaattien rakennelma. Ne ovat massateollisuuden ja urbanisaation hetkellinen hedelmä. Ne ovat niin laajalle levinneitä, että pidämme niitä itsestäänselvyyksinä. Uskon, että ne ovat kuitenkin vahingollisia ja antisosiaalisia, ja niiden kohtalo on hiipua hitaasti pois.

Mitä miespsykopaatit yli kaiken haluavat on toisten ihmisten hallinta. Kyse on harvoin rahasta. Kuten Frank Underwood toteaa Netflix-sarjassa \emph{House of Cards}:
\begin{quotation}
\noindent Hän valitsi vallan sijaan rahan. Mitä ajan haaskausta. Virhe, jonka melkein jokainen tässä kaupungissa tekee. Raha on Sarasotan McMansion, joka alkaa hajota kappaleiksi kymmenessä vuodessa. Valta on vanha kivirakennus, joka pysyy pystyssä vuosisatoja. En voi arvostamaan ihmistä, joka ei näe eroa.
\end{quotation}

\section{Miehet metsästämässä miehiä}

Valtapyramidi rekrytoi nuoria miehiä helposti. Kuten kaikkien vaistojemme kohdalla, näemme syyn evolutiivisessa historiassamme. Ihmismiehet muodostivat ryhmiä metsästäkseen suuria villieläimiä. Kyse on korkean riskin ja korkean hyödyn ryhmätyöskentelystä. Vanhemmat miehet jakavat tietämyksensä, ja nuoremmat miehet jakavat fyysiset kykynsä ja aikansa. Nuoret miehet, jotka reagoivat vanhempien miesten ``seuraa minua''-viestiin tulevat todennäköisemmin takaisin mukanaan lihaa. Tämä näkyy suoraan lisääntymismenestyksessä.

Tulevaisuuden palkintojen ``Seuraa minua''-lupaus vanhemmalta mieheltä on liipaisin. Mitä suurempi lupaus, sitä suurempi reaktio. Sen ei tarvitse olla looginen tai järkevä. Itseasiassa järjettömät lupaukset ovat usein \emph{houkuttelevampia} kuin järjelliset lupaukse. Järkevä ehdotus vaatii kovaa työtä ja kärsivällisyyttä. Järjetön ohdotus ei vaadi muuta kuin epäuskon suyrjään siirtämistä. ``Tiedän jotakin riskisijoittajia ja he investoivat miljoonia ideaasi!'' Tähän on vaikea sanoa ei.

On olemassa joukku ``seuraa minua''-liipaisimia, jotka antavat yhdelle miehelle mahdollisuuden ottaa toisia, jopa kokonaisen ryhmän, hallintaansa. Liipaisimet toimivat parhaiten nuoriin, alle nelikymppisiin miehiin, joilla ei ole lapsia. Tässä vat ne, jotka tiedän:
\begin{description}
\item[Yksinäinen lähestyminen.] Tämä kertoo itsevarmuudesta ja purka ryhmän luonnollisen puolustusjärjestelmän. Yksittäinen mies ei voi ola fyysinen uhka ryhmälle.
\item[Dominoivan kehonkielen näyttäminen,] erityisesti kohti senhetkistä dominoivaa miestä. Jos dominoiva mies ei tastele vastaan, hän on astunut alas, ainakin hetkeksi.
\item[Vanhemmalta ja viisaammalta vaikuttaminen.] Tämä liipaisee ``viisas vanha mies''-reaktion nuorissa miehissä. Viisaus on kallisarvoista, kunhan se vain on relevanttia.
\item[Keskustelun kontrolloiminen.] Dominoivat miehet ylläpitävät katsekontaktia, keskittyvät korkeamman statuksen miehiin, ja hymyilevät vähemmän. He puhuvat vähemmän, matalammalla ja hiljaisemmalla äänellä. Tämä pakottaa toiset kiinnittämään lähestä huomiota heihin. Tämä liipaisee ``tuo kaveri on dominoiva''-reaktion.
\item[Lupausten tekeminen potentiaalisista voitoista.] Nämä voivat olla suuria ja niin vaikeita kuin mahdollista. Mitä hullumpi, sen parempi. Nuoren miehen biologia tekee hänestä luonnollisen uhkapelaajan. Valtava potentiaalinen voitto liipaisee uhkapelireaktion, riippumatta siitä, kuinka pieni voiton mahdollisuus on.
\item[Yhteiseen viholliseen vetoaminen.] Tämä liipaisee puolustusreaktion. Se antaa ryhmälle keskittymistä ja energiaa, jonka ulkopuolinen voi omistaa ja jota se voi ohjata.
\item[Toiminnan vaatiminen ja suunnitelman ehdottaminen.] Tämä liipaisee ``seuraa minua''-reaktion. Jos valtaosa ryhmästä reagoi, pääsee ulkopuolinen johtoasemaan.
\item[Sisäisiä vihollisia vastaan hyökkääminen,] erityisesti vastaan vanhaa johtoa. Tämä liipaisee paranoia- ja kostoreaktiot. Jos uudella johtajalla on matkassa onnea, kykenee hän puhdistamaan hierarkian kaikista potentiaalisista uhista.
\end{description}
Kaikki kauniit, flirttailevat naiset eivät ole psykopaatteja, ja kaikki miehet, jotka käyttävät näitä tekniikoita, eivät ole psykopaatteja. Ero piilee lopputuloksissa. Näemmekö harhautusta ja hyväksikäyttöä vai rehellisyyttä ja yhteistä hyötymistä? Palavatko ihmiset loppuun ja masentuvat, vai tulevatko he onnellisemmiksi ja itsenäisemmiksi? Psykopatia piilottelee taidokkaasti, mutta silloin kun se organisoi ihmisiä omia tarkoitusperiänsä varten, vauriot kyllä jossakin vaiheessa tulevat esille.

Nämä liipaisimet kehittyivät pätevistä syistä. Kyky organisoitua karismaattisten johtajien ympärille pelasti edeltäjämme monta kertaa. Ja me harppaamme reagoidessamme. Jos liipaisimet ylipäänsä vaikuttavat meihin, biologinen määräys on olla ensimmäinen????.

Kun reaktio iskee, se kasvaa sopiakseen liipaisimeen. Luonnollinen johtaja kykenee kasvattaman stimulia johonkin pisteeseen, mutta ei siitä yli. Mallory kasvattaa ja kasvattaa, paljon yli normaalin ja tarpeellisen. Vaikutus rauhoittuu jossain vaiheessa. Kuitenkin tämä supernormaali stimulishokki jättää jäljen, joka kestää vuosia.

Mallory ohjaa ryhmää kohti itsetuhoa ja tyhjentää samalla kaapit. Kun hän sanoo: ``Seuraa minua!'' koukuttuu Bob tilanteeseen, jota Bob ei voi kontrolloida. Bob kokee, että hän ei voi lähteä pettämättä hänen ihanteitansa, ystäviänsä ja hänen omia investointejansa.

Olen nähnyt tämän satoja kertoja, monesti katastrofisten vaikutusten kanssa. Se saa aikaan loppuunpalamista: äärimmäistä uupumusta, inhotusta ja masennusta. Tänään tunnistamme tämän klassisena psykopaattisen suhteen lopputuloksena.

Mikä on selvin merkki siitä, kumpaa tyyppiä organisaatio on? Oman kokemukseni mukaan se on riippumattoman tiimiin koko. ``Riippumattomalla'' tarkoitan vapautta organisoitua ja työskennellä oman halun mukaan. Tusina tai vähemmän indikoi elävää järjestelmää. Enemmän kuin kaksitoista on todennäköisesti valtapyramidi tai sen osa.

Hyvä teoria mahdollistaa uusien deduktioiden ja päätelmien tekemisen. Kokeillaanpa muutamaa:
\begin{itemize}
\item Miksi valtapyramideissa on niin vähän naisia? Johtuuko se seksismistä ja syrjinnästä? Vaikka seksismi ja syrjintä rehottavat, en usko että ne selittävät tätä asiaa. Miehet tykkäävät työskennellä naisten kanssa ainakin yleensä. Syy on osittain siinä, että valtapyramidit eivät sovi yhteen kokopäiväisen vanhemmuuden kanssa, eivätkä varsinkaan äitiyden kanssa. Syy on osittain myös siinä, että naiset tuppaavat jättämään huomiotta ``seuraa minua''-liipaisimen, joka saa miehet uhraamaan perheaikansa.
\item Kykenevätkö psykopaattiset naiset nousemaan valtapyramideissa? Tämä vaikuttaa epätodennäköiseltä. Useimmat naispsykopaatit halveksivat miesten massavallan konseptia eivätkä näytä puhuvan miesprotokollan kieltä. Sekä mies- että naispsykopaateilta puuttuvat kyvyt valtapyramidien rakentamiseksi, ja niissä menestymiseksi, lukuun ottamatta ylöspäin suuntautuvaa valloitusta. Naispsykopaatit ovat taipuvaisia sihtaamaan vaikutusvaltaisia miehiä. Useimmat valtapyramideissa menestyvät miehet eivät ole psykopaatteja, ja ovat siten haavoittuvaisia.
\item Useimmat meistä pelkäävät valtapyramideja eivätkä luota niihin hyvästä syystä. Sellaiset organisaatiot tekevät miljardit ihmiset onnettomiksi, siitäkin huolimatta, että nämä vaikutukset ovat piilossa paljon suurempien elävien järjestelmien menestystarinoiden alla. Se ei tarkoita sitä, että kaikki bisnekset olisivat myrkyllisiä, ei sinne päinkään. Se ei tarkoita sitä, että kaikesta pitäisi syyttää vapaata markkinataloutta. Todelliset vapaat markkinat toimivat elävien järjestelmien konepeltien alla. Todelliset vapaat markkinat ovat valtapyramidien vihollinen.
\item Meillä on nyt käsillä evolutiivinen selitys ennenaikaiselle harmaantumiselle kaljuuntumiselle ja kaljuuntumiselle??. Harmaantuminen ja kaljuuntuminen kertovat kypsyydestä ja iästä. Ne liipaisevat ``viisas vanha mies''-reaktion nuorissa miehissä ja pyrkavat kilpailulliset vaistot. Kun mies harmaantuu tai kaljuuntuu ennen muita, hän jäljittelee vanhan iän signaaleja. Tämä voi antaa hänelle etulyöntiaseman, jos hän on tarpeeksi fiksu käyttääkseen sitä.
\end{itemize}

\section{Naiset metsästävät naisia}

Naispsykopaatit metsästävät muita naisia. Tämä on selvää. Kysymys on, millä tavoin, ei, että tapahtuuko sitä vai ei. Kuten mies-mies-kuviota, voi tätäkin ola vaikea nähdä. Se voi olla lähes kryptistä. Sokeutemme sukupuolen vinottamalle käytökselle saattaa tehdä tästä tutkimuksesta vaikeampaa kuin mitä sen pitäisi olla. Joka kerta, kuin kirjoitan ``miehet tekevät \(X\)'' tai ``naiset tekevät \(Y\)\vmq{,}`` se tekee kipeää. Ja silti nämä yleistykset ovat elintärkeä työkalu totuuden lähestymisessä.

Miesten suhteet tuppaavat olemaan halpoja, kovaäänisiä ja julkisia. Valtapyramidit näyttävät heijastavan miesvaltaa yli kokonaisten teollisuusalojen, talouksien ja maiden.

Naisten suhteet ovat puolestaan salailevia ja syvällisiä. Ne kuljettavat elintärkeää tietämystä ihmisistä ja tapahtumista. Ennenkaikkea ne ovat tärkeä puolustus Mallorya vastaan, olipa hänen sukupuolensa mikä hyvänsä. Jotta oppisimme, kuinka nais-Mallory metsästää Alisaa, meidän täytyy dekoodata naisprotokolla. Sitten meidän täytyy selvittää, kuinka sitä voi huijata.

Naisprotokolla esittää keskustelun kahden jo ennestään toisensa tuntevan naisen välillä. Nämä kaksi keskustelevat ihmisistä ja tapahtumista. Keskustelu ei ole mitä sattuu. Se on vaihtokauppaa. Dialogi jatkuu, kunnes kummallakin naisella on se, mitä he haluavat, ja sitten se päättyy.

Tämä on jokseenkin helppo nähdä. Kaksi naista, jotka tuntevat toisensa, ja jotka ovat olleet jonkin aikaa erossa toisistaan, istuvat alas ja juttelevat. He puhuvat ja kuuntelevat vuorottain, eikä kumpikaan ole dominoiva tai alistuva. Jutteluhetken jälkeen he siirtävät keskittymisensä toisistaan takaisin muuhun maailmaan. Olisi houkuttelevaa kutsua tätä ``juoruprotokollaksi\vmq{.}'' Mutta on tarkempaa kutsua sitä ``sukimisprotokollaksi'' (engl. \emph{grooming protocol}). Se on intiimi, mutta se ei ole seksuaalinen. Oman näkemykseni mukaan kyse on ihmisten versiosta siitä sukimiskäytöksestä, mitä muut kädelliset harjoittavat.

Myös miehet käyttävät sukimisprotokollaa suhteidensa syventämiseen. Se on kuitenkin ohutta verrattuna naisten versioon. Saippuaoopperioiden tuottajat tietävät tämän. Se on naismieli, joka on pakkomielteisen kiinostunut sosiaalisista juonista kertovista tarinoista. Miesten sukimisprotokolla ei ole paljoa enempää, kuin ``Moi, sulla menee varmaan ihan jees?'' jonka päälle juodaan pari kaljaa neutraaleissa olosuhteissa. Miehet vaihtavat palveluksia, mutta se on marginaalista.

Naisten sukimisprotokolla on keskeinen naisen identiteetin ja vallan kannalta. Vaikutusvaltaisella naisella on monia suhteita toisten naisten kanssa ja hän saa arvokasta tietoa aikaisessa vaiheessa. Arvokas tieto on oikea-aikaista, tarkkaa, salaista ja yksityiskohtaista. Heikolla naisella on vähän suhteita ja hänen tietämyksensä sosiaalisesta maailmasta on huonolla tolalla---siis epätarkkaa, hyvin tunnettua ja epätäydellistä.

Sukimisprotokollalla on kolme päätarkoitusta, mitkä kaikki toimivat samaan aikaan. Ensinnäkin, kuten minkä tahansa lajin tapauksessa, sukiminen luo luottamusta kahden yksilön välille. Toiseksi, protokolla levittää tarkkaa tietoa ihmisistä ja tapahtumista läpi ihmisten yhteiskunnan. Viimeiseksi, se tunnistaa ja rankaisee huijareita.

Kaikista aiheista mistä naiset tykkäävät puhua on seksuaalinen uskottomuus ykkösenä. Kyse ei ole niin yksinkertaisesta asiasta, kuin että ``huonot uutiset liikkuvat nopeasti\vmq{.}'' Seksuaalinen uskottomuus ei ole dataa. Pettäminen on, koska toistuva huono seksuaalinen käytös on psykopatian ykköstuntomerkkejä.

Mallory valehtelee ja liioittelee sukiessa. Se antaa hänen pysyä dominoivana suhteessa. Hänen tarpeensa kontrolloida narratiivia on punainen lippu, jos sen huomaa. Se on paremmin näkyvillä, kuin hänen tarjoamansa datan huono laatu. Mallory on uhri, satutettu ja välittämisen tarpeessa. Hän raportoi hänen mieskumppaninsa viimeisimmät kauheat teot. Hän anelee apua ja tukea. Hän imartelee ja hurmaa.

Vastapainoksi Mallory saa arvokasta tietoa muista ihmisistä. Alisa on eksyksissä. Hän saa jotakin, joka aluksi tuntuu arvokkaalta ja syvälliseltä ystävyydeltä. Se on kuitenkin tyhjä, ja ajan kanssa yhä hyväksikäyttävämpi.

Nämä näyttävät olevan pääliipaisimet:
\begin{description}
\item[Dramaattinen tarinankerronta.] Brasilialaisen saippuaoopperan draama. Hahmot ovat kauniita tai pahoja tai molempia. He ovat väkivaltaisia ja emotionaalisia, ylpeitä ja kovaäänisiä. Tarinat ovat epätosia ja loputtomia. Alisasta tuntuu kuin hän olisi viisivuotias ja hän olisi kuulemassa uskomatonta iltasatua.
\item[Avuttoman uhrin näytteleminen.] Syyllinen on kumppani, työnantaja tai viranomaiset. Rikokset ovat uskottomuus, väkivalta ja varkaus. ``Hän hakkasi minut ja lapseni, vei perheen rahat ja käytti sen huoriin\vmq{.}'' Alisa kokee olevansa vanhempi sisko, pakotettu tarjoamaan neuvoja ja apua.
\item[Kuulijan imarteleminen.] Tämä tarkoittaa kohteliaisuuksia, huomiota syntymäpäiviin ja henkilökohtaisiin tapahtumiin ja ylettömiä määriä huomiota. Tämä on yksi muoto ``rakkauspommitusta\vmq{,}'' jota tutkailen myöhemmin seuraavassa luvussa. Alisa kokee olevansa tärkeä, arvokas ja rakastettu.
\item[Toisten salaisuuksien paljastaminen.] Nämä ovat negatiivisia, intiimejä, yksityiskohtaisia ja usein keksittyjä. Kuulijasta tuntuu, kuin hän saisi harvinaista ja arvokasta tietämystä. Hän kokee olevansa vaikutusvaltainen. Mallory käyttää tätä erottamaan Alisan hänen ystävistään ja kollegoistaan.
\item[Äärimmäinen, kouriintuntuva vilpittömyys.] Mallory valehtelee usein mistä tahansa. Mutta hän ei näytä minkäänlaista stressireaktiota tai epäröintiä valehdellessaan. Hän vikuttaa olevan syvällisen vilpitön hänen äänensä, ilmeidensä ja kehonkielensä perusteella. Alisan reaktio on yliarvostaa kaikkea, mitä Mallory sanoo. Se ei ole pelkästään totta, se on hypertotta. Mitä kummallisempi Malloryn vale on, sitä todemmalta se tuntuu Alisasta.
\end{description}
Kuinka Alice vastaa tällaisiin liipaisimiin, jos hän ei erää ja lähde karkuun vihastuneena? Yleensä hän avautuu ja kertoo kaikki salaisuutensa. Hän kohtele Malloryä luotettavana BFF:nä. Alisa esittelee Malloryn muille kavereilleen ja kytkee hänet sosiaalisiin aktiviteetteihin. Todellisuus iskee vasta vuosia myöhemmin. Näiden lävistävien valheiden peruuttaminen vaatii monta vastaääntä. Sitten, kun Alisa alkaa kyseenalaistaa suhdetta, voivat vauriot olla syviä. Jos hän kykenee, hän postuu suhteesta häpeissään, eikä hän puhu Mallorystä enää koskaan.

\section{Katso, olen isäsi}

Olen käynyt läpi, kuinka Mallory metsästää toisia aikuisia. Yleisesti ottaen aikuisten metsästäminen on reilua peliä, ja aikuisten voi olettaa kykenevän puolustamaan itseään. Viranomaisilla ja suurella yleisöllä ei ole juurikaan sympatiaa aikuisia kohtaan. Kun lakia rikotaan, saattavat poliisi ja poikeusistuimet astua esiin. On olemassa kaksi yleistä tapausta, jotka herättävät enemmän vihaa ja inhoa. Ne ovat, kun Mallory vaanii nuoria tai vanhoja ihmisiä.

Tarkastaan ensin nuorten ihmisten tapausta. Näemme selviä, toistuvia haavoittuvuuden kuvioita. Uskon, että nämä sekä houkuttelevat että synnyttävät psykopaatteja.

Homma lähtee lapsista, jotka ovat leikkautuneet irti perheistänsä etäisyyden, eristämisen tai hylkäämisen johdosta. Vakaissa yhteiskunnissa orvot ja nuoret rikolliste kasvatetaan taloissa. Talousromahduksen murjomassa yhteiskunnassa nuoret juoksevat pakoon ja ryhtyvät katulapsiksi. Sodassa perheet saattavat jakautua ja nuoret päätyä pakolaisleireille.

Sitten näemme, kun Mallory astuu sisään ja alkaa rakentamaan hyväksikäyttäviä verkostoja. Hän saattaa esiintyä avustustyöntekijänä, uskonnollisena organisaationa tai nuorisotyöntekijänä. Tai hän saattaa odottaa kulkupisteillä odottamassa uusia saapujia, poimien kandidaatteja. ``Hei, näytät nälkäiseltä, haluaisitko jotakin purtavaa?''

Liipaisin on kuuluminen perheeseen. Nuoret ihmiset kaukana sukulaisistaan kokevat yksinäisyyttä ja turvattomuutta. He reagoivat aikuisiin, jotka käyttäytyvät vanhemman tavoin. Kuten muidenkin liipaisimien kanssa, Mallory voi liioitella käytöstänsä ja saada aikaan vahvempia reaktioita. Aikuinen näyttää itsevarmuutta ja antaa lisää väärennettyä huomiota. Nuori reagoi luottamalla voimakkaammin. Mallory voi venyttää tätä pidemmälle ja nopeammin kuin sosiaalinen ihminen.

Nuorista ihmisistä käydään valtavan suurta maailmanlaajuista kauppaa. Joskus se kutsuu itseään ``kulttuurilliseksi'' tai ``urheilulliseksi'' vaihdoksi. Nuoret tytöt Guatemalasta, jotka kuvittelevat ryhtyvänsä tanssijoiksi. Nuoret miehet Etelä-Afrikasta, jotka unelmoivat tulevaisuudesta eurooppalaisessa jalkapallossa.

Joskus kyse on epätoivoisista vanhemmista, jotka lähettävät lapsensa kohti ``parempaa tulevaisuutta\vmq{.}'' He maksavat välittäjille, jotta he veisivät heidän lapset Eurooppaan tai Amerikkan. Numeroista emme tiedä. Tätä kaupankäyntiä ei dokumentoida. Miljoona vuodessa? Kymmenen miljoonaa? Kukaan ei tiedä. Lapset vain katoavat.

Joskus se on räikeää orjakauppaa.\linkki{dsfds} Välittäjät matkustavat köyhiin kyliin, ostaen tai kidnapaten nuoria poikia ja tyttöjä. He siirtävät nämä lapset kauas pois ja laittavat heidät työskentelemään kodeissa, tehtaissa ja bordelleissa.

Mikä ikinä onkaan syynä etäisyyteen rakastavasta ja suojelevasta suvusta, tarkoittaa se haavoittuvuutta. Haavoittuvaiset lapset houkuttelevat aina Malloryä. Hän näkee raakamateriaalia jota hän voisi omistaa, muokata, käyttää ja myydä. Bobin ei pidä kuolla, hänen ei pidä juosta karkuun, ja hänestä pitää ola hyötyä Mallorylle. Tämä rajoittaa Malloryn mielenkiintoa.

Sitten Mallory rakentaa kauppaverkoston kaltaistensa kanssa. Hän aloittaa liikuttamaan nuoria ihmisiä ylös ja alas tässä verkossa. 
Hän erikoistuu ostamiseen. Tai kenties valitsemaan ja kouluttamaan lapsia erilaisia rooleja varten. Tai siirtämään heitä yli rajoijen, Eurooppaan ja Amerikkaan, missä heidän arvonsa on korkeampi.

Lasten salakuljettaminen rajojen yli on edelleen helppoa ja halpaa. Täytyy vain tietää, kuinka se tehdään. Väärennetty passi maksaa 500--2000 euroa riippuen maasta. Se on oikea lapsen passi, jossa on uhrin kuva. Tummat lapset näyttävät kaikki samalta, eikö? \emph{Päivitys: ainakin Belgiassa tämä porsaanreikä suljettiin vuonna 2015, kun alaikäisten passinhaltijoiden tarkastamista parannettiin.}

Ja sitten Mallory suodattaa porukasta nuoria potentiaalisia psykopaatteja. Hän valmentaa heitä, käyttäen muita lapsia harjoitusmateriaalina. Hän ylentää heitä ja tekee heistä oikeita käsiänsä.

Tämä lapsikauppa on vanha, häpeän vartioima ongelma, jonka ratkaiseminen on ollut vaikeaa. Ensin halveksitaan hikipajaa ja sen jälkeen pidetään päällä t-paitaa. Hyvin usein se piilottelee perheissä, perinteen ja rasismin kerrosten alla.

Kuinka ratkaista tämä onelma? Voimme toivoa vähentävämme sotia, vakauttavamme talouksia ja vahvistavamme perheitä. Voimme kasvattaa tietämystä hyväksikäytön mekanismeista. Voimme yrittää pitää pedot poissa haavoittuvaisista lapsista. Mutta emme voi poistaa psykopaatteja, tai kytkeä heidän petomaista luonnettaan pois päältä. Kuka vahtisi vahtijoita?

Näyttää siltä, että on olemassa toinenkin vastas, minkä uskon ilmaantuvan pikku hiljaa. Malloryn strategiassa on yksi heikkous: nuoren Alisan ja Bobin täytyy olla yksin, kaukana avusta. Ongelman ydin on eristys, mikä tarkoittaa sitä, että Malloryn väärennetylle vanhemmuuden maskille ei ole tarjolla vaihtoehtoa. Jos sinulla ei ole ketään, kelle puhua, Mallory näyttää ystävältä.

Meidän täytyy antaa lapsille työkaluja luoda heidän omia sosiaalisia verkostoja, kasvotusten tai Internetissä. Opettaa lapsille, kuinka pyytää apua, vertaisilta ja muilta.

Kyky pitää yhteyttä toisiin ihmisiin Internetin välityksellä, käyttäen henkilökohtaista laitetta, on elintärkeää. Se on yhtä tärkeää kuin puhdas vesi, koulutus ja pääsy terveydenhuoltoon. Jonain päivän teknologia on melkein ilmaista ja saatavilla jokaiselle planeetan lapselle.

\section{Hyvät kuuntelijat}

Elämän toissessa päässä on vanhuus. Tarkastellaan vanhuksia, joilla on omaisuutta. Voisi kuvitella, että mitä vanhemmiksi tulemme, sitä enemmän kykenemme vastustamaan huijareita. Mutta ei se ole niin. Vanhusten riisuminen omaisuudesta on melkein teollisuudenala. Se ei johdu vanhusten dementiasta, eikä siitä, että he olisivat keskimääräistä typerämpiä. Se johtuu siitä, että psykopaatit ovat hyviä tässä. Yksinäisyys tarkoittaa haavoittuvuutta.

Suurperheitä ei ole enää ainakaan useimmissa länsimaissa. Tämän takia moni vanhus jää yksin\linkki{sdfds} kotiinsa tai vanhainkotiin. Heidän lapsensa ovat aikuisia, joilla on omat perheet jossain kaukana. Kymmenet vuodet talouskasvua tarkoittaa sitä, että monella vanhuksella on omaisuutta. Tämä sukupolvi tarjoaa tuottoisan kohteen psykopaateille. Ja Mallory tähtää sitä huolellisesti ja tarkasti.

Tähän liittyy useamman tyyppisiä hyökkäyksiä, joita olen havainnut, ja jotka selostan:
\begin{description}
\item[Avulias neuvonantaja.] Mallory houkuttelee vanhemman Alisan uhkapelaamaan. Hän ottaa perheelle työskentelevän taloudellisen neuvonantajan roolin. Hän käyttää lasten ahneutta heidän äitiänsä vastaan. Ehkä hän ehdottaa, että Alisa ottaa lisää velkaa, käyttäen hänen taloaan panttina. Mallory saa hyvän komission. Lapset saavat rahaa. Alise huomaa myöhemmin olevansa kykenemätön maksamaan velkaa. Perhe menee vararikkoon. Liipaisimet ovat perheen paine ja ahneus.
\item[Avuton muukalainen.] Mallory kysyy vanhemmalta Bobilta rahaa. Mikä tahansa tekosyy päästää hänet olohuoneeseen. Pikkuinen juttuhetki luottamuksen saamiseksi. Sitten jotakin, vaikkapa: ``Nyky-yhteiskunnan ongelma on ihmisten itsekkyys. Kukaan ei enää välitä mistään. Äitinikin on kuolemassa syöpään, ja pankkimme yrittää potkia meidät ulos talostamme!'' Bob saataa kysyä, josko hän voisi auttaa. Mallory kieltäytyy suoraan, kyynelten valuessa hänen silmistään. Bob vaatii! Mallory kieltäytyy taas, sanoen, että hän kyllä löytää jonkun tavan. Bobin ehdotus loukkaa häntä. Mutta Bob on peräänantamaton, ja Mallory poistuu paikalta mukanaan kirjekuori täynnä rahaa. Tässä syyllisyydentunto on liipaisin.
\item[Vanhemman auktoriteetti.] Mallory saa täyden kontrollin vanhempaan Alisaan. Kontrolli on emotionaalista, fyysistä ja lopulta taloudellista. En sano, että kaikki yksityiset hoivakodit olisivat tällaisia. Vain tietty osa niistä. Painostavat Alisan toimimaan ja tuntemaan kuin lapsi. Hän vastaa auktoriteettiin, joka sanoo: ``Allekirjoita tämä dokumentti, kiitos\vmq{.}''
\item[Korvikelapsi.] Mallory käyttäytyy kuin vanhemman Bobin lapsi. Hän löytää tavan viettää aikaa Bobin kanssa. Hän kuuntelee, kysyy viattomia kysymyksiä. Vain olemalla huomaavainen ja alistuva hän liipaisee vanhempimaisen reaktion Bobissa. Sitten hän tulee huoliensa ja ongelmiensa kanssa. Bob yrittää ratkaista niitä. Mitä suurempi ongelma, sitä enemmän Bob auttaa.
\end{description}

\section{Arvaa, kuka tulee päivälliselle?}

On vielä yksi areena, jossa Mallory metsästää. Se on itse perhe. Perhe-elämä ei tuo usein sisään muukalaisia. Se kuitenkin luo hyvää suojaa. Perheet sietävät merkittävän määrän vallan epätasapainoa. Mallory voi saada aikaan erikoislaautisia vaurioita perheeseen sisältä päin. Kaikki tämä voi olla näkymättömissä ulkopuolisille, jopa aktiivisille havaitsijoille.

Voimme rikkoa tämän kahteen päätapaukseen. Joissakin tapauksissa Mallory hallitsee perhettä jo valmiiksi. Hän työskentelee säilyttääkseen ja laajentaakseen valtaansa. Useimmissa tapauksissa perhe ei ole\ldots\emph{saanut tartuntaa}\ldots ja Mallory yrittää astua sisään ja saada valtaa. Ensimmäinen tapaus koskee jokaisen uuden tulokkaan resurssien varastamista. Toinen taas koskee koko ryhmän resurssien varastamista.

Vauriot, joita Mallory voi saada aikaan perheelle, voivat olla äärimmäisiä. On henkilökohtaiset vauriot, trauma ja vallan menetys. On kollektiiviset vauriot, omaisuuden ja säästöjen menetys. Katsahda perhettä, joka on murtunut kiistojen takia. Katso tarkemmin, ja näet Malloryn työssään.

Kuinka Mallory astuukaan isäntäperheeseen? Selvin vaihtoehto on mennä naimisiin. Vanhempien reaktio uuteen kumppaniin on usein niin äärimmäinen, että siitä on tullut suosittu karikatyyri. Ottaen huomioon psykopaattien vaaran, epäilys ja vihamielisyys on normaalia. Kaikki muu olisi piittaamattomuutta.

Miesten on vaikea ymmärtää naisten todellisia motivaatioita ja luonnetta. Tiellä on liian paljon signaaleja ja liipaisimia. Joten uusien tyttöystävien arviointi jää äidin tehtäväksi. Vastaavasti naiset ovat liian huonoja arvioimaan miehiä. Niinpä uusien poikaystävien on voitettava puolellensa isän luottamus.

Klassinen kuvio alkaa kuulustelulla ja kredentiaalien?? tarkastuksella. Näitä seuraa joko ehdollinen hyväksyminen tai hylkäys. Sitten hetken koeaika. Sitten juhlat ja ehkä vauvoja, ainakin vanhempien mielessä.

Tämä draama toistuu uudelleen ja uudelleen, sekä todellisessa elämässä että populaarikulttuurissa. Meillä on vanhemmat ja heidän halunsa nähdä heidän tyttärensä ja poikansa saamassa lapsia. Meillä on heidän epäluottamuksensa uutta tulokasta kohtaan.

Meillä on nuoruus ja sen vatimukset itsenäisyyteen ja itsemääräämiseen. Nämä seikat mahdollistavat laajan skaalan hahmoja ja juonenkäänteitä.

Otetaan esimerkiksi paljon parjattu anoppi, joka on vitsien peruskohde joka ihmisyhteiskunnassa. Harva naimisissaoleva mies pitää anopistaan. On vaikea antaa anteeksi ihmiselle, joka lähtee siitä ajatuksesta, että olet psykopaatti. Ironia on rikasta. Anoppi joka ei kyseenalaista tyttärensä valintoja usein tarkoittaa ongelmia edessä.

Ero jakaa perheen ja paljastaa sen pedoille. Jos isä on poissa, tilanne on helpompi miespsykopaateille. Ero yleensä levittää omaisuutta. Joten siinä, missä eronjälkeiset perheet ovat helpompia kohteita, ne ovat yleensä myös vähemmän kannattavia kohteita.

Kun perhe on varakas maassa, jossa on heikko valtio, se on tuottoisa kohde. Vahvat perheet kehittyvissä maissa kehittävät järjestettyjen avioliittojen kulttuurin. Ne arvioivat kandidaatteja vainoharhaisesti. Yksi huono valinta voi tuhota sukupolvien verran kasautunutta varallisuutta.

Tämä malli antaa meidän tehdä ennusteen missä tahansa yhteiskunnassa: \emph{järjestettyjen avioliittojen määrä korreloi parin sosiaalisen statuksen kanssa.} Mitä korkeampi status, sitä vähemmän vapautta valita. Tämä näyttää pitävän paikkansa kaikissa yhteiskunnissa. Yhteiskuntien välillä heikompi valtio tarkoittaa suurempaa määrää järjestettyjä avioliittoja. Tämä johtuu siitä, että heikot valtiot eivät kykene suojelemaan perheiden omaisuutta pedoilta.

Liiton jälkeen seuraava tie perheen kaapistoihin käy viettelyksen kautta. Tällaisista suhteista tuppaa tulemaan yleistä tietoa. Kun nuori mies lounastaa lesken kanssa, tai kun nainen käy treffeillä kaksi kertaa hänen ikäisensä miehen kanssa, kysymme saman kysymyksen. ``Kuinka paljon rahaa on pöydällä?'' Jos vastaus on ``paljon\vmq{,}'' ajattelemme pahinta. Vain siinä tapauksessa, että rahaa ei ole pelissä, saatamme uskoa, että kyse on rakkaudesta.

Mallory kykenee viettelemään naimisissa olevan miehen saadakseen lahjoja ja rahaa. Hän käyttää seksuaalisia liipaisimiaan luodakseen kaiken kuluttavan riippuvuuden miehelle. Tämä toimii tehokkaammin, jos mies on liitossa, josta romantiikka on haihtunut aikapäivää sitten. Se on banaalia. Joskus Mallory ystävystyy vaimon kanssa. Hän vakuuttaa vaimon siitä, että huoleen ei ole syytä. Mallorystä tulee vaimon paras ystävä, ilman pienintäkään pelon pistosta tai katumusta.

Mies-mallorylle on vaikeampaa päästä tällä tavalla perheeseen. Hän voi vetää Bobin mukaan rahantekojuoniin, jotka päättyvät katastrofeihin ja häviöihi. Tällä tavoin Wall Streetin Malloryt tyhjentävät monia säästötilejä. Tai hän voi vietellä Bobin vaimon Alisan, ja usutaa Alisan varastamaan hänelle.

\section{Johtopäätökset}

Malory on kyltymätön, aina nälkäinen, ja ajattelee seuraavaa ateriaansa. Kun hän löytää uskottavia mahdolisuuksia, hän lähestyy ja vaihtaa tapojaan ja käytöstään. Nyt hän käyttäytyy arasti ja hillitysti. Sitten hän vaikuttaa dominoivalta ja ylimieliseltä. Hän näyttää hämmästyttävän skaalan käytösmalleja. Hän muuttaa ääntään ja kehonkieltään, asentoaan ja aksenttiaan. Hän muuttuu keneksi ikinä hänen tarvitseekaan muuttua päästäkseen lähemmäs saalistaan.

Muotoaan muuttavien maskien alla Malloryllä on todellinen ja johdonmukainen persoonallisuus. Se ei ole näkyvillä juuri koskaan. Jos se sattuu vilahtamaan, on sitä vaikea tunnistaa ihmiseksi. Psykologi Kathleen Vohs\linkki{sdfds} on näyttänyt kuinka ihmisten pohjustaminen ajattelemaan rahaa tekee heistä antisosiaalisia, epäempaattisia ja taipuvaisempia huijaamaan. Toisin sanoen enemmän Malloryn kaltaisia.

Mallory näkee maailman tällä tavoin koko ajan. Hän ei ole niinkuin muut ihmiset omassa mielessään. Ja muut eivät ole niinkuin hän. Muut ovat paperinohuita muotoja, joilla on yksinkertainen joukko ominaisuuksia. Jotkut ovat mehukkaista, jotkut ovat tylsiä, jotkut ovat hyödyllisiä, jotkut vaarallisia. Hän näkee maailman saaliina ja itsensä synnynnäisenä petona.













