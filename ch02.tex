\chapter{Metsästys}

\section{Juhlakalu}

\begin{tarina}

Hän vaeltaa lempipaikkaansa, joka on suuri baari sekä ravintola. Paikka on suosittu kovaäänisen hauskanpitoa etsivän nuorison keskuudessa. Kesäinen lauantai-ilta on alkamassa. Hän juttelee portsarin Miken kanssa. Suuri mies nauttii rutiinin keskeyttävästä hetkestä ja kertoo pieniä palasia elämästään. Ex-tyttöystävä ja odottamaton lapsi. Pomo. Hän heittää portsarin kanssa ylävitoset ja lopettaa keskusteluhetken. ``Törmätään myöhemmin, Mike, käyn hakemassa itselleni kaljan\vmq{.}'' Hän liikkuu baarin toiselle puolelle.

Ulkoa terassilta hän löytää suuren pyöreän pöydän ja istuu siihen kylmän pullonsa kanssa. Ihmisiä valuu sisään. Hän katsoo heitä. Sekalaista sakkia, niinkuin täällä päin Texasia voi olettaa. Maahanmuuttajat tulevat tänne joka puolelta Yhdysvaltoja ja kauempaa. Miehet tsekkailevat naisia. Naiset käyttäytyvät kuin he eivät huomaisi miehiä. Joitakin pareja. Jotkut seisovat yksin selkä seinää vasten, kehonkielen huutaessa, ``Kumpa olisin pitempi\vmq{.}''

Paikka on kohta täynnä. Pienehkö nuorten miesten porukka istuu rappusilla pyöreän pöydän vieressä. Valkoisia, mustia, latinoja. He näyttävät viihtyvän huonosti. Hän kääntyy ja kysyy: ``Mistäs te kaikki olette kotoisin?'' Sotilaita läheisestä armeijan tukikohdasta viettämässä iltaa. Hän viittoo käsillään pöytäänsä. ``Liittykää seuraan\vmq{,}'' hän sanoo, ``pöydän ääressä on parempi meininkin\vmq{,}'' hän hymyilee. Miehet hyväksyvät ehdotuksen ja nousevat ja siirtyvät pöytään. He iloitsevat tuoleista ja siitä, että he olivat tervetulleita pöytään.

Nämä nuoret miehet ovat fiksuja ja uteliaita. Heitä ei ole vielä lähetetty sotaan, ja he ovat optimistisia ja luottavaisia. Hän kertoo heille polveilevia tarinoita ulkomaan seikkailuistaan. He nauravat jännityksestä. Hän pysähtyy, asettaa kätensä pöydälle ja toteaa itsestäänselvyyden. ``Tarvitsemme naisia!'' Yksi sotilaista osoittaa leuallaan: ``Nuo kaksi?'' Hän kääntyy nähdäkseen kaksi nättiä tummahiuksista naista. He näyttävät tylsistyneiltä ja epävarmoilta. ``OK, älkää liikkuko!'' hän sanoo miehille ja nousee ylös.

``Hei, leidit, miten menee?'' hän kysyy heiltä, kuuntelematta vastausta, mikä on aina ``mikäs tässä'' tai ``ihan jees\vmq{.}'' Hän tarkkailee mahdollisia merkkejä kiusallisuudesta. Naiset vaikuttat pitävän miehen kanssa juttelemisesta. ``Odotatteko jotakuta?'' hän kysy, ja vastaus on kieltävä: he ovat liikkeellä kahdestaan. Hän rypistää otsaansa ja tarkastelee heidän piirteitään.

``Mistä olette kotoisin?'' hän kysyy. ``Arvaa\vmq{,}'' vastaa toinen nainen nauraen. Hän yrittää paikallistaa heidät. Tummat kulmakarvat, tummanvihreät silmät, vaalea iho, korkeat poskipäät. Libanon? Georgia (maa, ei osavaltio)? Naiset nauravat ja pudistelevat päitään. ``Ei\vmq{.}''

``Haluaisitteko liittyä seuraamme? Pöydässämme on tilaa\vmq{,}'' hän sanoo, pyyhkäisten kädellänsä ilmaan laajan kaaren. He katsovat komeaa, pystytukkaista miestä kasvoihin, kohauttavat olkiaan, ja hyväksyvät tarjouksen. ``Miksipä ei!''

Naiset istuvat hänen viereensä. Mies puhuu heille, arvaillen lisää väärin. Venäjä? Armenia? He nauravat. Sotilaat ostavat heille juomia. Kaikki ovat iloisia, juhlat ovat parhaimmillaan. Lopulta mies tunnustaa tappiansa, jolloin naiset kertovat olevansa kotoisin Intiasta. Hän on shokissa, vaikuttunut ja lumoissaan.

``Kaunein nainen, jonka olen koskaan tavannut, oli kotoisin Georgiasta. Juttelimme viitisen minuuttia, ja halusin hänen kanssaan naimisiin siinä paikassa. Teillä molemmilla on samat piirteet. Olin varma, että olette Georgiasta! Mutta että Intiasta\ldots vau, Intiasta!''

``Kyllä, Intiasta!'' he nauravat imarreltuina. He viihtyvät tilanteessa. He juttelevat läpi yön. Baari menee kiinni ja asiakkaat siirtyvät parkkipaikalle. He lähtevät viimeisenä. Sotilaat hyvästelevät ja menevät matkoihinsa, ja nämä kaksi naista jäävät miehen seuraan. ``Haluaisitteko mennä jonnekin muualle?'' hän kysyy. ``Autoni on tuolla\vmq{.}'' Hän painaa avaimen nappulaa ja uuden avo-Mustanging valot välähtävät.

Myöhemmin toinen naisista kysyy: ``Kuinka kauan oletkaan tuntenut nuo kaverit?'' Hän vastaa: ``Aa ne, en pitkään, tapasin heidät vasta tänä iltana\vmq{.}'' ``Mitä?!'' hän huudahtaa shokeeraantuneena. ``Luulimme, että olet tuntenut heidät vuosia! Olitte kuin parhaat ystävät!''

\end{tarina}

\section{Sosiaalinen peto}

Psykopaatit kykenevät kohdistamaan lumoavan määrän voimaa toisiin ihmisiin. Se on kuin kahden kultti. Kun tapaamme sellaisia ihmisiä ja elämämme alkaa kietoutua heidän elämäänsä, meistä tuntuu siltä, kuin kohtalo kuljettaisi meitä. Tunne on kummallinen sekoitus varmuutta ja kontrollin menetystä. Aivan kuin putoaisi kovassa tuulessa. Uskonnollisen fanaattisuuden kuuma tuli. Ja se näyttää aina päättyvän kyyneliin.

Kysymys, jota muut usen kysyvät, on ``Miksi?'' Suhteet psykopaatin ja sosiaalisen ihmisen välillä ovat niin tuhoisia ja karvaita. ``Miksi'' on hyvä ruutu aloittaa. Kun kykenemme vastaamaan siihen, voimme alkaa kysyä ``Kuinka?'' ja ``Kuka?'' ja muita syvempiä kysymyksiä.

Tarinani jokaisen luvun alussa kertovat aina jonkunlaisesta pedosta. Jokainen psykopaatti toimii tällä tavoin. Psykopaatit metsästävät toisia ihmisiä. He hyökkäävät ja sieppaavat uhrinsa. He elävät uhriensa ajasta, resursseista, vallasta ja energiasta. He hankkiutuvat eroon jäänteistä. Ja he liikkuvat eteenpäin.

Väkivalta on piilevää. Joskus se päättyy uhrin itsetuhoiseen käytökseen, jopa itsemurhaan. Yleensä se päättyy masennukseen. Jokainen suhde sosiaalisen ihmisen ja psykopaatin välillä noudattaa samaa kaavaa. Vaikuttaa siltä, että poikkeuksia ei ole; ``kivoja'' psykopaatteja ei ole olemassa. Psykopaattius tarkoittaa sitä, että on peto.

Tämä ei ole metafora. Tämä on avain psykopatian dekoodaamiseen. \emph{He ovat petoja tai loisia, jotka elävät toisten ihmisten kustannuksella.} Ilman tätä avainta psykopatia on mystinen ja hämmentävä ilmiö. Kuin muinainen kirjoitus täynnä symboleja ja glyphejä. Teksti, joka vaikuttaa niin moneen meistä, ja jonka avaaminen on kuitenkin mahdotonta. Avaimen avulla voimme lukea tarinat ja voimmme ymmärtää.

Siinä, missä kuvailuni koskevat usein yksittäisiä suhteita, kaavat toimivat monissa tilanteissa. Näemme ne kulteissa, hyväksikäyttävässä yritystoiminnassa ja muissa petomaisissa organisaatioissa.

\section{Mallory, Alisa ja Bob}

Tietoturva-alalla vihamielistä hyökkääjää kutsutaan silloin tällöin Malloryksi. Samalla viattomia uhreja kutsutaan Alisaksi ja Bobiksi. Käytän näitä nimiä tässä kirjassa, jotta teksti olisi helpommin luettavaa ja pureskeltavaa.

Mallory voi olla mies tai nainen. Kirjoitan naisesta tai miehestä sen mukaan, mikä sattuu sopimaan tilanteeseen. Hän on aikuinen, vähintäänkin 14--16-vuotias, ja alle 70-vuotias. Mallory on psykopaatti.

Alisa ja Bob ovat altruistisia, sosiaalisia ihmisiä. He ovat Malloryn huomion kohteita.

\section{Kävele näin}

Ensitapaamisesi Malloryn kanssa on intensiivinen, henkilökohtainen ja syvä kokemus. Siis Alisalle ja Bobille. Mallorylle se on merkityksetön, kasuaali refleksi. Kun hän sanoo ``Moi'' sadalle ihmiselle, 96\% vaikuttuu. Hymy, silmät, tuon yksinkertaisen tervehdyksen \emph{syvyys}. Mallorylle tervehdys ei tunnu missään, eikä se vaadi häneltä minkäänlaista emotionaalista panostusta.

Tämä on se ``karisma'' josta puhutaan. Se ilon heijastus kun tapaamme jonkun, josta välitämme kovasti. Alisa ja Bob eivät voi väärentää sitä. He näyttävät sen ainoastaan niille, joista he välittävät. Se, että välittää \emph{jokaisesta} tapaamastaan ihmisestä on äärimmäinen perspektiivi, jonka löytäminen vie vuosikymmeniä. Mallory jäljittelee tätä refleksinomaisesti nuoresta iästä lähtinen. Se ei vaadi opettelua. Se on hänen ensimmäinen selviytymisen sääntö: \emph{muiden on ihailtava sinua.}

Vaikutus on niin voimakas, että voit käyttää sitä psykopaattien tunnistamiseen luonnossa. Palaan tähän luvussa \ref{hunting-mallory}. Useimmat, jotka törmäävät Malloryyn, ovat taipuvaisia kokemaan pikkuruisia mielihyvän purkauksia. Jos henkilö on edes pikkuruisen yksinäinen, houkuttelee se hänet takaisin keskusteluun. Samalla Mallory skannailee uusia kohteita. Siihen ei tarvita keskustelua. Hän näkee ihmisten haavoittuvuudet kehonkielestä.

Sosiaaliset ihmiset voivat oppia tämän kyvyn vuosien harjoittelun myötä. Mallory ei tarvitse harjoittelua. Se on yksi hänen monista sisäsyntyisistä kyvyistä. \emph{Forbes}-lehti kirjoittaa:\linkki{sdfsd} ``Vaikuttaa siltä, että psykopaatit eivät tarvitse meditatiivista harjoitusta ollakseen suhteettoman tarkkaavaisia{\ldots} toisten heikkouksien suhteen\vmq{.}''

Heikkoutemme näkyy erityisesti kahdessa asiassa. Ensinnäkin yksinäisyys ja yksinäisyydestä kertova kehonkieli ovat osoituksia heikkoudesta. Toiseksi, pelon ja epävarmuuden näyttäminen ja hyväksikäytön kokemusten näyttäminen kertovat heikkoudesta.

Monet uskovat, että hyväksikäytön uhreista tulee itsekin hyväksikäyttäjiä. Prosenttiosuus on kuitenkin vain kymmenen luokkaa,\linkki{sdfds} paitsi jos hyväksikäyttäjä ja hyväksikäytetty ovat samasta perheestä. Siinä tapauksessa prosentti nousee merkittävästi. Ja joka kolmas näistä aikuisistä hyväksikäyttäjistä oli lapsena julma eläimiä kohtaan.

En epäile, etteikö psykopaatit hyväksikäyttäisi, laiminlöisi ja kiduttaisi henkisesti omia lapsiaan, ja etteikö moni näistä kasvaisi psykopaatiksi. Kyse on mekanismista, jonka olen havainnut, ja jonka selostan myöhemmin. Mutta tämä ``hyväksikäyttö aiheuttaa hyväksikäyttöä''-malli jättää huomiotta ne 90\% lapsista, jotka kokevat seksuaalista hyväksikäyttöä, ja joista kuitenkin kasvaa aikuisia, jotka eivät vahingoita toisia. Mielestäni tämä johtuu sosiaalityöntekijöistä, joita nuoret psykopaatit ovat onnistuneet huijaamaan. ``Isäni hyväksikäytti minua, ja siksi satutan muita\vmq{.}'' Psykopaatit eivät koskaan ota vastuuta teoistaan.

Todellisuudessa hyväksikäytön uhrit ovat taipuvaisia todistamaan oman elämänsä traumoja sanomatta sanaakaan. Tai, kuten Joanna Moore kirjoittaa kirjassa? \emph{The Faces of Narcissism},\linkki{sdsf} ``on helppoa syyttää vihaista uhria ja tukea rauhallista hyväksikäyttäjää\vmq{.}''

Menneisyydessä koettu hyväksikäyttö on ensisijainen indikaattori tulevaisuudessa koettavalle hyväksikäytölle. Hyväksikäyttävästä perheestä tulemin leimaa meidät pelolla ja epävarmuudella. Hyväksikäyttävä työnantaja tai puoliso saa aikaan saman. Pelkomme ja epävarmuutemme on kuin neonvaloa hohtava ``Syö minut!''-kyltti kaikille ohikulkevilli psykopaateille.

Toisten ihmisten pelko näkyy kehonkielessämme.\linkki{sdfds} Hyväksikäytön uhrit nostavat jalkojansa korkeammalle kävellessään. He ottavat keskimääräistä pitempiä tai lyhyempiä askelia. He sätkyttelevät käsiänsä ja jalkojansa. He välttelevät katsekontaktia. He näyttäväþ alistuvia ja puolustavia kehon asentoja.

Kaikki nämä vihjeet ovat helppoja luettavia, jos lukijalla on sopivanlainen mieli. Rikollisista psykopaateista tehdyt tutkimukset osoittavat, kuinka psykopaatit poimivat tällaisia vihjeitä.

Minulla ei ole numeroita siitä, kuinka nopeasti tämä tapahtuu, ainoastaan anekdoottisia kertomuksia. Voisin arvata, että Mallory kykenee käymään läpi sata ihmistä suurin piirtein kymmenessä minuutissa.

\section{Suuret, siniset munat}

Todennäköisiä kohteita etsivä Mallory kävelee läpi väkijoukon. Hän projisoi seksuaalisuuttaan vain hiukan muita naisia kovemnin. Hän etsi yksinäisiä, menestyviä miehiä. Hän tarkkailee, kuinka miehet katsovat häntä, ja näkee yhden hermostuneen reaktion, joka kiinnittää hänen huomionsa. Hän nousee ylös ja heilauttaa hiuksiansa, hengittää sisään, hymyilee hänelle. Hän katsoo miehen kasvoja. Hän katsoo hiukan liian pitkään. Hän hymyilee itselleen.

Valittuaan Bobin kaikkien potentiaalisten kohteiden joukosta, Mallory tekee siirron. Mitään näkyvää takaa-ajoa, juoksemista tai kirkumista ei tapahdu. Siirrot eivät kerro totuutta. Mallory liukuu sisään Bobin elämään kuin kauan kadoksissa ollut ystävä. Hän vaikuttaa niin kivalta, harmittomalta ja vilpittömältä.

Hän avaa pelin laajalla paletilla taktiikoita, joka riippuu kontekstista. Nämä hyökkäykset toimivat sekä hyökkääjässä että uhrissa vaistojen tasolla. Hän aloittaa laajalla, keskittämättömällä luotaamisella. Kun Bob vastaa vaistonvaraisesti, Mallory siirtää ja säätää peliään ja painaa kaasua.

Vaistonvaraisen käytöksen liipaisijat ovat yleensä yksinkertaisia karikatyyrejä. Evoluutio on tässä mielessä laiska. Esimerkiksi monet ihmiset pelkäävät hämähäkkejä niin kovasti, että ne saavat heidät kirkumaan. Liipaisin sijaitsee geeneissämme monella jalalla varustettuna mustana pisteenä. Pelkäämme tiettyä tapaa, jolla jalat liikkuvat. Piirretty hämähäkki, joka kävelee oikealla tavalla, on yhtä pelottava kuin oikea hämähäkki. Laita se kävelemään kuin ihminen?, ja se näyttää harmittomalta. Liioittele hämähäkkikävelyä ja piirretty hämähäkki on oikeaa pelottavampi.

Kun liipaisimen eristää ja sen voimakkuutta kasvattaa, kasvaa myös reaktio. Vain taivas on rajana. Otetaan esimerkiksi lajimme makeanhimo. Reagoimme vaistonvaraisesti fruktoosiin, jota kasvit pumppaavat hedelmiin pieninä annoksina. Makeus iskee samoihin aivojemme alueisiin kuin kokaiinin kaltainen huume. Luonnossa tämä ajoi kaksijalkaiset edeltäjämme syömään niin paljon hedelmiä kuin he vain kykenivät löytämään. Sitten jalostimme aina vain makeampia hedelmiä. Sitten opimme jalostamaan sokeria ja aloimme lataamaan sitä ruokavalioomme. Syömme satoja paunoja sokeria vuodessa, itsetuhoon asti.

Vaistonvaraisen fruktoosihimomme rajat eivät tulleet missään vaiheessa tätä tarinaa vastaan. Sen sijaan mitä enemmän kulutamme, sitä onnellisemmaksi näytämme tulevan.

Tämä eskaloituva reaktio tiivistettyyn liipaisimeen on tunnettu ilmiö nimeltä ``supernormaali stimuli\vmq{.}''

Pedot ja loiset ovat erikoistuneet käyttämään supernormaalia stimulia saaliiseensa. Se pakottaa itseään rankaisevaan ja epäloogiseen käytökseen, kunnes liipaisumekanismin ymmärtää. Joten parasiittinen lintulaji saattaa kaapata ne liipaisimet, joita nuoret poikaset käyttävät ruoan kinuamiseen. Esimerkiksi avoin punainen suu. Parasiitti imitoi ja liioittelee tätä liipaisinta, joka saa lintuemon ruokkimaan loispoikasta ennen sen omia poikasia.

Merien syvimmissä vesissä krotti roikuttaa pimeässä loistavaa kirkasta syöttiä. Tämä liipaisee saaliskalan uimaan kohti sen hampaista suuksi kutsuttua ansaa.

Tai mietitäänpä puussa pesivän laululinnun munia. Ne ovat usein vaaleansinisiä munia, joissa on tummanharmaita tai ruskeita pilkkuja. Tämä nimenomainen värikaava liipaisee naaraan tai koiraan istumaan munan päällä. Ehkä siksi, etteivät ne istuisi satunnaisten kivien tai toisen lintulajin munien päälle. Parasiittinen käki munii pesään suurempia, sinisempiä munia, joissa on tummempia pisteitä. Tämä saa laululinnun suosimaan käen munia omiensa sijaan.

Kilpavarustelu loisen ja isäntälajin välillä luo luonnollisen tasapainon. Liipaisimen liiallinen käyttäminen kääntyy parasiittia itseään vastaan. Jos käki tekisi munistaan liian houkuttelevia, haavoittuvaiset laululinnut eivät lisääntyisi ollenkaan. Laululinnut, jotka eivät reagoi liipaisimeen, saisivat etulyöntiaseman ja dominoisivat. Loiselle isäntälajin tappaminen on häviävä strategia. Tämä tarkoittaa sitä, että ainoastaan vastustuskykyiset isäntälajine edustajat lisääntyvät.

Niko Tinbergen, biologi joka löysi ja nimesi supernormaalin stimulin, rakensi muovisia munia. Hän havaitsi, että linnut suosivat muovimunia omiensa sijaan. Ne suosivat omiaan suurempia munia. Ne suosivat normaalia kylläisempiä värejä. Ja ne suosivat niiden omien munien kuvioita rajumpia kuvioita.

Joten pienen vaaleansinisen harmaantäplikkään munan sijaan hän tarjosi laululinnulle väärennöksen. Hänen munansa oli valtava, kirkkaan sininen, ja siinä oli suuria mustia täpliä. Lintu yritti istua munalle, yhä uudelleen ja uudelleen, ja putoili sen päältä pois.

Tämä saattaa naurattaa, mutta linnulle tämä on järjetöntä käytöstä. Näemme, että supernormaali stimuli voi tuottaa järjetöntä käytöstä järkevästi kehittyneistä vaistoista. Kyse on evolutiivisesta porsaanreiästä, jota monet pedot ja loiset käyttävät hyväkseen. Ihmispsykopaatit käyttävät sitä usein manipuloidakseen kohteitaan haluamaansa suuntaan.

\section{Avaussiirrot}

Vuonna 1989 Clatk ja Hatfield Floridan Yliopistosta? tekivät kuuluisan tutkimuksen.\linkki{dsfs} Heidän viehättävä tutkimusassistenttinsa käveli ympäri kampusta ja ehdotti treffejä ihmisille.

Tulokset ovat tunnettuja. Yli puolet miehistä hyväksyivät treffiehdotuksen, ja vielä useampi oli valmis menemään kotiin vieraan naisen kanssa. Kolme neljäsosaa hyväksyi suoran seksiehdotuksen. Naiset taas yleensä sanoivat ``ei\vmq{.}'' Opiskelijat eivät ehkä tyypillisiä esimerkkejä koko populaatiosta, mutta muut ovat saaneet samanlaisia tuloksia.

Tulen sukupuolien eroihin hetken päästä. Alkuperäinen tutkimus esitti, että naiset eivät harrasta satunnaista seksiä. Tiedämme, että tämä ei ole totta, ainakaan kaikissa konteksteissa. Ensimmäinen kysymykseni on: ``Kuinka niin moni mies on niin helposti koukutettavissa?'' Sitä voisi sanoa, että riski, jolle mies altistuu satunnaisessa seksissä on matala, mutta asia ei ole niin. Ilmiselviä riskejä ovat taudit ja yllätysvanhemmuus. Ja sitten on paljon suurempi vaara, että koko homma on lavastettu jonkinlaista huijausta varten.

Ja kuitenkin useimmat miehet sanova ``Ehdottomasti!'' Kuinka naisen viehätysvoima voi olla niin tehokas syötti? Ovatko miehet epätoivoisia, kiimaisia ja typeriä? Ovatko naiset fiksumpia? Ehkäpä vastaus on jotakin hieman hienovaraisempaa. Käy myös ilmi, että naiset eivät ole sen vastustuskykyisempiä kuin miehetkään. Kyse on vain siitä, käytetäänkö oikeanlaista syöttiä.

Kun vastaamme näihin kysymyksiin, näemme, kuinka merkittävä asia sukupuoli on. Sillä on oleellinen rooli psykopaatin avaussiirroissa. On olemassa neljä erillistä kaavaa: nainen miehelle, mies naiselle, mies miehelle ja nainen naiselle. Monet sosiaaliset vaistomme taipuvat kohti maskuliinista ja feminiinistä napaa, aivan kuten kehommekin.

Kehomme ja mielemme ovat lähtökohtaisesti feminiiniset. Kun mies kehittyy, ajoitetut testosteronipurkaukset siirtävät kehoa ja mieltä kohti miehuutta. Miehet ja naiset eroavat kehoiltansa ja mieliltänsä evoluutiota ajavien voimien vuoksi.

Joten kun sanon ``mies\vmq{,}'' se sisältää naiset, joilla on miehelle tyypilliset vaistot. Ja kun sanon ``nainen\vmq{,}'' se sisältää miehet, joilla on naiselle tyypilliset vaistot. Nämä avaussiirrot eivät oleta heteronormatiivisuutta. Psykopaateilla on usein nestemäinen seksuaali-identiteetti.\linkki{sdfds} He ovat yhtä itsevarmoja ja petomaisia homoseksuaaleina kuin heteroseksuaaleina.

\section{Naiset metsästämässä miehiä}

Psykopatian ``antisosiaalinen'' osa ei tarkoita sitä, etteikö henkilö haluaisi toisten seuraa.\linkki{sdfds} Se tarkoittaa haluttomuutta kunnioittaa sosiaalisia normeja ja käytäntöjä. Psykopaatit ovat taipuvaisia hypersosiaalisuuteen ja he etsivät pakkomielteisesti uusia ystäviä. Kyse on reviiristä?.

He voivat esiintyä henkilökohtaisesti ja hienotunteisesti, mikä yleensä piilottaa intensiivisen taustatyön. Web on tehnyt tästä paljon helpompaa, tarjotessaan monia tapoja puhua toisille yksityisesti.

Haluaisin louhia Facebook-dataa yksityisten keskustelujan, julkisten postauksien ja selfieiden takia. Arvaukseni on, että löytäisimme erillisen ryhmän käyttäjiä, joilla on paljon keskimääräistä enemmän yksityisiä keskusteluja, paljon keskimääräistä useamman eri ihmisen kanssa. Ennustan, että näkisimme kaksi päällekäistä kellokäyrää, emme vain yhtä.

Pedot tykkäävät mesästää sellaisissa paikoissa ja tapahtumissa, missä heillä on etulyöntiasema. Tilanteen täytyy tarjota tuore uusien kohteiden lähde. Kohteiden täytyy haluta jotakin, mitä peto voi hyödyntää. Kohteiden pitää tarjota potentiaalista hyötyä metsästäjälle. Tilanteen tulee tarjota suojaa ennen ja jälkeen minkä tahansa hyökkäyksen sekä sen aikana. Tilanteen pitää tehdä asioista jälkeenpäin puhuminen uhreille hankalaksi.

Deittailuskene on ilmiselvä mahdollisuus. Baarit, yökerhot ja deittisivustot ovat ideaalista aluetta sekä nais- että miespsykopaateille. Deittailun popkulttuuri on käsitellyt psykopatiaa jo tovin. He käyttävät kiertoilmaisua ``narsisti\vmq{.}''

Eräällä webbisivulla Susan Walsh käsittelee naispuolista narsismia\linkki{dfgf} ja listaa sellaisen henkilön piirteitä. Ensiksi, fyysinen olemus:
\begin{quotation}
\noindent Pukeutuu provokatiivisesti, rehennellen seksuaalisesti viihjaavilla kehonosilla; keskittää huomion meikkiin ja hiuksiin, jopa kaikista arkipäiväisimmissä tilanteissa; yli-itsevarma ulkonäöstään; pitää brändejä arvossa, ja kokee olevansa oikeutettu pukeutumaan ``parhaaseen;'' osta usein uusia vaatteita, eikä tee eroa haluille ja tarpeille; käyttää tavanomaista todennäköisemmin plastiikkakirurgiaa, tyypillisimmin rintojen suurennusta; tykkää olla valokuvauksen kohteena, ja usein pyytää muita ottamaan itsestään kuvan; jakaa innoissaan itsestään parhaat kuvat sosiaaliseen mediaan.
\end{quotation}
Toiseksi, persoonallisuus ja luonne:
\begin{quotation}
\noindent Vaatii päästä olemaan huomion keskipisteenä, monesti huoneen hurmaavin henkilö; hakee usein myötämielistä kohtelua, ja automaattista myöntymistä; uskoo olevansa erityinen; on erittäin materialistinen; on altis kateudelle, vaikka esittää itsensä äärimmäisen itsevarmana; etsii mahdollisuuksia toisten torpedoimiseen; on varma, että muut ovat kateellisia ja mustasukkaisia hänelle; häneltä puuttuu empatiakyky, ja jopa yleinen kohteliaisuus silloin tällöin; mollaa muita, mukaanlukien sinua; ei epäröi käyttää muita hyväksi; on kilpailuhenkinen; uskoo olevansa älyllisesti ylivertainen; syyttää muita ongelmista; ilmaisee ylimielistä asennetta silloin, kun hänen puolustus on alhaalla tai joku haastaa hänet; on epärehellinen ja usein valehtelee saadakseen mitä hän haluaa; on ``psyko\vmq{,}'' käyttäytyy riskialttiisti, ja omaa addiktoivan persoonallisuuden, ja on taipuvainen agressiiviseen käytökseen torjuttaessa; hänen mielialat ja teot ovat mahdottomia ennustaa.
\end{quotation}
Tämä sopii 95-prosenttisesti moniin naispsykopaatteihin joita tunnen tai olen tuntenut. Kirjoittaja sanoo: ``Pohjautuen kaikenikäisiin naisiin jotka olen tuntenut elämässäni, mielestäni 10\% on tarkka arvio narsistien osuudessa naispopulaatiossa\vmq{.}'' Numero on suuri, mutta se sopii arvoihin, joiden mukaan 10\% populaatiosta on sub-kliinisiä psykopaatteja. Olen vakuuttunut, että tämä petomainen ja tuhoisa ``narsismi\vmq{,}'' jota Walsh kuvailee, on yksi psykopatian maskeista.

Fyysinen olemus on kuin suuri ase, joka tähtää kohti miehen biologiaa. Sen vaikutus voi olla tuhoisa. Floridan yliopiston tutkimuksen mittauksen mukaan 75\% miehistä tarttuu sellaiseen syöttiin. Ehkä osuus on pienempi koko populaatiossa, kuin mitä se on yliopistokampuksella. Mutta deittailutilanteessa useimmat miehet etsivät satunnaista seksiä. Luku liukuu kohti sataa prosenttia.

Kun Mallory metsästää miehiä, hän ei vain kysy jokaista miestä drinkille. Se olisi turhan yksinkertaista. Hän tietää mitä hän etsii. Joten hän voi valita parhaat kohteet, jopa ennen kuin he näkevät hänen olevan paikalla.

Ihmiset reagoivat kuten mikä tahansa muukin elämänmuoto liipaisimiin ja supernormaaliin stimulin. Naiset, jotka pyrkivät vetämään miehiä puoleensa, investoivat oleellisten liipaisimien vahvistamiseen. Tässä on lista liipaisimista, jotka olen onnistunut tunnistamaan ja keräämään:
\begin{description}
\item[Vyöntärön suhde lantioon (WHR).] Tämä on ensisijaine signaali ihmisnaisen seksuaalisuudesta. Ideaalinen WHR liikkuu välillä 0,6--0,8 riippuen kulttuurista. Kapea vyötärö indikoi nuoruutta ja leveät lanteet hedelmällisyyttä. Yksinkertaisin WHR-huijaus on topata lantio. Sitten voi käyttää korsettia, ja sitten kauneuskirurgiaa. Rehellinen vastaliike on pitää tiukempia vaatteita ja näyttää enemmän paljasta pintaa.
\item[Rintojen koko ja muoto.] Ihmisnaisen rintojen kehityksestä käydään paljon väittelyä. Niiden koko ja muoto ei tarkoita suurempaa määrää tai parempilaatuista maitoa. Jotkut ajattelevat että rinnat kehittyivät vauvan tyynyiksi. Jotkut ovat sitä mieltq, että ne imitoivat pakaroita. Minun nähdäkseni ne kertovat naisellisesta nuoruudesta ja saatavuudesta, jotka molemmat ovat liipaisimia miehille. Ennen moderneja aikoja vauvoja imetettiin usein kahdesta kolmeen vuoteen. Rintaruokinta muuttaa rasvakudosta ja koossa pitävää kudosta? (Coopers's Ligaments). Joten rinnat näyttävät välittömästi, onko nainen jo saanut vauvoja vai ei. Vauvat tarkoittavat, että nainen on huonommin saatavilla, ja osoittaa kohti suojelevaa aviomiestä. Kuten WHR:n tapauksessa, huijari voi käyttää toppausta tai kirurgiaa. Ja rehellinen reaktio on taas kerran näyttää enemmän paljasta pintaa.
\item[Muut luotettavat nuoruuden indikaattorit.] Ensinnäkin sileä iho käsissä ja kasvoissa. Mitä sileämmät kasvot, sitä vahvemmin silmät, kulmakarvat ja huulet loistavat. Naiset voivat piilottaa kauneusvirheitä meikillä. He voivat liioitella silmien muotoa, huulia ja kulmakarvoja. Silmiinpistävät piirteet sileällä, virheettömällä iholla ovat liipaisin. Ja täydet huulet, pieni nenä ja korvat. Huulemme ohenevat iän myötä, ja nenämme ja korvamme kasvavat. Nainen voi vaikuttaa näihin ainoastaan plastiikkakirurgialla.
\item[Muut luotettavat hedelmällisyyden indikaattorit,] joita estrogeenihormooni ilmaisee. Tärkeimpiä ovat korkeat poskipäät ja ääni. Korkea, melodinen ääni kertoo hedelmällisyydestä ja nuoruudesta. Havaitse, kuinka jotkut naiset vaihtavat äänensä hiukan korkeammaks pyytäessään palvelusta. Sitä on lähes mahdoton tehdä kuulostamatta epäuskottavalta.
\item[Jalkojen suhde kehoon (LBR).] Mikä tahansa, mikä keskeyttää kasvamisen nuoruudesta aikuisuuteen vaikuttaa LBR:n. Tämä on hyvä geenien, dieetin ja terveyshistorian indikaattori. LBR ennustaa vastustuskykyä monille taudeille diabeteksesta moniin eri syöpiin. Pitkät jalat tarkoittavat terveyttä, ja terveys on seksikästä. Tämä on yksi harvoista liipaisimista, jotka toimivat molemmissa sukupuolissa. Naiset voivat väärentää LBR:ää käyttämällä korkeita korkoja ja lyhyitä hameita.
\item[Muut geneettisen resistanssin ja terveyshistorian indikaattorit.] Nämä ovat symmetriset kasvot, pitkä, puhdas tukka, kirkkaat, säkenöivät silmät, ja terveet kynnet. Tukka ja kynnet on nykyään helppo väärentää, eikä mitään todellista puollusta ole, lukuunottamatta tulitaukoa. Ehkä huivit ja muut ovat kehittyneet tämän takia??. Mascara voi saada silmät näyttämään valkoisemmilta ja säihkyvämmiltä.
\item[Haavoittuvuuden ja alistuvuuden indikaattorit.] Neito pulassa liipaisee petomaisen suojelevan reaktion miehissä.\linkki{sfdfs} Tekstitys kuuluu: ``Pelasta minut ja palkitsen sinut seksillä\vmq{.}'' Tummempi versi on: ``Olen yksin enkä voisi estää sinua vaikka haluaisin\vmq{.}'' Tähän liittyy useampia kehonkielisiä liipaisimia. Jalat yhdessä, ranteet näkyvillä tai velttoina. Pää alhaalla, katsekontaktin välttely. Tai pää alhaalla ja katse ylöspäin, näyttääkseen nuorelta. Ja viimeisenä humaltuneen näytteleminen. Palaan myöhemmin psykopaatteihin ja alkoholiin.
\item[Seksuaalisen saatavuuden ja himon merkit.] Toisin sanoen se, kun nainen kertoo miehelle: ``Haluan sinua, ja haluan seksiä kanssasi juuri nyt\vmq{.}'' Tähän liittyy ainakin kaksi liipaisinjoukkoa. Yksi on kosmeettinen huulien ja poskien punaaminen. Tämä matkii naisen kiihottumisen merkkejä (loistavat huulet ja kasvot). Toinen liipaisin on kehonkieli. Nainen pitää yllä katsekontaktia. Hän liikkuu lähemmäksi miestä. Hän käyttää paljastavaa vaatetusta. Hän vaihtaa asentoaan ja vaatetusta näyttääkseen enemmän paljasta pintaa. Hän koskee miehen käsivartta, leikkii hiuksillaan ja avaa huulensa. Hän nostaa kulmakarvojaan ja sulkee silmänsä puoliksi. Yleensäkin hän käyttäytyy kuin he olisiva sängyssä ja hän nauttisi siitä. Tämä on sosiaaliselle naiselle mahdotonta, lukuunottamatta turvallisissa olosuhteissa tapahtuvaa pelitilannetta. Torjutuksi tulemisen pelko on liian suuri. Psykopaateilla ei ole sellaista pelkoa, joten he voivat viedä tämän näytelmän äärimmäisyyksiin.
\item[Uutuudenviehätys.] Kutsutaan myös Coolidge-ilmiöksi. Siinä missä monogamiset parit ovat normi, monet ihmiset ovat opportunistisia pettäjiä. Useimmat miehet suosivat uusia avoimia seksipartnereita olemassaolevien ohi. Tämä ilmiö on yksi syy sille, miksi pornoteollisuus kaipaa niin kovasti uusia nuoria tähtiä. Kuinka nainen, joka ei pelkää joutuvansa naurunalaiseksi, voi näyttää joka viikko \emph{erilaiselta?} Se on yksinkertaista: hän vaihtaa hiustyyliä ja vaatteita. Uudet hiukset tarkoittavat uusia kasvoja. Uudet vaatteet ovat uusi keho. Molemmat saavat aikaan voimakkaampia reatkioita miehissä, jotka tuntevat naisen jo etukäteen.
\end{description}
Tämä joukko liipaisimia ja rajoittamattomat miesten reaktiot niihin selittää pornografian. Monille miehille se on lähes addiktoivaa. Se on kuin sokeri: jalostetun liipaisimen lähde. Porno näyttää nuorien, saatavilla olevien naisten virran, joka osuu kaikkiin kohtiin ylläolevassa listassa. Pornosivustotilastoissa maailmanlaajuisesti suosituin kategoria on ``teini\vmq{.}''\linkki{sdfds}

Miksi pakkomielle nuoruuteen? Deittailusivusto OKCupid havaitsi, että kaikenikäiset naiset suosivat omanikäistä partneiria. Kaikenikäiset miehet suosivat---silloin kun kukaan ei ole näkemässä---22-vuotiaita tai nuorempia naisia. Deittisivustoprofiilit ja haut voivat vaikuttaa heikolta pohjalta tieteelle. Turiun yliopiston Åbo Akademian? 12 000 suomalaista koskenut tutkimus havaitsi pitkälti saman asian: naiset suosivat omanikäisiään miehiä ja miehet suosivat noin 25-vuotiaita naisia.

Vastaus löytyy lajimme pitkäaikasen monogamisen suhteen mallista. Naisten hedelmällisyys on huipussaan nuorena, joten miehet ovat kehittyneet näkemään 16--22 vuoden iän (18--25, kun muut ovat näkemässä) ``seksikkyyden'' huippuna. Simpansseilla, läheisillä sukulaisillamme, on erillainen perhemalli. Ne elävät laajennetuissa perheissä, ilman monogamisia pareja. Niinpä urossimpanssit eivät liipaistu naisellisesta nuoruudesta. Ja simpanssinaaraat eivät esitä nuoruutta.

Jos olet koskaan pohtinut, miksi miehet ovat niin lumoutuneita naiskauneuteen, tiedät nyt vastauksen. Kuten myöhemmin selostan, naiset ovat aivan yhtä lumoutuneita miehisiin liipaisimiin.

Aikuiset miehet reagoivat oletusarvoisesti näihin supernormaaleihin stimuleihin. He ovat kuin laululintu, joka hoippuu jalkapallokentän kokoisen supersinisen munan pällä. He yrittävät uudelleen ja uudelleen käynnistää seksuaalisia suhteita naisen kanssa. Kohtelipa nainen heitä tai muita kuinka kaltoin, he kiipeävät takaisin munan päälle ja yrittävät uudelleen. Se näyttää järjettömältä. Se voi johtaa itsetuhoon.

Useimmat naiset jotka pukeutuvat vietelläkseen eivät ole psykopaatteja. Useimmissa tapauksissa se on aitoa ja terveellistä. Ero on siinä, kuinka syvällistä harhautus on. On niitä, jotka hiukan vääristävät totuutta, ja sitten on ammattilaisvalehtelijoita. Naispsykopaatit lähettävät seksuaalisuuttaan viekoitellakseen ja laajemmin kuin mihin sosiaalinen nainen kykenee. He käyttävät sitä hallitakseen narratiivia. He provosoivat off-the-charts?? reaktion, ja sitten he pidättelevät. Kaikki menee paljon peliä pidemmälle. Lupaus on: ``Minä ole täydellinen nainen ja minä olen sinun\vmq{.}'' Totuus on: ``Sinä kärsit ja maksat, etkä tule koskaan saamaan sitä ensimmäistä huumaa takaisin, ikinä\vmq{.}''

Haluaisin tarjota lähteitä tälle ilmiölle, mutta se on käsittääkseni dokumentoimaton asia. Olen kokenut sen ja havainnut sen tarpeeksi usein todetakseni, että se on todellinen ja tarkoituksellinen. Ja mekanismi vaikuttaa tutulta. Miehen reaktio naisellisiin seksuaalisiin signaaleihin asuu aivan riippuvuuden naapurissa.

Meidät on johdotettu kokemaan nautintoa näiden liipaisimien painamisesta. Biologiamme toimii sillä tavoin. Dopamiini osuu aivon emotionaalisiin keskuksiin. Ne kokevat iloa. Tämä vahvistaa sitä käytöstä mikä saikaan meidät alunperin liipaisimen äärelle. Kun joku vahvistaa liipaisinta, syntyy suurempi dopamiiniryöppy. Mieli kompensoi muuttumalla vähemmän herkemmäksi. Joten niinpä tarvitsemme lisää liipaisemista kokeaksemme saman ilon tunteen.

Addiktio ei tässä ole metafora. Se on psykopaatin kanssa tapahtuvan seksuaalisen suhteen ydin. Ja sellainen suhde on yhtä mahtava ja terveellinen kuin kokaiiniin tai raakaan viinaan pohjautuva elämä.

Psykopaatit suosivat sellaisia liipaisimia, joita he voivat jäljitellä tai vahvistaa keskitetyllä vaivannäöllä. Joten Mallory voi olla jokseenkin koruton, ja kuitenkin haltioiva silloin kuin hän niin haluaa. Selostin tapoja huijata useampia eri liipaisimia. Jokaiseta huijausta kohti---sanotaan vaikka, että nainen valehtelee ikänsä---on olemassa rehellinen pelaaja. Vaikkapa nuorempi nainen joka kilpailee samoista miehistä. Tämä on hidas, muinainen kilpavarustelu eri strategioiden välillä.

Naispsykopaatit näkevät enemmän vaivaa ulkonäön eteen, ja vähemmän vaivaa ystävien ja perheen eteen. He jäljittelevät liipaisimia, joita sosiaaliset naiset eivät voi tai halua jäljitellä. Sosiaaliset naiset sen sijaan kilpailevat heidän todellisilla antimillaan. Tämä kilpavarustelu kiittää autenttisia naisten painelemia miesten liipaisimia. Eli täydet rinnat, leveät lanteet, pitkät hiukset ja sileä iho. Ne ovat seksuaalivalinnan ja huijareiden ja rehellisten pelaajien välisen kilpavarustelun hedelmä.

\section{Miehet metsästämässä naisia}

Naiset ovat tottakai erilaisia, ja he ovat immuuneja halvalle imartelulle ja jäljitellyille pullistumille. Naisreaktiot vierailta miehiltä tuleviin seksiehdotuksiin ovat lähellä nollaa. Nainen käsittelee sellaista tarjousta yleensä vihamielisenä tekona. Hän on taipuvainen soittamaan apua poliisilta tai miespuolisilta ystäviltä.

Kyse on kuitenkin vain kontekstista. Kun liipaisimet ovat kohdallaan, useimmat naiset reagoivat. He kävelevät kohti heikkoja liipaisimia. He hölkkäävät kohti vahvoja liipaisimia. Ja he harppovat itsetuhoisen draaman saattelemana kohti supernormaaleita liipaisimia. Aivan kuten miehet.

Joten mitkä ovatkaan näitä liipaisimia? Mikä tekee miehistä viehättäviä naisten silmissä? Kyse on jossain mielessä iänikuisesta mysteeristä. Mutta vastaus in ilmeinen, kun sen vain näkee.

Biologia ja empiirinen tutkimus sulkee pois monia ilmiselviä vaihtoehtoja. Saatavuus ja haluukkuu seksiin eivät ole liipaisimia. Nuoruus ei ole liipaisin. Miehet näyttävät geeninsä, terveyshistoriansa ja hedelmällisyytensä aivan kuten naisetkin, joten jonkinlaista päällekkäisyyttä on. Pitkät jalat, hyvänmuotoiset pakarat, symmetriset kasvot, kiva tukka, vahva leka, korkeat poskipäät. Nämä toimivat samoin molemmissa sukupuolissa.

Mutta ulkonäkö toimii ainoastaan yhdessä muiden liipaisimien, kuten itsevarmuuden ja hurmaavuuden, kanssa. Yksin miehien ulkonäkö ei ole liipaisin naisille. Sosiaaliset naiset eivät koe hyvännäköistä mutta epävarmaa miestä viehättäväksi. Molemman sukupuoliset psykopaatit vaikuttavan pitävän sellaisilla miehillä leikkimisestä.

Mitä piirteitä naiset suosivat miehissä?

Useimmat naiset kilpailevat ollakseen kaikista kauneimman ja nuorimman näköinen. Kilpailu voi olla hunnutettua ja tiedostamatonta. Mutta se on läsnä kaikkialla, koska siihen miehet reagoivat.

Mistä miehet kilpailevat? Mikä on se asia, jonka kerryttämiseksi ja kiinnipitämiseksi miehet taistelevat elämänsä ajan? Se ei ole nuori ulkonäkö, lukuunottamatta joitakin kummallisia miehiä. Se ei ole pitkät hiukset. Eikä suorat piirteet eivätkä pitemmät jalat.

Miehet kilpailevat vallasta ja sen edustajasta, rahasta. Aivan kuten naisten ulkonäkökilpailussa, kilpailu voi ola hunnutettua ja jopa alitajuntaista. Mutta se on kaikkalla. Työssä, urheilussa, sosiaalisessa toiminnassa. Miesvalta ottaa monia eri muotoja. Se voi olla fyysistä, älyllistä, taloudellista. Jopa miehet, jotka eivät eksplisiittisesti kilpaile, ottavat kantaa.

Miesviehättävyydessä näyttää olevan kaksi pääteemaa. Yksi on hallitsevuus: pituus, matala ääni, itsevarmuus, näkyvät karvat kasvoissa, ja kilpailullisuus. Ainakin yksi tutkimus esittää,\linkki{sdfds} että tämä nousee ja laskee naisen hedelmällisyyssyklin mukaan. Toisin sanoen, kun nainen on korkeimmillaan hedelmällisyydessä, hän hakee todennäköisemmin opportunisista seksiä. Sitten naiset suosivat ``miehekkäitä'' miehiä, ja sosiaalisesti hallitsevia miehiä hyvien partnerien tai isien sijaan.

Toinen teema on persoonallisuus, mikä näyttäytyy sellaisina asioina kuin ``älykkyys\vmq{,}'' ``hyvä huumorintaju\vmq{,}'' ja ``on kiltti\vmq{.}'' Kun naiset etsivät pitkäaikaisia kumppaneita, he keskittyvät tähän teemaan. Toisin sanoen, potentiaalisesti hyviin kumppaneihin ja isiin. Jos muotoilet tämän uusiksi muotoon ``on empaattinen ja herkkä\vmq{,}'' se on koodi sille että mies ei ole Mallory.

Miesvalta on liipaisin naisille, ainakin osan ajasta, aivan kuten naishedelmällisyys on liipaisin miehille.








































